%\section{Conclus\~oes} \label{sec:conclusoes}
%
%Na dissertação realizada, foi conduzido um estudo abrangente sobre a previsão da demanda de água por meio da análise de séries temporais. Através da análise exploratória dos dados e da aplicação da decomposição STL, foram identificados padrões sazonais e tendências na demanda de água, fornecendo insights valiosos para o planejamento e gerenciamento eficiente do sistema de abastecimento de água.
%
%Com base nos resultados obtidos, conclui-se que a abordagem de séries temporais é uma ferramenta eficaz para prever a demanda futura de água. Os resultados também indicaram a importância de considerar as flutuações sazonais e as diferentes partes do dia ao determinar a vazão ótima e o volume mínimo de reserva no reservatório.
%
%Apesar dos avanços alcançados nesta pesquisa, é importante ressaltar que existem algumas limitações a serem consideradas. Primeiramente, a análise foi baseada em dados históricos de demanda de água de uma única região, limitando a generalização dos resultados para outras áreas geográficas. Além disso, o estudo não levou em conta fatores externos, como mudanças climáticas ou eventos imprevistos, que podem influenciar a demanda de água.
%
%Para pesquisas futuras, sugere-se abordar essas limitações e expandir o escopo do estudo. Uma proposta seria coletar dados de demanda de água de diferentes regiões e considerar variáveis climáticas e socioeconômicas para aprimorar a precisão das previsões. Além disso, seria interessante explorar técnicas de modelagem mais avançadas, como redes neurais artificiais ou métodos de aprendizado de máquina, a fim de melhorar ainda mais a precisão e eficiência das previsões.
%
%Outra proposta futura seria investigar estratégias adicionais para o gerenciamento eficiente dos recursos hídricos, como a implementação de sistemas de reúso de água, a promoção de práticas de conservação e o desenvolvimento de fontes alternativas de abastecimento. Essas medidas podem contribuir para a sustentabilidade do abastecimento de água e reduzir a dependência de recursos naturais limitados.
%
%Em resumo, esta dissertação proporcionou insights valiosos para a previsão da demanda de água e o gerenciamento eficiente do abastecimento hídrico. Apesar das limitações encontradas, as conclusões desta pesquisa fornecem uma base sólida para futuros estudos e aprimoramentos no campo da gestão dos recursos hídricos, visando garantir um abastecimento de água adequado, sustentável e resiliente às demandas futuras.
%
%\subsection{Propostas Futuras}

Apesar dos resultados promissores evidenciados por esta pesquisa, é essencial que se reconheçam suas limitações e que se instigue a exploração de novos horizontes em pesquisas subsequentes. Uma análise mais profunda e abrangente pode ser realizada, investigando modelos de redes neurais mais avançados. Além disso, a implementação de técnicas de otimização matemática mais refinadas, como o uso do método \textit{Covariance Matrix Adaptation Evolution Strategy} (CMAES), pode ser considerada. Seria prudente incluir cuidadosamente variáveis exógenas em todos os modelos pertinentes, como o uso de variáveis climáticas e dados de precipitação do tempo.
Implementa modelos que utilizam sistemas \textit{fuzzy} para aprimorar a previsão do tanque. Usa essa previsão juntamente com modelos existentes na literatura, como a otimização \textit{Bayesian Optimization Algorithm} (BOA), que não foi abordada neste contexto.


