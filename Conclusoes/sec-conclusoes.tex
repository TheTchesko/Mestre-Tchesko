\section{Conclus\~oes} \label{sec:conclusoes}

Na dissertação realizada, foi conduzido um estudo abrangente sobre a previsão da demanda de água por meio da análise de séries temporais. Através da análise exploratória dos dados e da aplicação da decomposição STL, foram identificados padrões sazonais e tendências na demanda de água.
No decorrer do trabalho, foram estabelecidas bases para responder às questões de pesquisa, com foco específico no consumo de água na cidade de Curitiba. As questões principais desta pesquisa foram abordadas com métodos específicos, divididas em dois estudos de caso. O primeiro estudo, dedicado à adequação da pressão e vazão em uma rede de distribuição de água, utilizou modelos ARIMA, DTR e XGBoost. As questões relacionadas a esse estudo foram eficientemente respondidas.

No segundo estudo de caso, que tratou do impacto do acionamento das bombas durante o horário de pico em uma rede de distribuição de água, a análise se concentrou nos horários em que as pessoas estão em casa e consomem mais água. A análise dos dados foi realizada considerando uma média de 24 horas por dia, o que revelou anomalias, especialmente no intervalo de 18h às 21h.

O objetivo geral do trabalho foi desenvolver modelos de previsão de séries temporais específicos para o abastecimento de água. Embora a literatura aborde diversos modelos de séries temporais, apenas alguns deles são aplicados ao contexto de abastecimento de água. Nesse sentido, foram comparados 19 tipos diferentes de modelos, excluindo o modelo LR devido às limitações observadas na análise de erros em vários passos futuros.

Com base nos resultados obtidos, conclui-se que a abordagem de séries temporais é uma ferramenta eficaz para prever a demanda futura de água. Os resultados também indicaram a importância de considerar as flutuações sazonais e as diferentes partes do dia ao determinar a vazão e o volume mínimo de reserva no reservatório.

Apesar dos avanços alcançados nesta pesquisa, é importante ressaltar que existem algumas limitações a serem consideradas. Primeiramente, a análise foi baseada em dados históricos de demanda de água de uma única região, limitando a generalização dos resultados para outras áreas geográficas. O estudo não levou em conta fatores externos, como mudanças climáticas ou eventos imprevistos, que podem influenciar a demanda de água.





\subsection{Propostas Futuras}

Apesar dos resultados promissores evidenciados por esta pesquisa, é essencial que se reconheçam suas limitações e que se instigue a exploração de novos horizontes em pesquisas subsequentes. Uma análise mais profunda e abrangente pode ser realizada, investigando modelos de redes neurais mais avançados. Além disso, a implementação de técnicas de otimização matemática mais refinadas, como o uso do método \textit{Covariance Matrix Adaptation Evolution Strategy} (CMAES), pode ser considerada. Seria prudente incluir cuidadosamente variáveis exógenas em todos os modelos pertinentes, como o uso de variáveis climáticas e dados de precipitação do tempo.
Implementa modelos que utilizam sistemas \textit{fuzzy} para aprimorar a previsão do tanque. Usa essa previsão juntamente com modelos existentes na literatura, como a otimização \textit{Bayesian Optimization Algorithm} (BOA), que não foi abordada neste contexto.


