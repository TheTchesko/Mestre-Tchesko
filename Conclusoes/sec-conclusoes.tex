\section{Conclus\~oes} \label{sec:conclusoes}

Nesta dissertação, o objetivo era mostrar a escassez de água que ocorreu em Curitiba, tornando possível uma decisão que foi uma adaptação do caso de 12 passos do \citeonline{de2013processo}, que busca e visa o ambiente para ter a visão de que não há interferência do ambiente, e se há esta interferência, ela foi listada como uma variável exógena nos modelos ARX, ARIMAX e SARIMAX, em modelos regressivos, mesmo bom para trabalhar com eles, eu não poderia incluí-los neste momento.  Se o projetista está procurando anomalias nos dados como foi feito aqui, procure os dados de 2020, que foi a grande anomalia na SANEPAR, estas anomalias explicadas nos resultados no capítulo \ref{sec:result}. 




\subsection{Propostas Futuras}

Apesar dos resultados promissores evidenciados por esta pesquisa, é essencial que se reconheçam suas limitações e que se instigue a exploração de novos horizontes em pesquisas subsequentes. Uma análise mais profunda e abrangente pode ser realizada, investigando modelos de redes neurais mais avançados. Além disso, a implementação de técnicas de otimização matemática mais refinadas, como o uso do método \textit{Covariance Matrix Adaptation Evolution Strategy} (CMAES), pode ser considerada. Seria prudente incluir cuidadosamente variáveis exógenas em todos os modelos pertinentes, como o uso de variáveis climáticas e dados de precipitação do tempo.
Implementa modelos que utilizam sistemas \textit{fuzzy} para aprimorar a previsão do tanque. Usa essa previsão juntamente com modelos existentes na literatura, como a otimização \textit{Bayesian Optimization Algorithm} (BOA), que não foi abordada neste contexto.


