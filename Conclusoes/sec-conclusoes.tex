\section{Conclus\~oes} \label{sec:conclusoes}

Nessa dissertação teve como objetivo, conduziu-se um estudo abrangente sobre a previsão da demanda d'água por meio da análise de séries temporais. Através da análise exploratória dos dados e da aplicação da decomposição STL, identificaram-se padrões sazonais e tendências na demanda d'água. Durante o estudo, os modelos DT e Prophet foram empregados para validar o estudo de caso da SANEPAR.

Os objetivos específicos do estudo foram alcançados, contribuindo significativamente para a compreensão e aplicação de modelos de previsão de séries temporais na gestão da demanda d'água.

Aplicar diferentes modelos de previsão de séries temporais utilizando dados provenientes do Bairro Alto em Curitiba, fornecidos pela SANEPAR.
A aplicação de diferentes modelos de previsão, utilizando dados específicos do Bairro Alto em Curitiba fornecidos pela SANEPAR, permitiu avaliar sua precisão, eficiência e capacidade de previsão em conjuntos de dados específicos, utilizando medidas de desempenho para análise.

Avaliar a precisão, eficiência e capacidade de previsão desses modelos em conjuntos de dados específicos, utilizando métricas para análise de desempenho.

Explorar estratégias de otimização baseadas em otimização Bayesiana, empregando o algoritmo TPE para ajustar os hiperparâmetros dos modelos de previsão de séries temporais.
A exploração de estratégias de otimização baseadas em otimização Bayesiana, empregando o algoritmo TPE para ajustar os hiperparâmetros dos modelos, resultou na identificação de combinações eficazes de modelos de previsão de séries temporais em conjunto com a configuração otimizada.

Identificar combinações eficazes de modelos de previsão de séries temporais em conjunto com a configuração otimizada. 

Avaliar o impacto das variáveis exógenas na melhoria da precisão dos modelos de previsão de séries temporais.

Os resultados obtidos demonstram que a abordagem de séries temporais é uma ferramenta eficaz para prever a demanda futura d'água, destacando a importância de considerar as flutuações sazonais e as diferentes partes do dia ao determinar a vazão e o volume mínimo de reserva no reservatório.

Esses objetivos específicos não apenas cumpriram sua função original, mas também forneceram informações para a otimização do planejamento e operação de sistemas de abastecimento d'água, promovendo a resiliência diante de variações sazonais e eventos imprevistos. Essa pesquisa pode ser aplicado na prática, auxiliando gestores e tomadores de decisão na implementação de estratégias mais eficientes para garantir o abastecimento d'água de forma sustentável e resiliente.

O estudo de casos separados em dois tipos de análise de padrões e consumo e o outro estratégia de economia de energia tem como foco melhorar o consumo d'água em horários de pico e otimizar a demanda d'água para que não seja afetada novamente no futuro e falte água novamente. Usando os modelos, consegue-se responder a três questões de pesquisa feitas nessa dissertação. A primeira foi entender os dados se existe tendência, padrão e sazonalidade, para o tratamento que foi realizado, o qual foi muito bom para entender os dados.

A segunda questão já teve a ver com o impacto no acionamento das bombas em horário de pico, confirmando assim que no horário de pico o gasto de energia e a demanda d'água são maiores, mas com uma pequena estratégia pode-se contornar isso facilmente. A terceira questão é quanto deve ser armazenado previamente no reservatório LT01 para que não seja necessário ter esse gasto de energia, $4.445$ litros é o suficiente para que o reservatório passe o horário de pico com o menor gasto energético.

 


\subsection{Propostas Futuras}

Apesar dos resultados promissores evidenciados por esta pesquisa, é essencial que se reconheçam suas limitações e que se instigue a exploração de novos horizontes em pesquisas subsequentes. Uma análise mais profunda e abrangente pode ser realizada, investigando modelos de redes neurais mais avançados. Além disso, a implementação de técnicas de otimização matemática mais refinadas, como o uso do método \textit{Covariance Matrix Adaptation Evolution Strategy} (CMAES), pode ser considerada. Seria prudente incluir cuidadosamente variáveis exógenas em todos os modelos pertinentes, como o uso de variáveis climáticas e dados de precipitação do tempo.
Implementa modelos que utilizam sistemas \textit{fuzzy} para aprimorar a previsão do tanque. Usa essa previsão juntamente com modelos existentes na literatura, como a otimização \textit{Bayesian Optimization Algorithm} (BOA), que não foi abordada neste contexto.


