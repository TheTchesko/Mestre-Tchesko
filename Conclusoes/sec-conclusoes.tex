\section{Conclus\~oes} \label{sec:conclusoes}

Na dissertação realizada, foi conduzido um estudo abrangente sobre a previsão da demanda d'água por meio da análise de séries temporais. Através da análise exploratória dos dados e da aplicação da decomposição STL, foram identificados padrões sazonais e tendências na demanda de água.
Ao longo do estudo, foram empregados os modelos ARIMA, DT e XGBoost para validar o estudo de caso da SANEPAR.

No segundo estudo de caso, que tratou do impacto do acionamento das bombas durante o horário de pico em uma rede de distribuição de água, a análise se concentrou nos horários em que as pessoas estão em casa e consomem mais água.
O objetivo geral do trabalho foi desenvolver modelos de previsão de séries temporais específicos para o abastecimento de água. Embora a literatura aborde diversos modelos de séries temporais, apenas alguns deles são aplicados ao contexto de abastecimento d'água. Nesse sentido, foram comparados 18 tipos diferentes de modelos.

Com base nos resultados obtidos, conclui-se que a abordagem de séries temporais é uma ferramenta eficaz para prever a demanda futura d'água. Os resultados também indicaram a importância de considerar as flutuações sazonais e as diferentes partes do dia ao determinar a vazão e o volume mínimo de reserva no reservatório.
Apesar dos progressos obtidos nesta pesquisa, é crucial destacar algumas limitações a serem consideradas. Primeiramente, a análise fundamentou-se em dados históricos de demanda d'água de uma única região, especificamente o maior bairro de Curitiba. O estudo não considerou fatores externos, como mudanças climáticas ou eventos imprevistos, que poderiam impactar a demanda d'água.





\subsection{Propostas Futuras}

Apesar dos resultados promissores evidenciados por esta pesquisa, é essencial que se reconheçam suas limitações e que se instigue a exploração de novos horizontes em pesquisas subsequentes. Uma análise mais profunda e abrangente pode ser realizada, investigando modelos de redes neurais mais avançados. Além disso, a implementação de técnicas de otimização matemática mais refinadas, como o uso do método \textit{Covariance Matrix Adaptation Evolution Strategy} (CMAES), pode ser considerada. Seria prudente incluir cuidadosamente variáveis exógenas em todos os modelos pertinentes, como o uso de variáveis climáticas e dados de precipitação do tempo.
Implementa modelos que utilizam sistemas \textit{fuzzy} para aprimorar a previsão do tanque. Usa essa previsão juntamente com modelos existentes na literatura, como a otimização \textit{Bayesian Optimization Algorithm} (BOA), que não foi abordada neste contexto.


