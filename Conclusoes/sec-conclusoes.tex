\section{Conclus\~oes} \label{sec:conclusoes}

Nesta dissertação, o objetivo foi analisar a escassez de água em Curitiba e propor uma abordagem baseada nos 12 passos descritos por \citeonline{de2013processo}. Essa abordagem busca compreender o ambiente sem interferências e, quando necessário, considera as variáveis exógenas nos modelos ARX, ARIMAX e SARIMAX. Embora os modelos regressivos sejam adequados para lidar com interferências, sua inclusão não foi realizada nesta etapa.

Para identificar anomalias nos dados, sugere-se consultar os registros de 2020, período em que ocorreu uma grande anomalia na SANEPAR. Os resultados detalhados dessas anomalias podem ser encontrados no capítulo \ref{sec:result}.

\subsection{Propostas Futuras}

Apesar dos resultados promissores evidenciados por esta pesquisa, é essencial que se reconheçam suas limitações e que se instigue a exploração de novos horizontes em pesquisas subsequentes. Uma análise mais profunda e abrangente pode ser realizada, investigando modelos de redes neurais mais avançados. Além disso, a implementação de técnicas de otimização matemática mais refinadas, como o uso do método \textit{Covariance Matrix Adaptation Evolution Strategy} (CMAES), pode ser considerada. Seria prudente incluir cuidadosamente variáveis exógenas em todos os modelos pertinentes, como o uso de variáveis climáticas e dados de precipitação do tempo.
Implementa modelos que utilizam sistemas \textit{fuzzy} para aprimorar a previsão do tanque. Usa essa previsão juntamente com modelos existentes na literatura, como a otimização \textit{Bayesian Optimization Algorithm} (BOA), que não foi abordada neste contexto.


