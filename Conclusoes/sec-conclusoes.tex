\section{Conclus\~oes} \label{sec:conclusoes}

%Na dissertação realizada, foi conduzido um estudo abrangente sobre a previsão da demanda d'água por meio da análise de séries temporais. Através da análise exploratória dos dados e da aplicação da decomposição STL, foram identificados padrões sazonais e tendências na demanda d'água.
%Ao longo do estudo, foram empregados os modelos  DT e Prophet para validar o estudo de caso da SANEPAR.
%
%No segundo estudo de caso, que tratou da estratégia de economia de energia impacto do acionamento das bombas durante o horário de pico em uma rede de distribuição d'água, a análise se concentrou nos horários em que as pessoas estão em casa e consomem mais água.
%O objetivo geral do trabalho foi desenvolver modelos de previsão de séries temporais específicos para o abastecimento d'água. Embora a literatura aborde diversos modelos de séries temporais, apenas alguns deles são aplicados ao contexto de abastecimento d'água. Nesse sentido, foram comparados 18 tipos diferentes de modelos.
%
%Com base nos resultados obtidos, conclui-se que a abordagem de séries temporais é uma ferramenta eficaz para prever a demanda futura d'água. Os resultados também indicaram a importância de considerar as flutuações sazonais e as diferentes partes do dia ao determinar a vazão e o volume mínimo de reserva no reservatório.
%Apesar dos progressos obtidos nesta pesquisa, é crucial destacar algumas limitações a serem consideradas. Primeiramente, a análise fundamentou-se em dados históricos de demanda d'água de uma única região, especificamente o maior bairro de Curitiba. O estudo não considerou fatores externos, como mudanças climáticas ou eventos imprevistos, que poderiam impactar a demanda d'água.


Na dissertação realizada, conduziu-se um estudo abrangente sobre a previsão da demanda d'água por meio da análise de séries temporais. Através da análise exploratória dos dados e da aplicação da decomposição STL, identificaram-se padrões sazonais e tendências na demanda d'água. Durante o estudo, os modelos DT e Prophet foram empregados para validar o estudo de caso da SANEPAR.

Os objetivos específicos do estudo foram alcançados, contribuindo significativamente para a compreensão e aplicação de modelos de previsão de séries temporais na gestão da demanda d'água.

A aplicação de diferentes modelos de previsão, utilizando dados específicos do Bairro Alto em Curitiba fornecidos pela SANEPAR, permitiu avaliar sua precisão, eficiência e capacidade de previsão em conjuntos de dados específicos, utilizando medidas de desempenho para análise.

A exploração de estratégias de otimização baseadas em otimização Bayesiana, empregando o algoritmo TPE para ajustar os hiperparâmetros dos modelos, resultou na identificação de combinações eficazes de modelos de previsão de séries temporais em conjunto com a configuração otimizada.

Os resultados obtidos demonstram que a abordagem de séries temporais é uma ferramenta eficaz para prever a demanda futura d'água, destacando a importância de considerar as flutuações sazonais e as diferentes partes do dia ao determinar a vazão e o volume mínimo de reserva no reservatório.

Esses objetivos específicos não apenas cumpriram sua função original, mas também forneceram informações para a otimização do planejamento e operação de sistemas de abastecimento d'água, promovendo a resiliência diante de variações sazonais e eventos imprevistos. Essa pesquisa pode ser aplicado na prática, auxiliando gestores e tomadores de decisão na implementação de estratégias mais eficientes para garantir o abastecimento d'água de forma sustentável e resiliente.


%Na dissertação realizada, conduziu-se um estudo abrangente sobre a previsão da demanda d'água por meio da análise de séries temporais. Através da análise exploratória dos dados e da aplicação da decomposição STL, identificaram-se padrões sazonais e tendências na demanda d'água. Durante o estudo, os modelos DT e Prophet foram empregados para validar o estudo de caso da SANEPAR.
%
%Os objetivos específicos do estudo foram alcançados, contribuindo significativamente para a compreensão e aplicação de modelos de previsão de séries temporais na gestão da demanda d'água.
%
%A aplicação de diferentes modelos de previsão, utilizando dados específicos do Bairro Alto em Curitiba fornecidos pela SANEPAR, permitiu avaliar sua precisão, eficiência e capacidade de previsão em conjuntos de dados específicos, utilizando medidas de desempenho para análise. 
%
%Além disso, a exploração de estratégias de otimização baseadas em otimização Bayesiana, empregando o algoritmo TPE para ajustar os hiperparâmetros dos modelos, resultou na identificação de combinações eficazes de modelos de previsão de séries temporais em conjunto com a configuração otimizada.
%
%Os resultados obtidos demonstram que a abordagem de séries temporais é uma ferramenta eficaz para prever a demanda futura d'água, destacando a importância de considerar as flutuações sazonais e as diferentes partes do dia ao determinar a vazão e o volume mínimo de reserva no reservatório.




\subsection{Propostas Futuras}

Apesar dos resultados promissores evidenciados por esta pesquisa, é essencial que se reconheçam suas limitações e que se instigue a exploração de novos horizontes em pesquisas subsequentes. Uma análise mais profunda e abrangente pode ser realizada, investigando modelos de redes neurais mais avançados. Além disso, a implementação de técnicas de otimização matemática mais refinadas, como o uso do método \textit{Covariance Matrix Adaptation Evolution Strategy} (CMAES), pode ser considerada. Seria prudente incluir cuidadosamente variáveis exógenas em todos os modelos pertinentes, como o uso de variáveis climáticas e dados de precipitação do tempo.
Implementa modelos que utilizam sistemas \textit{fuzzy} para aprimorar a previsão do tanque. Usa essa previsão juntamente com modelos existentes na literatura, como a otimização \textit{Bayesian Optimization Algorithm} (BOA), que não foi abordada neste contexto.


