%\subsection{Limita\c c\~oes da Pesquisa}
%
%
%Embora o estudo tenha alcançado resultados significativos e sobre o tema em questão, algumas limitações podem ser identificadas. Uma das principais restrições desta pesquisa reside na ausência de exploração de modelos avançados de redes neurais, como LSTM, RNN, GRU, ANN, CNN, Transformer e o modelo Prophet. Esses modelos, amplamente reconhecidos em problemas de processamento de linguagem natural, apresentam atributos distintos que podem potencializar o desempenho e a compreensão dos padrões presentes nos dados.



\subsection{Propostas Futuras}

Apesar dos resultados promissores evidenciados por esta pesquisa, é essencial que se reconheçam suas limitações e que se instigue a exploração de novos horizontes em pesquisas subsequentes. Uma análise mais profunda e abrangente pode ser realizada, investigando modelos de redes neurais mais avançados. Além disso, a implementação de técnicas de otimização matemática mais refinadas, como o uso do método CMAES (do inglês \textit{Covariance Matrix Adaptation Evolution Strategy}), pode ser considerada. Seria prudente incluir cuidadosamente variáveis exógenas em todos os modelos pertinentes, como o uso de variáveis climáticas e dados de precipitação do tempo.
Implementa modelos que utilizam lógica \textit{fuzzy} para aprimorar a previsão do tanque. Usa essa previsão juntamente com modelos existentes na literatura, como a otimização BOA, que não foi abordada neste contexto.


%A modelagem de séries temporais pode ser uma tarefa desafiadora. Por isso, alguns modelos sofrem com \textit{overfitting} e \textit{underfitting}, o que pode tornar os dados coletados enganosos, especialmente quando lidamos com \textit{outliers}. Cada modelo tem sua própria robustez, mas há a intenção de buscar métodos para contornar essas limitações, permitindo que os modelos apresentem melhores desempenhos no contexto em que são aplicados.

%Além disso, ao realizar modelos de redes neurais, cada um deles pode ser ajustado através do Optuna, sendo considerado o melhor dessa biblioteca para cada modelo. No entanto, essa otimização exige um processamento robusto, o que por sua vez implica em um treinamento demorado. Embora os modelos já tenham sido otimizados até o limite das capacidades computacionais atuais, existe a possibilidade de explorar ainda esses modelos.

%Assim, o primeiro passo seria verificar se cada um desses acontecimentos foi devidamente tratado, embora mesmo após a verificação, ainda possam ocorrer. A segunda etapa seria otimizar os modelos com maior precisão utilizando o Optuna, buscando explorar completamente o potencial dessas técnicas, apesar das limitações computacionais atuais.
%
%A incorporação dos modelos LSTM, RNN, GRU, ANN, CNN, Transformer e do modelo Prophet à pesquisa amplia significativamente o escopo da investigação. Notavelmente, o RNN demonstrou sua eficácia nesse contexto. No entanto, é imprescindível salientar a necessidade de uma exploração contínua sobre como melhor integrar variáveis exógenas em todos esses modelos. Essa lacuna no conhecimento ressalta a importância de investigações contínuas neste domínio.

