\subsection{Limita\c c\~oes da Pesquisa e Propostas Futuras}



Embora o estudo tenha obtido resultados significativos e fornecido insights valiosos sobre o tema abordado, algumas limitações podem ser identificadas. Uma das principais limitações dessa pesquisa reside na falta de exploração de modelos de rede neural LSTM (Long Short-Term Memory), CNN (Convolutional Neural Network) e RNN (Recurrent Neural Network), que têm sido amplamente utilizados em problemas de processamento de linguagem natural. Esses modelos possuem características específicas que podem melhorar o desempenho e a compreensão dos padrões presentes nos dados.

Outra limitação desse estudo está relacionada à otimização matemática dos algoritmos de aprendizado de máquina utilizados. Embora tenham sido empregadas técnicas comuns, como a busca em grade (Grid Search) e a validação cruzada (Cross Validation), existem métodos mais avançados que podem ser explorados no futuro. Sugere-se uma análise mais aprofundada de técnicas de otimização, como Optuna, Grid Search com validação cruzada, busca aleatória (Randomized Search) e BayesSearchCV, para encontrar de forma mais eficiente os melhores hiperparâmetros dos modelos e melhorar ainda mais o desempenho preditivo.

Para estudos futuros, recomenda-se também investigar a influência de outros fatores e características nos modelos de aprendizado de máquina aplicados à detecção de fraudes em transações financeiras. Por exemplo, explorar o impacto de informações demográficas dos usuários, dados geográficos ou histórico de comportamento de transações anteriores. Além disso, uma análise mais aprofundada sobre técnicas de feature engineering e seleção de variáveis pode ser realizada, visando identificar quais atributos são mais relevantes para a detecção de fraudes e, assim, melhorar a precisão dos modelos.

Em suma, embora este estudo tenha alcançado resultados promissores, é importante reconhecer suas limitações e abrir caminho para pesquisas futuras que explorem modelos de rede neural mais avançados, técnicas de otimização matemática e fatores adicionais que podem aprimorar a detecção de fraudes em transações financeiras. Essas investigações têm o potencial de aprimorar ainda mais as estratégias de segurança e proteção de instituições financeiras, contribuindo para a mitigação de perdas e prejuízos causados por atividades fraudulentas.