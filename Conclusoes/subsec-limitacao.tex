\subsection{Limita\c c\~oes da Pesquisa}


Embora o estudo tenha alcançado resultados significativos e sobre o tema em questão, algumas limitações podem ser identificadas. Uma das principais restrições desta pesquisa reside na ausência de exploração de modelos avançados de redes neurais, como LSTM, RNN, GRU, ANN, CNN, Transformer e o modelo Prophet. Esses modelos, amplamente reconhecidos em problemas de processamento de linguagem natural, apresentam atributos distintos que podem potencializar o desempenho e a compreensão dos padrões presentes nos dados.

Para estudos subsequentes, recomenda-se também investigar a influência de outros fatores e características nos modelos de aprendizado de máquina aplicados à detecção de fraudes em transações financeiras. Por exemplo, é pertinente explorar o impacto de informações demográficas dos usuários, dados geográficos e histórico de comportamento de transações anteriores. Além disso, uma análise mais aprofundada das técnicas de engenharia de recursos e seleção de variáveis pode ser realizada, objetivando identificar quais atributos possuem maior relevância para a detecção de fraudes e, dessa forma, aprimorar a precisão dos modelos.

\subsection{Propostas Futuras}

Apesar dos resultados promissores evidenciados por esta pesquisa, é essencial reconhecer suas limitações e instigar a exploração de novos horizontes em pesquisas subsequentes. Para aprimorar ainda mais a detecção de fraudes em transações financeiras, recomenda-se uma análise mais profunda e abrangente, que investigue modelos de redes neurais mais avançados, a implementação de técnicas de otimização matemática mais refinadas e a inclusão cuidadosa de variáveis exógenas em todos os modelos pertinentes.

A incorporação dos modelos LSTM, RNN, GRU, ANN, CNN, Transformer e do modelo Prophet à pesquisa amplia significativamente o escopo da investigação. Notavelmente, o RNN demonstrou sua eficácia nesse contexto. No entanto, é imprescindível salientar a necessidade de uma exploração contínua sobre como melhor integrar variáveis exógenas em todos esses modelos. Essa lacuna no conhecimento ressalta a importância de investigações contínuas neste domínio.

Tais pesquisas prospectivas não apenas fortalecerão as estratégias de segurança e proteção das instituições financeiras, mas também têm o potencial de contribuir substancialmente para a mitigação de perdas e danos originados por atividades fraudulentas. Ao abordar as limitações identificadas e ao direcionar o foco para áreas como aprimoramento de modelos, otimização matemática avançada e utilização eficaz de variáveis exógenas, as futuras investigações podem desempenhar um papel crucial na evolução das abordagens de detecção de fraudes no cenário das transações financeiras.