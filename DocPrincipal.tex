
%% COMPLETAR OS DADOS NO ARQUIVO DE INFORMAÇÕES

%% PREAMBULO
\documentclass[12pt, a4paper]{article} %% TIPO DO ARQUIVO
%\documentclass[12pt, a4paper, article]{abntex2} %% TIPO DO ARQUIVO
\usepackage[english, brazil]{babel}     %% LÍNGUA
\usepackage[utf8x]{inputenc}             %% FORMATO DO TEXTO
\usepackage{ucs}
\usepackage[T1]{fontenc}
\usepackage[top=3cm, left=3cm, right=2cm, bottom=2cm]{geometry} %% MARGENS
\usepackage{setspace}                   %% PERMITE ESPAÇO SIMPLES
\usepackage{graphicx,amsmath,dsfont, amssymb,amsfonts}    %% GRÁFICOS, EQUAÇÕES E FONTES
\setlength{\parindent}{1.25cm}          %% RECUO DE 1.25
\usepackage{indentfirst}                %% RECUO DA PRIMEIRA LINHA
\renewcommand{\baselinestretch}{1.3}    %% ESPAÇAMENTO 1.5
\pagestyle{myheadings}                  %% PAGINAÇÃO ABNT
\usepackage{enumitem}                   %% PACOTE DE NUMERAÇÃO
\usepackage{lipsum}                     %% LOREM IPSUM
\usepackage[table,xcdraw]{xcolor}       %% TABELA COLORIDA
\usepackage{multirow}                   %% LINHAS MESCLADAS
\usepackage{float}                      %% PREVENIR FLOAT
\usepackage{placeins}                   %% CRIAR FLOTBARRIER
\usepackage{longtable}                  %% TABELA CONTÍNUA
\newcolumntype{C}[1]{>{\centering\arraybackslash}p{#1}} %% FAZER QUEBRA DE LINHA DENTRO DE TABELA
\usepackage{tikz}                       %% PLOTAR GRÁFICOS TIKZ
\usetikzlibrary{arrows,positioning}     %% ARCOS NO TIKZ
\tikzset{
    %Define standard arrow tip
    >=stealth',
    %Define style for boxes
    punkt/.style={
           rectangle,
           rounded corners,
           draw=black, very thick,
           text width=6.5em,
           minimum height=2em,
           text centered},
    % Define arrow style
    pil/.style={
           ->,
           thick,
           shorten <=2pt,
           shorten >=2pt,}
}
%\usepackage{subfig}                  %% ATIVAR SUBFIGURAS
%\PassOptionsToPackage{subfigure}{tocloft} %% EVITAR ERROS DE SUBFIGURAS
%\usepackage{fixmetodonotes}             %% EVITAR ERROS DE SUBFIGURAS
\usepackage{hyperref}                   %% HYPERLINK
\hypersetup{
    %bookmarks=true,         % show bookmarks bar?
    unicode=false,          % non-Latin characters in Acrobat’s bookmarks
    pdftoolbar=true,        % show Acrobat’s toolbar?
    pdfmenubar=true,        % show Acrobat’s menu?
    pdffitwindow=false,     % window fit to page when opened
    pdfstartview={FitH},    % fits the width of the page to the window
    %pdftitle={My title},    % title
    %pdfauthor={Author},     % author
    %pdfsubject={Subject},   % subject of the document
    %pdfcreator={Creator},   % creator of the document
    %pdfproducer={Producer}, % producer of the document
    %pdfkeywords={keyword1, key2, key3}, % list of keywords
    pdfnewwindow=true,      % links in new PDF window
    colorlinks=true,        % false: boxed links; true: colored links
    linkcolor=black,        % color of internal links (change box color with linkbordercolor)
    citecolor=black,        % color of links to bibliography
    filecolor=black,        % color of file links
    urlcolor=black          % color of external links
}
\usepackage{tocloft}                    %% PONTOS NO SUMÁRIO
\renewcommand{\cftsecleader}{\cftdotfill{\cftdotsep}} %% PONTOS NO SUMÁRIO 
\usepackage[alf, abnt-emphasize=bf]{abntex2cite} %% CITAÇÃO PADRÃO ABNT
\usepackage{pdfpages}                   %% IMPORTAR PÁGINAS PDF

\usepackage{titlesec}                   %% HABILITAR OS TÓPICOS QUATERNÁRIOS
\titleclass{\subsubsubsection}{straight}[\subsection]
\newcounter{subsubsubsection}[subsubsection]
\renewcommand\thesubsubsubsection{\thesubsubsection.\arabic{subsubsubsection}}
\renewcommand\theparagraph{\thesubsubsubsection.\arabic{paragraph}} %% optional; useful if paragraphs are to be numbered
\titleformat{\subsubsubsection}
  {\normalfont\normalsize\bfseries}{\thesubsubsubsection}{1em}{}
\titlespacing*{\subsubsubsection}
{0pt}{3.25ex plus 1ex minus .2ex}{1.5ex plus .2ex}
\makeatletter
\renewcommand\paragraph{\@startsection{paragraph}{5}{\z@}%
  {3.25ex \@plus1ex \@minus.2ex}%
  {-1em}%
  {\normalfont\normalsize\bfseries}}
\renewcommand\subparagraph{\@startsection{subparagraph}{6}{\parindent}%
  {3.25ex \@plus1ex \@minus .2ex}%
  {-1em}%
  {\normalfont\normalsize\bfseries}}
\def\toclevel@subsubsubsection{4}
\def\toclevel@paragraph{5}
\def\toclevel@paragraph{6}
\def\l@subsubsubsection{\@dottedtocline{4}{7em}{4em}}
\def\l@paragraph{\@dottedtocline{5}{10em}{5em}}
\def\l@subparagraph{\@dottedtocline{6}{14em}{6em}}
\makeatother
\setcounter{secnumdepth}{4}
\setcounter{tocdepth}{4}
\usepackage{algorithmic} %% Colocar simbolos matematicos diferente
\usepackage{textcomp} %% TEXTO DA TABELA DIFERENTE
\usepackage{xcolor} %% TABELA COM MAIS COLORIDA
\usepackage{booktabs}
\usepackage{pdflscape}  %% orientação da paigina

\usepackage{siunitx}
\usepackage{ragged2e}

\usepackage{newfloat}



\newenvironment{dedicatoria}{%
	\par\raggedleft\itshape\footnotesize%
}{%
	\par%
}


\usepackage{epigraph} 

% Definindo a lista de símbolos
\usepackage{nomencl}
\makenomenclature

\newcommand{\fonte}[1]{{Fonte: {#1}}}

\usepackage{quoting}
\usepackage{subcaption}
\usepackage{caption}


\usepackage{chngcntr}

\numberwithin{equation}{section} % Numeração das equações em relação à seção
%\numberwithin{equation}{subsection} % Numeração das equações em relação à subseção
%\counterwithin{equation}{subsubsection} % Numeração das equações em relação à subsubseção


\usepackage{fancyhdr}
%
%\pagestyle{fancy}
%\fancyhead[R]{Capítulo \thesection} % Adicione "Seção X" ao cabeçalho
%\renewcommand{\headrulewidth}{0pt} % Remove a linha horizontal superior
%\fancyfoot[C]{} % Limpa o rodapé central
%
%\pagestyle{fancy}
%\fancyhf{} % Limpa todos os cabeçalhos e rodapés anteriores
%\fancyhead[R]{\leftmark} % Exibe o título do capítulo ou seção no cabeçalho direito
%\fancyfoot[C]{\thepage} % Exibe o número da página no rodapé central
%%\usepackage{fancyhdr}
%%%\usepackage{lastpage}
%
\pagestyle{fancy}
\fancyhead[R]{\thepage}
\renewcommand{\headrulewidth}{0pt} % Remove a linha horizontal superior
\fancyfoot[C]{} % Limpa o rodapé central
%
%% Defina um estilo de página personalizado para remover os números de página
%\fancypagestyle{mystyle}{
%	\fancyhf{} % Limpa todos os cabeçalhos e rodapés
%	\renewcommand{\headrulewidth}{0pt} % Remove a linha horizontal no cabeçalho
%	\fancyfoot[C]{} % Limpa o rodapé central
%}


\usepackage{longtable}


% Defina a largura da primeira coluna (Informação)
\newcolumntype{P}[1]{>{\centering\arraybackslash}p{#1}}


\usepackage{makecell}


\usepackage{geometry}
\usepackage{booktabs}
\usepackage{siunitx}
\usepackage{icomma}

\usepackage{icomma}
\def\mycomma{\futurelet\hmmac\mycommabehavior}
\def\mycommabehavior{\ifx,\hmmac\expandafter\mycommaelse\else\expandafter\mycommado\fi}
\def\mycommaelse{\ifmmode\comma\else,\fi\space}
\def\mycommado#1{\ifmmode\comma\else,#1\fi\space}


\usepackage{array}

\usepackage{multirow}
\usepackage{caption}

% Configuração para usar a vírgula como separador decimal
\sisetup{output-decimal-marker={,}}

%% INFORMAÇÕES DO DOCUMENTO
%% COMPLETAR COM OS DADOS

\def\universidade{Pontifícia Universidade Católica do Paraná}

\def\departamento{Escola Politécnica}

\def\curso{Programa de Pós-Graduação em Engenharia de Produção e Sistemas (PPGEPS)}

\def\cidade{Curitiba}

\def\ano{2023}

\def\datadefesa{\today} %% 

\def\aluno{Franchesco Sanches Dos Santos}

\def\orientador{Dr. Leandro dos Santos Coelho}

\def\coorientador{Dr. Viviana Cocco Mariani}

\def\convidadoa{Convidado A} %%

\def\univconvidadoa{Instituição A} %% 

\def\convidadob{Convidado B} %% 

\def\univconvidadob{Instituição B} %% 

\def\titulo{Explorando a Eficiência dos Modelos de Previsão de Séries Temporais no Abastecimento de Água}


%-----------------------------------------------------------------
%% INÍCIO DO DOCUMENTO
\begin{document}

%-----------------------------------------------------------------
%% ELEMENTOS PRE-TEXTUAIS

%% CAPA
\input{Packages/espacamento_paginacao1}
\begin{figure}[H]
	\centering
	\includegraphics[width=0.3\linewidth]{Pretextuais/pucpr}
\end{figure}

\begin{center}
    {\singlespacing
    \MakeUppercase{\universidade} 

    \MakeUppercase{\departamento} 

    \MakeUppercase{\curso} \\ [1.5cm] 
    
    \MakeUppercase{\textbf{\aluno}} \\ [6cm]
    
    \MakeUppercase{\titulo}
    
    \vfill
    
    \MakeUppercase{\cidade} \\ 
    \ano}
\end{center}

%% FOLHA DE ROSTO
\include{Pretextuais/folha_rosto1} %% QUALIFICAÇÃO
%\include{Pretextuais/folha_rosto2} %% DISSERTAÇÃO

%% FICHA CATALOGRÁFICA
%\includepdf{Pretextuais/ficha.pdf}

%% FOLHA DE APROVAÇÃO
\include{Pretextuais/folha_aprovacao} %% NÃO ASSINADA
%\includepdf{Pretextuais/folha_aprovacao1.pdf} %% ASSINADA

%% DEDICATÓRIA
\vspace*{\fill}

\begin{dedicatoria}
	Dedico essa dissertação de mestrado à Deus, essa força maior, que me guia e ilumina meus pensamentos para que eu desenvolva minha luz.
\end{dedicatoria}


%% AGRADECIMENTOS
\begin{center}
    \textbf{AGRADECIMENTOS}
\end{center}

Em primeiro lugar, agradeço a Deus por tudo o que ele tem a oferecer, pois abriu o caminho para mim e me deu forças para superar esse desafio, sem ele nada seria possível.

À minha família, eles sempre me apoiaram e me incentivaram a seguir em frente com a cabeça erguida e buscar um estado mais elevado.

Ao Professor Leandro dos Santos Coelho, agradeço por me dar a oportunidade de trabalhar com ele e de compartilhar seu conhecimento e experiência ao longo do Mestrado, sempre em busca do meu crescimento profissional e pessoal que tornou este trabalho possível.

A Professora Viviana Cocco Mariani, obrigada pela disponibilidade e paciência em me ajudar com minhas deficiências e por utilizar seus conhecimentos para contribuir com o desenvolvimento da pesquisa.

Agradeço à equipe da Pontifícia Universidade Católica do Paraná (PUCPR) e demais professores, em especial a secretária Denise da Mata Medeiros (PPGEPS), por cuidar de mim com paciência e carinho e me ajudar inúmeras vezes, ao invés de medir o esforço despendido.

Aos meus amigos que estiveram torcendo, assim como aos novos amigos que fiz nesta caminhada, que proporcionaram grandes momentos de alegria na batalha.

Graças ao investimento em bolsas concedidas pela Coordenação de Aperfeiçoamento de Pessoal de Nível Superior (CAPES), esta etapa da minha carreira profissional e acadêmica foi concluída.

%% EPÍGRAFE

\vspace*{\fill}
\begin{quote}
	\epigraph{\textit{A imaginação é mais importante que o conhecimento. \\ Pois o conhecimento é limitado, enquanto a imaginação abraça o mundo inteiro.}}{- Albert Einstein}
\end{quote}

%% LISTA DE ABREVIAÇÕES
%\newpage 
%\printnomenclature
%% inserir lista de abreviaturas e siglas

\section*{Lista de Abreviaturas e Siglas}

\begin{tabular}{cp{0.8\textwidth}}
	AdaBoost & Impulso ou Estímulo adaptativo (do inglês \textit{Adaptive Boosting}) \\
	AR & Auto-Regressivo\\
	ARIMA & Média Móvel Integrada Auto-Regressiva (do inglês \textit{autoregressive integrated moving average}) \\
	ARX & Auto-Regressivo com variável Exógena (do inglês \textit{autoregressive with eXogeneous inputs})\\ 
	BrownBoost & Algoritmo de aumento\\
	CNN & Rede Neural Convolucional (do inglês \textit{Convolutional Neural network ou ConvNet})\\
	DBN & Rede de Crenças Profundas (do inglês \textit{Deep Belief Network}) \\
	EFB & Pacote de características exclusivas (do inglês \textit{Exclusive Feature Bundling})\\
	FT & flow transmitter (Transmissor de fluxo)\\
	Hz & Hertz\\
	INMET & Instituto Nacional de Meteorologia\\
	LGBMRegressor & Regressão Ligth GBM\\
	Light GBM & Máquina de Impulso de Gradiente Leve (do inglês Light \textit{Gradient Boosting Machine}) \\
	LogitBoost & Representa uma aplicação de técnicas de regressão logísticas\\
	LPBoost & Reforço da Programação Linear (do inglês \textit{Linear Programming Boosting}) \\
	LR & Regressão linear (do inglês \textit{linear regression})\\
	LSTM & Memória de longo curto prazo (do inglês \textit{Long short-term memory})\\
	$m^3 $ & Metros cúbicos\\
	$m^3/h $ & Metros cúbicos por hora\\
	MadaBoost & Modificando o sistema de ponderação da AdaBoost\\
	MAE & Erro Médio Absoluto (do inglês \textit{Mean Absolute Error})\\
	MAPE & Erro Percentual Médio Absoluto (do inglês \textit{Mean Absolute Percentage Error})\\
	$mca$ & Metros coluna d’água\\
	ML & Aprendizado de máquina (do inglês \textit{machine learning})\\
	mm & Milímetros\\
	MSE & Erro médio quadrático (do inglês \textit{Mean Squared Error})\\
	PR & Estado do Paraná
\end{tabular}

\begin{tabular}{cp{0.8\textwidth}}
	RBAL & Recalque Bairro Alto\\
	RFR & Random Forest Regression\\
	RMSE & Erro de Raiz Média Quadrática (do inglês \textit{Root Mean Squared Error})\\
	RNN & Rede Neural Recorrente (do inglês \textit{Recurrent Neural Network})\\
	SANEPAR & Companhia de Saneamento do Paraná \\
	SARIMA & Auto-Regressivos Integrados de Médias Móveis com Sazonalidade (do inglês \textit{Integrated Auto-Regressive Moving Averages with Seasonality}) \\
	SARIMAX &  Média Móvel Integrada Auto-Regressiva Sazonal com regressores eXogenous (do inglês \textit{Seasonal Auto-Regressive Integrated Moving Average with eXogenous regressors}) \\
	SVM-VAR & Máquinas de vetor de suporte - Vetores Auto-Regressivos\\
	Totalboost & Impulso total\\
	XGBoost & Impulso Gradiente Extremo (do inglês \textit{eXtreme Gradient Boosting})\\
	XGBRegressor & Regressão XGBoost
\end{tabular}




%% RESUMO
\begin{abstract} 
	\noindent Este estudo explora a previsão de séries temporais para a tomada de decisões relacionadas à demanda de água, visando de alguma forma auxiliar no controle eficaz dos recursos hídricos em um ambiente competitivo. Relacionado ao fornecimento de água. A abordagem proposta envolve a utilização de modelos de previsão de séries temporais para melhorar a precisão das estimativas de demanda de água.
	Neste estudo, o abastecimento de água pela SANEPAR (Companhia de Saneamento do Paraná) para o Bairro Alto da cidade de Curitiba é explorado.	
	Avaliando-se modelos de redes neurais artificiais, como GRU (do inglês \textit{Gated Recurrent Unit}), LSTM (do inglês \textit{Long Short-Term Memory}), RNN (do inglês \textit{Recurrent Neural Network}), \textit{Transformer} e Facebook Prophet. Além disso, modelos do tipo ARIMA (do inglês \textit{Auto-Regressive Integrated Moving Average}), técnicas de \textit{boosting} como XGBoost (do inglês \textit{eXtreme Gradient Boosting}) e LightGBM (do inglês \textit{Light Gradient Boosting Machine}), regressão linear e RFR (do inglês \textit{Random Forest Regression}). A eficácia dos modelos é  avaliada por meio de métricas como sMAPE (do inglês \textit{Symmetric Mean Absolute Percentage Error}), MAE (do inglês \textit{Mean Absolute Error}) e RRMSE (do inglês \textit{Root Relative Mean Square Error}).
	A análise e comparação de todos os casos, ficou evidente que o modelo RNN obteve o menor erro em todas as métricas avaliadas, incluindo-se SMAPE, MAE e RRMSE. É interessante notar que o desempenho do modelo RNN foi excepcional, com erros de previsão consistentemente menor que 1\% em todas as análises. 
	
\hspace{1cm}


    \noindent \textbf{Palavras-chave:} Previsão de séries temporais, Abastecimento de água, Aprendizado de Máquina, Redes Neurais Artificiais, Modelos de Previsão.
\end{abstract}



%% ABSTRACT
{\selectlanguage{english}
\begin{abstract}
\noindent  Time series forecasting is very important for decision making.
In this dissertation the problem of water demand that occurred in the city of Curitiba -- PR will be addressed, there was a period of data collected in the years 2018 to 2020, aiming for the year 2020 that was the year that occurred the highest water demand, causing the reservoirs to suffer from this, several factors. 
In the decision making of this problem in question, some methods found in the review that was conducted during this work are used, to be predicted in some forecast horizons, the horizon addressed here is a way to solve the issue of water demand and thus validate the models to see which one is the most efficient, the horizon adopted was the forecast of 1, 7, 14 and 30 days ahead, so that each method will deal with the data over time.
In order to mitigate and solve the problem that SANEPAR faced in the year 2020, so that it doesn't occur again or that it doesn't catch us unprepared in the next event that may arise. With the isolated event that happened in the year in question and may not be repeated in future years, this work aims to improve the use of water.     
The methods derived from ARIMA models, thus listing the models are AR, ARX, MA, ARMA, ARIMA, SARIMA, SARIMAX and ARIMAX, as each model has its own particularity the models with exogenous variables may seem graphically better to be predicted than the ARIMA models without exogenous variables. Gradient boosting models are the best models to predict with the lowest errors. The models called boosting or gradient regression tree, the following models LR, XGboost random forest regression and Light GBM were used, these models for time series are listed as the best models because some of them use the gradient way of predicting.     
It is obtained in some error metrics, the smaller the error the better for decision making. The metrics adopted in this work is MAPE, MAE and RMSE, in time series these metrics are more frequent, with better or more effective forecasting models in some circumstances with in forecasting no future horizon, The XGBoost model has $0.013\%$ error in the MAPE metric just analyzing over this metric, and the LR model has the highest error of $21\%$ in the longest forecast horizon (30 days), the MA model comes with $11.57\%$ error and the LR model with $548.59\%$ error. Thus, the LR model for a smaller data set can be more efficient than the other models, since it works with a small volume of data and the errors get higher as the horizon increases.

    \noindent \textbf{Keywords:} Forecasting, Water economy, Time series, Time series analysis.
\end{abstract}
}


%% LISTA DE TABELAS
\newpage 
\pdfbookmark[1]{\listtablename}{lot}
\listoftables
\cleardoublepage

%% LISTA DE FIGURAS
\newpage 
\pdfbookmark[1]{\listfigurename}{lof}
\listoffigures 


% ---
% inserir lista de símbolos
% ---
%	\newpage
%\section*{Lista de Símbolos}
%
%\begin{tabular}{cp{0.6\textwidth}}
%	$x$ & position \\
%	$v$ & velocity \\
%	$a$ & acceleration \\
%	$t$ & time \\
%	$F$ & force
%\end{tabular}\\


%% SUMÁRIO
\newpage
\pdfbookmark[1]{\contentsname}{toc}
\tableofcontents 

%-----------------------------------------------------------------
%% ELEMENTOS TEXTUAIS

%% INTRODUÇÃO
\setlength{\parskip}{1pt} %% ESPAÇO DEPOIS DE 6pt
\pagenumbering{arabic}  %% PAGINAÇÃO INICIA AQUI
\setcounter{page}{16} % Reiniciar contagem de página para 1
%\pagestyle{plain} % Restaurar numeração de página com estilo "plain"
\clearpage
\pagestyle{fancy}


\section{Introdu{\c c}{\~a}o} \label{sec:int}

Este capítulo apresenta a introdução do que é abordado nesta dissertação, usando modelos ML, dentro destes modelos será abordada a previsão futura dos dados coletados na SANEPAR Curitiba no estado do Paraná, estes dados foram coletados no bairro superior nos anos 2018 a 2020 houve uma escassez de água que afetou a todos em Curitiba.

No uso de séries temporais pensando neste contexto de tomada de decisão, pode-se pensar como a aplicação de modelos ML em séries temporais, usando os modelos mais clássicos encontrados durante uma revisão sistemática do conteúdo, para tabular alguns modelos que são usados na literatura.  




    \subsection{Contexto da Pesquisa} \label{subsec:contexto}
%Torna a análise de séries temporais e previsões valiosas ferramentas para apoiar o processo de tomada de decisão a curto, médio e longo prazo. Devido às não linearidades, sazonalidade, tendência e que podem ocorrer em séries temporais de abastimento de água nos dados temporais, o desenvolvimento de modelos de previsão eficientes é uma tarefa desafiadora \cite{mateus}.
%
%Na Figura \ref{fig:paradigma-ml}, são mostradas as etapas de como deve ocorrer a análise de dados e a seleção dos modelos. Essa seleção pode ser feita de forma que se tenha que escolher o que deve ser previsto na variável. Feito isso, temos a primeira etapa que será dos dados, depois que cada um foi identificado com seus rótulos de entrada e saída. Os dados não podem conter \textbf{NaN} (do inglês \textit{not a number}) ou dados ausentes, o que evita falsos positivos. 

Torna-se evidente que a análise de séries temporais e previsões são ferramentas valiosas para apoiar o processo de tomada de decisão em curto, médio e longo prazo. Devido às não linearidades, sazonalidades e tendências que podem ocorrer nos dados temporais de abastecimento de água, o desenvolvimento de modelos de previsão eficientes torna-se uma tarefa desafiadora \cite{mateus}.

Na Figura \ref{fig:paradigma-ml}, as etapas de como a análise de dados e a seleção dos modelos devem ocorrer são apresentadas. Essa escolha pode ser conduzida de modo a determinar o que deve ser previsto na variável. Feito isso, a primeira etapa envolve a preparação dos dados, garantindo que cada um tenha sido identificado com seus rótulos de entrada e saída. É imperativo que os dados não contenham \textbf{NaN} (do inglês \textit{not a number}) ou dados ausentes, evitando assim falsos positivos.

\begin{figure}[!htb]
	\centering
	\caption{Paradigma de aprendizado de máquina}
	\includegraphics[width=\linewidth]{Introducao/Figuras/paradigma-ml}
	
	\fonte{Adaptado de \cite{apmonitor}}
	\label{fig:paradigma-ml}
\end{figure}

Ao realizar essa etapa, a pessoa deve visualizar os dados para garantir que estejam carregados corretamente e em um tamanho tolerável, o que é conhecido como avaliação dos dados. Uma vez que os dados estejam limpos e devidamente carregados, sem falsos positivos, a divisão dos dados pode ser efetuada.
A otimização dos dados para os modelos pode ocorrer de diversas maneiras, como a utilização da biblioteca Optuna em Python, que emprega a otimização Bayesiana para cada modelo pré-listado, reduzindo assim o tempo de processamento.

A validação é uma prática comum em conjuntos de dados extensos, permitindo que os modelos interajam mais eficientemente com os dados e proporcionem resultados mais precisos. Após essa etapa, na escolha dos modelos, há a possibilidade de determinar se o modelo é de série temporal, classificação, agrupamento ou regressão. Posteriormente, ao listar os modelos, cada um deles deve passar por uma avaliação com métricas específicas para verificar a precisão de seus resultados.

%Fazendo isso, os dados podem ser visualizados para garantir que estejam bem carregados e que estejam em um tamanho tolerável. Isso é chamado de avaliação dos dados. Com os dados limpos e bem carregados, sem falsos positivos, a divisão dos dados pode ser feita. 
%
%A otimização dos dados para os modelos pode ser realizada de várias formas, como o uso da biblioteca do python Optuna que usa otimização Bayesiana para cada modelo pré-listado, reduzindo assim o tempo de processamento. 
%
%A validação é comum em conjuntos de dados muito grandes para permitir que os modelos trabalhem mais com os dados, proporcionando resultados mais precisos. Após essa etapa, na escolha dos modelos, há a possibilidade de escolher o modelo de série temporal, se o modelo é de classificação, agrupamento ou regressão. Após listar os modelos, cada um deles deve ser avaliado em métricas para verificar a veracidade de cada um.



  
      
\subsubsection{Motiva\c c\~ao da Pesquisa} \label{subsubsec:motivacao}

A motivação desta pesquisa baseia-se na situação enfrentada por Curitiba e região metropolitana, conforme destacado por \cite{vasconcelos_2020}. A região passou por um rodízio de abastecimento de água, com períodos de 36 horas com abastecimento de água, seguidos por 36 horas sem abastecimento. A média geral dos reservatórios na região estava em torno de $27,96\%$ de sua capacidade. Além disso, a quantidade de chuva nos anos anteriores, em $2020$, foi de $1.704$ mm, superando a média anual de precipitação de $1.490$ mm.

Diante dessa situação, a pesquisa tem como abordagem principal a previsão do abastecimento de água, associada a condições de seca ou decorrentes das consequências da COVID-19. A partir dos dados coletados pela SANEPAR, é possível realizar uma análise detalhada, com o objetivo de prever e evitar a ocorrência de escassez de água. 
 
%A motivação desta pesquisa é baseada na situação enfrentada por Curitiba e região metropolitana, conforme apontado por \cite{vasconcelos_2020}. A região passou por um rodízio de abastecimento de água, com períodos de 36 horas com abastecimento de água seguidos por 36 horas sem abastecimento de água. A média geral dos reservatórios na região estava em torno de $27,96\%$ de sua capacidade. Além disso, a quantidade de chuva nos anos anteriores, de $2020$, foi de $1.704$ mm, superando a média anual de precipitação de $1.490$ mm.
% 	
%Diante dessa situação, a pesquisa tem como abordagem principal a previsão do abastecimento de água, que pode ser associada a condições de seca ou decorrentes das consequência da COVID-19. A partir dos dados coletados pela SANEPAR, é possível realizar uma análise detalhada, com o objetivo de prever e evitar a ocorrência de escassez de água. 
%    
     
    
          
    \subsection{Objetivo geral} \label{subsec:objetivos}

O objetivo desta pesquisa é identificar o modelo mais adequado de séries temporais para abordar a escassez de água em Curitiba. Ao longo da dissertação, serão avaliados diversos modelos de regressão, com destaque para os modelos de redes neurais e o Prophet, conforme listados. É importante mencionar que a pesquisa enfatizará os modelos de \textit{gradient boosting}, amplamente reconhecidos na literatura por sua eficácia na previsão de séries temporais. Os principais modelos analisados incluem o ARIMA e suas variações mais contemporâneas. Além das previsões, também serão realizadas análises de anomalias nos dados, buscando compreender as causas subjacentes a essas ocorrências
 
    
    
    \subsubsection{Objetivos espec\'ificos e quest\~ao de pesquisa} \label{subsubsec:obespec}
    
Neste estudo, busca-se identificar e compreender possíveis anomalias nos dados, bem como investigar as causas por trás dessas ocorrências. O objetivo é responder às perguntas de pesquisa relacionadas a essas anomalias.

\begin{enumerate}[start=1, label={\textbf{Q} \arabic*}]
	\item \label{q1} Qual é a adequação da pressão atual para atender à demanda diária?
	\item \label{q2} Qual é o volume mínimo de água necessário no reservatório para evitar o acionamento das bombas durante o horário de pico? 
	\item \label{q3} Qual é a vazão ótima para atender à demanda diária?
	\item \label{q4} Como encontrar o ponto de equilíbrio entre a demanda e a vazão?
	\item \label{q5} Qual é o impacto do acionamento das bombas durante o horário de pico?
	 
	\begin{enumerate}[label=\alph*.]
	\item \label{q5:a} Qual é o nível ideal no reservatório para evitar a ativação das bombas da SANEPAR durante o período de maior demanda, das 18h às 21h, sem comprometer o abastecimento de água para a população? Além disso, como variam as médias das vazões nos horários críticos (18h às 21h) para as diferentes estações do ano (Outono, Inverno, Primavera, Verão)? 
	\item \label{q5:b} Existe tendência, padrão, sazonalidade para os dados destes três anos do Bairro Alto?
	\item \label{q5:c} Identificar quais os horários de maior demanda das $18$ às $21$?
	\item \label{q5:d} Quanto tenho que armazenar previamente no reservatório para não acionar as bombas no horário de pico?
	\item \label{q5:e} Se a vazão cresce e a pressão decresce temos uma ANOMALIA na rede (com base no histórico).	
	\end{enumerate}
\end{enumerate}

    
    \subsection{Descri\c c\~ao do Problema} \label{subsec:descricao}

A descrição do problema é fundamental para obter uma compreensão mais precisa do que está sendo abordado neste trabalho. É por meio dessa descrição que as variáveis-chave são expostas e o objetivo da previsão é estabelecido de forma clara. Sem um plano estruturado para determinar o que deve ser previsto, torna-se difícil justificar o uso de modelos de previsão de dados. Portanto, é essencial estabelecer um propósito claro e definir as metas da previsão antes de aplicar os modelos adequados.

\begin{itemize}
	\item Bombas de sucção (B1, B2 e B3) – valor máximo da frequência 60 Hz
	
	\item Nível do Reservatório (Câmara 1) LT01 $ (m^3) $ - \textbf{PREVER}
	
	\item Vazão de entrada (FT01) $ (m^3/h) $
	
	\item Vazão de gravidade (FT02) $ (m^3/h) $
	
	\item Vazão de recalque (FT03) $ (m^3/h) $
	
	\item Pressão de Sucção (PT01SU) (mca)
	
	\item Pressão de Recalque (PT02RBAL) (mca)
\end{itemize}

A pesquisa fará uso da variável LT01, que representa o nível do reservatório e desempenha um papel de extrema importância, como evidenciado pelas Figuras \ref{fig:dados-todos} e \ref{fig:2020-a-frente}. Essas figuras retratam as anomalias ocorridas durante o período em que a capital paranaense foi afetada pela escassez de chuvas, resultando na redução do nível dos reservatórios e na implementação de rodízios periódicos, conforme discutido na subseção \ref{subsubsec:motivacao}. Assim, tais observações permitem uma compreensão mais aprofundada das perspectivas futuras.

\subsection{Procedimentos Metodol{\'o}gicos} \label{subsec:metod}

Com o intuito de realizar previsões e fazer comparações entre os modelos obtidos na revisão sistemática, será adotado um processo metodológico bem definido. Tal processo está detalhado na subseção \ref{subsubsec:etp} desta seção, onde foram estabelecidas as etapas a serem seguidas. Isso inclui a definição do que será previsto, bem como a seleção dos métodos a serem utilizados na Análise Exploratória de Dados (EDA).
   

\subsubsection{Etapas da Pesquisa}\label{subsubsec:etp}

%A pesquisa foi conduzida seguindo as etapas delineadas:
%
%\begin{figure}[!htpb]
%	\centering
%	\caption{Mapa das Etapas}
%	\label{fig:etapas}
%	\includegraphics[width=1\linewidth]{Introducao/Figuras/Etapas}
%	
%	\fonte{De autoria própria}
%\end{figure}

\begin{enumerate}[start=1, label={\textbf{Etapa} \arabic*}]
	
	\item \label{etp:1} \textbf{Análise Exploratória de Dados (EDA)}: Nesta etapa inicial, compreende-se abrangentemente as características dos dados. As tarefas envolvem a identificação de valores ausentes, a observação de padrões temporais e a detecção de anomalias. Gráficos de linha são comuns para visualizar a convergência dos dados e desvios potenciais \cite{Rostam2021108249}.
	
	\item \label{etp:2} \textbf{Definição de Variáveis Preditoras e Variável Alvo (MISO)}: Na segunda etapa, as variáveis preditoras e a variável alvo para a previsão de Múltiplas Entradas e Uma Saída (MISO) são selecionadas. Diferentes modelos, podem incorporar variáveis exógenas na modelagem. Essas variáveis adicionais aprimoram as capacidade de previsão do modelo, especialmente quando o horizonte de previsão se estende além dos dados históricos \cite{PAWLOWSKI202298}. 
	
	\item \label{etp:3} \textbf{Decomposição STL}: O método de decomposição STL (do inglês \textit{Seasonal and Trend Decomposition Using Loess}) separa uma série temporal em três componentes: sazonalidade, tendência e resíduo. Essa decomposição permite uma analisa separada das diferentes influências presentes nos dados. A componente sazonal representa variações periódicas e repetitivas, a componente de tendência indica a direção geral dos dados ao longo do tempo, e a componente de resíduo captura variações não explicadas pelas componentes anteriores \cite{DEOLIVEIRA2018776}.
	
	\item \label{etp:4} \textbf{Divisão dos Dados}: É prática comum dividir o conjunto de dados em conjuntos de treinamento, validação e teste para avaliar o desempenho do modelo. Essa divisão permite uma análise abrangente e objetiva das habilidades de generalização dos modelos, evitando problemas de ajuste excessivo ou insuficiente. A proporção de alocação pode variar, mas uma abordagem comum é alocar 70\% para treinamento e validação, e os 30\% restantes para o conjunto de testes. A porção de treinamento e validação pode ser subdividida em 80\% para treinamento e 20\% para validação \cite{Tao2020}.
	
	\item \label{etp:5} \textbf{Modelagem e Seleção do Modelo}: Nesta etapa, diversos modelos são construídos e avaliados. Alguns modelos comumente usados para previsão de séries temporais incluem ARX (do inglês \textit{Auto-Regressive with Exogenous Inputs}), AR (do inglês \textit{Auto-Regressive}), MA (do inglês \textit{Moving Average}), ARIMA, SARIMA (do inglês \textit{Seasonal Auto-Regressive Integrated Moving Averages}), SARIMAX (ARIMA Sazonal com variáveis exógenas) e modelos de aprendizado de máquina como RNN, LSTM (do inglês \textit{Long Short-Term Memory}), GRU (do inglês \textit{Gated Recurrent Unit}), Transformer (Transformador), DTR (do inglês \textit{Decision tree regressor}), LR (do inglês \textit{Linear Regression}), XGBoost (do inglês \textit{eXtreme Gradient Boosting}), Light GBM (do inglês \textit{Light Gradient Boosting Machine}) além do Prophet. A escolha do modelo final baseia-se em critérios como desempenho na validação, simplicidade do modelo e interpretabilidade dos resultados.
	
	\item \label{etp:6} \textbf{Validação e Ajuste do Modelo}: Após a construção do modelo, é importante avaliar seu desempenho usando dados de validação. Métricas de avaliação como Erro Médio Absoluto (MAE), Erro Médio Percentual Absoluto Simétrico (sMAPE) e Raiz do Erro Médio Quadrático Relativo (RRMSE) podem ser usadas para comparar e selecionar o melhor modelo. Além disso, técnicas de ajuste como otimização de hiperparâmetros e refinamento do modelo usando dados de treinamento e validação combinados podem melhorar o desempenho do modelo selecionado.
	
	\item \label{etp:7} \textbf{Previsão e Avaliação}: Com o modelo final ajustado, é possível fazer previsões para o conjunto de testes, que representa dados futuros não observados. Essas previsões são comparadas com os valores reais correspondentes para avaliar a qualidade e precisão do modelo. Métricas de desempenho mencionadas anteriormente (MAE, RRMSE, sMAPE) podem quantificar a precisão do modelo e compará-lo com outros modelos ou abordagens.
	
	\item \label{etp:8} \textbf{Teste de Significância}: Aplicar os modelos de previsão e fazer comparativo baseado em testes de significância estatística (\textit{Friedman e Nemenjy})

	
\end{enumerate}

Cada uma dessas etapas desempenha um papel crucial na pesquisa e no processo de modelagem de séries temporais, contribuindo para a compreensão dos dados, construção e validação dos modelos, além de previsões precisas.




    
    
    \subsection{Justificativa da pesquisa} \label{subsec:justif}

No decorrer dessa dissertação ocorre da seguinte forma, para que possa ser previsto e para que seja evitado a efetiva falta d'água, e como pode ser solucionado esse problema para não voltar a acontecer.

\subsubsection{Contribui\c c\~oes} \label{subsubsec:Contribuição}

Seguindo as questões de pesquisa feito na subseção \ref{subsubsec:obespec} tem duas contribuições, a primeira levando em conta a demanda d'água na cidade de Curitiba, entre a \ref{q1} a \ref{q4} é feito a previsão da demanda d'água, as outras ficam em como é o consumo d'água na cidade e gasto com energia no período de pico, mostrado na \ref{q5}\ref{q5:a} a \ref{q5}\ref{q5:e}.

Assim usando os métodos escolhido de previsão de series temporais, como os modelos ARIMA e ARIMA atualizado, como os modelos ARMA, SARIMA, ARIMAX e SARIMAX, outros modelos mais simples que vem do modelo ARIMA, como, por exemplo, os modelos AR, ARX e MA para previsão mais precisa como na \ref{q5} em diante os modelos regressivo ou modelos de gradiente, modelos regressivo testado aqui foi os modelos LR e floresta aleatória, para os modelos de gradiente foi usado XGBoost e Ligth GBM se torna uma opção mais viável na hora de tomar a decisão em meio aos gastos de energia e água que a empresa SANEPAR teve e com o intuito de minimizar esses gasto. Foi estabelecido os horizonte de previsão para que possa ser tomado a melhor decisão a respeito da demanda d'água.

Em ambas das contribuições foi realizado o tabelamento tanto em curto prazo (1 a 30 dias, um mês) até longo prazo (30 a 60 dias, dois meses). Para que assim o melhor modelo tanto em curto quanto em longo prazo seja mostrado e evidenciado. Os modelos ARIMA para o problema em questão em horizonte de previsão de longo prazo se sai melhor que os modelos de reforço de gradiente, modelos de gradiente é mais viável em previsão de curto prazo, por exemplo de 1 dia a frente até uma semana. E ainda sim os modelos ARIMA ou que os modelos que se vem dele supera os gradiente.


    
    \subsection{Estrutura do trabalho} \label{subsec:estrutura}

 Este documento está estruturado em~\ref{sec:conclusoes} capítulos, divididos da seguinte forma:
   
    \begin{figure}[H]
    	\centering
    	\caption{Estrutura da dissertação}
    	\label{fig:estrutura}
    	\includegraphics[width=0.7\linewidth]{Introducao/Figuras/Estrutura}
    	
    	Fonte: Elaboração própria 
    \end{figure}
O capítulo~\ref{sec:int} apresenta a introdução do trabalho, contendo a contextualização, a motivação, o objetivo geral, os objetivos específicos e a questão de pesquisa, a descrição do problema, a metodologia utilizada, a justificativa da pesquisa, as contribuições e a organização do trabalho.
O capítulo~\ref{sec:refteo} revisão teórica do trabalho, fazendo uma visão geral dos principais pesquisadores sobre as questões abordadas na pesquisa.
O capítulo~\ref{sec:base} apresenta os modelos que serão trabalhados nos dados coletados.
O capítulo~\ref{sec:result} apresenta os resultados da pesquisa, assim como uma análise dos resultados gerados.
O capítulo~\ref{sec:conclusoes}, finalmente, apresenta as considerações finais da pesquisa e algumas propostas para pesquisas futuras.



    



%% REFERENCIAL


%% Revisão Sitematica da Literatura
\section{Referencial}\label{sec:refteo}


Este capítulo apresentará a base da literatura que foi coletada durante a preparação desta dissertação, embora os resultados sejam um pouco menores do que os de uma tese, ainda são relevantes para o trabalho realizado aqui.


 

\subsection{Detec\c cão de anomalias} \label{subsec:detec}


Detectar anomalias em séries temporais representa um desafio significativo para os previsores, pois requer habilidade em identificar mudanças nos dados mesmo quando não estão claramente evidentes. Nesse contexto, a coleta de dados realizada ao longo do tempo pela empresa SANEPAR revela anomalias mais expressivas do que inicialmente imaginado. A escassez de água que afetou a cidade de Curitiba se prolongou por vários dias, como evidenciado pelos gráficos de linha utilizados na etapa de trabalho mencionada (\ref{etp:1}). Esses gráficos oferecem uma representação visual clara das variações nos níveis de água ao longo do tempo, auxiliando na compreensão da extensão do problema e na necessidade de uma abordagem adequada.

\begin{figure}[H]
	\centering
	\caption{Dados completos com uma frequência média de 24 horas}
	\label{fig:dados-todos}
	\includegraphics[width=0.9\linewidth]{"Introducao/Figuras/dados todos"}
	
	Fonte: Elaboração própria a partir de dados da SANEPAR (2018 a 2020)
\end{figure}

\begin{figure}[H]
	\centering
	\caption{Plotagem de dados para o ano de 2020}
	\label{fig:2020-a-frente}
	\includegraphics[width=0.9\linewidth]{"Introducao/Figuras/2020 a frente"}
	
	Fonte: Elaboração própria a partir de dados da SANEPAR (2018 a 2020)
\end{figure}

Os dados coletados possuem uma dimensão de $26306$ linhas  $9$ colunas. Essa ampla quantidade de dados será utilizada nos modelos descritos na subseção \ref{subsec:metod} para que seja possível prever e analisar as anomalias evidenciadas nas Figuras \ref{fig:dados-todos} e \ref{fig:2020-a-frente}. Essas figuras ilustram visualmente as variações e padrões observados nos dados ao longo do tempo, destacando a importância de explorá-los de maneira apropriada a fim de compreender as anomalias e embasar a tomada de decisões.








\subsection{Revis\~ao sistem\'atica da literatura} \label{subsec:revisão}

As séries temporais aparecem em vários campos do conhecimento, tais como Economia (preços de estoque diários, taxa de desemprego mensal, produção industrial), Medicina (eletrocardiograma, eletroencefalograma), Epidemiologia (número mensal de novos casos de meningite), Meteorologia (chuvas, temperatura diária, velocidade do vento), etc. Ao longo dos anos tem usado ferramentas computacionais para tornar esta previsão mais eficiente, com aprendizagem de máquinas e algumas características que podem ser aplicadas em linguagem computacional através da linguagem \textit{python e R}, as melhores linguagens para trabalhar com séries temporais hoje em dia.

Para entender melhor este conceito de série temporal, suponhamos que um maratonista que esteja correndo há vários anos e uma pessoa sedentária se submeta a uma corrida de, no máximo, $5$ km, ambos corram ao mesmo tempo para que tenham um monitor de frequência cardíaca para que possa ser monitorado por um médico se você pegar os dados desde o início e compará-los com o final da corrida, o maratonista terá uma série mais estacionária porque ele tem o hábito de correr regularmente enquanto a pessoa sedentária terá uma série não estacionária como mostrado na Figura \ref{fig:series}.


\begin{figure}[H]
	\centering
	\caption{Exemplo de séries temporais}
	\label{fig:series}
	\includegraphics[width=1\linewidth]{Revisao/Figuras/séries}
	
	Fonte: \cite{brandão_2020}
\end{figure}


Na figura \ref{fig:series} observa-se que o eixo $x$ representa os dados observados e $t$ para o tempo percorrido.
Além disso, as séries temporais são processos estocásticos por leis probabilísticas, o que significa que há a possibilidade de ser pensado como um conjunto de todas as trajetórias possíveis na Figura \ref{fig:series} é capaz de ser observado para uma variável alvo. Por exemplo, se você lançar um dado qualquer valor inteiro entre 1 e 6, mas apenas um número ocorrerá. Da mesma forma, em séries temporais existem infinitas possibilidades, entre elas apenas uma de acordo com as características que atenderam a esse período e que de fato ocorrerão.

\begin{figure}[H]
	\centering
	\caption{Processo estocástico}
	\label{fig:serie}
	\includegraphics[width=0.8\linewidth]{Revisao/Figuras/serie}
	
	\fonte{Adaptado de \citeonline{pinheiro_2022}}
\end{figure}

Com $Y(t)$ os dados fictícios e $Tempo \ (t)$ a linha do tempo da Figura \ref{fig:series}.

De repente é pensado como um conjunto de todas as trajetórias possíveis que poderiam ser para observar uma variável.


Esta revisão sistemática da literatura, com o tema abordado até agora é sobre séries temporais, considerando o contexto aqui exposto este tema pode ser de grande relevância em diversas áreas, como mostrado na Figura \ref{fig:areas}. Realizando esta análise de séries temporais nos últimos 6 anos para poder observar as melhores realizações neste tema abordado aqui um curto período, mas tendo o tempo não muito a favor, então teve a opção de deixar este tempo específico para buscar artigos.

O objetivo desta revisão é analisar uma literatura menor, mas muito relevante. Como a própria série temporal procura analisar e modelar a dependência e considerando a ordem apresentada nas bases, por exemplo, os maiores autores e o ano de atividade que mais publicaram nos países que têm o maior número de publicações na apresentação das palavras-chave que serão mostradas, o objetivo é rever cada coisa que pode ser usada em uma aplicação de aprendizagem de máquina.

Em todos os artigos observados que tem uma contribuição científica neste trabalho é a análise do conceito de série temporal com o melhor uso das palavras-chave mesmo não tendo uma grande relação na aprendizagem de máquinas podem ser usados estes artigos como base para outros pesquisadores, aqui algumas análises muito simples para alguns leitores. Entretanto, é um ponto de partida para muitos que não conhecem o conceito de séries cronológicas ou revisão sistemática da literatura.


\subsection{Problematiza\c c\~ao da Revis\~ao} \label{subsec: problematização da revisão}

Nesta subseção, é discutido um problema de pesquisa que pode ser compreendido por diversos leitores. A Figura \ref{fig:serie-temporal} apresenta um mapa conceitual das publicações, destacando a importância dos autores como base para esta revisão. Os modelos propostos por esses autores são fundamentais para abordar o problema em questão, uma vez que a previsão em séries temporais é um desafio de grande significado por si só.

\begin{figure}[H]
	\centering
	\caption{Mapa conceitual do problema de pesquisa}
	\label{fig:serie-temporal}
	\includegraphics[width=0.9\linewidth]{Revisao/Figuras/"Série temporal"}
	
	Fonte: Elaboração própria 
\end{figure}

O mapa conceitual apresentado na Figura \ref{fig:serie-temporal} ilustra a relação entre as palavras-chave que estão relacionadas ao problema em questão, proporcionando uma visão clara do que será abordado ao longo do trabalho. Esse mapa contribui para a identificação dos principais tópicos de pesquisa e das questões que serão exploradas posteriormente.

\begin{enumerate}[start=1, label = {\textbf{Q} \arabic* } ]
	\item \label{questão:rev1}Quais os autores que mais publicam sobre o assunto de séries temporais?
	\item \label{questão:rev2}Quais os países que mais publicam sobre o assunto? 
	\item \label{questão:rev3}Quais as áreas que mais publicam sobre o tema?
	\item \label{questão:rev4}Quais são as obras mais influentes na análise de séries temporais?
\end{enumerate}

\subsection{Metodologia}\label{subsec:met da revisão}

Nesta subseção, é fornecida uma explicação detalhada de como a revisão foi conduzida, abrangendo desde a análise do banco de dados até a conclusão final da revisão. São apresentados os passos e critérios adotados para a seleção dos artigos, bem como os procedimentos utilizados para a extração e análise dos dados. A subseção visa esclarecer de forma clara e objetiva todo o processo metodológico empregado durante a realização da revisão.

\begin{figure}[H]
	\centering
	\caption{Etapas da Revisão.}
	\label{fig:rsl}
	\includegraphics[width=0.9\linewidth]{Revisao/Figuras/RSL}
	
	Fonte: Adaptado de \citeonline{MARTINS201671}
\end{figure}


\begin{enumerate}[start=1, label={\textbf{Etapa} \arabic*}]
	
	\item \label{etp:rev-1} A Figura \ref{fig:rsl} apresenta uma adaptação da metodologia proposta por \citeonline{MARTINS201671} para a realização desta revisão sistemática. Inicialmente, foram realizadas buscas nos bancos de dados Scopus, Web of Science e Lens, selecionando algumas bases relevantes para o tema da pesquisa.
	
	
\textbf{Scopus campo de busca}

\textbf{\textit{TITLE-ABS-KEY (``time series forecasting")  AND  TITLE-ABS-KEY (``time series analysis")  AND  ( LIMIT-TO ( DOCTYPE ,  ``ar" ) )  AND  ( LIMIT-TO ( LANGUAGE ,  ``English" ) )  AND  ( LIMIT-TO ( PUBYEAR ,  2022 )  OR LIMIT-TO ( PUBYEAR ,  2021 )  OR  LIMIT-TO ( PUBYEAR ,  2020 )  OR  LIMIT-TO ( PUBYEAR ,  2019 )  OR  LIMIT-TO ( PUBYEAR ,  2018 )  OR  LIMIT-TO ( PUBYEAR ,  2017 ) )}}

\textbf{Web of Science campo de busca}

\textit{\textbf{``times series forecasting" (All Fields) and ``time series analysis" (All Fields)}} (Publication Years: 2022 or 2021 or 2020 or 2019 or 2018 or 2017) (Document Types: Articles) (Languages: English)

\textbf{Lens campo de busca}

\textit{\textbf{Scholarly Works (11) = ( ``time series forecasting" ) AND ( ( ``time series analysis" ) AND ( ``nonlinear forecasting" ) ) }}
Filters: Year Published = ( 2016 - 2022  ) Publication Type = ( journal article  )\\
	
	Para todas as bases de busca, foram considerados os últimos 6 anos, com exceção do Lens, que retornava poucos artigos. Nesta etapa, foram utilizadas palavras-chave que se adequam melhor à pesquisa, como \textit{time series forecasting and time series analysis and nonlinear forecasting}.
	
	\item \label{etp:rev-2} No cruzamento das palavras-chave, obteve-se um número considerável de artigos, sem restringir a área em que cada um pode ser publicado. A Tabela \ref{tb1} apresenta a tabulação dos resultados obtidos, sem excluir duplicatas, que serão tratadas na seção \ref{subesec:resul da revisão}.
	
	\item \label{etp:rev-3} Nesta etapa, é realizada uma avaliação preliminar de cada artigo obtido, sem aplicar nenhum filtro anual nas buscas. Analisar todos os artigos dessa maneira resultaria em um número elevado, por exemplo, no banco de dados Scopus seriam 498 artigos, na Web of Science seriam 140 artigos e no Lens, que retorna poucos artigos, seriam 11 artigos, totalizando 649 artigos sem remover duplicatas. É importante ressaltar que esses artigos passaram apenas pelo filtro de idioma inglês e de serem artigos, visando aprimorar a busca e a tomada de decisões. Ao aplicar o filtro dos últimos 6 anos, obteve-se um número mais gerenciável de artigos para análise. Levando em consideração a diferença entre essa estimativa apresentada na Tabela \ref{tb1} e a quantidade de artigos restantes após a remoção de duplicatas, temos menos de 356 artigos para análise. É válido lembrar que, ao remover as duplicatas, esse número pode diminuir ainda mais, atingindo o objetivo proposto neste trabalho.
	
	\item \label{etp:rev-4} Nesta etapa, é realizada uma análise mais aprofundada do conteúdo dos artigos selecionados, levando em consideração as áreas de especialização e correlação com séries temporais. Como esta revisão está inserida no contexto de um programa de mestrado em Engenharia de Produção e Sistemas, vale a pena analisar a correlação com áreas como Matemática. A Figura \ref{fig:areas} mostra que as áreas mais relevantes para a pesquisa são \textbf{Informática, Engenharia e Matemática}, representando 50\% das publicações. Portanto, a pesquisa está alinhada com a utilização de conceitos matemáticos básicos para realizar uma estimativa do número de artigos que podem ser eliminados. Estima-se que cerca de 481 artigos possam ser excluídos, porém essa estim
	
	ativa não possui uma base sólida. Utilizando o software Mendeley Desktop para obter o número exato de artigos sem duplicatas, chegou-se a um total de 308 artigos.
	
\end{enumerate}

\subsection{Resultados da Busca de Revis\~ao}\label{subesec:resul da revisão}


Nesta seção, são apresentados os resultados da pesquisa, utilizando um software para melhor aproveitamento de cada banco de dados utilizado no trabalho. Inicialmente, é realizada uma análise no \textit{software VOSviewer}.

\begin{figure}[!htb]
	\centering
	\caption{Palavras-chave mais populares na Scopus}
	\label{fig:scopus-09-08}
	\includegraphics[width=0.8\linewidth]{Revisao/Figuras/"scopus 09-08"}
	
	\fonte{Elaboração própria a partir de dados da Scopus (2016 a 2022)}
\end{figure}

A Figura \ref{fig:scopus-09-08} mostra uma lista das palavras mais frequentemente utilizadas como sinônimos ou em conjunto com "time series analysis" nos artigos. A análise da base de dados do Scopus é feita com uma ferramenta que exibe as palavras-chave relacionadas em cada campo de busca, proporcionando uma visão abrangente das correlações com as palavras-chave principais.

Nesse primeiro momento, são obtidas 3.484 palavras-chave, sendo que 212 delas atingem o limite estabelecido. É importante destacar que as palavras-chave base utilizadas são ``\textit{time series forecasting and time series analysis}'' no Scopus.

\begin{figure}[!htb]
	\centering
	\caption{Palavras-chave mais populares na Web of Science}
	\label{fig:web-09-08}
	\includegraphics[width=0.8\linewidth]{Revisao/Figuras/"web 09-08"}
	
	
	
	\fonte{Elaboração própria a partir de dados da Web of Science (2016 a 2022)}
\end{figure}

A análise do banco de dados Web of Science, apresentada na Figura \ref{fig:web-09-08}, também é realizada por meio de uma ferramenta que mostra as palavras-chave relacionadas em cada campo de busca. Mais uma vez, é possível obter uma visão ampla das correlações com as palavras-chave principais.

Nesse primeiro momento, são obtidas 305 palavras-chave, sendo que 13 delas atingem o limite estabelecido. É importante ressaltar que as palavras-chave base utilizadas são \textit{``time series forecasting and time series analysis''} na Web of Science.

O banco de dados Lens não é apresentado aqui, pois, embora seja uma excelente fonte, não retornou muitos resultados na pesquisa realizada. O site do Lens retorna apenas 11 artigos com os filtros aplicados. Na \ref{etp:rev-1} apresenta o campo de busca utilizado nessa pesquisa, resultando nos 11 artigos encontrados.


\begin{table}[!htb]
	\caption{Cruzamento de palavras-chave através da aplicação de filtros de ano e de linguagem}\label{tb1}
	\centering
	\begin{tabular}{@{}cp{2cm}p{1cm}p{2cm}p{1cm}p{2cm}p{2cm}p{2cm}@{}}
		\toprule
		Bases                             & \multicolumn{5}{c}{Palavras Chaves}                                                         & Resultado \\ \midrule
		\multirow{2}{*}{Scopus}           & time   series forecasting & AND & time   series analysis    &     &                         & 490       \\
		& nonlinear forecasting     & AND & time   series forecasting &     &                         & 8         \\
		\multirow{2}{*}{Web   of Science} & time   series forecasting & AND & time   series analysis    &     &                         & 126       \\
		& nonlinear forecasting     & AND & time   series forecasting &     &                         & 14        \\
		Lens                              & time   series forecasting & AND & time   series analysis    & AND & nonlinear   forecasting & 11        \\
		\multicolumn{6}{c}{Total}                                                                                                       & 649       \\ \bottomrule
	\end{tabular}
	
	\fonte{Elaboração própria a partir de dados da Scopus, Lens e Web of Science (2016 a 2022)}
\end{table}


A Tabela \ref{tb1} apresenta as palavras-chave utilizadas em cada base de dados, juntamente com o número de artigos encontrados inicialmente. No entanto, é importante ressaltar que esses dados ainda não foram processados para remover duplicatas. Após a utilização do \textit{software Mendeley} para eliminar as duplicações, restam 308 artigos únicos, os quais serão considerados nesta revisão.


\begin{figure}[htp!]
	\centering
	\caption{Analise das quantidades de artigos em relação aos anos.}
	\label{fig:regressao-linear-dos-artigos-baseados-nos-anos}
	\includegraphics[width=0.9\linewidth]{Revisao/Figuras/"regressão linear dos artigos baseados nos anos"}
	
	\fonte{Elaboração própria a partir de dados da SANEPAR (2018 a 2020)}
\end{figure}


A Figura \ref{fig:regressao-linear-dos-artigos-baseados-nos-anos} apresenta um gráfico que ilustra a relação entre o número de artigos publicados e os anos correspondentes. Foi realizada uma análise utilizando regressão linear para examinar essa relação ao longo do tempo.

A equação de regressão linear obtida é a seguinte:

\begin{eqnarray}
	y(x) &=& 8,3571x - 16,803 \quad \text{com } R^2 = 0,3062\label{eq1}
\end{eqnarray}

Na equação \eqref{eq1}, $y(x)$ representa a equação da reta, onde $x$ é a variável independente que corresponde aos anos. O coeficiente angular da reta é de $ 8,3571$, e o coeficiente linear é de -16.803, indicando o ponto de intersecção com o eixo $y$.

O coeficiente de determinação, $R^2$, é utilizado para avaliar a proporção da variação na variável dependente (número de artigos) que pode ser explicada pela variação na variável independente (anos). Nesse caso, o valor de $R^2$ é de $0,3062$, o que indica que aproximadamente $30,62\%$ da variação nos números de artigos pode ser explicada pela passagem do tempo.

O coeficiente de determinação mede a relação entre a variável dependente e as variáveis independentes, representando a porcentagem da variação explicada pela regressão em relação à variação total. Quando $R^2$ é igual a 1, todos os pontos observados estão exatamente na reta de regressão, indicando um ajuste perfeito, ou seja, todas as variações em $y$ são totalmente explicadas pela variação em $x_n$ através da função especificada, sem desvios em torno da função estimada. Por outro lado, quando $R^2$ é igual a 0, conclui-se que as variações em $y$ são exclusivamente aleatórias e a inclusão das variáveis $x_n$ no modelo não fornece nenhuma informação sobre as variações em $y$.

A fórmula do coeficiente de determinação $R^2$ é dada pela equação:
\begin{equation}
	R^{2}=\frac{\left(\sum X \cdot Y-\frac{\sum X \cdot \sum Y}{n}\right)^{2}}{\left[\sum X^{2}-\frac{\left(\sum X\right)^{2}}{n}\right] \cdot\left[\sum Y^{2}-\frac{\left(\sum Y\right)^{2}}{n}\right]}=(r)^{2}\label{eq2}
\end{equation}
Na equação \eqref{eq2}, $X$ e $Y$ representam as coordenadas no plano cartesiano, como, por exemplo, o par ordenado $(x,y)$. Na análise realizada com a relação entre o número de artigos e os anos, obteve-se um valor de $R^2=30\%$, o que implica que a linha de regressão é influenciada pelo valor encontrado de $R^2$.

Embora seja uma análise simples da relação entre o número de artigos e os anos, essa é uma validação significativa para observar o teste F de significância, que deve ser sempre inferior a $5\%$, também conhecido como valor-p. Com base nesses valores, é possível analisar o significado da linha de regressão e observar que o ano de 2021 foi o ano em que a maioria dos artigos foi publicada sobre o tema das séries temporais.

\begin{table}[H]
	\centering
	\caption{Fator de impacto.}\label{tb2}
	\begin{tabular}{@{}cp{3cm}p{3cm}c@{}}
		\toprule
		Revista cientíica      & Quantidade de plubicação & Qualidade da revista & H-INDEX \\\midrule
		Neurocomputing         & 27                         & A1                     & 143     \\
		IEEE Access            & 18                         & A1                     & 127     \\
		Applied Soft Computing & 12                         & A1                     & 143     \\
		Energies               & 11                         & A2                     & 93      \\
		Energy                 & 11                         & A1                     & 343     \\ \bottomrule
	\end{tabular}
	
	
	\vspace{0.2cm}
	Fonte: Elaboração própria a partir de dados da Scopus, Lens e Web of Science (2016 a 2022)
\end{table}

A Tabela \ref{tb2} apresenta as revistas que mais publicam artigos sobre o tema em questão. É importante destacar que muitas dessas revistas estão localizadas fora do Brasil e têm seus nomes em inglês. No entanto, todas as revistas listadas, incluindo aquelas com um alto fator de impacto, como a categoria Q1, apresentam uma correlação significativa com as áreas de \textbf{informática, engenharia e matemática}.

Essa observação ressalta a importância dessas áreas de especialização na pesquisa sobre séries temporais, uma vez que estão fortemente representadas nas principais revistas científicas. Essas revistas desempenham um papel fundamental na disseminação do conhecimento e no avanço do campo, garantindo a qualidade e o impacto dos artigos publicados. Portanto, é valioso direcionar a atenção para essas revistas, uma vez que são reconhecidas como fontes confiáveis e respeitadas dentro da comunidade científica.


\begin{figure}[H]
	\centering
	\caption{Autores relação entre artigos publicados.	}
	\label{fig:autores-relacao-entre-artigos-publicados}
	\includegraphics[width=1\linewidth]{Revisao/Figuras/"Autores Relação entre artigos publicados"}
	\vspace{0.2cm}
	Fonte: Elaboração própria a partir de dados da Scopus (2016 a 2022)
\end{figure}

\begin{figure}[H]
	\centering
	\caption{Ligação bibliográfica entre os autores}
	\label{fig:autores}
	\includegraphics[width=1\linewidth]{Revisao/Figuras/Autores}
	
	\vspace{0.2cm}
	Fonte: Elaboração própria a partir de dados da Scopus (2016 a 2022)
\end{figure}

Em resposta à questão colocada anteriormente \eqref{questão:rev1}, foi utilizada a Figura \ref{fig:autores-relacao-entre-artigos-publicados} para visualizar de forma mais clara os autores que mais publicaram sobre o tema em análise. O gráfico apresenta um histograma que destaca os autores cujo número de publicações é maior que 4 durante o período de 2016 a 2022. Essa abordagem visa evitar a inclusão de todos os autores e destacar aqueles que tiveram uma contribuição significativa no campo, considerando o critério estabelecido de pelo menos 4 publicações. Dessa forma, é possível identificar os principais autores que se destacam nesse tópico específico, fornecendo uma visão geral da distribuição da produção científica entre os pesquisadores.


\begin{figure}[!htb]
	\centering
	\caption{Mapa mundial da publicação de artigos em todo o mundo}
	\label{fig:mapa-mundi-artigos}
	\includegraphics[width=0.87\linewidth]{Revisao/Figuras/"mapa mundi artigos"}
	
	
	\fonte{Elaboração própria a partir de dados da Scopus, Lens e Web of Sicence (2016 a 2022)}
\end{figure}

A pergunta de pesquisa \eqref{questão:rev2} foi abordada por meio da análise da Figura \ref{fig:mapa-mundi-artigos}, que apresenta os países com maior número de publicações sobre o assunto em escala, ordenados de forma decrescente. Os principais países que se destacam nessa análise são os seguintes: China, com $119$ publicações; Estados Unidos, com $67$ publicações; Índia, com $57$ publicações; Brasil, com $32$ publicações; Espanha, com $28$ publicações; Reino Unido, com $25$ publicações; Austrália, com $24$ publicações; Irã, com $18$ publicações; Malásia, com $17$ publicações; e Canadá, com $16$ publicações.

É importante ressaltar que o mapa não exibe todos os países e seus respectivos números de publicações, mas destaca aqueles com maior produção nesse contexto específico. Essa análise ajuda a identificar os países com maior contribuição científica nessa área de estudo, fornecendo insights sobre os locais onde a pesquisa sobre séries temporais tem sido mais ativa.

\begin{figure}[htpb!]
	\centering
	\caption{Áreas de aplicação do tema}
	\label{fig:areas}
	\includegraphics[width=0.8\linewidth]{Revisao/Figuras/areas}
	\vspace{0.2cm}
	
	\fonte{Elaboração própria a partir de dados da Scopus, Lens e Web of Sicence (2016 a 2022)}
\end{figure}


Para responder à pergunta de pesquisa \eqref{questão:rev3}, foi criado um gráfico circular, apresentado na Figura \ref{fig:areas}, que ilustra as áreas com maior número de publicações durante o período analisado na revisão. A Tabela \ref{tb3} complementa o gráfico, fornecendo os valores específicos de cada área e a quantidade de publicações correspondente.

O gráfico circular oferece uma representação visual clara das áreas que se destacam em termos de produção científica no campo das séries temporais. Ao examinar a tabela, é possível identificar as áreas com maior número de publicações, permitindo uma compreensão aprofundada das principais áreas de conhecimento relacionadas ao tema. Essa análise contribui para uma melhor compreensão da distribuição de publicações e áreas de pesquisa ao longo do período estudado.

\begin{table}[htpb!]
	\centering
	\caption{Áreas e seus valores respetivos de artigos em cada área.}\label{tb3}
	\begin{tabular}{@{}ll@{}}
		\toprule
		Informática                      & 240 \\ \midrule
		Engenharia                       & 174 \\
		Ciências Ambientais              & 94  \\
		Matemática                       & 67  \\
		Neurociência                     & 40  \\
		Medicina                         & 38  \\
		Ciências sociais                 & 38  \\
		Ciências dos Materias            & 34  \\
		Negócios, Gestão e Contabilidade & 33  \\
		Outros                           & 204 \\ \bottomrule
	\end{tabular}

	
	\fonte{Elaboração própria a partir de dados da Scopus, len e Web of Sicence (2016 a 2022)}
\end{table}

Na última pergunta de pesquisa, referente à \eqref{questão:rev4}, foi realizada uma investigação dos artigos mais influentes na revisão. Esses artigos retratam alguns dos métodos utilizados por renomados autores \citeonline{Golyandina2020, Kumar2021, Xie2019, Lara-Benitez2021, Ahmad2018, CarvalhoJr.2019, Tan2021, Liu2019, Liu2021, Rossi2018, Soyer, Martinovic2020a, Ursu2016, Wang2016, Shih2019a, Moon2019, Chou2018, Bergmeir2018, Boroojeni2017, Chou2018a, Coelho2017, Du2020, Sadaei2019, Salgotra2020, Tyralis2017, Vlachas2020, Yang2019a, Shen2020, Sezer2020, Chen2018, Buyuksahin2019, Li2020, Kulshreshtha2020, Samanta2020, Xu2019, Graff2017, Taieb2016}.

Esses artigos abordam diferentes métodos usados pelos autores para previsão de séries temporais e análise não-linear dessas previsões. Eles representam contribuições significativas para o avanço do conhecimento e aplicação prática das séries temporais, oferecendo insights valiosos sobre abordagens eficazes nesse campo. Ao incluir esses estudos influentes na análise, obtém-se uma visão abrangente dos métodos e técnicas mais relevantes na previsão de séries temporais.

No estudo conduzido por \citeonline{Xu2019}, um modelo híbrido foi proposto, combinando o modelo linear AR e LR com o modelo não-linear ARIMA e o modelo DBN. Essa abordagem permite capturar tanto os comportamentos lineares quanto os não-lineares de uma série temporal. Por outro lado, \citeonline{Li2020} comparou o desempenho de previsão da abordagem MAELS com outros modelos de aprendizado de máquina de última geração, como ANN, CNN, RNN, LSTM, GRU, Transformer, Prophet ARIMA e SVM-VAR. As abordagens ANN, CNN, RNN, GRU, Transformer e LSTM são capazes de lidar com dados multivariados de entrada e saída, enquanto o ARIMA utiliza informações passadas para prever o futuro com base em características como autocorrelação e médias móveis.

Dessa forma, por meio dessa revisão sistemática e análise de conteúdo, a pergunta de pesquisa formulada no início do capítulo foi respondida.
Além desses modelos mencionados, também será utilizada a versão atualizada do ARIMA nesta dissertação. Os modelos SARIMA e SARIMAX também serão comparados para determinar qual deles é o mais adequado. Além disso, serão empregados os modelos Light GBM e XGBoost. Quanto às métricas de erro, serão utilizadas MAE, sMAPE e RRMSE, que são amplamente adotadas na literatura. O coeficiente de determinação ($R^2$), mencionado na equação \eqref{eq2}, não é tão comumente utilizado para comparação de modelos de previsão futura.

\subsection{Principais conclus\~ao} \label{subsec:conclusão da revisão}

A conclusão abrangente da pesquisa de revisão revela que diversas bases de dados foram consultadas, como Scopus, Web of Science e Lens. Cada uma dessas bases proporcionou uma quantidade significativa de artigos relevantes, que foram minuciosamente analisados. Essa abordagem permitiu responder à pergunta de pesquisa formulada no início da revisão.

Apesar da base de dados Lens ser menor em comparação com as demais, também foram encontrados artigos relevantes que contribuíram para enriquecer o processo de dissertação. Além disso, o uso de software especializado foi essencial para lidar com a grande quantidade de artigos e suas inter-relações.

No âmbito específico da revisão sistemática, a análise de séries temporais recebeu uma ênfase particular, com um enfoque aprofundado e atualizado nos últimos seis anos. Os resultados obtidos foram altamente relevantes e significativos. Por meio do cruzamento de palavras-chave e da aplicação de filtros específicos, foram selecionados 308 artigos publicados entre 2016 e 2022.

Com o objetivo de refinar ainda mais a análise, foi realizado um filtro adicional com base em áreas de interesse, como matemática, engenharia e informática. Isso resultou na seleção de 481 artigos relacionados a essas áreas, excluindo aqueles de outras áreas não pertinentes.

A pesquisa de revisão realizada foi minuciosa e abrangente, proporcionando uma base sólida de artigos relevantes para o desenvolvimento da dissertação. Os resultados obtidos foram fundamentais para orientar as próximas etapas do trabalho e para alcançar uma compreensão aprofundada do tema das séries temporais.











\section{Base Te\'orica}\label{sec:base}

Para um bom resultado, uma base sólida é fundamental. Este capítulo cobre métricas de erro e modelos regressivos de previsão de modelos, entre outros.

\subsection{M\'etricas de Erros}\label{subsec:metrica}


A métrica MSE é uma das mais utilizadas em aprendizado de máquina. Seu cálculo é feito da seguinte forma:

\begin{eqnarray}
	M S E=\dfrac{1}{n} \sum\left(y_i-\hat{y}_i\right)^2\label{eq:mse}
\end{eqnarray}

MSE é a média da somatória do erro ao quadrado. Subtraímos o que aconteceu, $y_i$, do valor que foi projetado, $\hat{y}_i$. O resultado é o cálculo do erro. Ao elevarmos o erro ao quadrado, estamos evitando que os erros fiquem negativos e, portanto, se subtraiam na somatória.

\textbf{RMSE}

\begin{eqnarray}
	R M S E=\sqrt{\dfrac{1}{n} \sum\left(y_i-\hat{y}_i\right)^2}\label{eq:rmse}
\end{eqnarray}

A vantagem de utilizarmos o RMSE é que, ao computar a raiz quadrada, o erro passa a ter a mesma escala do indicador que estamos trabalhando. Um RMSE baixo, significa que a performance do modelo foi boa, pois o erro se aproxima de zero.

\textbf{MAE}

O MAE é calculado usando o módulo da subtração, obtida entre o valor do que realmente aconteceu e o valor projetado (previsto) e dividi tudo pelo número $n$ de amostras. Com isso, obtêm o erro médio absoluto. Equação do MAE:

\begin{eqnarray}
	M A E=\dfrac{1}{n} \sum\left|y_i-\hat{y}_i\right|\label{eq:mae}
\end{eqnarray}

Sua interpretação é comparável ao RMSE, onde o erro se dá no mesma escala/ordem de grandeza da variável estudada.

Não é possível dizer se o MAE é um indicador melhor ou pior que os dois anteriores.

\textbf{MAPE}

Conhecido como MAPE, é a porcentagem relativa ao valor observado. O cálculo é feito obtendo a somatória da diferença entre o valor que realmente ocorreu com o valor previsto (resultado do erro), dividido pelo valor observado.

\begin{eqnarray}
	M A P E=\dfrac{1}{n} \Sigma\left|\frac{y_i-\hat{y}_i}{y_i}\right|\label{eq:mape}
\end{eqnarray}

O problema é quando o valor observado $y_i$ é igual a $0$, pois é matematicamente impossível dividir por $0$. Sendo uma medição de erro, porcentagens menores são melhores.

Se fizer $1 -$ \textbf{MAPE}, tem a porcentagem de acerto.

\subsection{ARIMA, SARIMA e SARIMAX}\label{subsec:arima}

A previsão da série temporal é um problema difícil sem resposta fácil. Existem inúmeros modelos estatísticos que afirmam superar uns aos outros, mas nunca está claro qual modelo é o melhor.

Dito isto, modelos baseados em ARMA são muitas vezes um bom modelo para começar. Eles podem alcançar pontuações decentes na maioria dos problemas de séries temporais e são bem adequados como um modelo de linha de base em qualquer problema de séries temporais.


O modelo ARIMA, vamos dividi-lo em AR, I e MA.

\subsubsection{Componente auto-regressivo -- AR(p)}

O componente auto regressivo do modelo ARIMA é representado por AR(p), com o parâmetro $ p $ determinando o número de séries defasadas que é utilizado.

\begin{eqnarray}
	Y_t&=&c+\sum_{n=1}^{p} \alpha_n y_{t-n} + \varepsilon_t\label{AR}
\end{eqnarray}

Dos dados pode ser obtido a seguinte previsão no modelo AR(7)

\begin{figure}[H]
	\centering
	\caption{Modelo AR(7)  }
	\label{fig:1-ar}
	\includegraphics[width=1\linewidth]{Modelos/Figuras/0-AR}
	
	Fonte: Elaboração própria a partir de dados da SANEPAR (2018 a 2020)
\end{figure}

\begin{figure}[H]
	\centering
	\caption{ARX (7)}
	\label{fig:1-arx}
	\includegraphics[width=1\linewidth]{Modelos/Figuras/0-ARX}
	
	Fonte: Elaboração própria a partir de dados da SANEPAR (2018 a 2020)
\end{figure}



Onde em \eqref{AR} o $\varepsilon_t$ é ruído branco. Isso é como uma regressão múltipla, mas com valores defasados de $y_t$ como preditores. é referido a isso como um $\operatorname{AR}(p)$ modelo, um modelo auto regressivo de ordem $p$

Da Figura \ref{fig:1-ar}, tem como objetivo mostrar uma previsão de um passo a frente (um dia) nos apêndices \ref{sec:ararxma24} pode ver uma comparação dos AR, MA e o ARX

O modelo ARX é um modelo similar ao AR só coloca as variáveis exógenas do conjunto de dados para melhorar a previsão futura.

O modelo AR pode ser visivelmente um modelo adequado para a previsão que está sendo feito, mas como é um modelo auto regressivo ainda assim com o passar do tempo e da previsão ele vai prever de uma forma linear e não convencional, para um analise mais rápido da série pode se considerar um modelo adequado. Logo mais adiante tem exemplos de casos gerais que pode ocorrer nesse método.

\textbf{AR(0): Ru\'ido branco}

Se definir o parâmetro $p$ como zero (AR($0$)), sem termos autorregressivos. Esta série de tempo é apenas um ruído branco. Cada ponto de dados é amostrado a partir de uma distribuição com uma média de $0$ e uma variância de sigma-quadrado. Isso resulta em uma sequência de números aleatórios que não podem ser previstos. Isso é realmente útil, pois pode servir como uma hipótese nula, e proteger nossas análises de aceitar padrões falso-positivos.

\textbf{AR(1): Caminhadas aleat\'orias e Oscila\c c\~oes}



Com o parâmetro p definido para $1$, vai levar em conta o medidor de tempo anterior ajustado por um multiplicador e, em seguida, adicionando ruído branco. Se o multiplicador é $0$, então temos ruído branco, e se o multiplicador é 1, teremos uma caminhada aleatória. Se o multiplicador estiver entre $ 0 < \alpha < 1 $, então a série temporal exibirá reversão média. Isso significa que os valores tendem a pairar em torno de 0 e reverter para a média depois de regredir a partir dele.

\textbf{AR(p): Termos de ordem superior}

Aumentar ainda mais o parâmetro $p$ significa apenas ir mais para trás e adicionar mais medidores de tempo ajustados por seus próprios multiplicadores. Pode ir o mais longe que poder, mas à medida que aproxima é mais provável que usa parâmetros adicionais, como a média móvel (MA($q$)).

\subsubsection{M\'edia M\'ovel -- MA(q)}\label{subsubsec:ma}
Este componente não é uma média de rolamento, mas sim os atrasos no ruído branco. \citeonline{signal}


MA(q) é o modelo de média móvel e q é o número de termos de erro de previsão defasados na previsão. Em um modelo MA(1), na previsão é um termo constante mais o termo de ruído branco anterior vezes um multiplicador, adicionado com o termo de ruído branco atual. Esta é apenas simples probabilidade mais estatísticas, pois estamos ajustando nossa previsão com base em termos anteriores de ruído branco.

\begin{eqnarray}
	y_t=c+\varepsilon_t+\theta_1 \varepsilon_{t-1}+\theta_2 \varepsilon_{t-2}+\cdots+\theta_q \varepsilon_{t-q}\label{eq:ma}
\end{eqnarray}

De \eqref{eq:ma} onde $\varepsilon_t$ é ruído branco. Refere nos a isto como um modelo de $MA(q)$, um modelo de ordem média móvel $q$. Claro que não observamos os valores de $\varepsilon_t$, por isso não é realmente uma regressão no sentido habitual.

\begin{figure}[H]
	\centering
	\caption{Modelo MA(7) }
	\label{fig:1-ma}
	\includegraphics[width=1\linewidth]{Modelos/Figuras/0-MA}
	
	Fonte: Elaboração própria a partir de dados da SANEPAR (2018 a 2020)
\end{figure}

O modelo MA com o mesmo valor do AR para comparação e torna o modelo mais fácil de ser previsto. Como observado na Figura \ref{fig:1-ma} a previsão graficamente é parecido com o modelo da Figura \ref{fig:1-ar}, só não se compara com a Figura \ref{fig:1-arx}, perceba que esse modelo aparente prever perfeitamente o tempo que foi listado.  

\subsubsection{Modelos ARMA e ARIMA}\label{subsubsec:arma}
As arquiteturas ARMA e ARIMA são apenas os componentes AR (Autoregressive) e MA (Moving Average) juntos.

\textbf{ARMA}

O modelo ARMA é uma constante mais a soma de lags AR e seus multiplicadores, além da soma dos lags ma e seus multiplicadores mais ruído branco. Esta equação é a base de todos os modelos que vêm a seguir e é uma estrutura para muitos modelos de previsão em diferentes domínios.

\begin{figure}[H]
	\centering
	\caption{ARMA (7,7)  }
	\label{fig:1-arma}
	\includegraphics[width=1\linewidth]{Modelos/Figuras/0-ARMA}
	
	Fonte: Elaboração própria a partir de dados da SANEPAR (2018 a 2020)
\end{figure}

Da Figura \ref{fig:1-arma} é a junção dos modelos AR e MA esse modelos juntos pode ocorrer a redução do erro em escala mais significativa, nos apêndice \ref{sec:comtb24} e \ref{sec:comtb18} pode ser notado a comparação de alguns passos a mais do que mostrado aqui.

\textbf{ARIMA}

\begin{eqnarray}
	Y_t = \beta_2 + \omega_1\varepsilon_{t-1} + \omega_2 \varepsilon_{t-2} +\ldots+ \omega_q \varepsilon_{t-q} + \varepsilon_t \label{arima}
\end{eqnarray}


Onde em \eqref{arima} o $Y_t$ é a série diferenciada (pode ter sido diferente mais de uma vez). Os ``preditores" no lado direito incluem ambos os valores defasados de $Y_t e$ erros defasados. Chamamos isso de ARIMA( $p, d, q$ ).

O modelo ARIMA é um modelo ARMA ainda com uma etapa de pré-processamento incluída no modelo que representamos usando I(d). I(d) é a ordem de diferença, que é o número de transformações necessárias para tornar os dados estacionários. Assim, um modelo ARIMA é simplesmente um modelo ARMA na série de tempo diferente.

\begin{figure}[H]
	\centering
	\caption{ARIMA (7,1,7)  }
	\label{fig:1-arima}
	\includegraphics[width=1\linewidth]{Modelos/Figuras/0-ARIMA}
	
	Fonte: Elaboração própria a partir de dados da SANEPAR (2018 a 2020)
\end{figure}

Olhando a Figura \ref{fig:1-arima} podemos perceber que não tem muita diferença visual com os outros métodos mostrados até agora, visualmente o método ARX ainda esta melhor que os outros.  


\subsubsection{Modelos SARIMA, ARIMAX e SARIMAX}

Os modelos ARIMA são ótimos, mas incluir variáveis sazonais e exógenas no modelo pode ser muito poderoso. Como o modelo ARIMA assume que a série temporal é estacionária, precisamos usar um modelo diferente.
\textbf{SARIMA}

\begin{eqnarray}
	Y_t&=&c+\sum_{n=1}^p \alpha_n y_{t-n}+\sum_{n=1}^q \theta_n \epsilon_{t-n}+\sum_{n=1}^P \phi_n y_{t-s n}+\sum_{n=1}^Q \eta_n \epsilon_{t-s n}+\epsilon_t \label{sarima}
\end{eqnarray}

O modelo é muito semelhante ao modelo ARIMA, com um conjunto adicional de componentes autorregressivos e de média móvel. O atraso extra é compensado pela frequência sazonal (por exemplo, 12 - mensal, 24 - por hora). 

\begin{figure}[H]
	\centering
	\caption{SARIMA $(7,1,7) (2,1,1)_{12}$}
	\label{fig:1-sarima}
	\includegraphics[width=1\linewidth]{Modelos/Figuras/0-SARIMA}
	
	Fonte: Elaboração própria a partir de dados da SANEPAR (2018 a 2020)
\end{figure}

Na Figura \ref{fig:1-sarima} pode ser observado como a previsão em vermelho esta mais próxima do observado em preto, só acionando o termo de sazonalidade na previsão. 

Os modelos SARIMA permitem diferenciar dados por frequências sazonais e não sazonais. Uma estrutura de pesquisa automatizada de parâmetros, como pmdarina, pode ajudar a entender quais são os melhores parâmetros.

\textbf{ARIMAX e SARIMAX}

\begin{eqnarray}
	d_t=c+\sum_{n=1}^p \alpha_n d_{t-n}+\sum_{n=1}^q \theta_n \epsilon_{t-n}+\sum_{n=1}^r \beta_n x_{n_t}+\sum_{n=1}^P \phi_n d_{t-s n}+\sum_{n=1}^Q \eta_n \epsilon_{t-s n}+\epsilon_t \label{eq:sarmax}
\end{eqnarray}

Em \eqref{eq:sarmax} está o modelo SARIMAX. Este modelo tem em conta variáveis exógenas, ou por outras palavras, utiliza dados externos na nossa previsão. É interessante pensar que todos os factores exógenos ainda são tecnicamente indiretamente modelados na previsão histórica do modelo. Dito isto, se incluirmos dados externos, o modelo responderá muito mais rapidamente ao seu efeito do que se confia na influência de termos desfasados.

\begin{figure}[H]
	\centering
	\caption{ARIMAX $(7,1,7)$  }
	\label{fig:1-arimax}
	\includegraphics[width=1\linewidth]{Modelos/Figuras/0-ARIMAX}
	
	Fonte: Elaboração própria a partir de dados da SANEPAR (2018 a 2020)
\end{figure}

\begin{figure}[H]
	\centering
	\caption{SARIMAX $(7,1,7) (2,1,1)_{12}$  }
	\label{fig:1-sarimax}
	\includegraphics[width=1\linewidth]{Modelos/Figuras/0-SARIMAX}
	
	Fonte: Elaboração própria a partir de dados da SANEPAR (2018 a 2020)
\end{figure}


Entre os modelos com variáveis exógenas os modelos da Figura \ref{fig:1-arimax} e \ref{fig:1-sarimax}, é possível perceber que a previsão está mais completa do que nos modelos sem a variável exógena.

\subsection{Modelos Regressivo}\label{subsec:reg}

\subsubsection{Regress\~ao Linear (LR)}

Segundo \citeonline{korstanje2021} nos modelos de aprendizado de máquina supervisionados, você tenta identificar relações entre diferentes variáveis:

\begin{itemize}
	\item Variável de destino: a variável que você tenta prever
	\item Variáveis explicativas: Variáveis que ajudam você a prever o alvo variável
\end{itemize}

Para a previsão, é importante entender quais tipos de variáveis explicativas você pode ou não usar. Como exemplo, aqui vai ser usado as variáveis \textbf{Pressão de Succção (PT01SU)} como variável $x$ e \textbf{Nível do Reservatório (Câmara 1) LT01} como variável $y$ pois na correlação de Pearson mostrado na Figura \ref{fig:person}, o coeficiente mostra a relação que tem entre o eixo $x$ e $y$ com a seguinte fórmula.



\begin{eqnarray}
	r=\frac{\sum\left(x_i-\bar{x}\right)\left(y_i-\bar{y}\right)}{\sqrt{\left(\sum\left(x_i-\bar{x}\right)^2\right)\left(\sum\left(y_i-\bar{y}\right)^2\right)}}\label{eq:pearson}
\end{eqnarray}
De \eqref{eq:pearson} sejam $x_i \in y_i$ os valores das variáveis $X$ e $Y$.  $\bar{x}$ e $\bar{y}$ são respectivamente as médias dos valores $x_i \in y_i$.

A fórmula do coeficiente de correlação de Pearson é então,

\begin{figure}[H]
	\centering
	\caption{Corelação de Pearson }
	\label{fig:person}
	\includegraphics[width=0.9\linewidth]{Apendices/Figuras/modelagem-24h/person}
	
	Fonte: Elaboração própria a partir de dados da SANEPAR (2018 a 2020)
\end{figure}

Como mostra a Figura \ref{fig:person} essa imagem é meramente ilutação da correlação que tem relação no conjunto de dados que esta sendo trabalhado aqui. E com isso também pode ser respondido a \ref{q1} da pesquisa. porque a correlação entre essas variáveis é forte.

\textbf{Definição do modelo}

A regressão linear é definida da seguinte forma:
\begin{eqnarray}
y&=&\beta_0+\beta_1 x_1+\cdots+\beta_p x_p+\varepsilon\label{eq:lr}
\end{eqnarray}
Da \eqref{eq:lr} têm a seguinte variáveis:

\begin{itemize}
	\item  Há $p$ variáveis explicativas, chamadas $x$.
\item Existe uma variável alvo chamada $y$.
\item  O valor para $y$ é calculado como uma constante $\left(\beta_0\right)$ mais os valores do $x$ variáveis multiplicadas pelos seus coeficientes $\beta_1$ para $\beta_p$.
\end{itemize}

\begin{figure}[H]
	\centering
	\caption{Regressão linear LT01 vs PT01 correlação 98\%}
	\label{fig:lr-lt01-m3}
	\includegraphics[width=1\linewidth]{"Modelos/Figuras/LR LT01 (m³)"}
	
	Fonte: Elaboração própria a partir de dados da SANEPAR (2018 a 2020)
\end{figure}



A Figura \ref{fig:lr-lt01-m3} mostra como interpretar $\beta_0$ e $\beta_1$ visualmente. Mostra que para um aumento de $1$ na variável $x$, o aumento na variável $x$ representa $\beta_1$. O valor para $0$ é o valor para $x$ quando $y$ é $0$.

Para poder utilizar a regressão linear, é necessário estimar os coeficientes (betas) sobre um conjunto de dados de formação. Os coeficientes podem então ser estimados utilizando a seguinte fórmula, em notação matricial:

\begin{eqnarray}
	\hat{\beta}&=&\left(X^T X\right)^{-1} X^T y\label{eq:ols}
\end{eqnarray}

\citeonline{korstanje2021} esta fórmula é conhecida como \textbf{OLS}: o método dos mínimos quadrados ordinários (Ordinary Least Squares method). Este modelo é muito rápido para caber, uma vez que requer apenas cálculos matriciais para calcular os betas. Embora fácil para caber, é menos adequado para processos mais complexos. Afinal de contas, é um modelo linear, e pode portanto, só se encaixam em processos lineares.

\begin{figure}[H]
	\centering
	\caption{Regressão linear (LR) um passo a frente}
	\label{fig:1-regressao-linear}
	\includegraphics[width=1\linewidth]{Modelos/Figuras/0-regressão-linear}
	
	Fonte: Elaboração própria a partir de dados da SANEPAR (2018 a 2020)
\end{figure}


\subsubsection{Floresta Aleat\'oria} \label{subsubsec:rf}

Pode entender que ter exatamente a mesma árvore de decisão 1000 vezes não tem valor agregado do que usar essa árvore de decisão apenas uma vez.  Em um modelo de conjunto, cada modelo individual deve ser ligeiramente diferente do outro. Existem dois métodos bem conhecidos de criação de coleções: ensacamento e reforço.  Random Forest usa ensacamento para criar um conjunto de árvores de decisão

\begin{figure}[H]
	\centering
	\caption{Regressão da Floresta Aleatória (RFA)}
	\label{fig:1-regressao-rfa}
	\includegraphics[width=1\linewidth]{Modelos/Figuras/0-regressão-rfa}
	
	Fonte: Elaboração própria a partir de dados da SANEPAR (2018 a 2020)
\end{figure}

Segundo \citeonline{Pelletier2016156} Cada árvore é construída executando um algoritmo de aprendizado individual que divide o conjunto de variáveis de entrada em subconjuntos com base em um teste de valor de atributo (por exemplo, o coeficiente de Gini). Ao contrário das árvores de decisão (DT) clássicas, as árvores de RFA são construídas sem poda e selecionando aleatoriamente em cada nó um subconjunto de variáveis de entrada. Atualmente, esse número de variáveis utilizadas para dividir um nó de RFA (denotado por $m$) corresponde à raiz quadrada do número de variáveis de entrada.

\begin{figure}[H]
	\centering
	\caption{Esquema da Floresta Aleatória}
	\label{fig:rf}
	\includegraphics[width=1\linewidth]{Modelos/Figuras/RF}
	
	Fonte: Elaboração própria
\end{figure}


\subsubsection{LightGBM e XGboost}\label{subsubsec:lgbxgb}

O aumento de gradiente combina vários pequenos modelos de árvore de decisão para fazer previsões. É claro que essas pequenas árvores de decisão são diferentes umas das outras, caso contrário não há vantagem em usar mais árvores de decisão. O conceito importante a ser entendido aqui é como essas árvores de decisão se tornam diferentes umas das outras. outros. Isto é conseguido através de um processo chamado elevação. Boosting e bagging são dois métodos principais que são aprendidos juntos.  Boosting é um processo iterativo. Ele adiciona cada vez mais modelos fracos ao conjunto de modelos de maneira inteligente. Em cada etapa, pontos de dados individuais são ponderados.  

Pontos de dados que já estão bem previstos não são importantes para o aluno adicionar. Portanto, novos modelos fracos se concentrarão em aprender coisas que ainda não são compreendidas, melhorando assim o conjunto.

Pode se ver uma visão geral esquemática do processo de reforço na Figura \ref{fig:xgboos}. Com essa abordagem, você ajusta iterativamente modelos fracos que se concentram nas partes dos dados que ainda
não são compreendidas. Ao fazer isso, você mantém todos os modelos fracos intermediários. O modelo ensemble é a combinação de todos esses modelos fracos.


\begin{figure}[H]
	\centering
	\caption{Impulsionando gradiente com XGBoost e LightGBM}
	\label{fig:xgboos}
	\includegraphics[width=1\linewidth]{Modelos/Figuras/xgboos}
	
	Fonte: Adaptação de \citeonline{korstanje2021}
\end{figure}



\subsubsection{O Gradiente em Gradiente de Boosting (Refor\c co)} \label{subsubsec:boosting}

\citeonline{korstanje2021} esse processo iterativo é chamado de aumento de gradiente por um motivo. Um gradiente é um termo matemático que se refere ao campo vetorial de derivadas parciais que apontam na direção da inclinação mais acentuada. Em termos simples, muitas vezes comparamos gradientes com declives de estradas em aclive: quanto maior a inclinação, mais íngreme a colina. Os gradientes são calculados tomando derivadas, ou derivadas parciais, de uma função.

No aumento de gradiente, ao adicionar árvores adicionais ao modelo, o objetivo é adicionar uma árvore que melhor explique a variação que não foi explicada pelas árvores anteriores. O destino de sua nova árvore é, portanto.

\begin{eqnarray}
	y-\hat{y}\label{eq:xb}
\end{eqnarray}

De \eqref{eq:xb} isso pode ser denotado reescrito como a derivada parcial negativa da função de perda em relação às previsões de $y$:

\begin{eqnarray}
	y-\hat{y}&=&-\dfrac{\partial L}{\partial \hat{y}}\label{eq:xb2}
\end{eqnarray}

Você define isso como o destino da nova árvore para garantir que a adição da árvore explicará uma quantidade máxima de variação adicional no modelo geral de aumento de gradiente. Isso explica por que o modelo é chamado de aumento de gradiente boosting.

\subsubsection{Algoritmos de boosting de gradiente}

Existem muitos algoritmos que executam versões ligeiramente diferentes de aumento de gradiente. Quando o método de aumento de gradiente foi inventado, o algoritmo não Muito desempenho, mas mudou com o advento do algoritmo AdaBoost: o primeiro algoritmo que pode se adaptar a modelos fracos. 

O algoritmo de aumento de gradiente é uma das ferramentas de aprendizado de máquina com melhor desempenho no mercado. Depois do AdaBoost, uma longa lista de algoritmos de aumento ligeiramente diferentes foi adicionada à literatura, incluindo XGBoost, LightGBM, LPBoost, BrownBoost, MadaBoost, LogitBoost e TotalBoost. Ainda existem muitas contribuições para melhorar a teoria do aumento de gradiente. Nesta subseção, dois algoritmos são apresentados: XGBoost e LightGBM.

O \textbf{XGBoost} é um dos algoritmos de aprendizado de máquina mais usados. O XGBoost é uma maneira rápida de obter bons desempenhos. Como é fácil de usar e tem alto desempenho, é o primeiro algoritmo para muitos profissionais de aprendizado de máquina.

\textbf{LightGBM} é outro algoritmo de aumento de gradiente que é importante conhecer. No momento, é um pouco menos difundido que o XGBoost, mas está ganhando popularidade seriamente.
A vantagem esperada do LightGBM sobre o XGBoost é um ganho de velocidade e uso de memória.
Nesta subseção, você descobrirá as implementações de ambos os algoritmos de aumento de gradiente.

\subsubsection{A diferen\c ca entre XGBoost e LightGBM}

Se você for usar esses dois algoritmos de aumento de gradiente, é importante entender de
que maneira eles diferem. Isso também pode fornecer uma visão dos tipos de diferença que fazem um número tão grande de modelos no mercado.

É importante entender se você planeja usar os dois algoritmos de aumento de gradiente
Como eles são diferentes. Isso também fornece informações sobre as várias diferenças que acompanham tantos modelos no mercado.

A diferença aqui é a forma como eles identificam as melhores divisões entre os azarões. (árvores de decisão individuais). Lembre-se de que uma divisão em uma árvore de decisão é quando sua árvore precisa encontrar a divisão que mais melhora seu modelo.

A ideia intuitiva e mais simples para encontrar a melhor divisão é iterar todos os ajustes possíveis e encontrar a melhor divisão. No entanto, isso leva muito tempo e algoritmos recentes apresentam alternativas melhores.
Uma alternativa proposta pelo XGBoost é usar a segmentação baseada em histograma. Nesse caso, ao invés de iterar sobre todas as partições possíveis, o modelo constrói um histograma de cada partição.
variáveis e use-as para encontrar a melhor divisão de variáveis. A melhor divisão geral é então mantida.

LightGBM foi inventado pela Microsoft e tem uma maneira mais eficiente de definir partições. Essa abordagem é chamada de amostragem unilateral baseada em gradiente (GOSS). O GOSS calcula o gradiente de cada ponto de dados e o usa para filtrar pontos de dados com gradientes baixos. Afinal, pontos de dados com gradientes baixos já são bem compreendidos, enquanto indivíduos com gradientes altos precisam ser melhor aprendidos.

O LightGBM também usa uma abordagem chamada Exclusive Feature Bundling (EFB), que permite acelerar a seleção de muitas variáveis correlacionadas. Outra diferença é que o modelo LightGBM é adequado para crescimento de folhas (preferencialmente preferido), enquanto o XGBoost cultiva árvores como árvores. A diferença pode ser vista na Figura \ref{fig:xgboost}.

Essa diferença é um recurso que teoricamente favoreceria o LightGBM em termos de precisão, mas apresenta um risco maior de overfitting (sobreajustamento) no caso de poucos dados disponíveis.

\begin{figure}[H]
	\centering
	\caption{Crescimento em folha versus crescimento em nível}
	\label{fig:xgboost}
	\includegraphics[width=0.7\linewidth]{Modelos/Figuras/xgboost}
	
	Fonte: Adaptação de \citeonline{korstanje2021}
\end{figure}


Na Figura \ref{fig:xgboost} pode ser visto como cada modelo é ajustado, no crescimento da árvore em folha e em nível.

\begin{figure}[H]
	\centering
	\caption{XGBoost e LigthGBM regressão}\label{fig:1-xgb-regressao}
	\includegraphics[width=1\linewidth]{Modelos/Figuras/0-xgb-regressão}
	
\end{figure}
	
\begin{figure}[H]
	\centering
	\includegraphics[width=1\linewidth]{Modelos/Figuras/0-lgbm-regressão}	
	
	
	Fonte: Elaboração própria a partir de dados da SANEPAR (2018 a 2020)

\end{figure}

Na Figura \ref{fig:1-xgb-regressao} é um modelo baseado nos dados coletados da SANEPAR.



%\include{Referencial/sec-referencial}

%% RESULTADOS
\section{Resultados} \label{sec:result}

Neste capítulo, é fornecida uma síntese dos resultados obtidos neste estudo aplicando os diferentes modelos de previsão, citados no capítulo prévio, bem como a aplicação de testes e métricas estatísticas. 



\subsection{Descri\c c\~ao das Etapas}

\textbf{An\'alise explorat\'oria dos dados (EDA)}
a partir da \ref{etp:1}, foi realizado o EDA (do inglês \textit{Exploratory Data Analysis}) para processar os dados obtidos até o momento. A análise exploratória de dados foi promovida por John Tukey \cite{Bandara2021} como uma abordagem para explorar os dados, resumir suas principais características e formular hipóteses que possam direcionar a coleta adicional de dados e experimentos. No contexto de análises de dados, várias técnicas de EDA têm sido adotadas.
Nessa análise da EDA, serão abordadas várias análises, como a correlação de Pearson, para iniciar e verificar quais variáveis podem ser excluídas como ruído e não têm correlação com a variável LT01. Nesse caso, as variáveis retiradas são consideradas negativas e têm pouca correlação com o LT01, as únicas variáveis que apresentam correlação negativa são B3 e FT02, mesmo sendo uma correlação negativa, essas variáveis possuem uma correlação inviável ou próxima de $0$, levando à decisão de removê-las.

Uma análise bem feita do EDA mostra tudo que os dados podem ter. Esses dados fornecidos pela companhia SANEPAR são coletados em campo por hora, como por exemplo, a cada hora possui um valor esperado. Nisso, pode haver os famosos NaN, que representam a falta de dado coletado, como em um dia em que as bombas podem ter ido para manutenção. Esses NaNs também podem ser registrados como uma anomalia nos dados.

A Figura \ref{fig:person} mostra a correlação entre as variáveis no conjunto de dados em questão. Essa imagem representa graficamente a relação entre as variáveis e é usada para evidenciar a existência de uma correlação forte no valor de $0,97$ é considerada forte entre elas, quanto mais próximo do valor $1$ a correlação é sempre forte e $0,9$ para mais ou para menos indica uma correlação muito forte, 0 a $0,3$ positivo ou negativo indica uma correlação desprezível.

\begin{figure}[!htb]
	\centering
	\caption{Correlação de Pearson}
	\label{fig:person}
	\includegraphics[width=0.7\linewidth]{Resultados/Figuras/person}
	
	
\end{figure}

Nesse conjunto de dados que está sendo trabalhado, há uma forte correlação da variável PT01 nos modelos de LR e DTR (do inglês \textit{Decision Tree Regressor}). Esses modelos foram escolhidos para trabalhar com o LT01 como entrada e o PT01 como saída. A Figura \ref{fig:lr-lt01-m3} fornece uma representação dos coeficientes $\beta_0$ e $\beta_1$. Um aumento de $1$ na variável $x$ está associado a um aumento proporcional de $\beta_1$ na variável $y$. O valor de $\beta_0$ representa o valor de $y$ quando $x$ é igual a $0$.

\begin{figure}[!htb]
	\centering
	\caption{Regressão linear LT01 vs PT01 correlação 97\%}
	\label{fig:lr-lt01-m3}
	\includegraphics[width=0.7\linewidth]{Resultados/Figuras/LR}
	
	
\end{figure}








A Tabela \ref{tb:est}, o desvio padrão é representado pela sigla STD, que corresponde à expressão em inglês \textit{standard deviation}. Assim como em qualquer empresa de tratamento de água, é utilizado um mecanismo de acionamento automático denominado ``trava de segurança'' para evitar que o nível do tanque se aproxime de zero e haja falta de água nos locais abastecidos por esse tanque. O nível máximo que o tanque pode alcançar é de $5,26 m^3$ (equivalente a $5264.56$ litros). As bombas são ativadas em sua potência máxima para evitar que sejam acionadas quando o nível do tanque estiver dentro dessa faixa. No entanto, a bomba $1$ ainda estaria operando para completar o nível do tanque caso esteja dentro dessa faixa. Nessa Tabela \ref{tb:est}, foram filtrados os horários de pico nos quais pode ocorrer a maior demanda d'água.

\begin{table}[!htb]
	\centering
	\caption{Descrição estatística dos dados com o filtro aplicado das 18h às 21h}\label{tb:est}
	\begin{tabular}{@{}cccccccccc@{}}
		\toprule
		\multicolumn{1}{l}{\textbf{18h a 21h}} & \multicolumn{1}{c}{\textbf{B1}} & \multicolumn{1}{c}{\textbf{B2}} & \multicolumn{1}{c}{\textbf{B3}} & \multicolumn{1}{c}{\textbf{LT01}} & \multicolumn{1}{c}{\textbf{FT01}} & \multicolumn{1}{c}{\textbf{FT02}} & \multicolumn{1}{c}{\textbf{FT03}} & \multicolumn{1}{c}{\textbf{PT01}} & \multicolumn{1}{c}{\textbf{PT02}} \\ \midrule
		\textbf{Contagem}                      & 2921                            & 2921                            & 2921                            & 2921                              & 2921                              & 2921                              & 2921                              & 2921                              & 2921                              \\
		\textbf{Média}                         & 55,98                           & 30,60                           & 5,26                            & 3,18                              & 86,52                             & 132,87                            & 117,62                            & 4,04                              & 22,55                             \\
		\textbf{STD}                           & 10,00                           & 15,63                           & 15,57                           & 0,67                              & 124,41                            & 18,48                             & 12,83                             & 0,71                              & 2,92                              \\
		\textbf{Min.}                          & 0                               & 0                               & 0                               & 0,29                              & 0                                 & 0                                 & 0,03                              & 0,88                              & 0                                 \\
		\textbf{25\%}                          & 57,99                           & 31,63                           & 0                               & 2,77                              & 0                                 & 123,40                            & 112,52                            & 3,61                              & 21,98                             \\
		\textbf{50\%}                          & 57,99                           & 35,20                           & 0                               & 3,24                              & 0,12                              & 132,88                            & 118,18                            & 4,10                              & 22,87                             \\
		\textbf{75\%}                          & 57,99                           & 38,17                           & 0                               & 3,71                              & 258,48                            & 142,89                            & 124,48                            & 4,59                              & 23,05                             \\
		\textbf{Max.}                          & 59,99                           & 59,99                           & 59,99                           & 4,39                              & 370,35                            & 277,94                            & 167,78                            & 5,31                              & 28,08                             \\ \bottomrule
	\end{tabular}
	
	
\end{table}



A realização da EDA consegue mostrar que os dados estão sendo coletados de maneira significativa, ao observar as correlações e possibilita trabalhar com as variáveis que apresentam correlação maior ou não negativa. Nos modelos ARIMA e seus antecessores, ele utiliza o ACF (do inglês \textit{\textit{Auto-Correlation Function}}) e o PACF (do inglês \textit{Partial Auto-correlation Function}) para analisar esses métodos. A análise destes métodos para o modelo ARIMA permite otimizar os parâmetros do modelo.

O ACF é uma medida estatística utilizada para identificar a presença de correlação serial em uma série temporal. Ele calcula a autocorrelação entre os valores da série em diferentes defasagens, ou seja, a correlação entre os valores atuais e os valores passados da série. 

O ACF é útil para analisar a dependência temporal dos dados e identificar padrões de sazonalidade, tendência ou outros efeitos temporais. Por meio do ACF, é possível avaliar se a série exibe autocorrelação significativa em defasagens específicas, o que pode indicar a presença de não estacionariedade ou estrutura temporal que precisa ser considerada na análise ou modelagem da série temporal.


A estatística ADF (do inglês \textit{Augmented Dickey-Fuller}) de $-12,515$ indica a evidência de estacionariedade na série temporal. Quanto mais negativo for o valor da estatística ADF, maior é a evidência de estacionariedade nos dados.

O valor de p, aproximadamente de $0,0000000000000000000000262$, expresso de forma mais concisa como $2,62\times 10^{-23}$ usando a notação científica, está associado ao teste ADF. Este valor-p representa a probabilidade de obter um resultado igual ou mais extremo do que o observado, sob a suposição de que a hipótese nula seja verdadeira. No contexto do teste ADF, a hipótese nula é a presença de raiz unitária na série temporal, indicando não estacionariedade. Portanto, um valor de p baixo, geralmente abaixo de um nível de significância predefinido, como $0,05$, sugere que a série temporal é estacionária, enquanto um valor de p alto sugere não estacionariedade. Dado o valor de p de $2,62\times 10^{-23}$, evidencia-se uma probabilidade muito baixa, indicando forte suporte contra a hipótese nula e sugerindo que a série temporal é estacionária. Na Tabela \ref{tb:adf}, são apresentados todos os dados do teste para estacionalidade.
Os resultados indicam fortes evidências contra a hipótese nula. Com um teste ADF de $-12,515 $ e um valor de p extremamente baixo de $2,62 \times 10^{-23}$, rejeita-se a hipótese nula de presença de raiz unitária. Os $44$ atrasos utilizados e as $17.477$ observações corroboram a análise estatística.

\begin{table}[!htb]
	\centering
	\caption{Teste de Dickey-Fuller Aumentado}\label{tb:adf}
	\begin{tabular}{lc}
		\hline
		Teste ADF & $-12,515$ \\ \hline
		Valor de p & $2,62 \times 10^{-23}$ \\
		Atrasos utilizados & $44$ \\
		Número de observações & $17.477$ \\
		Valor crítico $(1\%)$ & $-3,431$ \\
		Valor crítico $(5\%)$ & $-2,862$ \\
		Valor crítico $(10\%)$ & $-2,567$ \\
		\hline
	\end{tabular}
\end{table}




Ao comparar a estatística de teste ADF com os valores críticos, observa-se que está significativamente abaixo deles em todos os níveis de significância $(1\%, 5\%, 10\%)$. Portanto, a conclusão é de que os dados não possuem raiz unitária, indicando que são estacionários.

Na Figura \ref{fig:acfa}, pode-se observar a diferença entre a autocorrelação (ACF) exibida na Figura \ref{fig:acfa} e a autocorrelação parcial (PACF) exibida na Figura \ref{fig:pacf}. A autocorrelação é uma medida da correlação entre os valores da série temporal em diferentes defasagens, levando em consideração tanto a correlação direta quanto a correlação indireta. Por outro lado, a autocorrelação parcial mede apenas a correlação direta entre os valores, desconsiderando a influência das defasagens intermediárias. Essas análises são úteis para identificar padrões e relações de dependência entre os valores da série temporal, fornecendo informações importantes para a modelagem e previsão desses dados.
O intervalo de confiança padrão de 95\% é representado pela marca azul nas Figuras \ref{fig:acfa} e \ref{fig:pacf}. As observações que estão fora desse intervalo são consideradas estatisticamente correlacionadas, indicando a presença de padrões ou estrutura na série temporal.

A correlação visualizada na Figura \ref{fig:acfa} é fundamental para a interpretação do teste ADF. Em uma série de ruído branco, os valores são completamente aleatórios e não apresentam correlação significativa. Portanto, quando há correlação presente na série, isso indica a existência de padrões ou dependências entre os valores, o que pode ser explorado para a modelagem e previsão da série temporal.
\begin{figure}[!htb]
	\centering
	\caption{Autocorrelação}\label{fig:acfa}
	\includegraphics[width=0.7\linewidth]{Resultados/Figuras/acf} 
	
	
\end{figure}


\begin{figure}[!htb]
	\centering
	\caption{Autocorrelação parcial}\label{fig:pacf}
	\includegraphics[width=0.7\linewidth]{Resultados/Figuras/pacf}
	
	
	
\end{figure}



Demonstrar que uma série temporal tem ou pode ter um ruído branco também é conveniente para a análise da EDA.
Na Figura \ref{fig:ruido-branco}, é possível observar uma série temporal que pode ser caracterizada como ruído branco. Uma série temporal é considerada ruído branco se suas variáveis forem independentes e distribuídas de forma idêntica, com média zero. Isso implica que todas as variáveis possuem a mesma variância ($\sigma^2$) e que cada valor não possui correlação com os demais valores da série.

\begin{figure}[!htb]
	\centering
	\caption{Ruído branco}
	\label{fig:ruido-branco}
	\includegraphics[width=\linewidth]{Resultados/Figuras/ruido-branco}
	
	
\end{figure}

Nesse exemplo, ao utilizar os dados da SANEPAR, a série temporal trabalhada é estacionária e também apresenta ruído branco (do inglês \textit{white noise}).



Com base na forte evidência contra a hipótese nula, podemos rejeitar a hipótese nula. A Figura \ref{fig:hist}, podemos notar um aumento na demanda durante essas horas durante o ano de 2019.
Conforme mencionado na subseção \ref{subsubsec:motivacao}, as anomalias climáticas ocorridas em 2020, especialmente a falta de chuvas e devido ao COVID-19, tiveram um impacto significativo nos resultados. Isso contribuiu para as mudanças observadas na demanda de água ao longo desse período.

\begin{figure}[!htb]
	\centering
	\caption{Violino no nível do reservatório}
	\label{fig:hist}
	\includegraphics[width=0.7\linewidth]{Resultados/Figuras/viol}
	
	
\end{figure}





A Figura \ref{fig:ft03} mostra como a vazão pode ser afetada pelo nível do tanque. É interessante observar que a vazão de recalque tem um impacto mais significativo no nível do tanque em comparação com as outras vazões. Isso ocorre porque a vazão de recalque está associada à injeção de água diretamente no tanque por meio da bomba localizada próxima à base do tanque. Por outro lado, as demais vazões apresentam alguns valores ausentes, o que limita sua influência na análise geral.	


\begin{figure}[!htb]
	\centering
	\caption{Violino da vazão de recalque}
	\label{fig:ft03}
	\includegraphics[width=0.7\linewidth]{Resultados/Figuras/ft03}
	
	
\end{figure}




\textbf{M\'ultiplas entradas e sa\'ida \'unica (MISO)}
na etapa \ref{etp:2}, foi explorado o modelo MISO (do inglês \textit{Multiple Inputs, Single Output}) na dissertação. O modelo ARIMA, juntamente com suas variantes e extensões, foi amplamente estudado durante a pesquisa, assim como modelos regressivos que envolvem múltiplas variáveis de entrada e uma variável de saída, neste caso, a LT01. As demais variáveis foram utilizadas como suporte para melhorar o modelo do tipo ARIMAX ou modelos com variáveis exógenas. Quando aplicado sem o uso de variáveis exógenas, o modelo ARIMA apresenta apenas uma entrada, semelhante ao modelo de LR. No entanto, ao incluir variáveis exógenas, o modelo se torna MISO, permitindo uma modelagem  abrangente e considerando a interação de várias variáveis para prever a variável de interesse.


\textbf{Decomposi\c c\~ao STL}
através da decomposição, é possível analisar se a série apresenta tendência, sazonalidade e resíduos. Ao observar a Figura \ref{fig:stl}, é evidente que os dados exibem ambos os padrões. Isso indica que a série é estacionária, como confirmado pelo seguinte teste ADF anterior.



\begin{figure}[!htb]
	\centering
	\caption{Decomposição STL aditiva dos dados coletados}
	\label{fig:stl}
	\includegraphics[width=0.8\linewidth]{Resultados/Figuras/STL}
	
	
	
\end{figure}






\textbf{Separa\c c\~ao dos Dados}
na etapa \ref{etp:4}, os dados foram divididos em conjuntos de treinamento, teste e validação. Essa prática é comum entre profissionais de aprendizado de máquina, pois permite avaliar o desempenho do modelo em conjuntos de dados diferentes \cite{raschka2015practical, geron2017hands_on}.

Quanto à divisão dos dados, foi adotada uma estratégia básica em que $70\%$ dos dados foram destinados ao conjunto de treinamento e os $30\%$ restantes foram reservados para o conjunto de teste. Dentro dos $70\%$ de treinamento, foi realizada uma subdivisão em que $80\%$ desses dados foram usados novamente para treinamento e os $20\% $ restantes foram utilizados para validação. Essa abordagem foi implementada em linguagem de programação para facilitar o processo e evitar a necessidade de recalculá-la a cada modificação do modelo.

\textbf{Modelagem e sele\c c\~ao do modelo}
a estratégia recursiva é mencionada por \citeonline{PETROPOULOS2022705} como uma abordagem eficaz na previsão de séries temporais de múltiplos passos. De acordo com o autor, essa estratégia envolve o uso de previsões anteriores como entradas para prever os próximos passos da série temporal. A abordagem recursiva tem demonstrado potencial para melhorar a acurácia das previsões de séries temporais de longo prazo.

Na Etapa \ref{etp:5}, discute-se a previsão dos dados em uma janela de horizonte de previsão estendida, abrangendo diferentes períodos de tempo, como uma hora, seis horas, doze horas e um dia. Essa estratégia de previsão recorrente permite a comparação entre modelos de regressão e modelos ARIMA em diferentes horizontes temporais.

Essa abordagem é vantajosa, pois cada modelo possui suas próprias características e desempenho ao lidar com previsões de curto prazo, como um dia, e previsões de prazo mais longo, como um dia. Ao utilizar uma janela de previsão mais ampla, é possível observar e avaliar melhor as diferenças entre os modelos e analisar seu desempenho em horizontes de tempo variados.

Além desses modelos, vários outros foram implementados no documento, tais como DTR, RFR, XGBRegressor, LGBMRegressor, LSTM, GRU, Prophet, RNN, Transformer, CNN e ANN, a fim de obter o melhor resultado para a previsão de séries temporais de abastecimento de água.

\textbf{Validação e ajuste do modelo}
na etapa \ref{etp:6}, o horizonte de previsão foi personalizado com base no método recursivo de previsão de série temporal e na previsão do nível do tanque LT01. Foram selecionados os seguintes passos para a previsão à frente: uma hora, seis horas, doze horas e um dia. Essa escolha do horizonte de previsão foi feita levando em consideração a estratégia recursiva e os objetivos específicos do estudo. Identifica-se que essa janela de tempo proporciona uma análise mais adequada e comparável entre os modelos utilizados.

Foram utilizados os parâmetros obtidos pelo autoARIMA, que são $(p = 7, d = 0, q = 0) (P = 2, D = 1, Q = 1)_{M = 12}$, mas foram ajustados para obter um melhor resultado, sendo $(p = 7, d = 1, q = 7) (P = 2, D = 1, Q = 1)_{M = 12}$. Na Tabela \ref{tab:autoarima_params}, são exibidos todos os modelos obtidos por esse método do ``autoARIMA'' e ajustados para que obtenham o melhor resultado.
\(p\): Ordem do componente AR (\textit{Auto-Regressivo}),
\(d\): Número de diferenciações não sazonais,
\(q\): Ordem do componente MA (\textit{Média Móvel}),
\(P\): Ordem do componente AR sazonal,
\(D\): Número de diferenciações sazonais,
\(Q\): Ordem do componente MA sazonal,
\(M\): Período sazonal (número de observações em um ciclo sazonal).
Na Tabela \ref{tb:resltsar} mostra como a biblioteca do Python autoARIMA obteve os resultados dos parâmetros, exibindo o STD e os intervalos de confiança nos quais o modelo alcançou o melhor desempenho. O leve ajuste realizado não altera significativamente os parâmetros obtidos nesta biblioteca, permitindo que cada modelo seja trabalhado de maneira eficiente.

\begin{table}[!htb]
	\centering
	\caption{Parâmetros utilizados nos modelos ARIMA e seus antecessores obtidos pelo ``autoARIMA'' do Python.}
	\label{tab:autoarima_params}
	\small
	\begin{tabular}{
			>{\centering\arraybackslash}p{5.5cm}
			>{\centering\arraybackslash}p{6cm}
			>{\centering\arraybackslash}p{3cm}
		}
		\toprule
		\textbf{Modelo} & \textbf{Parâmetros Utilizados} & \textbf{Método de Estimação} \\
		\midrule
		AR(p) & \( p = 7 \) & AutoARIMA \\
		ARX(p) & \( p = 7 \) & AutoARIMA \\
		MA(q) & \( q = 7 \) & AutoARIMA  \\
		ARMA(p, q) & \( p = 7 \), \( q = 7 \) & AutoARIMA  \\
		ARIMA(p, d, q) & \( p = 7 \), \( d = 1 \), \( q = 7 \) & AutoARIMA  \\
		ARIMAX(p, d, q) & \( p = 7 \), \( d = 1 \), \( q = 7 \) & AutoARIMA  \\
		SARIMA(p, d, q)(P, D, Q) & \( p = 7 \), \( d = 1 \), \( q = 7 \), \( P = 2 \), \( D = 1 \), \( Q = 1 \), \( M = 12 \) & AutoARIMA  \\
		SARIMAX(p, d, q)(P, D, Q, M) & \( p = 7 \), \( d = 1 \), \( q = 7 \), \( P = 2 \), \( D = 1 \), \( Q = 1 \), \( M = 12 \) & AutoARIMA  \\
		\bottomrule
	\end{tabular}
\end{table}



\begin{table}[!htb]
	\centering
	\caption{SARIMAX$(7, 0, 0)\times(2, 1, [1], 12)$ Results} \label{tb:resltsar}
	\begin{tabular}{
			l
			S[table-format=1.4]
			S[table-format=1.4]
			S[table-format=3.3]
			S[table-format=1.3]
			S[table-format=1.3]
			S[table-format=1.3]
		}
		\toprule
		& {Coef} & {STD Err} & {z} & {P$>|z|$} & {[0,025} & {0,975]} \\
		\midrule
		Intercept & 0,0003 & 0,000 & 1,053 & 0,292 & -0,000 & 0,001 \\
		ar.L1 & 1,6149 & 0,011 & 141,865 & 0,000 & 1,593 & 1,637 \\
		ar.L2 & -0,8879 & 0,021 & -42,045 & 0,000 & -0,929 & -0,847 \\
		ar.L3 & 0,3167 & 0,024 & 13,033 & 0,000 & 0,269 & 0,364 \\
		ar.L4 & -0,1056 & 0,027 & -3,961 & 0,000 & -0,158 & -0,053 \\
		ar.L5 & -0,1099 & 0,028 & -3,928 & 0,000 & -0,165 & -0,055 \\
		ar.L6 & 0,1431 & 0,027 & 5,368 & 0,000 & 0,091 & 0,195 \\
		ar.L7 & -0,0673 & 0,015 & -4,583 & 0,000 & -0,096 & -0,039 \\
		ar.S.L12 & -0,1222 & 0,016 & -7,705 & 0,000 & -0,153 & -0,091 \\
		ar.S.L24 & 0,1692 & 0,014 & 12,244 & 0,000 & 0,142 & 0,196 \\
		ma.S.L12 & -0,8728 & 0,012 & -74,569 & 0,000 & -0,896 & -0,850 \\
		sigma2 & 0,0157 & 0,000 & 60,022 & 0,000 & 0,015 & 0,016 \\
		\bottomrule
	\end{tabular}
\end{table}


Para os modelos de gradiente \textit{boosting} e redes neurais artificiais, os hiperparâmetros foram otimizados usando a biblioteca Optuna do Python. Nesse contexto, são empregadas técnicas bayesianas, especificamente o algoritmo TPE, visando uma otimização mais eficiente.

Os modelos XGBoost e LightGBM tem como parâmetros e hiperparâmetros mostrado na Tabela \ref{tab:hiperparametros} a otimização dos paramétrios dos modelos XGBoost, LightGBM, RFR e DTR. Esses modelos, devido à sua semelhança, exibem tempos de desempenho próximos um do outro. 



\begin{table}[!htb]
	\centering
	\caption{Hiperparâmetros dos modelos}
	\label{tab:hiperparametros}
	\begin{tabular}{
			>{\centering\arraybackslash}p{2.2cm}
			>{\centering\arraybackslash}p{2.8cm}
			>{\centering\arraybackslash}p{1.9cm}
			>{\centering\arraybackslash}p{1.9cm}
			>{\centering\arraybackslash}p{1.9cm}
			>{\centering\arraybackslash}p{1.9cm}
			>{\centering\arraybackslash}p{1.9cm}
		}
		\toprule
		\textbf{Modelo} & \textbf{Estimadores} & \textbf{Profund. Máxima} & \textbf{Min. Amostras Divisão} & \textbf{Min. Amostras por Folha} & \textbf{Máx. Recursos} & \textbf{Taxa de Aprendizado} \\
		\midrule
		XGB Regressor & 503 & 5 & 7 & 2 & ``sqrt'' & 0,034 \\
		LGBM Regressor & 820 & 10 & 3 & 5 & ``auto'' & 0,014 \\
		Random Forest Regressor & 135 & 10 & 4 & 2 & None & N/A \\
		Decision Tree Regressor & N/A & 229 & 32 & 20 & None & N/A \\
		\bottomrule
	\end{tabular}
\end{table}



Os modelos de rede neural artificial, como RNN, ANN, CNN, GRU, LSTM e Transformer, obtidos na otimização do Optuna do Python, tiveram seus hiperparâmetros melhorados, conforme exibido na Tabela \ref{tab:hyperparameters_summary}. Esses modelos, por serem modelos de rede neural artificial, são melhores para otimizar do que os outros. 

\begin{table}[!htb]
	\centering
	\caption{Resumo dos Hiperparâmetros dos Modelos de Redes Neurais}
	\label{tab:hyperparameters_summary}
	\small
	\begin{tabular}{
			>{\centering\arraybackslash}p{1.8cm}
			>{\centering\arraybackslash}p{2cm}
			>{\centering\arraybackslash}p{2cm}
			>{\centering\arraybackslash}p{2cm}
			>{\centering\arraybackslash}p{2cm}
			>{\centering\arraybackslash}p{1.5cm}
			>{\centering\arraybackslash}p{2.5cm}
		}
		\toprule
		\textbf{Modelo} & \textbf{Unidades/ Layers} & \textbf{Heads/ Dimensões} & \textbf{Tamanho do Batch} & \textbf{Épocas} & \textbf{Dropout/ Learning Rate} & \textbf{Outros Parâmetros} \\
		\midrule
		LSTM & 128 & -- & 32 & 77 & -- & -- \\
		
		GRU & -- & -- & 32 & 50 & -- & -- \\
		
		Transformers & -- & 8 heads, 217; 433 & -- & 50 & -- & 2 camadas \\
		
		RNN & 79 & -- & 16 & 50 & 0,0008612 & -- \\
		
		CNN & -- & -- & 61 & 10 & 0,2799; 0,00052 & Kernel: 7, Densas: 1, Verbosidade: 1 \\
		
		ANN & 125 & -- & 27 & 96 & 0,4135, 0,0004057 & Densas: 1, Verbosidade: 0 \\
		\bottomrule
	\end{tabular}
\end{table}


\textbf{Previs\~ao e avalia\c c\~ao}
a partir da etapa \ref{etp:7}, foram empregadas três métricas amplamente utilizadas na literatura para avaliar e comparar os modelos ARIMA e os modelos de regressão, conforme detalhado na seção \ref{subsec:metrica}.

Na análise dos modelos desenvolvidos, observou-se que o modelo DTR obteve o melhor desempenho, tanto para previsões de curto prazo, durante as horas de pico entre 18h e 21h, quanto para outros períodos. Além disso, os modelos MA, AR, SARIMA, ARIMA, SARIMAX, ARIMAX, ARX, LGBMRegressor, XGBRegressor, RFR, RNN, ANN, CNN, GRU, LSTM, Prophet e Transformer também apresentaram resultados satisfatórios, seguindo uma ordem decrescente de desempenho.

No âmbito das previsões de longo prazo, abrangendo casos de um dia, os modelos ARMA, AR, MA, ARIMA, ARIMAX, ARX, SARIMA, SARIMA, XGBRegressor, RFR, LGBMRegressor, DTR, RNN, ANN, CNN, GRU, LSTM, Prophet e Transformer foram avaliados. Uma observação recorrente foi a superioridade dos modelos que incorporam variáveis exógenas em termos de capacidade de previsão, evidenciada nas Figuras de \ref{fig:1-ar-arx-ma} a \ref{fig:prophet1} e nas Tabelas de \ref{tb:apd-trn} a \ref{tb:apd-int}, onde os valores menores foram destacados em \textbf{negrito} para facilitar a análise. O modelo RNN destacou-se tanto nos conjuntos de treinamento quanto na avaliação global, consolidando-se como o modelo mais eficaz nas previsões realizadas.

Cada figura, desde a \ref{fig:1-ar-arx-ma} até a \ref{fig:prophet1}, ilustra cenários distintos de previsão e comparação entre modelos semelhantes. Os modelos Prophet e RNN, sendo este último a escolha superior, são apresentados de forma isolada. A decisão de não incluir o modelo LR na comparação baseou-se na constância observada em suas previsões a longo prazo.

Ao avaliar os modelos de previsão, tanto nas tabelas quanto nas imagens, o modelo RNN destaca-se como a opção mais eficaz. No caso específico da SANEPAR, esse modelo demonstra um desempenho superior em comparação com os demais modelos de previsão adotados. partir da etapa \ref{etp:7}, três métricas amplamente utilizadas na literatura foram empregadas para avaliar e comparar os modelos ARIMA e os modelos de regressão, conforme detalhado na seção \ref{subsec:metrica}.

Na análise dos modelos desenvolvidos, verificou-se que o modelo DTR alcançou o melhor desempenho, tanto para previsões de curto prazo, durante as horas de pico entre 18h e 21h, quanto para outros períodos. Adicionalmente, os modelos MA, AR, SARIMA, ARIMA, SARIMAX, ARIMAX, ARX, LGBMRegressor, XGBRegressor, RFR, RNN, ANN, CNN, GRU, LSTM, Prophet e Transformer também apresentaram resultados satisfatórios, seguindo uma ordem decrescente de desempenho.

No contexto das previsões de longo prazo, abrangendo períodos de um dia, os modelos ARMA, AR, MA, ARIMA, ARIMAX, ARX, SARIMA, SARIMA, XGBRegressor, RFR, LGBMRegressor, DTR, RNN, ANN, CNN, GRU, LSTM, Prophet e Transformer foram avaliados. Destacou-se a superioridade dos modelos que incorporam variáveis exógenas em termos de capacidade de previsão, evidenciada nas Figuras de \ref{fig:1-ar-arx-ma} a \ref{fig:prophet1} e nas Tabelas de \ref{tb:apd-trn} a \ref{tb:apd-int}, onde os valores menores foram destacados em \textbf{negrito} e \textit{itálico} para facilitar a análise. O modelo RNN destacou-se tanto nos conjuntos de treinamento quanto na avaliação global, consolidando-se como o modelo mais eficaz nas previsões realizadas.

Cada Figura, desde a \ref{fig:1-ar-arx-ma} até a \ref{fig:prophet1}, ilustra cenários distintos de previsão e comparação entre modelos semelhantes. Os modelos Prophet e RNN, sendo este último a escolha superior, são apresentados de forma isolada. A decisão de não incluir o modelo LR na comparação baseou-se na constância observada em suas previsões a longo prazo.

Ao avaliar os modelos de previsão, tanto nas tabelas quanto nas imagens, o modelo RNN destaca-se como a opção mais eficaz. No caso específico da SANEPAR, esse modelo demonstra um desempenho superior em comparação com os demais modelos de previsão adotados.


\begin{figure}[!htb]
	\centering
	\caption{Comparação dos modelos AR, ARX e MA}
	\label{fig:1-ar-arx-ma}
	\includegraphics[width=0.6\linewidth]{Resultados/Figuras/1-AR-ARX-MA}
	
\end{figure}
\begin{figure}[!htb]
	\centering
	\caption{Comparação do modelos ARIMAX, SARIMA e SARIMAX}
	\label{fig:1-arimax-sarima-sarimax}
	\includegraphics[width=0.6\linewidth]{Resultados/Figuras/1-ARIMAX-SARIMA-SARIMAX}
	
\end{figure}
\begin{figure}[!htb]
	\centering
	\caption{Comparação dos modelos ARMA e ARIMA}
	\label{fig:1-arma-arima}
	\includegraphics[width=0.6\linewidth]{Resultados/Figuras/1-ARMA-ARIMA}
	
\end{figure}
\begin{figure}[!htb]
	\centering
	\caption{Comparação dos modelos DTR, RFR, XGBoost, Light GBM}
	\label{fig:lr-xgb-lgbm-rf}
	\includegraphics[width=0.6\linewidth]{Resultados/Figuras/LR-XGB-LGBM-RF}
	
\end{figure}
\begin{figure}[!htb]
	\centering
	\caption{Modelo RNN e os vários horizontes }
	\label{fig:rnn}
	\includegraphics[width=0.6\linewidth]{Resultados/Figuras/RNN}
\end{figure}
\begin{figure}[!htb]
	\centering
	\caption{Previsões do modelo Prophet para o reservatório LT01}\label{fig:prophet1}
	\includegraphics[width=0.7\linewidth]{Resultados/Figuras/prophet1}
	
	
\end{figure}



\begin{landscape}
	
	\begin{table}[!htb]
		\centering
		\small % Reduzir o tamanho da fonte
		\setlength{\tabcolsep}{4pt} % Reduzir o espaçamento entre as colunas
		\caption{Comparação dos modelos de previsão com as métricas de desempenho \textbf{treino}}\label{tb:apd-trn}
		\begin{tabular}{@{}cclllllllllllllllllll@{}}
			\toprule
			&          & \multicolumn{12}{c}{Modelos Treino}                                                                                                                                                                                                                                                           & \multicolumn{1}{c}{\textit{}} & \multicolumn{1}{c}{\textit{}} & \multicolumn{1}{c}{\textit{}} & \multicolumn{1}{c}{\textit{}} & \multicolumn{1}{c}{\textit{}} & \multicolumn{1}{c}{\textit{}} & \multicolumn{1}{c}{\textit{}} \\ \midrule
			Horizontes                         & Métricas & \multicolumn{1}{c}{A} & \multicolumn{1}{c}{B} & \multicolumn{1}{c}{C} & \multicolumn{1}{c}{D} & \multicolumn{1}{c}{E} & \multicolumn{1}{c}{F} & \multicolumn{1}{c}{G} & \multicolumn{1}{c}{H} & \multicolumn{1}{c}{I} & \multicolumn{1}{c}{J} & \multicolumn{1}{c}{K} & \multicolumn{1}{c}{L} & \multicolumn{1}{c}{M}         & \multicolumn{1}{c}{N}         & \multicolumn{1}{c}{O}         & \multicolumn{1}{c}{P}         & \multicolumn{1}{c}{Q}         & \multicolumn{1}{c}{R}         & \multicolumn{1}{c}{S}         \\ \toprule
			\multirow{3}{*}{1 hora à frente}   & sMAPE    & 3,91                  & 4,01                  & 4,03                  & 3,91                  & 3,92                  & 3,89                  & 3,82                  & 3,86                  & 8,85                  & 9,31                  & 9,52                  & 9,37                  & 35,4                          & 35,8                          & 9                             & \textbf{0,0665}               & 16,8                          & 23                            & 23                            \\ 
			& MAE      & \textbf{0,25}         & \textbf{0,25}         & \textbf{0,26}         & \textbf{0,25}         & \textbf{0,25}         & \textbf{0,25}         & \textbf{0,24}         & \textbf{0,25}         & 0,36                  & 0,65                  & 0,67                  & 0,65                  & 1,42                          & 1,44                          & \textbf{0,2}                  & \textit{0,0023}               & 0,55                          & 0,83                          & 0,83                          \\
			& RRMSE    & \textbf{0,09}         & \textbf{0,10}         & \textbf{0,09}         & \textbf{0,09}         & \textbf{0,09}         & \textbf{0,09}         & \textbf{0,09}         & \textbf{0,09}         & \textbf{0,21}         & \textbf{0,21}         & \textbf{0,21}         & \textbf{0,21}         & 2,3                           & 0,65                          & \textbf{0,2}                  & \textit{0,0008}               & 0,31                          & 0,48                          & 0,48                          \\ \toprule
			\multirow{3}{*}{6 horas à frente}  & sMAPE    & 9,97                  & 10,1                  & 9,7                   & 9,98                  & 9,97                  & 10                    & 10,1                  & 9,99                  & 6,99                  & 12,4                  & 12,7                  & 9,369                 & 66,2                          & 83,9                          & 20                            & \textit{0,0230}               & 16,7                          & 20,6                          & 20,6                          \\
			& MAE      & 0,64                  & 0,65                  & 0,62                  & 0,64                  & 0,64                  & 0,64                  & 0,65                  & 0,64                  & 0,59                  & 0,9                   & 0,93                  & 0,651                 & 3,37                          & 4,95                          & 0,6                           & \textit{0,0007}               & 0,55                          & 0,72                          & 0,72                          \\
			& RRMSE    & \textbf{0,23}         & \textbf{0,23}         & \textbf{0,23}         & \textbf{0,23}         & \textbf{0,23}         & \textbf{0,23}         & \textbf{0,23}         & \textbf{0,23}         & \textbf{0,16}         & 0,32                  & 0,33                  & \textbf{0,209}        & 5,02                          & 1,71                          & 0,6                           & \textit{0,0006}               & 0,31                          & 0,45                          & 0,45                          \\ \toprule
			\multirow{3}{*}{12 horas à frente} & sMAPE    & 11,6                  & 11,6                  & 11,3                  & 11,6                  & 11,5                  & 11,7                  & 11,8                  & 11,6                  & 6,99                  & 12,4                  & 12,7                  & 9,369                 & 72                            & 98,6                          & 25                            & \textbf{0,0683}               & 16,8                          & 29,2                          & 29,2                          \\
			& MAE      & 0,75                  & 0,75                  & 0,74                  & 0,75                  & 0,75                  & 0,76                  & 0,77                  & 0,75                  & 0,59                  & 0,9                   & 0,93                  & 0,651                 & 3,83                          & 6,69                          & 0,8                           & \textit{0,0022}               & 0,55                          & 1,11                          & 1,11                          \\
			& RRMSE    & \textbf{0,27}         & \textbf{0,27}         & \textbf{0,26}         & \textbf{0,27}         & \textbf{0,26}         & \textbf{0,27}         & \textbf{0,27}         & \textbf{0,27}         & \textbf{0,16}         & 0,32                  & 0,33                  & \textbf{0,209}        & 5,69                          & 2,25                          & 0,9                           & \textit{0,0009}               & 0,31                          & 0,55                          & 0,55                          \\ \toprule
			\multirow{3}{*}{24 horas à frente} & sMAPE    & 6,77                  & 6,85                  & 6,67                  & 6,77                  & 6,69                  & 6,82                  & 6,86                  & 6,82                  & 6,99                  & 12,4                  & 12,7                  & 9,369                 & 74,4                          & 104                           & 26                            & \textbf{0,2328}               & 16,8                          & 26,8                          & 26,8                          \\
			& MAE      & 0,43                  & 0,44                  & 0,43                  & 0,43                  & 0,43                  & 0,44                  & 0,44                  & 0,43                  & 0,59                  & 0,9                   & 0,93                  & 0,651                 & 4,04                          & 7,5                           & 0,8                           & \textit{0,0079}               & 0,55                          & 1                             & 1                             \\
			& RRMSE    & \textbf{0,17}         & \textbf{0,17}         & \textbf{0,17}         & \textbf{0,17}         & \textbf{0,17}         & \textbf{0,17}         & \textbf{0,17}         & \textbf{0,17}         & \textbf{0,16}         & 0,32                  & 0,33                  & \textbf{0,209}        & 5,99                          & 2,5                           & 1                             & \textit{0,0024}               & 0,31                          & 0,52                          & 0,52                          \\ \cmidrule(l){1-21} 				
		\end{tabular}
		
		\captionsetup{justification=centering} % Centralizar a legenda
		Legenda para os Modelos de Previsão: A - AR, B - ARX, C - MA, D - ARMA, E - ARIMA, F - SARIMA, G - ARIMAX, H - SARIMAX, I - Decision Tree Regressor, J - Random Forest Regressor, K - XGBRegressor, L - LGBMRegressor, M - LSTM, N - GRU, O - Prophet, P - RNN, Q - Transformer, R - CNN, S - ANN.
	\end{table}
	
	\newpage
	
	\begin{table}[!htb]
		\centering
		\small % Reduzir o tamanho da fonte
		\setlength{\tabcolsep}{4pt} % Reduzir o espaçamento entre as colunas
		\caption{Comparação dos modelos de previsão com as métricas de desempenho \textbf{teste}}\label{tb:apd-tst}
		\begin{tabular}{@{}cclllllllllllllllllll@{}}
			\toprule
			&          & \multicolumn{12}{c}{Modelos Teste}                                                                                                                                                                                                                                                            & \multicolumn{1}{c}{\textit{}} & \multicolumn{1}{c}{\textit{}} & \multicolumn{1}{c}{\textit{}} & \multicolumn{1}{c}{\textit{}} & \multicolumn{1}{c}{\textit{}} & \multicolumn{1}{c}{\textit{}} & \multicolumn{1}{c}{\textit{}} \\ \midrule
			Horizontes                         & Métricas & \multicolumn{1}{c}{A} & \multicolumn{1}{c}{B} & \multicolumn{1}{c}{C} & \multicolumn{1}{c}{D} & \multicolumn{1}{c}{E} & \multicolumn{1}{c}{F} & \multicolumn{1}{c}{G} & \multicolumn{1}{c}{H} & \multicolumn{1}{c}{I} & \multicolumn{1}{c}{J} & \multicolumn{1}{c}{K} & \multicolumn{1}{c}{L} & \multicolumn{1}{c}{M}         & \multicolumn{1}{c}{N}         & \multicolumn{1}{c}{O}         & \multicolumn{1}{c}{P}         & \multicolumn{1}{c}{Q}         & \multicolumn{1}{c}{R}         & \multicolumn{1}{c}{S}         \\ \toprule
			\multirow{3}{*}{1 hora à frente}   & sMAPE    & 3,93                  & 4,15                  & 3,99                  & 3,93                  & 3,92                  & 3,91                  & 4,16                  & 4,16                  & 7,76                  & 8,46                  & 8,68                  & 8,45                  & 15,6                          & 15,9                          & 9                             & \textbf{0,0744}               & 15,1                          & 20,6                          & 20,6                          \\
			& MAE      & \textbf{0,26}         & \textbf{0,27}         & \textbf{0,26}         & \textbf{0,26}         & \textbf{0,26}         & \textbf{0,26}         & \textbf{0,27}         & \textbf{0,27}         & 0,40                  & 0,61                  & 0,63                  & 0,61                  & 0,53                          & 0,54                          & \textbf{0,2}                  & \textit{0,0024}               & 0,52                          & 0,76                          & 0,76                          \\
			& RRMSE    & \textbf{0,09}         & \textbf{0,10}         & \textbf{0,09}         & \textbf{0,09}         & \textbf{0,09}         & \textbf{0,09}         & \textbf{0,10}         & \textbf{0,10}         & \textbf{0,18}         & \textbf{0,19}         & \textbf{0,20}         & \textbf{0,19}         & 1,01                          & 0,33                          & \textbf{0,2}                  & \textit{0,0029}               & 0,34                          & 0,5                           & 0,5                           \\ \toprule
			\multirow{3}{*}{6 horas à frente}  & sMAPE    & 9,74                  & 9,94                  & 9,44                  & 9,74                  & 9,71                  & 9,76                  & 9,96                  & 9,96                  & 6,36                  & 10,7                  & 11                    & 8,446                 & 59,5                          & 72,7                          & 20                            & \textit{0,0308}               & 15,1                          & 17,3                          & 17,3                          \\
			& MAE      & 0,65                  & 0,66                  & 0,63                  & 0,65                  & 0,65                  & 0,65                  & 0,66                  & 0,66                  & 0,56                  & 0,8                   & 0,82                  & 0,609                 & 2,97                          & 4,04                          & 0,6                           & \textit{0,0007}               & 0,51                          & 0,62                          & 0,62                          \\
			& RRMSE    & \textbf{0,23}         & \textbf{0,23}         & \textbf{0,22}         & \textbf{0,23}         & \textbf{0,23}         & \textbf{0,23}         & \textbf{0,23}         & \textbf{0,23}         & \textbf{0,14}         & \textbf{0,28}         & \textbf{0,29}         & \textbf{0,191}        & 4,9                           & 1,42                          & 0,6                           & \textit{0,0033}               & 0,34                          & 0,46                          & 0,46                          \\ \toprule
			\multirow{3}{*}{12 horas à frente} & sMAPE    & 11,1                  & 11,2                  & 10,9                  & 11,1                  & 11,1                  & 11,2                  & 11,2                  & 11,3                  & 6,36                  & 10,8                  & 11                    & 8,446                 & 68,4                          & 94,1                          & 25                            & \textbf{0,0745}               & 15,1                          & 18,8                          & 18,8                          \\
			& MAE      & 0,74                  & 0,75                  & 0,73                  & 0,74                  & 0,74                  & 0,75                  & 0,75                  & 0,75                  & 0,56                  & 0,8                   & 0,82                  & 0,609                 & 3,67                          & 6,31                          & 0,8                           & \textit{0,0023}               & 0,52                          & 0,68                          & 0,68                          \\
			& RRMSE    & \textbf{0,26}         & \textbf{0,26}         & \textbf{0,25}         & \textbf{0,26}         & \textbf{0,26}         & \textbf{0,26}         & \textbf{0,26}         & \textbf{0,26}         & \textbf{0,14}         & \textbf{0,28}         & \textbf{0,29}         & \textbf{0,191}        & 6,01                          & 2,11                          & 0,9                           & \textit{0,0032}               & 0,34                          & 0,48                          & 0,48                          \\ \toprule
			\multirow{3}{*}{24 horas à frente} & sMAPE    & 6,15                  & 6,34                  & 6,08                  & 6,15                  & 6,14                  & 6,24                  & 6,36                  & 6,37                  & 6,36                  & 10,7                  & 11                    & 8,446                 & 71,5                          & 102                           & 26                            & \textbf{0,2385}               & 15,1                          & 18,1                          & 18,1                          \\
			& MAE      & 0,4                   & 0,41                  & 0,4                   & 0,4                   & 0,4                   & 0,41                  & 0,42                  & 0,42                  & 0,56                  & 0,8                   & 0,83                  & 0,609                 & 3,92                          & 7,36                          & 0,8                           & \textit{0,0081}               & 0,52                          & 0,65                          & 0,65                          \\
			& RRMSE    & \textbf{0,16}         & \textbf{0,16}         & \textbf{0,16}         & \textbf{0,16}         & \textbf{0,16}         & \textbf{0,16}         & \textbf{0,16}         & \textbf{0,16}         & \textbf{0,14}         & \textbf{0,28}         & \textbf{0,29}         & \textbf{0,191}        & 6,42                          & 2,43                          & 1                             & \textit{0,0041}               & 0,34                          & 0,47                          & 0,47                          \\ \cmidrule(l){1-21} 
		\end{tabular}
		
		\captionsetup{justification=centering} % Centralizar a legenda
		Legenda para os Modelos de Previsão: A - AR, B - ARX, C - MA, D - ARMA, E - ARIMA, F - SARIMA, G - ARIMAX, H - SARIMAX, I - Decision Tree Regressor, J - Random Forest Regressor, K - XGBRegressor, L - LGBMRegressor, M - LSTM, N - GRU, O - Prophet, P - RNN, Q - Transformer, R - CNN, S - ANN.
	\end{table}
	
	\newpage
	
	\begin{table}[!htb]
		\centering
		\small % Reduzir o tamanho da fonte
		\setlength{\tabcolsep}{4pt} % Reduzir o espaçamento entre as colunas
		\caption{Comparação dos modelos de previsão com as métricas de desempenho \textbf{validação}}\label{tb:apd-vld}
		\begin{tabular}{@{}cclllllllllllllllllll@{}}
			\toprule
			&          & \multicolumn{12}{c}{Modelos Validação}                                                                                                                                                                                                                                                        & \multicolumn{1}{c}{\textit{}} & \multicolumn{1}{c}{\textit{}} & \multicolumn{1}{c}{\textit{}} & \multicolumn{1}{c}{\textit{}} & \multicolumn{1}{c}{\textit{}} & \multicolumn{1}{c}{\textit{}} & \multicolumn{1}{c}{\textit{}} \\ \midrule
			Horizontes                         & Métricas & \multicolumn{1}{c}{A} & \multicolumn{1}{c}{B} & \multicolumn{1}{c}{C} & \multicolumn{1}{c}{D} & \multicolumn{1}{c}{E} & \multicolumn{1}{c}{F} & \multicolumn{1}{c}{G} & \multicolumn{1}{c}{H} & \multicolumn{1}{c}{I} & \multicolumn{1}{c}{J} & \multicolumn{1}{c}{K} & \multicolumn{1}{c}{L} & \multicolumn{1}{c}{M}         & \multicolumn{1}{c}{N}         & \multicolumn{1}{c}{O}         & \multicolumn{1}{c}{P}         & \multicolumn{1}{c}{Q}         & \multicolumn{1}{c}{R}         & \multicolumn{1}{c}{S}         \\ \toprule
			\multirow{3}{*}{1 hora à frente}   & sMAPE    & 4,08                  & 4,28                  & 4,20                  & 4,09                  & 4,10                  & 4,20                  & 4,26                  & 4,29                  & 8,54                  & 10,47                 & 10,66                 & 10,45                 & 29,8                          & 29,4                          & 9                             & \textbf{0,0675}               & 17,4                          & 18,3                          & 18,3                          \\
			& MAE      & \textbf{0,25}         & \textbf{0,26}         & \textbf{0,26}         & \textbf{0,25}         & \textbf{0,25}         & \textbf{0,26}         & \textbf{0,26}         & \textbf{0,26}         & 0,32                  & 0,72                  & 0,74                  & 0,72                  & 1,1                           & 1,08                          & \textbf{0,2}                  & \textit{0,0023}               & 0,56                          & 0,6                           & 0,6                           \\
			& RRMSE    & \textbf{0,10}         & \textbf{0,10}         & \textbf{0,10}         & \textbf{0,10}         & \textbf{0,10}         & \textbf{0,10}         & \textbf{0,10}         & \textbf{0,10}         & \textbf{0,20}         & \textbf{0,23}         & \textbf{0,24}         & \textbf{0,23}         & 1,87                          & 0,56                          & \textbf{0,2}                  & \textit{0,0008}               & 0,33                          & 0,39                          & 0,39                          \\ \toprule
			\multirow{3}{*}{6 horas à frente}  & sMAPE    & 10,9                  & 11,1                  & 10,6                  & 10,9                  & 10,9                  & 11                    & 11,1                  & 11,1                  & 6,8                   & 13,9                  & 14,2                  & 10,45                 & 67,9                          & 84                            & 20                            & \textit{0,0229}               & 17,4                          & 20,5                          & 20,5                          \\
			& MAE      & 0,68                  & 0,69                  & 0,66                  & 0,68                  & 0,68                  & 0,69                  & 0,69                  & 0,69                  & 0,57                  & 1,01                  & 1,04                  & 0,721                 & 3,39                          & 4,81                          & 0,6                           & \textit{0,0007}               & 0,56                          & 0,69                          & 0,69                          \\
			& RRMSE    & \textbf{0,25}         & \textbf{0,25}         & \textbf{0,24}         & \textbf{0,25}         & \textbf{0,25}         & \textbf{0,25}         & \textbf{0,25}         & \textbf{0,25}         & \textbf{0,16}         & \textbf{0,36}         & \textbf{0,37}         & \textbf{0,233}        & 4,98                          & 1,72                          & 0,6                           & \textit{0,0005}               & 0,34                          & 0,44                          & 0,44                          \\ \toprule
			\multirow{3}{*}{12 horas à frente} & sMAPE    & 12,7                  & 12,8                  & 12,4                  & 12,7                  & 12,6                  & 12,8                  & 12,8                  & 12,8                  & 6,8                   & 13,9                  & 14,2                  & 10,45                 & 74,4                          & 100                           & 25                            & \textbf{0,0689}               & 17,4                          & 22,9                          & 22,9                          \\
			& MAE      & 0,8                   & 0,81                  & 0,79                  & 0,8                   & 0,8                   & 0,81                  & 0,81                  & 0,81                  & 0,57                  & 1,01                  & 1,04                  & 0,721                 & 3,92                          & 6,71                          & 0,8                           & \textit{0,0022}               & 0,56                          & 0,79                          & 0,79                          \\
			& RRMSE    & \textbf{0,29}         & \textbf{0,29}         & \textbf{0,28}         & \textbf{0,29}         & \textbf{0,29}         & \textbf{0,29}         & \textbf{0,29}         & \textbf{0,29}         & \textbf{0,16}         & \textbf{0,36}         & \textbf{0,37}         & \textbf{0,233}        & 5,73                          & 2,33                          & 0,9                           & \textit{0,0008}               & 0,33                          & 0,48                          & 0,48                          \\ \toprule
			\multirow{3}{*}{24 horas à frente} & sMAPE    & 7,3                   & 7,45                  & 7,19                  & 7,3                   & 7,27                  & 7,37                  & 7,43                  & 7,46                  & 6,8                   & 13,9                  & 14,2                  & 10,45                 & 76,9                          & 106                           & 26                            & \textbf{0,2342}               & 17,4                          & 22,9                          & 22,9                          \\
			& MAE      & 0,46                  & 0,46                  & 0,45                  & 0,46                  & 0,45                  & 0,46                  & 0,46                  & 0,46                  & 0,57                  & 1,01                  & 1,04                  & 0,721                 & 4,14                          & 7,59                          & 0,8                           & \textit{0,0077}               & 0,56                          & 0,79                          & 0,79                          \\
			& RRMSE    & \textbf{0,18}         & \textbf{0,18}         & \textbf{0,18}         & \textbf{0,18}         & \textbf{0,18}         & \textbf{0,18}         & \textbf{0,18}         & \textbf{0,18}         & \textbf{0,16}         & \textbf{0,36}         & \textbf{0,37}         & \textbf{0,233}        & 6,04                          & 2,61                          & 1                             & \textit{0,0024}               & 0,33                          & 0,48                          & 0,48                          \\ \cmidrule(l){1-21} 
		\end{tabular}
		
		\captionsetup{justification=centering} % Centralizar a legenda
		Legenda para os Modelos de Previsão: A - AR, B - ARX, C - MA, D - ARMA, E - ARIMA, F - SARIMA, G - ARIMAX, H - SARIMAX, I - Decision Tree Regressor, J - Random Forest Regressor, K - XGBRegressor, L - LGBMRegressor, M - LSTM, N - GRU, O - Prophet, P - RNN, Q - Transformer, R - CNN, S - ANN.
	\end{table}
	
	\newpage
	
	\begin{table}[!htb]
		\centering
		\small % Reduzir o tamanho da fonte
		\setlength{\tabcolsep}{4pt} % Reduzir o espaçamento entre as colunas
		\caption{Comparação dos modelos de previsão com as métricas de desempenho \textbf{inteiro}}\label{tb:apd-int}
		\begin{tabular}{@{}cclllllllllllllllllll@{}}
			\toprule
			&          & \multicolumn{12}{c}{Modelos Inteiros}                                                                                                                                                                                                                                                         & \multicolumn{1}{c}{\textit{}} & \multicolumn{1}{c}{\textit{}} & \multicolumn{1}{c}{\textit{}} & \multicolumn{1}{c}{\textit{}} & \multicolumn{1}{c}{\textit{}} & \multicolumn{1}{c}{\textit{}} & \multicolumn{1}{c}{\textit{}} \\ \midrule
			Horizontes                         & Métricas & \multicolumn{1}{c}{A} & \multicolumn{1}{c}{B} & \multicolumn{1}{c}{C} & \multicolumn{1}{c}{D} & \multicolumn{1}{c}{E} & \multicolumn{1}{c}{F} & \multicolumn{1}{c}{G} & \multicolumn{1}{c}{H} & \multicolumn{1}{c}{I} & \multicolumn{1}{c}{J} & \multicolumn{1}{c}{K} & \multicolumn{1}{c}{L} & \multicolumn{1}{c}{M}         & \multicolumn{1}{c}{N}         & \multicolumn{1}{c}{O}         & \multicolumn{1}{c}{P}         & \multicolumn{1}{c}{Q}         & \multicolumn{1}{c}{R}         & \multicolumn{1}{c}{S}         \\ \toprule
			\multirow{3}{*}{1 hora à frente}   & sMAPE    & 3,94                  & 4,08                  & 4,05                  & 3,93                  & 3,95                  & 3,91                  & 4,05                  & 4,05                  & 8,51                  & 9,22                  & 9,43                  & 9,244                 & 17,1                          & 17,4                          & 9                             & \textbf{0,0690}               & 16,4                          & 22,5                          & 22,5                          \\
			& MAE      & \textbf{0,25}         & \textbf{0,26}         & \textbf{0,26}         & \textbf{0,25}         & \textbf{0,25}         & \textbf{0,25}         & \textbf{0,26}         & \textbf{0,26}         & 0,36                  & 0,65                  & 0,67                  & 0,648                 & 0,57                          & 0,58                          & \textbf{0,2}                  & \textit{0,0023}               & 0,54                          & 0,81                          & 0,81                          \\
			& RRMSE    & \textbf{0,09}         & \textbf{0,1}          & \textbf{0,09}         & \textbf{0,09}         & \textbf{0,09}         & \textbf{0,09}         & \textbf{0,1}          & \textbf{0,1}          & \textbf{0,2}          & \textbf{0,21}         & \textbf{0,21}         & \textbf{0,207}        & 1,01                          & 0,31                          & \textbf{0,2}                  & \textit{0,0017}               & 0,32                          & 0,49                          & 0,49                          \\ \toprule
			\multirow{3}{*}{6 horas à frente}  & sMAPE    & 10                    & 10,2                  & 9,75                  & 10                    & 10                    & 10,1                  & 10,2                  & 10,1                  & 6,77                  & 12,1                  & 12,4                  & 12,07                 & 61,7                          & 74,6                          & 20                            & \textit{0,0253}               & 16,3                          & 20                            & 20                            \\
			& MAE      & 0,65                  & 0,66                  & 0,63                  & 0,65                  & 0,65                  & 0,65                  & 0,66                  & 0,65                  & 0,58                  & 0,89                  & 0,91                  & 0,885                 & 3,04                          & 4,08                          & 0,6                           & \textit{0,0007}               & 0,54                          & 0,7                           & 0,7                           \\
			& RRMSE    & \textbf{0,23}         & \textbf{0,23}         & \textbf{0,23}         & \textbf{0,23}         & \textbf{0,23}         & \textbf{0,23}         & \textbf{0,23}         & \textbf{0,23}         & \textbf{0,16}         & \textbf{0,32}         & \textbf{0,32}         & \textbf{0,316}        & 4,65                          & 1,45                          & 0,6                           & \textit{0,0019}               & 0,33                          & 0,46                          & 0,46                          \\ \toprule
			\multirow{3}{*}{12 horas à frente} & sMAPE    & 11,6                  & 11,7                  & 11,4                  & 11,6                  & 11,6                  & 11,7                  & 11,8                  & 11,7                  & 6,77                  & 12,1                  & 12,4                  & 12,12                 & 70,7                          & 96                            & 25                            & \textbf{0,0703}               & 16,4                          & 28,7                          & 28,7                          \\
			& MAE      & 0,76                  & 0,76                  & 0,74                  & 0,76                  & 0,75                  & 0,76                  & 0,77                  & 0,76                  & 0,58                  & 0,89                  & 0,91                  & 0,889                 & 3,75                          & 6,38                          & 0,8                           & \textit{0,0023}               & 0,54                          & 1,09                          & 1,09                          \\
			& RRMSE    & \textbf{0,27}         & \textbf{0,27}         & \textbf{0,26}         & \textbf{0,27}         & \textbf{0,27}         & \textbf{0,27}         & \textbf{0,27}         & \textbf{0,27}         & \textbf{0,16}         & \textbf{0,32}         & \textbf{0,32}         & \textbf{0,317}        & 5,69                          & 2,16                          & 0,9                           & \textit{0,0019}               & 0,32                          & 0,56                          & 0,56                          \\ \toprule
			\multirow{3}{*}{24 horas à frente} & sMAPE    & 6,66                  & 6,79                  & 6,57                  & 6,66                  & 6,6                   & 6,71                  & 6,82                  & 6,8                   & 6,77                  & 12,1                  & 12,4                  & 12,21                 & 73,8                          & 104                           & 26                            & \textbf{0,2347}               & 16,4                          & 26,2                          & 26,2                          \\
			& MAE      & 0,43                  & 0,43                  & 0,42                  & 0,43                  & 0,42                  & 0,43                  & 0,44                  & 0,43                  & 0,58                  & 0,89                  & 0,92                  & 0,897                 & 4,01                          & 7,44                          & 0,8                           & \textit{0,0080}               & 0,54                          & 0,98                          & 0,98                          \\
			& RRMSE    & \textbf{0,17}         & \textbf{0,17}         & \textbf{0,17}         & \textbf{0,17}         & \textbf{0,17}         & \textbf{0,17}         & \textbf{0,17}         & \textbf{0,17}         & \textbf{0,16}         & \textbf{0,32}         & \textbf{0,32}         & \textbf{0,319}        & 6,07                          & 2,49                          & 1                             & \textit{0,0030}               & 0,32                          & 0,53                          & 0,53                          \\ \cmidrule(l){1-21} 
		\end{tabular}
		
		\captionsetup{justification=centering} % Centralizar a legenda
		Legenda para os Modelos de Previsão: A - AR, B - ARX, C - MA, D - ARMA, E - ARIMA, F - SARIMA, G - ARIMAX, H - SARIMAX, I - Decision Tree Regressor, J - Random Forest Regressor, K - XGBRegressor, L - LGBMRegressor, M - LSTM, N - GRU, O - Prophet, P - RNN, Q - Transformer, R - CNN, S - ANN.
	\end{table}
	
\end{landscape}

\textbf{Teste de signific\^ancia}
na etapa \ref{etp:8}, realizou-se o teste de Friedman e o teste de Nemenyi para comparar as classificações médias entre os diversos classificadores. O teste de Nemenyi é uma ferramenta de comparação múltipla frequentemente empregada após a aplicação de testes não paramétricos com três ou mais fatores.

A matriz de comparação entre os classificadores, apresentada na Tabela \ref{tb:nemeyi}, exibe os valores de comparação múltipla de Nemenyi, onde as entradas evidenciam as diferenças significativas entre os pares de classificadores.

A Tabela \ref{tb:nemeyi} apresenta os resultados do teste de Nemenyi, um método utilizado para comparar as classificações médias entre diferentes classificadores após a aplicação de testes não paramétricos com três ou mais fatores. Cada célula da tabela mostra os valores de comparação múltipla de Nemenyi, que indicam as diferenças significativas entre os pares de classificadores. O valor na interseção da linha $i$ e da coluna $j$ representa a diferença significativa entre os classificadores $i$ e $j$.

\begin{table}[!htb]
	\centering
	\caption{Teste Nemenyi}\label{tb:nemeyi}
	\begin{tabular}{@{}clllllllll@{}}
		\toprule
		\multicolumn{1}{l}{\textbf{Nemenyi}} & \multicolumn{1}{c}{\textbf{B1}} & \multicolumn{1}{c}{\textbf{B2}} & \multicolumn{1}{c}{\textbf{B3}} & \multicolumn{1}{c}{\textbf{LT01}} & \multicolumn{1}{c}{\textbf{FT01}} & \multicolumn{1}{c}{\textbf{FT02}} & \multicolumn{1}{c}{\textbf{FT03}} & \multicolumn{1}{c}{\textbf{PT01}} & \multicolumn{1}{c}{\textbf{PT02}} \\ \midrule
		\textbf{B1}                         & 1.000                          & 0.001                          & 0.001                          & 0.001                          & 0.001                          & 0.001                          & 0.001                          & 0.001                          & 0.001                          \\
		\textbf{B2}                         & 0.001                          & 1.000                          & 0.001                          & 0.001                          & 0.001                          & 0.001                          & 0.001                          & 0.001                          & 0.001                          \\
		\textbf{B3}                         & 0.001                          & 0.001                          & 1.000                          & 0.001                          & 0.001                          & 0.001                          & 0.001                          & 0.001                          & 0.001                          \\
		\textbf{LT01}                       & 0.001                          & 0.001                          & 0.001                          & 1.000                          & 0.001                          & 0.001                          & 0.001                          & 0.001                          & 0.001                          \\
		\textbf{FT01}                       & 0.001                          & 0.001                          & 0.001                          & 0.001                          & 1.000                          & 0.001                          & 0.131                          & 0.001                          & 0.001                          \\
		\textbf{FT02}                       & 0.001                          & 0.001                          & 0.001                          & 0.001                          & 0.001                          & 1.000                          & 0.001                          & 0.001                          & 0.001                          \\
		\textbf{FT03}                       & 0.001                          & 0.001                          & 0.001                          & 0.001                          & 0.131                          & 0.001                          & 1.000                          & 0.001                          & 0.001                          \\
		\textbf{PT01}                       & 0.001                          & 0.001                          & 0.001                          & 0.001                          & 0.001                          & 0.001                          & 0.001                          & 1.000                          & 0.001                          \\
		\textbf{PT02}                       & 0.001                          & 0.001                          & 0.001                          & 0.001                          & 0.001                          & 0.001                          & 0.001                          & 0.001                          & 1.000                          \\ \bottomrule
	\end{tabular}
\end{table}



No contexto do estudo, os resultados da análise comparativa revelaram diferenças estatisticamente significativas entre vários pares de classificadores, como indicado pelas entradas da tabela. Isso sugere que pelo menos um modelo é considerado estatisticamente superior aos demais, com base nas comparações realizadas.

O valor crítico CD foi utilizado para determinar se dois classificadores eram significativamente diferentes entre si. Esse valor é calculado com base no valor crítico obtido da Tabela \ref{tb:nemeyi} de teste de Nemenyi, o número de classificadores e o número total de amostras. O valor CD é uma métrica que auxilia na interpretação das diferenças entre os classificadores, ajudando a identificar quais pares de classificadores apresentam diferenças estatisticamente significativas.

Os resultados da pesquisa indicaram a existência de evidências estatísticas que sugerem a superioridade de pelo menos um modelo em relação aos demais. Além disso, a análise de comparação significativa entre os modelos revelou pares de classificadores que apresentam diferenças estatisticamente significativas em seus desempenhos. Essas informações são valiosas para a seleção e avaliação dos modelos de classificação, permitindo uma compreensão mais precisa das diferenças de desempenho entre os classificadores avaliados no estudo. Na Tabela \ref{tab:metrics} é mostrado como cada modelos foi comparado entre si em 24 passos à frente.

\begin{table}[!htb]
	\centering
	\caption{Métricas de avaliação dos modelos}
	\label{tab:metrics}
	\small
	
	\begin{tabular}{cccc}
		\hline
		
		\textbf{Modelo} & \textbf{sMAPE} & \textbf{MAE} & \textbf{RRMSE} \\
		\hline
		Prophet & 25,67 & 0,844 & 0,975 \\
		Transformer & 16,39 & 0,544 & 0,324 \\
		ANN & 26,22 & 0,980 & 0,531 \\
		CNN & 26,22 & 0,980 & 0,531 \\
		\textbf{RNN} & \textbf{0,235} & \textbf{0,008} & \textbf{0,003} \\
		LSTM & 73,75 & 4,010 & 6,068 \\
		GRU & 103,57 & 7,443 & 2,485 \\
		AR & 6,66 & 0,428 & 0,169 \\
		ARX & 6,79 & 0,434 & 0,173 \\
		MA & 6,57 & 0,423 & 0,166 \\
		ARMA & 6,66 & 0,428 & 0,169 \\
		ARIMA & 6,60 & 0,424 & 0,167 \\
		SARIMA & 6,71 & 0,432 & 0,170 \\
		ARIMAX & 6,82 & 0,436 & 0,173 \\
		SARIMAX & 6,80 & 0,435 & 0,173 \\
		DTR & 6,77 & 0,577 & 0,158 \\
		RFR & 12,09 & 0,886 & 0,316 \\
		XGBRegressor & 12,41 & 0,916 & 0,323 \\
		LGBMRegressor & 12,21 & 0,897 & 0,319 \\
		\hline
	\end{tabular}
\end{table}




\noindent\textbf{Modelo com menor valor em cada métrica:}	
Primeiramente, os diversos modelos de previsão de séries temporais foram avaliados para um horizonte de previsão de um dia. Para cada métrica (\textbf{sMAPE}, \textbf{MAE} e \textbf{RRMSE}), identificou-se o modelo que apresentou o menor valor.
A métrica \textbf{sMAPE} apontou que o modelo \textbf{RNN} obteve o menor valor.
Quanto à métrica \textbf{MAE}, novamente o modelo \textbf{RNN} demonstrou o menor valor.
A métrica \textbf{RRMSE} também indicou que o modelo \textbf{RNN} teve o menor valor.


\noindent\textbf{Evidências estatísticas de que pelo menos um modelo é superior:}
Para validar estatisticamente as diferenças entre os modelos, foi realizado um teste estatístico denominado \textbf{Teste de Friedman}. Esse teste avalia o desempenho dos modelos em todas as métricas simultaneamente. O resultado do teste de Friedman revelou \textbf{evidências estatísticas} que pelo menos um dos modelos apresenta superioridade estatística em relação aos demais, considerando um nível de significância de $0.05$.

\noindent\textbf{Comparação significativa entre modelos - Teste de Nemenyi:}	
A fim de determinar quais modelos apresentam diferenças estatisticamente significativas entre si, foi conduzido o \textbf{teste de comparações múltiplas de Nemenyi}. Esse teste avalia todos os pares possíveis de modelos e identifica quais deles possuem diferenças estatisticamente significativas. Os resultados indicaram \textbf{diferenças estatisticamente significativas} entre vários pares de modelos. Especificamente:

O modelo \textbf{RNN} apresentou diferenças significativas em relação aos modelos \textbf{LSTM} e \textbf{GRU}.
O modelo \textbf{LSTM} apresentou diferenças significativas em relação ao modelo \textbf{RNN}.
O modelo \textbf{GRU} exibiu diferenças significativas em relação ao modelo \textbf{RNN}.
Com base na análise estatística de Friedman e no teste de comparações múltiplas de Nemenyi, conclui-se que o modelo \textbf{RNN} apresenta o melhor desempenho geral em relação às métricas consideradas (\textbf{sMAPE}, \textbf{MAE} e \textbf{RRMSE}) para um horizonte de previsão de um dia, utilizando os dados completos.


\subsubsection{Compara\c c\~ao dos Modelos}

Com o objetivo de obter uma análise mais aprofundada do desempenho de cada modelo, foi realizada uma comparação por meio de um gráfico de violino e de barra. Dessa forma, pôde-se observar qual dos modelos apresentava o melhor desempenho.



Ao examinar os modelos representados nas Figuras \ref{fig:modelos-arima} e \ref{fig:violin-lr-xgb-lgbm-rf}, identifico os modelos que se destacam em relação à natureza dos dados. Na Figura \ref{fig:basic_comparar}, que compara os modelos ARIMA e XGBoost com outros, torna-se evidente que os modelos ARIMA como AR, ARX, MA, ARMA, ARIMAX e SARIMAX demonstram um desempenho sólido. Além disso, os modelos baseados em gradientes e regressão, como o XGBoost, exibem resultados comparáveis, beneficiando-se da otimização por meio do Optuna, uma abordagem de bayesiana usando o metodo TPE.

Na Figura \ref{fig:rrmse_comparar}, que contrasta as redes neurais com o modelo Prophet, é importante destacar que os modelos de redes neurais, incluindo RNN, LSTM, GRU, ANN, CNN e Transformer, foram avaliados em conjunto com o modelo Prophet. A análise estatística também demonstrou que o modelo RNN se sobressai como o vencedor entre as métricas avaliadas. Essa conclusão é respaldada pelas evidências de que pelo menos um modelo é superior aos demais. Os modelos com valores de p-valor abaixo de $0,05$ foram realçados em \textit{itálico} para enfatizar sua significância.

\begin{figure}[!htb]
	\centering
	\caption{Comparação dos modelos ARIMA}\label{fig:modelos-arima}
	\includegraphics[width=\linewidth]{Resultados/Figuras/modelos-arima}
	
	
\end{figure}

Na Figura \ref{fig:violin-lr-xgb-lgbm-rf}, é feita uma comparação entre os modelos de gradiente e regressor. Esses modelos, por serem mais robustos e utilizar técnicas de otimização mais avançadas, mostram-se superiores aos modelos comparados. O modelo XGBoost, em particular, é identificado como superior em relação aos outros modelos na análise.

\begin{figure}[!htb]
	\centering
	\caption{Comparação de modelos de regressão}\label{fig:violin-lr-xgb-lgbm-rf}
	\includegraphics[width=\linewidth]{Resultados/Figuras/violin-LR-XGB-LGBM-RF}
	
\end{figure}

Na Figura \ref{fig:rrmse_comparar}, nota-se que todos os modelos trabalhados aqui, exceto o modelo LR, foram comparados em relação às métricas de desempenho. Mesmo sendo muito robustos, esses modelos não conseguiram obter um resultado tão bom quanto o RNN.
\begin{figure}[!htb]
	\centering
	\caption{Comparação dos modelos na métrica RRMSE\label{fig:rrmse_comparar}}
	\includegraphics[width=\linewidth]{Resultados/Figuras/rrmse_comparar}
\end{figure}

\begin{figure}[!htb]
	\centering
	\caption{Comparação dos modelos nas métricas sMAPE, MAE e RRMSE\label{fig:basic_comparar}}
	\includegraphics[width=\linewidth]{Resultados/Figuras/basic_comparar}
\end{figure}



A avaliação da eficácia dos modelos ARIMA em previsões de longo prazo emprega o teste de Ljung-Box, conforme detalhado nas Tabelas \ref{tb:lbtrn} a \ref{tb:lbcm} ilustram a acurácia dos modelos ARIMA ao longo do tempo, com valores menores sendo destacados em \textbf{negrito}  para facilitar a interpretação. Modelos como ARX, ARIMAX e SARIMAX, que incorporam variáveis exógenas, demonstram um desempenho superior nesse contexto. Esses modelos não lineares apresentam uma capacidade de previsão robusta em horizontes temporais mais longos, diferenciando-se positivamente dos outros modelos ARIMA. Na Figura \ref{fig:modelos-arima}, são selecionados os modelos ARIMA e seus antecessores. Esses modelos têm suas limitações, tanto para horizontes de previsão de curto prazo quanto para horizontes de longo prazo. Nessa comparação no gráfico de violino, são combinados vários outros gráficos em um só, como o gráfico de barras e o \textit{boxplot}. Esse gráfico pode fornecer várias informações, mas o objetivo aqui é identificar apenas o melhor modelo entre os modelos ARIMA.

Como essa série não apresentou uma estacionariedade bem definida e os dados não a tornaram estacionária, os modelos que não têm sazonalidade mostraram-se superiores, tais como AR, MA, ARX, ARMA, ARIMA e ARIMAX. O modelo ARIMAX demonstrou ser bastante robusto para este caso, mas mesmo assim, modelos mais básicos como AR e MA ainda apresentaram resultados melhores.

\begin{table}[H]
	\centering		
	\caption{Comparação dos modelos Ljung Box: Modelos ARIMA com defasagem de 10 para previsão de longo prazo na demanda de água}
	
	\begin{subtable}{0.46\linewidth}
		\centering
		\caption{\textbf{Treinamento}} \label{tb:lbtrn}
		\begin{tabular}{@{}ccc@{}}
			\toprule
			\multirow{5}{*}{\begin{tabular}[c]{@{}c@{}}Ljung \\ Box\end{tabular}} & \multirow{5}{*}{\begin{tabular}[c]{@{}c@{}}Estatística\\ de Teste\end{tabular}} & \multirow{5}{*}{\begin{tabular}[c]{@{}c@{}}Valor \\ De p\end{tabular}} \\
			& & \\
			& & \\
			& & \\
			& & \\ \midrule
			ARX & 59,677 & \textbf{0,000} \\
			AR & 52,312 & \textbf{0,265} \\
			MA & 57,268 & \textbf{0,000} \\
			ARMA & \textbf{6,945} & \textbf{0,731} \\
			ARIMA & 16,724 & 0,081 \\
			SARIMA & 48,505 & \textbf{0,000} \\
			ARIMAX & 89,931 & \textbf{0,000} \\
			SARIMAX & 29,093 & \textbf{0,000} \\ \bottomrule
		\end{tabular}
	\end{subtable}
	\hfill
	\begin{subtable}{0.46\linewidth}
		\centering
		\caption{\textbf{Teste}} \label{tb:lbtst}
		\begin{tabular}{@{}ccc@{}}
			\toprule
			\multirow{5}{*}{\begin{tabular}[c]{@{}c@{}}Ljung \\ Box\end{tabular}} & \multirow{5}{*}{\begin{tabular}[c]{@{}c@{}}Estatística\\ de Teste\end{tabular}} & \multirow{5}{*}{\begin{tabular}[c]{@{}c@{}}Valor \\ De p\end{tabular}} \\
			& & \\
			& & \\
			& & \\
			& & \\ \midrule
			ARX & 47,177 & \textbf{0,000} \\
			AR & 49,965 & 0,444 \\
			MA & 77,884 & \textbf{0,000} \\
			ARMA & \textbf{1,545} & 0,999 \\
			ARIMA & \textbf{5,354} & 0,866 \\
			SARIMA & 24,663 & \textbf{0,006} \\
			ARIMAX & 36,738 & \textbf{0,000} \\
			SARIMAX & 21,236 & \textbf{0,020} \\ \bottomrule
		\end{tabular}
	\end{subtable}
	
	\vspace{1em}
	
	\begin{subtable}{0.46\linewidth}
		\centering
		\caption{\textbf{Validação}} \label{tb:lbvld}
		\begin{tabular}{@{}ccc@{}}
			\toprule
			\multirow{5}{*}{\begin{tabular}[c]{@{}c@{}}Ljung \\ Box\end{tabular}} & \multirow{5}{*}{\begin{tabular}[c]{@{}c@{}}Estatística\\ de Teste\end{tabular}} & \multirow{5}{*}{\begin{tabular}[c]{@{}c@{}}Valor \\ De p\end{tabular}} \\
			& & \\
			& & \\
			& & \\
			& & \\ \midrule
			ARX & \textbf{5,108} & 0,884 \\
			AR & 4,360 & 0,930 \\
			MA & 46,252 & \textbf{0,000} \\
			ARMA & \textbf{7,515} & 0,676 \\
			ARIMA & \textbf{7,738} & 0,654 \\
			SARIMA & 28,998 & \textbf{0,001} \\
			ARIMAX & \textbf{6,115} & \textbf{0,000} \\
			SARIMAX & \textbf{4,443} & 0,925 \\ \bottomrule
		\end{tabular}
	\end{subtable}
	\hfill
	\begin{subtable}{0.46\linewidth}
		\centering
		\caption{\textbf{Inteiro}} \label{tb:lbcm}
		\begin{tabular}{@{}ccc@{}}
			\toprule
			\multirow{5}{*}{\begin{tabular}[c]{@{}c@{}}Ljung \\ Box\end{tabular}} & \multirow{5}{*}{\begin{tabular}[c]{@{}c@{}}Estatística\\ de Teste\end{tabular}} & \multirow{5}{*}{\begin{tabular}[c]{@{}c@{}}Valor \\ De p\end{tabular}} \\
			& & \\
			& & \\
			& & \\
			& & \\ \midrule
			ARX & 48,870 & \textbf{0,000} \\
			AR & 49,432 & \textbf{0,035} \\
			MA & 57,629 & \textbf{0,000} \\
			ARMA & \textbf{10,053} & \textbf{0,436} \\
			ARIMA & \textbf{10,053} & \textbf{0,436} \\
			SARIMA & \textbf{10,053} & \textbf{0,436} \\
			ARIMAX & 70,458 & \textbf{0,000} \\
			SARIMAX & \textbf{2,897} & \textbf{0,000} \\ \bottomrule
		\end{tabular}
	\end{subtable}
	
	
	\vspace{0.5cm}
	
\end{table}







\subsection{Aplica\c c\~ao}\label{subsec:estudo-reslt}


A previsão da demanda d'água é uma preocupação fundamental para muitas organizações e autoridades responsáveis pelo abastecimento d'água. Neste estudo de caso, explorou-se como a análise de séries temporais pode ser aplicada para prever a demanda d'água ao longo do tempo.


\subsubsection{Estudo de Caso 1 - An\'alise de Padr\~oes de Consumo}\label{subsubsec:quest-est}

\textbf{Questão de Pesquisa:} Existe tendência, padrão, sazonalidade para os dados destes três anos do Bairro Alto?
 
Na análise dos dados dos últimos 2 anos do Bairro Alto utilizando o método de decomposição STL mostrado na Figura \ref{fig:stl}, constatou-se que identificar tendências, sazonalidades e padrões de consumo d'água não é uma tarefa fácil. Essa dificuldade em compreender os dados torna-se um desafio na hora de planejar o abastecimento de forma eficiente.

O gráfico de barras apresentado na Figura \ref{fig:grafico-barras-demanda} mostra a demanda média das variáveis de fluxo (Vazão de Entrada -- FT01, Vazão de Gravidade -- FT02 e Vazão de Recalque -- FT03) durante o intervalo das 18h às 21h. Cada barra representa a média da demanda para cada variável em um horário específico dentro desse intervalo. A altura de cada barra indica a magnitude da demanda média para a respectiva variável. Essa visualização permite que seja identificado o horário em que as variáveis de fluxo apresentaram maior demanda, o que é útil para o planejamento e gerenciamento adequado do sistema.

\begin{figure}[!htb]
	\centering
	\caption{Demanda média das variáveis de fluxo.}
	\includegraphics[width=0.7\linewidth]{Resultados/Figuras/grafico-barras-demanda}
	
	\label{fig:grafico-barras-demanda}
	
	
\end{figure}

\subsubsection{Estudo de Caso 2 -- Estrat\'egias de Economia de Energia}

\textbf{Questão de Pesquisa:} Qual é o impacto do acionamento das bombas durante o horário de pico?

Confirma-se que a ativação das bombas de sucção durante o período de 18h às 21h resulta em um maior custo energético para a SANEPAR. Portanto, é recomendado evitar o acionamento das bombas durante esse período, utilizando estratégias de armazenamento e gerenciamento eficientes.

\textbf{Questão de Pesquisa:} Qual é o nível ideal no reservatório para evitar a ativação das bombas da SANEPAR durante o período de maior demanda, das 18h às 21h, sem comprometer o abastecimento d'água para a população?

Verifica-se que, para evitar o acionamento das bombas durante o horário de pico (18h às 21h) sem comprometer o abastecimento d'água para a população, é necessário manter o nível do reservatório acima de $4.445$ litros.


A Tabela \ref{tb:dem} apresenta os resultados para as três variáveis estudadas: vazão de entrada -- FT01, vazão de gravidade -- FT02 e vazão de recalque -- FT03. Os resultados destacam os horários específicos em que cada variável apresentou maior demanda dentro do intervalo das 18h às 21h, fornecendo importantes informações para o planejamento e gerenciamento adequado do sistema.



\begin{table}[!htb]
	\centering
	\caption{Demanda d'água.}\label{tb:dem}
	\begin{tabular}{@{}lll@{}}
		\toprule
		\textbf{Vazões}         & \textbf{Horário de Maior Demanda} & \textbf{Demanda} \\ \midrule
		Entrada -- FT01   & 2020/10/08 21:00:00               & $383,87 \ m^3/h$                   \\
		Gravidade -- FT02 & 2020/10/20 18:00:00               & $326,17 \ m^3/h$                    \\
		Recalque -- FT03  & 2020/11/26 19:00:00               & $194,35 \ m^3/h$                    \\ \bottomrule
	\end{tabular}
	
	
\end{table}

Durante as horas de pico, é necessário que o nível do reservatório esteja mantido dentro na média de $4.445$ litros para evitar o acionamento das bombas. Manter o nível do reservatório dentro dessa faixa permitirá que o sistema opere de forma eficiente, atendendo à demanda d'água sem a necessidade de acionar as bombas.

É importante destacar que a vazão de recalque exerce um impacto significativo no nível do reservatório em comparação com as outras vazões. Essa diferença se deve ao fato de que a vazão de recalque está diretamente relacionada à injeção d'água no reservatório por meio da bomba localizada próxima à sua base. Em contraste, as demais vazões possuem alguns valores ausentes, o que limita sua influência na análise geral do sistema.




%% CONCLUSÕES
\section{Conclus\~oes} \label{sec:conclusoes}

Na dissertação realizada, foi conduzido um estudo abrangente sobre a previsão da demanda de água por meio da análise de séries temporais. Através da análise exploratória dos dados e da aplicação da decomposição STL, foram identificados padrões sazonais e tendências na demanda de água.
No decorrer do trabalho, foram estabelecidas bases para responder às questões de pesquisa, com foco específico no consumo de água na cidade de Curitiba. As questões principais desta pesquisa foram abordadas com métodos específicos, divididas em dois estudos de caso. O primeiro estudo, dedicado à adequação da pressão e vazão em uma rede de distribuição de água, utilizou modelos ARIMA, DTR e XGBoost. As questões relacionadas a esse estudo foram eficientemente respondidas.

No segundo estudo de caso, que tratou do impacto do acionamento das bombas durante o horário de pico em uma rede de distribuição de água, a análise se concentrou nos horários em que as pessoas estão em casa e consomem mais água. A análise dos dados foi realizada considerando uma média de 24 horas por dia, o que revelou anomalias, especialmente no intervalo de 18h às 21h.

O objetivo geral do trabalho foi desenvolver modelos de previsão de séries temporais específicos para o abastecimento de água. Embora a literatura aborde diversos modelos de séries temporais, apenas alguns deles são aplicados ao contexto de abastecimento de água. Nesse sentido, foram comparados 19 tipos diferentes de modelos, excluindo o modelo LR devido às limitações observadas na análise de erros em vários passos futuros.

Com base nos resultados obtidos, conclui-se que a abordagem de séries temporais é uma ferramenta eficaz para prever a demanda futura de água. Os resultados também indicaram a importância de considerar as flutuações sazonais e as diferentes partes do dia ao determinar a vazão e o volume mínimo de reserva no reservatório.

Apesar dos avanços alcançados nesta pesquisa, é importante ressaltar que existem algumas limitações a serem consideradas. Primeiramente, a análise foi baseada em dados históricos de demanda de água de uma única região, limitando a generalização dos resultados para outras áreas geográficas. O estudo não levou em conta fatores externos, como mudanças climáticas ou eventos imprevistos, que podem influenciar a demanda de água.





\subsection{Propostas Futuras}

Apesar dos resultados promissores evidenciados por esta pesquisa, é essencial que se reconheçam suas limitações e que se instigue a exploração de novos horizontes em pesquisas subsequentes. Uma análise mais profunda e abrangente pode ser realizada, investigando modelos de redes neurais mais avançados. Além disso, a implementação de técnicas de otimização matemática mais refinadas, como o uso do método \textit{Covariance Matrix Adaptation Evolution Strategy} (CMAES), pode ser considerada. Seria prudente incluir cuidadosamente variáveis exógenas em todos os modelos pertinentes, como o uso de variáveis climáticas e dados de precipitação do tempo.
Implementa modelos que utilizam sistemas \textit{fuzzy} para aprimorar a previsão do tanque. Usa essa previsão juntamente com modelos existentes na literatura, como a otimização \textit{Bayesian Optimization Algorithm} (BOA), que não foi abordada neste contexto.





%-----------------------------------------------------------------
%% ELEMENTOS POS-TEXTUAIS

%% BIBLIOGRAFIA
\addcontentsline{toc}{section}{Refer\^encias} %% ADICIONA REFERÊNCIAS NO SUMÁRIO
	\bibliography{Bibliografia/Dissert2}
	

%% APÊNDICES
\appendix

%% TABELAS


%\begin{landscape}


\section{Ap\^endice - Compara\c c\~ao dos modelos de previs\~ao de series temporais m\'edia de 24h}\label{sec:comtb24}

Nas tabelas do apêndice, todos os valores são multiplicados por 100, portanto, os valores têm um erro percentual.

$(p = 7,d = 1,q = 7) (P = 2,D = 1,Q = 1)_{M = 12}$ Média 24h
	\begin{table}[H]
	\centering
	\caption{Comparação dos modelos com 1 dia de antecedência 24h \textbf{Treinamento} }\label{tb:1-24trn}
	\begin{tabular}{@{}clll@{}}
		\toprule
		\multirow{2}{*}{\textbf{Modelos}} & \multicolumn{3}{c}{\textbf{Erros}}                                                                       \\ \cmidrule(l){2-4} 
		& \multicolumn{1}{c}{\textbf{MAPE}} & \multicolumn{1}{c}{\textbf{MAE}} & \multicolumn{1}{c}{\textbf{RMSE}} \\ \hline
\textbf{AR}                       & 0,096                             & 0,306                            & 0,419                             \\
\textbf{ARX}                      & 0,118                             & 0,377                            & 0,513                             \\
\textbf{MA}                       & 0,093                             & 0,296                            & 0,403                             \\
\textbf{ARMA}                     & 0,102                             & 0,325                            & 0,435                             \\
\textbf{ARIMA}                    & 0,095                             & 0,302                            & 0,405                             \\
\textbf{SARIMA}                   & 0,105                             & 0,342                            & 0,450                             \\
\textbf{ARIMAX}                   & 0,119                             & 0,378                            & 0,511                             \\
\textbf{SARIMAX}                  & 0,118                             & 0,377                            & 0,512                             \\
\textbf{LR}                       & 0,015                             & 0,069                            & 0,077                             \\
\textbf{RFR}                      & 0,190                             & 0,624                            & 0,672                             \\
\textbf{XGBRegressor}             & 0,207                             & 0,683                            & 0,720                             \\
\textbf{LGBMRegressor}            & 0,184                             & 0,599                            & 0,655                             \\ \bottomrule
	\end{tabular}

Fonte: Elaboração própria a partir de dados da SANEPAR (2018 a 2020)
\end{table}

\begin{table}[H]
	\centering
	\caption{Comparação dos modelos com 1 dia de antecedência 24h \textbf{Validação} }\label{tb:1-24vld}
	\begin{tabular}{@{}clll@{}}
		\toprule
		\multirow{2}{*}{\textbf{Modelos}} & \multicolumn{3}{c}{\textbf{Erros}}                                                                       \\ \cmidrule(l){2-4} 
		& \multicolumn{1}{c}{\textbf{MAPE}} & \multicolumn{1}{c}{\textbf{MAE}} & \multicolumn{1}{c}{\textbf{RMSE}} \\ \hline
\textbf{AR}                       & 0,084                             & 0,285                            & 0,366                             \\
\textbf{ARX}                      & 0,103                             & 0,354                            & 0,459                             \\
\textbf{MA}                       & 0,082                             & 0,278                            & 0,361                             \\
\textbf{ARMA}                     & 0,086                             & 0,295                            & 0,372                             \\
\textbf{ARIMA}                    & 0,082                             & 0,280                            & 0,351                             \\
\textbf{SARIMA}                   & 0,097                             & 0,333                            & 0,421                             \\
\textbf{ARIMAX}                   & 0,102                             & 0,353                            & 0,458                             \\
\textbf{SARIMAX}                  & 0,104                             & 0,358                            & 0,463                             \\
\textbf{LR}                       & 0,014                             & 0,066                            & 0,073                             \\
\textbf{RFR}                      & 0,172                             & 0,587                            & 0,633                             \\
\textbf{XGBRegressor}             & 0,192                             & 0,658                            & 0,692                             \\
\textbf{LGBMRegressor}            & 0,166                             & 0,564                            & 0,616                             \\ \bottomrule
	\end{tabular}

Fonte: Elaboração própria a partir de dados da SANEPAR (2018 a 2020)
\end{table}

\begin{table}[H]
	\centering
	\caption{Comparação dos modelos com 1 dia de antecedência 24h \textbf{Teste} }\label{tb:1-24tst}
	\begin{tabular}{@{}clll@{}}
		\toprule
		\multirow{2}{*}{\textbf{Modelos}} & \multicolumn{3}{c}{\textbf{Erros}}                                                                       \\ \cmidrule(l){2-4} 
		& \multicolumn{1}{c}{\textbf{MAPE}} & \multicolumn{1}{c}{\textbf{MAE}} & \multicolumn{1}{c}{\textbf{RMSE}} \\ \hline
\textbf{AR}                       & 0,100                             & 0,329                            & 0,424                             \\
\textbf{ARX}                      & 0,137                             & 0,462                            & 0,586                             \\
\textbf{MA}                       & 0,102                             & 0,336                            & 0,431                             \\
\textbf{ARMA}                     & 0,102                             & 0,340                            & 0,433                             \\
\textbf{ARIMA}                    & 0,103                             & 0,346                            & 0,440                             \\
\textbf{SARIMA}                   & 0,118                             & 0,398                            & 0,501                             \\
\textbf{ARIMAX}                   & 0,137                             & 0,461                            & 0,587                             \\
\textbf{SARIMAX}                  & 0,138                             & 0,464                            & 0,590                             \\
\textbf{LR}                       & 0,018                             & 0,087                            & 0,098                             \\
\textbf{RFR}                      & 0,153                             & 0,494                            & 0,587                             \\
\textbf{XGBRegressor}             & 0,170                             & 0,560                            & 0,643                             \\
\textbf{LGBMRegressor}            & 0,145                             & 0,465                            & 0,568                             \\ \bottomrule
	\end{tabular}

Fonte: Elaboração própria a partir de dados da SANEPAR (2018 a 2020)
\end{table}

\begin{table}[H]
	\centering
	\caption{Comparação dos modelos com 1 dia de antecedência 24h \textbf{Completo} }\label{tb:1-24cm}
	\begin{tabular}{@{}clll@{}}
		\toprule
		\multirow{2}{*}{\textbf{Modelos}} & \multicolumn{3}{c}{\textbf{Erros}}                                                                       \\ \cmidrule(l){2-4} 
		& \multicolumn{1}{c}{\textbf{MAPE}} & \multicolumn{1}{c}{\textbf{MAE}} & \multicolumn{1}{c}{\textbf{RMSE}} \\ \hline
\textbf{AR}                       & 0,093                             & 0,302                            & 0,165                             \\
\textbf{ARX}                      & 0,123                             & 0,402                            & 0,283                             \\
\textbf{MA}                       & 0,107                             & 0,344                            & 0,460                             \\
\textbf{ARMA}                     & 0,097                             & 0,316                            & 0,424                             \\
\textbf{ARIMA}                    & 0,094                             & 0,303                            & 0,406                             \\
\textbf{SARIMA}                   & 0,106                             & 0,350                            & 0,448                             \\
\textbf{ARIMAX}                   & 0,120                             & 0,394                            & 0,521                             \\
\textbf{SARIMAX}                  & 0,122                             & 0,401                            & 0,530                             \\
\textbf{LR}                       & 0,016                             & 0,074                            & 0,084                             \\
\textbf{RFR}                      & 0,176                             & 0,579                            & 0,642                             \\
\textbf{XGBRegressor}             & 0,194                             & 0,643                            & 0,694                             \\
\textbf{LGBMRegressor}            & 0,170                             & 0,554                            & 0,624                             \\ \bottomrule
	\end{tabular}

Fonte: Elaboração própria a partir de dados da SANEPAR (2018 a 2020)
\end{table}


\begin{table}[H]
	\centering
	\caption{Comparação dos modelos com 7 dias de antecedência 24h \textbf{Treinamento} }\label{tb:10-24trn}
	\begin{tabular}{@{}clll@{}}
		\toprule
		\multirow{2}{*}{\textbf{Modelos}} & \multicolumn{3}{c}{\textbf{Erros}}                                                                       \\ \cmidrule(l){2-4} 
		& \multicolumn{1}{c}{\textbf{MAPE}} & \multicolumn{1}{c}{\textbf{MAE}} & \multicolumn{1}{c}{\textbf{RMSE}} \\ \hline
\textbf{AR}                       & 13,103                            & 40,563                           & 52,800                            \\
\textbf{ARX}                      & 15,660                            & 48,295                           & 63,852                            \\
\textbf{MA}                       & 13,485                            & 41,753                           & 54,611                            \\
\textbf{ARMA}                     & 13,198                            & 40,905                           & 52,954                            \\
\textbf{ARIMA}                    & 13,476                            & 42,156                           & 53,959                            \\
\textbf{SARIMA}                   & 14,057                            & 44,732                           & 57,526                            \\
\textbf{ARIMAX}                   & 15,658                            & 48,252                           & 63,875                            \\
\textbf{SARIMAX}                  & 15,566                            & 48,111                           & 63,736                            \\
\textbf{LR}        & 116,813                           & 502,867                          & 502,939                           \\
\textbf{RFR}  & 24,430                            & 73,299                           & 88,773                            \\
\textbf{XGBRegressor}             & 21,727                            & 64,472                           & 80,657                            \\
\textbf{LGBMRegressor}            & 27,197                            & 82,446                           & 97,262                            \\ \bottomrule
	\end{tabular}

Fonte: Elaboração própria a partir de dados da SANEPAR (2018 a 2020)
\end{table}

\begin{table}[H]
	\centering
	\caption{Comparação dos modelos com 7 dias de antecedência 24h \textbf{Validação} }\label{tb:10-24vld}
	\begin{tabular}{@{}clll@{}}
		\toprule
		\multirow{2}{*}{\textbf{Modelos}} & \multicolumn{3}{c}{\textbf{Erros}}                                                                       \\ \cmidrule(l){2-4} 
		& \multicolumn{1}{c}{\textbf{MAPE}} & \multicolumn{1}{c}{\textbf{MAE}} & \multicolumn{1}{c}{\textbf{RMSE}} \\ \hline
\textbf{AR}                       & 8,478                             & 30,338                           & 36,972                            \\
\textbf{ARX}                      & 10,964                            & 38,216                           & 50,968                            \\
\textbf{MA}                       & 9,847                             & 35,076                           & 43,061                            \\
\textbf{ARMA}                     & 9,560                             & 34,375                           & 42,890                            \\
\textbf{ARIMA}                    & 10,061                            & 36,208                           & 43,299                            \\
\textbf{SARIMA}                   & 11,950                            & 43,141                           & 64,342                            \\
\textbf{ARIMAX}                   & 10,994                            & 38,333                           & 51,037                            \\
\textbf{SARIMAX}                  & 11,340                            & 39,533                           & 52,336                            \\
\textbf{LR}        & 106,577                           & 497,911                          & 497,962                           \\
\textbf{RFR}  & 13,705                            & 46,546                           & 59,266                            \\
\textbf{XGBRegressor}             & 11,925                            & 40,432                           & 52,141                            \\
\textbf{LGBMRegressor}            & 15,301                            & 52,126                           & 65,675                            \\ \bottomrule
	\end{tabular}

Fonte: Elaboração própria a partir de dados da SANEPAR (2018 a 2020)
\end{table}

\begin{table}[H]
	\centering
	\caption{Comparação dos modelos com 7 dias de antecedência 24h \textbf{Teste} }\label{tb:10-24tst}
	\begin{tabular}{@{}clll@{}}
		\toprule
		\multirow{2}{*}{\textbf{Modelos}} & \multicolumn{3}{c}{\textbf{Erros}}                                                                       \\ \cmidrule(l){2-4} 
		& \multicolumn{1}{c}{\textbf{MAPE}} & \multicolumn{1}{c}{\textbf{MAE}} & \multicolumn{1}{c}{\textbf{RMSE}} \\ \hline
\textbf{AR}                       & 11,205                            & 39,296                           & 50,127                            \\
\textbf{ARX}                      & 11,488                            & 41,102                           & 56,274                            \\
\textbf{MA}                       & 11,369                            & 39,945                           & 49,176                            \\
\textbf{ARMA}                     & 11,813                            & 41,897                           & 52,989                            \\
\textbf{ARIMA}                    & 11,807                            & 41,990                           & 51,958                            \\
\textbf{SARIMA}                   & 12,383                            & 44,613                           & 54,671                            \\
\textbf{ARIMAX}                   & 11,514                            & 41,166                           & 56,360                            \\
\textbf{SARIMAX}                  & 11,651                            & 41,735                           & 57,201                            \\
\textbf{LR}        & 105,378                           & 497,224                          & 497,271                           \\
\textbf{RFR}  & 12,071                            & 40,281                           & 55,844                            \\
\textbf{XGBRegressor}             & 10,754                            & 35,751                           & 50,403                            \\
\textbf{LGBMRegressor}            & 13,609                            & 45,772                           & 61,690                            \\ \bottomrule
	\end{tabular}

Fonte: Elaboração própria a partir de dados da SANEPAR (2018 a 2020)
\end{table}

\begin{table}[H]
	\centering
	\caption{Comparação dos modelos com 7 dias de antecedência 24h \textbf{Completo} }\label{tb:10-24cm}
	\begin{tabular}{@{}clll@{}}
		\toprule
		\multirow{2}{*}{\textbf{Modelos}} & \multicolumn{3}{c}{\textbf{Erros}}                                                                       \\ \cmidrule(l){2-4} 
		& \multicolumn{1}{c}{\textbf{MAPE}} & \multicolumn{1}{c}{\textbf{MAE}} & \multicolumn{1}{c}{\textbf{RMSE}} \\ \hline
\textbf{AR}                       & 11,543                            & 37,520                           & 23,623                            \\
\textbf{ARX}                      & 14,443                            & 47,192                           & 38,330                            \\
\textbf{MA}                       & 12,296                            & 40,064                           & 51,184                            \\
\textbf{ARMA}                     & 11,541                            & 37,812                           & 48,308                            \\
\textbf{ARIMA}                    & 11,441                            & 37,524                           & 47,972                            \\
\textbf{SARIMA}                   & 12,983                            & 43,670                           & 54,496                            \\
\textbf{ARIMAX}                   & 14,462                            & 47,257                           & 61,956                            \\
\textbf{SARIMAX}                  & 14,433                            & 47,261                           & 61,891                            \\
\textbf{LR}        & 111,919                           & 500,465                          & 500,534                           \\
\textbf{RFR}  & 19,187                            & 59,557                           & 76,339                            \\
\textbf{XGBRegressor}             & 17,033                            & 52,412                           & 69,104                            \\
\textbf{LGBMRegressor}            & 21,418                            & 67,099                           & 83,833                            \\ \bottomrule
	\end{tabular}

Fonte: Elaboração própria a partir de dados da SANEPAR (2018 a 2020)
\end{table}


\begin{table}[H]
	\centering
	\caption{Comparação dos modelos com 14 dias de antecedência 24h \textbf{Treinamento} }\label{tb:30-24trn}
	\begin{tabular}{@{}clll@{}}
		\toprule
		\multirow{2}{*}{\textbf{Modelos}} & \multicolumn{3}{c}{\textbf{Erros}}                                                                       \\ \cmidrule(l){2-4} 
		& \multicolumn{1}{c}{\textbf{MAPE}} & \multicolumn{1}{c}{\textbf{MAE}} & \multicolumn{1}{c}{\textbf{RMSE}} \\ \hline
\textbf{AR}                       & 13,434                            & 41,516                           & 55,038                            \\
\textbf{ARX}                      & 14,776                            & 46,180                           & 64,258                            \\
\textbf{MA}                       & 13,618                            & 42,216                           & 55,568                            \\
\textbf{ARMA}                     & 12,813                            & 39,654                           & 52,850                            \\
\textbf{ARIMA}                    & 13,179                            & 41,239                           & 54,480                            \\
\textbf{SARIMA}                   & 13,831                            & 43,379                           & 56,533                            \\
\textbf{ARIMAX}                   & 14,759                            & 46,106                           & 64,262                            \\
\textbf{SARIMAX}                  & 14,689                            & 45,931                           & 63,708                            \\
\textbf{LR}        & 254,843                           & 1098,535                         & 1098,567                          \\
\textbf{RFR}  & 24,628                            & 73,935                           & 89,385                            \\
\textbf{XGBRegressor}             & 21,473                            & 63,647                           & 79,914                            \\
\textbf{LGBMRegressor}            & 27,577                            & 83,811                           & 98,249                            \\ \bottomrule
	\end{tabular}

Fonte: Elaboração própria a partir de dados da SANEPAR (2018 a 2020)
\end{table}

\begin{table}[H]
	\centering
	\caption{Comparação dos modelos com 14 dias de antecedência 24h \textbf{Validação} }\label{tb:30-24vld}
	\begin{tabular}{@{}clll@{}}
		\toprule
		\multirow{2}{*}{\textbf{Modelos}} & \multicolumn{3}{c}{\textbf{Erros}}                                                                       \\ \cmidrule(l){2-4} 
		& \multicolumn{1}{c}{\textbf{MAPE}} & \multicolumn{1}{c}{\textbf{MAE}} & \multicolumn{1}{c}{\textbf{RMSE}} \\ \hline
\textbf{AR}                       & 13,434                            & 41,516                           & 55,038                            \\
\textbf{ARX}                      & 14,776                            & 46,180                           & 64,258                            \\
\textbf{MA}                       & 13,618                            & 42,216                           & 55,568                            \\
\textbf{ARMA}                     & 12,813                            & 39,654                           & 52,850                            \\
\textbf{ARIMA}                    & 13,179                            & 41,239                           & 54,480                            \\
\textbf{SARIMA}                   & 13,831                            & 43,379                           & 56,533                            \\
\textbf{ARIMAX}                   & 14,759                            & 46,106                           & 64,262                            \\
\textbf{SARIMAX}                  & 14,689                            & 45,931                           & 63,708                            \\
\textbf{LR}        & 254,843                           & 1098,535                         & 1098,567                          \\
\textbf{RFR}  & 24,628                            & 73,935                           & 89,385                            \\
\textbf{XGBRegressor}             & 21,473                            & 63,647                           & 79,914                            \\
\textbf{LGBMRegressor}            & 27,577                            & 83,811                           & 98,249                            \\ \bottomrule
	\end{tabular}

Fonte: Elaboração própria a partir de dados da SANEPAR (2018 a 2020)
\end{table}

\begin{table}[H]
	\centering
	\caption{Comparação dos modelos com 14 dias de antecedência 24h \textbf{Teste} }\label{tb:30-24tst}
	\begin{tabular}{@{}clll@{}}
		\toprule
		\multirow{2}{*}{\textbf{Modelos}} & \multicolumn{3}{c}{\textbf{Erros}}                                                                       \\ \cmidrule(l){2-4} 
		& \multicolumn{1}{c}{\textbf{MAPE}} & \multicolumn{1}{c}{\textbf{MAE}} & \multicolumn{1}{c}{\textbf{RMSE}} \\ \hline
\textbf{AR}                       & 7,205                             & 24,492                           & 36,504                            \\
\textbf{ARX}                      & 7,096                             & 24,283                           & 40,578                            \\
\textbf{MA}                       & 7,918                             & 27,309                           & 39,147                            \\
\textbf{ARMA}                     & 6,790                             & 23,355                           & 33,748                            \\
\textbf{ARIMA}                    & 6,699                             & 23,057                           & 33,202                            \\
\textbf{SARIMA}                   & 8,199                             & 28,612                           & 39,813                            \\
\textbf{ARIMAX}                   & 7,100                             & 24,282                           & 40,535                            \\
\textbf{SARIMAX}                  & 7,079                             & 24,191                           & 40,501                            \\
\textbf{LR}        & 231,425                           & 1092,891                         & 1092,913                          \\
\textbf{RFR}  & 12,243                            & 40,893                           & 56,474                            \\
\textbf{XGBRegressor}             & 10,663                            & 35,417                           & 50,227                            \\
\textbf{LGBMRegressor}            & 13,875                            & 46,821                           & 62,137                            \\ \bottomrule
	\end{tabular}

Fonte: Elaboração própria a partir de dados da SANEPAR (2018 a 2020)
\end{table}

\begin{table}[H]
	\centering
	\caption{Comparação dos modelos com 14 dias de antecedência 24h \textbf{Completo} }\label{tb:30-24cm}
	\begin{tabular}{@{}clll@{}}
		\toprule
		\multirow{2}{*}{\textbf{Modelos}} & \multicolumn{3}{c}{\textbf{Erros}}                                                                       \\ \cmidrule(l){2-4} 
		& \multicolumn{1}{c}{\textbf{MAPE}} & \multicolumn{1}{c}{\textbf{MAE}} & \multicolumn{1}{c}{\textbf{RMSE}} \\ \hline
\textbf{AR}                       & 11,633                            & 37,422                           & 24,692                            \\
\textbf{ARX}                      & 11,953                            & 38,601                           & 31,117                            \\
\textbf{MA}                       & 11,877                            & 38,382                           & 50,293                            \\
\textbf{ARMA}                     & 10,399                            & 33,438                           & 45,908                            \\
\textbf{ARIMA}                    & 10,473                            & 33,770                           & 45,742                            \\
\textbf{SARIMA}                   & 11,301                            & 37,080                           & 49,417                            \\
\textbf{ARIMAX}                   & 11,937                            & 38,537                           & 55,792                            \\
\textbf{SARIMAX}                  & 11,882                            & 38,374                           & 55,405                            \\
\textbf{LR}        & 244,820                           & 1096,132                         & 1096,164                          \\
\textbf{RFR}  & 19,376                            & 60,188                           & 76,964                            \\
\textbf{XGBRegressor}             & 16,844                            & 51,783                           & 68,532                            \\
\textbf{LGBMRegressor}            & 21,758                            & 68,359                           & 84,646                            \\ \bottomrule
	\end{tabular}

Fonte: Elaboração própria a partir de dados da SANEPAR (2018 a 2020)
\end{table}


\begin{table}[H]
	\centering
	\caption{Comparação dos modelos com 30 dias de antecedência 24h \textbf{Treinamento} }\label{tb:60-24trn}
	\begin{tabular}{@{}clll@{}}
		\toprule
		\multirow{2}{*}{\textbf{Modelos}} & \multicolumn{3}{c}{\textbf{Erros}}                                                                       \\ \cmidrule(l){2-4} 
		& \multicolumn{1}{c}{\textbf{MAPE}} & \multicolumn{1}{c}{\textbf{MAE}} & \multicolumn{1}{c}{\textbf{RMSE}} \\ \hline
\textbf{AR}                       & 14,313                            & 44,675                           & 55,915                            \\
\textbf{ARX}                      & 15,119                            & 47,527                           & 65,408                            \\
\textbf{MA}                       & 14,075                            & 43,874                           & 55,592                            \\
\textbf{ARMA}                     & 14,710                            & 45,993                           & 58,021                            \\
\textbf{ARIMA}                    & 14,963                            & 47,121                           & 59,252                            \\
\textbf{SARIMA}                   & 14,851                            & 47,013                           & 62,084                            \\
\textbf{ARIMAX}                   & 15,110                            & 47,450                           & 65,409                            \\
\textbf{SARIMAX}                  & 15,048                            & 47,272                           & 64,923                            \\
\textbf{LR}        & 570,339                           & 2460,060                         & 2460,074                          \\
\textbf{RFR}  & 23,900                            & 71,535                           & 87,190                            \\
\textbf{XGBRegressor}             & 20,622                            & 60,945                           & 77,146                            \\
\textbf{LGBMRegressor}            & 27,382                            & 83,382                           & 96,982                            \\ \bottomrule
	\end{tabular}

Fonte: Elaboração própria a partir de dados da SANEPAR (2018 a 2020)
\end{table}

\begin{table}[H]
	\centering
	\caption{Comparação dos modelos com 30 dias de antecedência 24h \textbf{Validação} }\label{tb:60-24vld}
	\begin{tabular}{@{}clll@{}}
		\toprule
		\multirow{2}{*}{\textbf{Modelos}} & \multicolumn{3}{c}{\textbf{Erros}}                                                                       \\ \cmidrule(l){2-4} 
		& \multicolumn{1}{c}{\textbf{MAPE}} & \multicolumn{1}{c}{\textbf{MAE}} & \multicolumn{1}{c}{\textbf{RMSE}} \\ \hline
\textbf{AR}                       & 8,947                             & 31,696                           & 39,134                            \\
\textbf{ARX}                      & 5,826                             & 19,805                           & 34,048                            \\
\textbf{MA}                       & 9,221                             & 32,614                           & 40,521                            \\
\textbf{ARMA}                     & 9,335                             & 33,060                           & 41,158                            \\
\textbf{ARIMA}                    & 9,476                             & 32,411                           & 43,537                            \\
\textbf{SARIMA}                   & 11,150                            & 39,153                           & 48,524                            \\
\textbf{ARIMAX}                   & 6,080                             & 20,706                           & 34,072                            \\
\textbf{SARIMAX}                  & 6,509                             & 22,314                           & 34,614                            \\
\textbf{LR}        & 524,857                           & 2455,103                         & 2455,113                          \\
\textbf{RFR}  & 13,467                            & 45,678                           & 58,497                            \\
\textbf{XGBRegressor}             & 11,831                            & 40,234                           & 51,521                            \\
\textbf{LGBMRegressor}            & 16,358                            & 56,280                           & 67,611                            \\ \bottomrule
	\end{tabular}

Fonte: Elaboração própria a partir de dados da SANEPAR (2018 a 2020)
\end{table}

\begin{table}[H]
	\centering
	\caption{Comparação dos modelos com 30 dias de antecedência 24h \textbf{Teste} }\label{tb:60-24tst}
	\begin{tabular}{@{}clll@{}}
		\toprule
		\multirow{2}{*}{\textbf{Modelos}} & \multicolumn{3}{c}{\textbf{Erros}}                                                                       \\ \cmidrule(l){2-4} 
		& \multicolumn{1}{c}{\textbf{MAPE}} & \multicolumn{1}{c}{\textbf{MAE}} & \multicolumn{1}{c}{\textbf{RMSE}} \\ \hline
\textbf{AR}                       & 10,061                            & 34,647                           & 46,429                            \\
\textbf{ARX}                      & 9,470                             & 32,927                           & 52,354                            \\
\textbf{MA}                       & 9,112                             & 31,105                           & 42,858                            \\
\textbf{ARMA}                     & 11,540                            & 40,297                           & 51,763                            \\
\textbf{ARIMA}                    & 11,556                            & 40,173                           & 52,104                            \\
\textbf{SARIMA}                   & 13,024                            & 45,855                           & 59,020                            \\
\textbf{ARIMAX}                   & 9,540                             & 33,161                           & 52,488                            \\
\textbf{SARIMAX}                  & 9,604                             & 33,394                           & 52,420                            \\
\textbf{LR}        & 519,532                           & 2454,417                         & 2454,426                          \\
\textbf{RFR}  & 11,819                            & 39,362                           & 55,016                            \\
\textbf{XGBRegressor}             & 10,589                            & 35,311                           & 49,626                            \\
\textbf{LGBMRegressor}            & 14,593                            & 49,692                           & 63,239                            \\ \bottomrule
	\end{tabular}

Fonte: Elaboração própria a partir de dados da SANEPAR (2018 a 2020)
\end{table}

\begin{table}[H]
	\centering
	\caption{Comparação dos modelos com 30 dias de antecedência 24h \textbf{Completo} }\label{tb:60-24cm}
	\begin{tabular}{@{}clll@{}}
		\toprule
		\multirow{2}{*}{\textbf{Modelos}} & \multicolumn{3}{c}{\textbf{Erros}}                                                                       \\ \cmidrule(l){2-4} 
		& \multicolumn{1}{c}{\textbf{MAPE}} & \multicolumn{1}{c}{\textbf{MAE}} & \multicolumn{1}{c}{\textbf{RMSE}} \\ \hline
\textbf{AR}                       & 11,797                            & 37,971                           & 24,622                            \\
\textbf{ARX}                      & 14,245                            & 46,963                           & 40,100                            \\
\textbf{MA}                       & 11,569                            & 37,184                           & 48,853                            \\
\textbf{ARMA}                     & 11,876                            & 38,578                           & 50,109                            \\
\textbf{ARIMA}                    & 11,893                            & 38,619                           & 50,375                            \\
\textbf{SARIMA}                   & 13,373                            & 44,344                           & 57,734                            \\
\textbf{ARIMAX}                   & 14,165                            & 46,667                           & 63,257                            \\
\textbf{SARIMAX}                  & 14,216                            & 46,857                           & 63,063                            \\
\textbf{LR}        & 548,593                           & 2457,658                         & 2457,672                          \\
\textbf{RFR}  & 18,782                            & 58,175                           & 75,047                            \\
\textbf{XGBRegressor}             & 16,354                            & 50,285                           & 66,588                            \\
\textbf{LGBMRegressor}            & 21,967                            & 69,388                           & 84,215                            \\ \bottomrule
	\end{tabular}

Fonte: Elaboração própria a partir de dados da SANEPAR (2018 a 2020)
\end{table}

%
\section{Ap\^endice - Compara\c c\~ao dos modelos de previs\~ao com o m\'eto do Ljung Box}\label{sec:comtb18}



Modelos ARIMAS para previsão de longo tempo 

	\begin{table}[H]
		\centering
		\caption{Comparação dos modelos Ljung Box \textbf{Treino} }\label{tb:lbtrn}
	\begin{tabular}{@{}ccc@{}}
		\toprule
		\textbf{Ljung Box} & \textbf{Estatística de Teste} & \textbf{Valor De p} \\ \midrule
		\textbf{ARX}       & 0,328                         & 1                   \\
		\textbf{AR}        & 3,419                         & 0,97                \\
		\textbf{MA}        & 8,706                         & 0,56                \\
		\textbf{ARMA}      & 1,463                         & 0,999               \\
		\textbf{ARIMA}     & 0,681                         & 1                   \\
		\textbf{SARIMA}    & 2,019                         & 0,996               \\
		\textbf{ARIMAX}    & 0,254                         & 1                   \\
		\textbf{SARIMAX}   & 1,169                         & 1                   \\ \bottomrule
	\end{tabular}

Fonte: Elaboração própria a partir de dados da SANEPAR (2018 a 2020)
	\end{table}

\begin{table}[H]
	\centering
	\caption{Comparação dos modelos Ljung Box \textbf{Validação} }\label{tb:lbvld}
	\begin{tabular}{@{}ccc@{}}
		\toprule
		\textbf{Ljung Box} & \textbf{Estatística de Teste} & \textbf{Valor De p} \\ \midrule
		\textbf{ARX}       & 3,205                         & 0,976               \\
		\textbf{AR}        & 4,58                          & 0,917               \\
		\textbf{MA}        & 2,658                         & 0,988               \\
		\textbf{ARMA}      & 6,237                         & 0,795               \\
		\textbf{ARIMA}     & 5,543                         & 0,852               \\
		\textbf{SARIMA}    & 14,345                        & 0,158               \\
		\textbf{ARIMAX}    & 0,194                         & 1                   \\
		\textbf{SARIMAX}   & 0,465                         & 1                   \\ \bottomrule
	\end{tabular}

Fonte: Elaboração própria a partir de dados da SANEPAR (2018 a 2020)
\end{table}

\begin{table}[H]
	\centering
	\caption{Comparação dos modelos Ljung Box \textbf{Teste} }\label{tb:lbtst}
	\begin{tabular}{@{}ccc@{}}
		\toprule
		\textbf{Ljung Box} & \textbf{Estatística de Teste} & \textbf{Valor De p} \\ \midrule
		\textbf{ARX}       & 1,826                         & 0,998               \\
		\textbf{AR}        & 0,075                         & 0,434               \\
		\textbf{MA}        & 22,404                        & 0,013               \\
		\textbf{ARMA}      & 2,089                         & 0,996               \\
		\textbf{ARIMA}     & 7,647                         & 0,663               \\
		\textbf{SARIMA}    & 6,531                         & 0,769               \\
		\textbf{ARIMAX}    & 0,568                         & 1                   \\
		\textbf{SARIMAX}   & 0,8                           & 1                   \\ \bottomrule
	\end{tabular}
	
	Fonte: Elaboração própria a partir de dados da SANEPAR (2018 a 2020)
\end{table}

\begin{table}[H]
	\centering
	\caption{Comparação dos modelos Ljung Box \textbf{Completo} }\label{tb:lbcm}
	\begin{tabular}{@{}ccc@{}}
		\toprule
		\textbf{Ljung Box} & \textbf{Estatística de Teste} & \textbf{Valor De p} \\ \midrule
		\textbf{ARX}       & 0,744                         & 1                   \\
		\textbf{AR}        & 6,607                         & 0,762               \\
		\textbf{MA}        & 71,338                        & 0                   \\
		\textbf{ARMA}      & 24,353                        & 0,007               \\
		\textbf{ARIMA}     & 24,353                        & 0,007               \\
		\textbf{SARIMA}    & 24,353                        & 0,007               \\
		\textbf{ARIMAX}    & 0,253                         & 1                   \\
		\textbf{SARIMAX}   & 1,595                         & 0,999               \\ \bottomrule
	\end{tabular}

	
	Fonte: Elaboração própria a partir de dados da SANEPAR (2018 a 2020)
\end{table}



%\end{landscape}
%% FIGURAS
%
\section{Apêndice - Modelos AR(6), ARX (6) e MA (6) 18h a 21h}\label{sec:ararxma18}

\begin{figure}[H]
	\centering
	\caption{Comparação dos modelos AR, ARX e MA, 1 dia a frente }
	\label{fig:1-AR-ARX-MA}
	\includegraphics[width=1\linewidth]{Apendices/Figuras/modelagem-18-a-21h/1-AR-ARX-MA}
	
	Fonte: Autoria própria.
\end{figure}

\begin{figure}[H]
	\centering
	\caption{Comparação dos modelos AR, ARX e MA, 10 dias a frente }
	\label{fig:10-AR-ARX-MA}
	\includegraphics[width=1\linewidth]{Apendices/Figuras/modelagem-18-a-21h/10-AR-ARX-MA}
	
	Fonte: Autoria própria.
\end{figure}


\begin{figure}[H]
	\centering
	\caption{Comparação dos modelos AR, ARX e MA, 30 dias a frente }
	\label{fig:30-AR-ARX-MA}
	\includegraphics[width=1\linewidth]{Apendices/Figuras/modelagem-18-a-21h/30-AR-ARX-MA}
	
	Fonte: Autoria própria.
\end{figure}

\begin{figure}[H]
	\centering
	\caption{Comparação dos modelos AR, ARX e MA, 60 dias a frente }
	\label{fig:60-AR-ARX-MA}
	\includegraphics[width=1\linewidth]{Apendices/Figuras/modelagem-18-a-21h/60-AR-ARX-MA}
	
	Fonte: Autoria própria.
\end{figure}




\section{Ap\^endice - Modelos ARMA(6,6) e ARIMA (6,1,6) 18h a 21h}\label{sec:armarima18}

\begin{figure}[H]
	\centering
	\caption{Comparação dos modelos ARMA e ARIMA, 1 dia a frente }
	\label{fig:1-ARMA-ARIMA}
	\includegraphics[width=1\linewidth]{Apendices/Figuras/modelagem-18-a-21h/1-ARMA-ARIMA}
	
	Fonte: Autoria própria.
\end{figure}

\begin{figure}[H]
	\centering
	\caption{Comparação dos modelos ARMA e ARIMA, 10 dias a frente }
	\label{fig:10-ARMA-ARIMA}
	\includegraphics[width=1\linewidth]{Apendices/Figuras/modelagem-18-a-21h/10-ARMA-ARIMA}
	
	Fonte: Autoria própria.
\end{figure}


\begin{figure}[H]
	\centering
	\caption{Comparação dos modelos ARMA e ARIMA, 30 dias a frente }
	\label{fig:30-ARMA-ARIMA}
	\includegraphics[width=1\linewidth]{Apendices/Figuras/modelagem-18-a-21h/30-ARMA-ARIMA}
	
	Fonte: Autoria própria.
\end{figure}

\begin{figure}[H]
	\centering
	\caption{Comparação dos modelos ARMA e ARIMA, 60 dias a frente }
	\label{fig:60-ARMA-ARIMA}
	\includegraphics[width=1\linewidth]{Apendices/Figuras/modelagem-18-a-21h/60-ARMA-ARIMA}
	
	Fonte: Autoria própria.
\end{figure}


\section{Apêndice - Modelos ARIMAX (6,1,6), SARIMA (6,1,6) (2,1,1,7) e SARIMAX (6,1,6) (2,1,1,7) 18h a 21h}\label{sec:arimaxsarimasarimax18}

\begin{figure}[H]
	\centering
	\caption{Comparação dos modelos ARIMAX, SARIMA e SARIMAX, 1 dia a frente }
	\label{fig:1-ARIMAX-SARIMA-SARIMAX}
	\includegraphics[width=1\linewidth]{Apendices/Figuras/modelagem-18-a-21h/1-ARIMAX-SARIMA-SARIMAX}
	
	Fonte: Autoria própria.
\end{figure}

\begin{figure}[H]
	\centering
	\caption{Comparação dos modelos ARIMAX, SARIMA e SARIMAX, 10 dias a frente }
	\label{fig:10-ARIMAX-SARIMA-SARIMAX}
	\includegraphics[width=1\linewidth]{Apendices/Figuras/modelagem-18-a-21h/10-ARIMAX-SARIMA-SARIMAX}
	
	Fonte: Autoria própria.
\end{figure}


\begin{figure}[H]
	\centering
	\caption{Comparação dos modelos ARIMAX, SARIMA e SARIMAX, 30 dias a frente }
	\label{fig:30-ARIMAX-SARIMA-SARIMAX}
	\includegraphics[width=1\linewidth]{Apendices/Figuras/modelagem-18-a-21h/30-ARIMAX-SARIMA-SARIMAX}
	
	Fonte: Autoria própria.
\end{figure}

\begin{figure}[H]
	\centering
	\caption{Comparação dos modelos ARIMAX, SARIMA e SARIMAX, 60 dias a frente }
	\label{fig:60-ARIMAX-SARIMA-SARIMAX}
	\includegraphics[width=1\linewidth]{Apendices/Figuras/modelagem-18-a-21h/60-ARIMAX-SARIMA-SARIMAX}
	
	Fonte: Autoria própria.
\end{figure}


\section{Apêndice - Modelos Regressão linear, XGB Regressão, Ligth GBM Regressão e Regressão de Floresta Aleatória 18h a 21h}\label{sec:lrxgblgbmrf18}

\begin{figure}[H]
	\centering
	\caption{Comparação dos modelos Regressão linear, XGB Regressão, Ligth GBM Regressão e Regressão de Floresta Aleatória, 1 dia a frente }
	\label{fig:1-LR-XGB-LGBM-RF}
	\includegraphics[width=1\linewidth]{Apendices/Figuras/modelagem-18-a-21h/1-LR-XGB-LGBM-RF}
	
	Fonte: Autoria própria.
\end{figure}

\begin{figure}[H]
	\centering
	\caption{Comparação dos modelos Regressão linear, XGB Regressão, Ligth GBM Regressão e Regressão de Floresta Aleatória, 10 dias a frente }
	\label{fig:10-LR-XGB-LGBM-RF}
	\includegraphics[width=1\linewidth]{Apendices/Figuras/modelagem-18-a-21h/10-LR-XGB-LGBM-RF}
	
	Fonte: Autoria própria.
\end{figure}


\begin{figure}[H]
	\centering
	\caption{Comparação dos modelos Regressão linear, XGB Regressão, Ligth GBM Regressão e Regressão de Floresta Aleatória, 30 dias a frente }
	\label{fig:30-LR-XGB-LGBM-RF}
	\includegraphics[width=1\linewidth]{Apendices/Figuras/modelagem-18-a-21h/30-LR-XGB-LGBM-RF}
	
	Fonte: Autoria própria.
\end{figure}

\begin{figure}[H]
	\centering
	\caption{Comparação dos modelos Regressão linear, XGB Regressão, Ligth GBM Regressão e Regressão de Floresta Aleatória, 60 dias a frente }
	\label{fig:60-LR-XGB-LGBM-RF}
	\includegraphics[width=1\linewidth]{Apendices/Figuras/modelagem-18-a-21h/60-LR-XGB-LGBM-RF}
	
	Fonte: Autoria própria.
\end{figure}



\section{Ap\^endice - Modelos AR(7), ARX (7) e MA (7) 24h }\label{sec:ararxma24}

\begin{figure}[H]
	\centering
	\caption{Comparação dos modelos AR, ARX e MA, 1 dia a frente }
	\label{fig:1-AR-ARX-MA24}
	\includegraphics[width=1\linewidth]{Apendices/Figuras/modelagem-24h/1-AR-ARX-MA}
	
Fonte: Elaboração própria a partir de dados da SANEPAR (2018 a 2020)
\end{figure}

\begin{figure}[H]
	\centering
	\caption{Comparação dos modelos AR, ARX e MA, 7 dias a frente }
	\label{fig:10-AR-ARX-MA24}
	\includegraphics[width=1\linewidth]{Apendices/Figuras/modelagem-24h/7-AR-ARX-MA}
	
Fonte: Elaboração própria a partir de dados da SANEPAR (2018 a 2020)
\end{figure}


\begin{figure}[H]
	\centering
	\caption{Comparação dos modelos AR, ARX e MA, 14 dias a frente }
	\label{fig:30-AR-ARX-MA24}
	\includegraphics[width=1\linewidth]{Apendices/Figuras/modelagem-24h/14-AR-ARX-MA}
	
Fonte: Elaboração própria a partir de dados da SANEPAR (2018 a 2020)
\end{figure}

\begin{figure}[H]
	\centering
	\caption{Comparação dos modelos AR, ARX e MA, 30 dias a frente }
	\label{fig:60-AR-ARX-MA24}
	\includegraphics[width=1\linewidth]{Apendices/Figuras/modelagem-24h/30-AR-ARX-MA}
	
Fonte: Elaboração própria a partir de dados da SANEPAR (2018 a 2020)
\end{figure}




\section{Ap\^endice - Modelos ARMA(7,7) e ARIMA (7,1,7) 24h}\label{sec:armaarima24}

\begin{figure}[H]
	\centering
	\caption{Comparação dos modelos ARMA e ARIMA, 1 dia a frente }
	\label{fig:1-ARMA-ARIMA24}
	\includegraphics[width=1\linewidth]{Apendices/Figuras/modelagem-24h/1-ARMA-ARIMA}
	
	Fonte: Autoria própria.
\end{figure}

\begin{figure}[H]
	\centering
	\caption{Comparação dos modelos ARMA e ARIMA, 7 dias a frente }
	\label{fig:10-ARMA-ARIMA24}
0	\includegraphics[width=1\linewidth]{Apendices/Figuras/modelagem-24h/7-ARMA-ARIMA}
	
	Fonte: Autoria própria.
\end{figure}


\begin{figure}[H]
	\centering
	\caption{Comparação dos modelos ARMA e ARIMA, 14 dias a frente }
	\label{fig:30-ARMA-ARIMA24}
	\includegraphics[width=1\linewidth]{Apendices/Figuras/modelagem-24h/14-ARMA-ARIMA}
	
	Fonte: Autoria própria.
\end{figure}

\begin{figure}[H]
	\centering
	\caption{Comparação dos modelos ARMA e ARIMA, 30 dias a frente }
	\label{fig:60-ARMA-ARIMA24}
	\includegraphics[width=1\linewidth]{Apendices/Figuras/modelagem-24h/30-ARMA-ARIMA}
	
	Fonte: Autoria própria.
\end{figure}

%
%\section{Ap\^endice - Modelos ARIMAX, SARIMA e SARIMAX }\label{sec:arimaxsarimasarimax24}
%
%\begin{figure}[H]
%	\centering
%	\caption{Comparação dos modelos ARIMAX, SARIMA e SARIMAX, 1 dia à frente }
%	\label{fig:1-ARIMAX-SARIMA-SARIMAX24}
%	\includegraphics[width=1\linewidth]{Apendices/Figuras/modelagem-24h/1-ARIMAX-SARIMA-SARIMAX}
%	
%
%\end{figure}
%
%\begin{figure}[H]
%	\centering
%	\caption{Comparação dos modelos ARIMAX, SARIMA e SARIMAX, 7 dias à frente }
%	\label{fig:10-ARIMAX-SARIMA-SARIMAX24}
%	\includegraphics[width=1\linewidth]{Apendices/Figuras/modelagem-24h/7-ARIMAX-SARIMA-SARIMAX}
%	
%
%\end{figure}
%
%
%\begin{figure}[H]
%	\centering
%	\caption{Comparação dos modelos ARIMAX, SARIMA e SARIMAX, 14 dias à frente }
%	\label{fig:30-ARIMAX-SARIMA-SARIMAX24}
%	\includegraphics[width=1\linewidth]{Apendices/Figuras/modelagem-24h/14-ARIMAX-SARIMA-SARIMAX}
%	
%
%\end{figure}
%
%\begin{figure}[H]
%	\centering
%	\caption{Comparação dos modelos ARIMAX, SARIMA e SARIMAX, 30 dias à frente }
%	\label{fig:60-ARIMAX-SARIMA-SARIMAX24}
%	\includegraphics[width=1\linewidth]{Apendices/Figuras/modelagem-24h/30-ARIMAX-SARIMA-SARIMAX}
%	
%
%\end{figure}
%
%\newpage
%
%\section{Ap\^endice - Modelos RNN e Prophet }\label{sec:rnnprophet}
%
%\begin{figure}[H]
%	\centering
%	\caption{A rede neural recorrente (RNN) com todos os horizontes }
%	\label{fig:rnn}
%	\includegraphics[width=1\linewidth]{Apendices/Figuras/modelagem-24h/RNN}
%	
%
%\end{figure}
%
%\begin{figure}[H]
%	\centering
%	\caption{Previsões do modelo Prophet para diferentes horizontes}
%	\label{fig:prophet}
%	\includegraphics[width=1\linewidth]{Apendices/Figuras/modelagem-24h/prophet}
%	
%
%\end{figure}


\section{Ap\^endice - Modelos Regress\~ao linear, XGB Regress\~ao, Ligth GBM Regress\~ao e Regress\~ao de Floresta Aleat\'oria 24h}\label{sec:lrxgblgbmrf24}

\begin{figure}[H]
	\centering
	\caption{Comparação dos modelos Regressão linear, XGB Regressão, Ligth GBM Regressão e Regressão de Floresta Aleatória, 1 dia a frente }
	\label{fig:1-LR-XGB-LGBM-RF24}
	\includegraphics[width=1\linewidth]{Apendices/Figuras/modelagem-24h/1-LR-XGB-LGBM-RF}
	
	Fonte: Autoria própria.
\end{figure}

\begin{figure}[H]
	\centering
	\caption{Comparação dos modelos Regressão linear, XGB Regressão, Ligth GBM Regressão e Regressão de Floresta Aleatória, 7 dias a frente }
	\label{fig:10-LR-XGB-LGBM-RF24}
	\includegraphics[width=1\linewidth]{Apendices/Figuras/modelagem-24h/7-LR-XGB-LGBM-RF}
	
Fonte: Elaboração própria a partir de dados da SANEPAR (2018 a 2020)
\end{figure}


\begin{figure}[H]
	\centering
	\caption{Comparação dos modelos Regressão linear, XGB Regressão, Ligth GBM Regressão e Regressão de Floresta Aleatória, 14 dias a frente }
	\label{fig:30-LR-XGB-LGBM-RF24}
	\includegraphics[width=1\linewidth]{Apendices/Figuras/modelagem-24h/14-LR-XGB-LGBM-RF}
	
Fonte: Elaboração própria a partir de dados da SANEPAR (2018 a 2020)
\end{figure}

\begin{figure}[H]
	\centering
	\caption{Comparação dos modelos Regressão linear, XGB Regressão, Ligth GBM Regressão e Regressão de Floresta Aleatória, 30 dias a frente }
	\label{fig:60-LR-XGB-LGBM-RF24}
	\includegraphics[width=1\linewidth]{Apendices/Figuras/modelagem-24h/30-LR-XGB-LGBM-RF}
	
Fonte: Elaboração própria a partir de dados da SANEPAR (2018 a 2020)
\end{figure}


%-----------------------------------------------------------------
%% FIM DO DOCUMENTO
\end{document}