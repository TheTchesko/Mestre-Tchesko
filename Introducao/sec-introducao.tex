\setlength{\parskip}{1pt} %% ESPAÇO DEPOIS DE 6pt
\pagenumbering{arabic}  %% PAGINAÇÃO INICIA AQUI
\setcounter{page}{16} % Reiniciar contagem de página para 1
%\pagestyle{plain} % Restaurar numeração de página com estilo "plain"
\clearpage
\pagestyle{fancy}


\section{Introdu{\c c}{\~a}o} \label{sec:int}

Este capítulo apresenta o conteúdo abordado nesta dissertação, que se concentra na utilização de modelos de Aprendizado de Máquina (ML) para prever futuramente os dados coletados pela SANEPAR. Os dados coletados referem-se ao abastecimento de água no bairro alto durante o período de 2018 a 2020, quando ocorreu uma escassez que afetou toda a população da capital paranaense.

Dentro do contexto de análise de séries temporais e tomada de decisão, foram explorados modelos de ML para aplicação nesses dados. Por meio de uma revisão sistemática da literatura, foram identificados e tabulados os modelos clássicos mais comumente utilizados para análise de séries temporais.


\subsection{Contexto da Pesquisa} \label{subsec:contexto}
%Torna a análise de séries temporais e previsões valiosas ferramentas para apoiar o processo de tomada de decisão a curto, médio e longo prazo. Devido às não linearidades, sazonalidade, tendência e que podem ocorrer em séries temporais de abastimento de água nos dados temporais, o desenvolvimento de modelos de previsão eficientes é uma tarefa desafiadora \cite{mateus}.
%
%Na Figura \ref{fig:paradigma-ml}, são mostradas as etapas de como deve ocorrer a análise de dados e a seleção dos modelos. Essa seleção pode ser feita de forma que se tenha que escolher o que deve ser previsto na variável. Feito isso, temos a primeira etapa que será dos dados, depois que cada um foi identificado com seus rótulos de entrada e saída. Os dados não podem conter \textbf{NaN} (do inglês \textit{not a number}) ou dados ausentes, o que evita falsos positivos. 

Torna-se evidente que a análise de séries temporais e previsões são ferramentas valiosas para apoiar o processo de tomada de decisão em curto, médio e longo prazo. Devido às não linearidades, sazonalidades e tendências que podem ocorrer nos dados temporais de abastecimento de água, o desenvolvimento de modelos de previsão eficientes torna-se uma tarefa desafiadora \cite{mateus}.

Na Figura \ref{fig:paradigma-ml}, as etapas de como a análise de dados e a seleção dos modelos devem ocorrer são apresentadas. Essa escolha pode ser conduzida de modo a determinar o que deve ser previsto na variável. Feito isso, a primeira etapa envolve a preparação dos dados, garantindo que cada um tenha sido identificado com seus rótulos de entrada e saída. É imperativo que os dados não contenham \textbf{NaN} (do inglês \textit{not a number}) ou dados ausentes, evitando assim falsos positivos.

\begin{figure}[!htb]
	\centering
	\caption{Paradigma de aprendizado de máquina}
	\includegraphics[width=\linewidth]{Introducao/Figuras/paradigma-ml}
	
	\fonte{Adaptado de \cite{apmonitor}}
	\label{fig:paradigma-ml}
\end{figure}

Ao realizar essa etapa, a pessoa deve visualizar os dados para garantir que estejam carregados corretamente e em um tamanho tolerável, o que é conhecido como avaliação dos dados. Uma vez que os dados estejam limpos e devidamente carregados, sem falsos positivos, a divisão dos dados pode ser efetuada.
A otimização dos dados para os modelos pode ocorrer de diversas maneiras, como a utilização da biblioteca Optuna em Python, que emprega a otimização Bayesiana para cada modelo pré-listado, reduzindo assim o tempo de processamento.

A validação é uma prática comum em conjuntos de dados extensos, permitindo que os modelos interajam mais eficientemente com os dados e proporcionem resultados mais precisos. Após essa etapa, na escolha dos modelos, há a possibilidade de determinar se o modelo é de série temporal, classificação, agrupamento ou regressão. Posteriormente, ao listar os modelos, cada um deles deve passar por uma avaliação com métricas específicas para verificar a precisão de seus resultados.

%Fazendo isso, os dados podem ser visualizados para garantir que estejam bem carregados e que estejam em um tamanho tolerável. Isso é chamado de avaliação dos dados. Com os dados limpos e bem carregados, sem falsos positivos, a divisão dos dados pode ser feita. 
%
%A otimização dos dados para os modelos pode ser realizada de várias formas, como o uso da biblioteca do python Optuna que usa otimização Bayesiana para cada modelo pré-listado, reduzindo assim o tempo de processamento. 
%
%A validação é comum em conjuntos de dados muito grandes para permitir que os modelos trabalhem mais com os dados, proporcionando resultados mais precisos. Após essa etapa, na escolha dos modelos, há a possibilidade de escolher o modelo de série temporal, se o modelo é de classificação, agrupamento ou regressão. Após listar os modelos, cada um deles deve ser avaliado em métricas para verificar a veracidade de cada um.



  
      
\subsubsection{Motiva\c c\~ao da Pesquisa} \label{subsubsec:motivacao}

A motivação desta pesquisa baseia-se na situação enfrentada por Curitiba e região metropolitana, conforme destacado por \cite{vasconcelos_2020}. A região passou por um rodízio de abastecimento de água, com períodos de 36 horas com abastecimento de água, seguidos por 36 horas sem abastecimento. A média geral dos reservatórios na região estava em torno de $27,96\%$ de sua capacidade. Além disso, a quantidade de chuva nos anos anteriores, em $2020$, foi de $1.704$ mm, superando a média anual de precipitação de $1.490$ mm.

Diante dessa situação, a pesquisa tem como abordagem principal a previsão do abastecimento de água, associada a condições de seca ou decorrentes das consequências da COVID-19. A partir dos dados coletados pela SANEPAR, é possível realizar uma análise detalhada, com o objetivo de prever e evitar a ocorrência de escassez de água. 
 
%A motivação desta pesquisa é baseada na situação enfrentada por Curitiba e região metropolitana, conforme apontado por \cite{vasconcelos_2020}. A região passou por um rodízio de abastecimento de água, com períodos de 36 horas com abastecimento de água seguidos por 36 horas sem abastecimento de água. A média geral dos reservatórios na região estava em torno de $27,96\%$ de sua capacidade. Além disso, a quantidade de chuva nos anos anteriores, de $2020$, foi de $1.704$ mm, superando a média anual de precipitação de $1.490$ mm.
% 	
%Diante dessa situação, a pesquisa tem como abordagem principal a previsão do abastecimento de água, que pode ser associada a condições de seca ou decorrentes das consequência da COVID-19. A partir dos dados coletados pela SANEPAR, é possível realizar uma análise detalhada, com o objetivo de prever e evitar a ocorrência de escassez de água. 
%    
     
    
          
\subsection{Objetivo geral} \label{subsec:objetivos}

O objetivo desta pesquisa é identificar o modelo mais adequado de séries temporais para abordar a escassez de água em Curitiba. Ao longo da dissertação, serão avaliados diversos modelos de regressão, com destaque para os modelos de redes neurais e o Prophet, conforme listados. É importante mencionar que a pesquisa enfatizará os modelos de \textit{gradient boosting}, amplamente reconhecidos na literatura por sua eficácia na previsão de séries temporais. Os principais modelos analisados incluem o ARIMA e suas variações mais contemporâneas. Além das previsões, também serão realizadas análises de anomalias nos dados, buscando compreender as causas subjacentes a essas ocorrências
 
    
    
    \subsubsection{Objetivos espec\'ificos e quest\~ao de pesquisa} \label{subsubsec:obespec}
    
Neste estudo, busca-se identificar e compreender possíveis anomalias nos dados, bem como investigar as causas por trás dessas ocorrências. O objetivo é responder às perguntas de pesquisa relacionadas a essas anomalias.

\begin{enumerate}[start=1, label={\textbf{Q} \arabic*}]
	\item \label{q1} Qual é a adequação da pressão atual para atender à demanda diária?
	\item \label{q2} Qual é o volume mínimo de água necessário no reservatório para evitar o acionamento das bombas durante o horário de pico? 
	\item \label{q3} Qual é a vazão ótima para atender à demanda diária?
	\item \label{q4} Como encontrar o ponto de equilíbrio entre a demanda e a vazão?
	\item \label{q5} Qual é o impacto do acionamento das bombas durante o horário de pico?
	 
	\begin{enumerate}[label=\alph*.]
	\item \label{q5:a} Qual é o nível ideal no reservatório para evitar a ativação das bombas da SANEPAR durante o período de maior demanda, das 18h às 21h, sem comprometer o abastecimento de água para a população? Além disso, como variam as médias das vazões nos horários críticos (18h às 21h) para as diferentes estações do ano (Outono, Inverno, Primavera, Verão)? 
	\item \label{q5:b} Existe tendência, padrão, sazonalidade para os dados destes três anos do Bairro Alto?
	\item \label{q5:c} Identificar quais os horários de maior demanda das $18$ às $21$?
	\item \label{q5:d} Quanto tenho que armazenar previamente no reservatório para não acionar as bombas no horário de pico?
	\item \label{q5:e} Se a vazão cresce e a pressão decresce temos uma ANOMALIA na rede (com base no histórico).	
	\end{enumerate}
\end{enumerate}

   
\subsection{Descri\c c\~ao do Problema} \label{subsec:descricao}

A descrição do problema é fundamental para obter uma compreensão mais precisa do que está sendo abordado neste trabalho. É por meio dessa descrição que as variáveis-chave são expostas e o objetivo da previsão é estabelecido de forma clara. Sem um plano estruturado para determinar o que deve ser previsto, torna-se difícil justificar o uso de modelos de previsão de dados. Portanto, é essencial estabelecer um propósito claro e definir as metas da previsão antes de aplicar os modelos adequados.

\begin{itemize}
	\item Bombas de sucção (B1, B2 e B3) – valor máximo da frequência 60 Hz
	
	\item Nível do Reservatório (Câmara 1) LT01 $ (m^3) $ - \textbf{PREVER}
	
	\item Vazão de entrada (FT01) $ (m^3/h) $
	
	\item Vazão de gravidade (FT02) $ (m^3/h) $
	
	\item Vazão de recalque (FT03) $ (m^3/h) $
	
	\item Pressão de Sucção (PT01SU) (mca)
	
	\item Pressão de Recalque (PT02RBAL) (mca)
\end{itemize}

A pesquisa fará uso da variável LT01, que representa o nível do reservatório e desempenha um papel de extrema importância, como evidenciado pelas Figuras \ref{fig:dados-todos} e \ref{fig:2020-a-frente}. Essas figuras retratam as anomalias ocorridas durante o período em que a capital paranaense foi afetada pela escassez de chuvas, resultando na redução do nível dos reservatórios e na implementação de rodízios periódicos, conforme discutido na subseção \ref{subsubsec:motivacao}. Assim, tais observações permitem uma compreensão mais aprofundada das perspectivas futuras.

\subsection{Procedimentos Metodol{\'o}gicos} \label{subsec:metod}

Com o intuito de realizar previsões e fazer comparações entre os modelos obtidos na revisão sistemática, será adotado um processo metodológico bem definido. Tal processo está detalhado na subseção \ref{subsubsec:etp} desta seção, onde foram estabelecidas as etapas a serem seguidas. Isso inclui a definição do que será previsto, bem como a seleção dos métodos a serem utilizados na Análise Exploratória de Dados (EDA).
   

\subsubsection{Etapas da Pesquisa}\label{subsubsec:etp}

%A pesquisa foi conduzida seguindo as etapas delineadas:
%
%\begin{figure}[!htpb]
%	\centering
%	\caption{Mapa das Etapas}
%	\label{fig:etapas}
%	\includegraphics[width=1\linewidth]{Introducao/Figuras/Etapas}
%	
%	\fonte{De autoria própria}
%\end{figure}

\begin{enumerate}[start=1, label={\textbf{Etapa} \arabic*}]
	
	\item \label{etp:1} \textbf{Análise Exploratória de Dados (EDA)}: Nesta etapa inicial, compreende-se abrangentemente as características dos dados. As tarefas envolvem a identificação de valores ausentes, a observação de padrões temporais e a detecção de anomalias. Gráficos de linha são comuns para visualizar a convergência dos dados e desvios potenciais \cite{Rostam2021108249}.
	
	\item \label{etp:2} \textbf{Definição de Variáveis Preditoras e Variável Alvo (MISO)}: Na segunda etapa, as variáveis preditoras e a variável alvo para a previsão de Múltiplas Entradas e Uma Saída (MISO) são selecionadas. Diferentes modelos, podem incorporar variáveis exógenas na modelagem. Essas variáveis adicionais aprimoram as capacidade de previsão do modelo, especialmente quando o horizonte de previsão se estende além dos dados históricos \cite{PAWLOWSKI202298}. 
	
	\item \label{etp:3} \textbf{Decomposição STL}: O método de decomposição STL (do inglês \textit{Seasonal and Trend Decomposition Using Loess}) separa uma série temporal em três componentes: sazonalidade, tendência e resíduo. Essa decomposição permite uma analisa separada das diferentes influências presentes nos dados. A componente sazonal representa variações periódicas e repetitivas, a componente de tendência indica a direção geral dos dados ao longo do tempo, e a componente de resíduo captura variações não explicadas pelas componentes anteriores \cite{DEOLIVEIRA2018776}.
	
	\item \label{etp:4} \textbf{Divisão dos Dados}: É prática comum dividir o conjunto de dados em conjuntos de treinamento, validação e teste para avaliar o desempenho do modelo. Essa divisão permite uma análise abrangente e objetiva das habilidades de generalização dos modelos, evitando problemas de ajuste excessivo ou insuficiente. A proporção de alocação pode variar, mas uma abordagem comum é alocar 70\% para treinamento e validação, e os 30\% restantes para o conjunto de testes. A porção de treinamento e validação pode ser subdividida em 80\% para treinamento e 20\% para validação \cite{Tao2020}.
	
	\item \label{etp:5} \textbf{Modelagem e Seleção do Modelo}: Nesta etapa, diversos modelos são construídos e avaliados. Alguns modelos comumente usados para previsão de séries temporais incluem ARX (do inglês \textit{Auto-Regressive with Exogenous Inputs}), AR (do inglês \textit{Auto-Regressive}), MA (do inglês \textit{Moving Average}), ARIMA, SARIMA (do inglês \textit{Seasonal Auto-Regressive Integrated Moving Averages}), SARIMAX (ARIMA Sazonal com variáveis exógenas) e modelos de aprendizado de máquina como RNN, LSTM (do inglês \textit{Long Short-Term Memory}), GRU (do inglês \textit{Gated Recurrent Unit}), Transformer (Transformador), DTR (do inglês \textit{Decision tree regressor}), LR (do inglês \textit{Linear Regression}), XGBoost (do inglês \textit{eXtreme Gradient Boosting}), Light GBM (do inglês \textit{Light Gradient Boosting Machine}) além do Prophet. A escolha do modelo final baseia-se em critérios como desempenho na validação, simplicidade do modelo e interpretabilidade dos resultados.
	
	\item \label{etp:6} \textbf{Validação e Ajuste do Modelo}: Após a construção do modelo, é importante avaliar seu desempenho usando dados de validação. Métricas de avaliação como Erro Médio Absoluto (MAE), Erro Médio Percentual Absoluto Simétrico (sMAPE) e Raiz do Erro Médio Quadrático Relativo (RRMSE) podem ser usadas para comparar e selecionar o melhor modelo. Além disso, técnicas de ajuste como otimização de hiperparâmetros e refinamento do modelo usando dados de treinamento e validação combinados podem melhorar o desempenho do modelo selecionado.
	
	\item \label{etp:7} \textbf{Previsão e Avaliação}: Com o modelo final ajustado, é possível fazer previsões para o conjunto de testes, que representa dados futuros não observados. Essas previsões são comparadas com os valores reais correspondentes para avaliar a qualidade e precisão do modelo. Métricas de desempenho mencionadas anteriormente (MAE, RRMSE, sMAPE) podem quantificar a precisão do modelo e compará-lo com outros modelos ou abordagens.
	
	\item \label{etp:8} \textbf{Teste de Significância}: Aplicar os modelos de previsão e fazer comparativo baseado em testes de significância estatística (\textit{Friedman e Nemenjy})

	
\end{enumerate}

Cada uma dessas etapas desempenha um papel crucial na pesquisa e no processo de modelagem de séries temporais, contribuindo para a compreensão dos dados, construção e validação dos modelos, além de previsões precisas.




    
    
\subsection{Justificativa da pesquisa} \label{subsec:justif}

No decorrer dessa dissertação ocorre da seguinte forma, para que possa ser previsto e para que seja evitado a efetiva falta d'água, e como pode ser solucionado esse problema para não voltar a acontecer.

\subsubsection{Contribui\c c\~oes} \label{subsubsec:Contribuição}

Seguindo as questões de pesquisa feito na subseção \ref{subsubsec:obespec} tem duas contribuições, a primeira levando em conta a demanda d'água na cidade de Curitiba, entre a \ref{q1} a \ref{q4} é feito a previsão da demanda d'água, as outras ficam em como é o consumo d'água na cidade e gasto com energia no período de pico, mostrado na \ref{q5}\ref{q5:a} a \ref{q5}\ref{q5:e}.

Assim usando os métodos escolhido de previsão de series temporais, como os modelos ARIMA e ARIMA atualizado, como os modelos ARMA, SARIMA, ARIMAX e SARIMAX, outros modelos mais simples que vem do modelo ARIMA, como, por exemplo, os modelos AR, ARX e MA para previsão mais precisa como na \ref{q5} em diante os modelos regressivo ou modelos de gradiente, modelos regressivo testado aqui foi os modelos LR e floresta aleatória, para os modelos de gradiente foi usado XGBoost e Ligth GBM se torna uma opção mais viável na hora de tomar a decisão em meio aos gastos de energia e água que a empresa SANEPAR teve e com o intuito de minimizar esses gasto. Foi estabelecido os horizonte de previsão para que possa ser tomado a melhor decisão a respeito da demanda d'água.

Em ambas das contribuições foi realizado o tabelamento tanto em curto prazo (1 a 30 dias, um mês) até longo prazo (30 a 60 dias, dois meses). Para que assim o melhor modelo tanto em curto quanto em longo prazo seja mostrado e evidenciado. Os modelos ARIMA para o problema em questão em horizonte de previsão de longo prazo se sai melhor que os modelos de reforço de gradiente, modelos de gradiente é mais viável em previsão de curto prazo, por exemplo de 1 dia a frente até uma semana. E ainda sim os modelos ARIMA ou que os modelos que se vem dele supera os gradiente.


    
\subsection{Estrutura do trabalho} \label{subsec:estrutura}

 Este documento está estruturado em~\ref{sec:conclusoes} capítulos, divididos da seguinte forma:
   
    \begin{figure}[H]
    	\centering
    	\caption{Estrutura da dissertação}
    	\label{fig:estrutura}
    	\includegraphics[width=0.7\linewidth]{Introducao/Figuras/Estrutura}
    	
    	Fonte: Elaboração própria 
    \end{figure}
O capítulo~\ref{sec:int} apresenta a introdução do trabalho, contendo a contextualização, a motivação, o objetivo geral, os objetivos específicos e a questão de pesquisa, a descrição do problema, a metodologia utilizada, a justificativa da pesquisa, as contribuições e a organização do trabalho.
O capítulo~\ref{sec:refteo} revisão teórica do trabalho, fazendo uma visão geral dos principais pesquisadores sobre as questões abordadas na pesquisa.
O capítulo~\ref{sec:base} apresenta os modelos que serão trabalhados nos dados coletados.
O capítulo~\ref{sec:result} apresenta os resultados da pesquisa, assim como uma análise dos resultados gerados.
O capítulo~\ref{sec:conclusoes}, finalmente, apresenta as considerações finais da pesquisa e algumas propostas para pesquisas futuras.



    