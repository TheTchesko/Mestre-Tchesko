\subsection{Contexto da pesquisa} \label{subsec:contexto}
\citeonline{mateus} A necessidade de desenvolvimento do planejamento estratégico no mundo corporativo e no dia-a-dia torna a análise de séries temporais e previsões valiosas ferramentas para apoiar o processo de tomada de decisão a curto, médio e longo prazo. Devido a não linearidades, sazonalidade, tendência e ciclicidade nos dados temporais, o desenvolvimento de modelos de previsão eficientes é uma tarefa desafiadora. 

 Em séries temporais o aprendizado de máquinas é frequentemente utilizado para grandes processamentos de dados, com o conjunto de dados SANEPAR da cidade de Curitiba no estado do Paraná há um volume significativo no consumo de água e com a escassez que a cidade experimentou é necessário avaliar os dados para ter certeza do que está acontecendo, quando há escassez de água e picos que ocorrem entre horas e dias.
 
 Entre os modelos preditivos que serão apresentados em uma revisão sistemática, avaliar o melhor modelo que podemos utilizar e validar quando e como ocorre a escassez de água. Estas análises estarão em \textit{python}.
 
 Explorando o que são séries temporais e aprendizagem de máquinas, séries temporais são dados armazenados ao longo do tempo que permitem a um observador analisar anomalias nos dados. Nas séries cronológicas, a classificação dos dados por ano ou dia é crítica, e se os dados forem atribuídos aleatoriamente, pode tornar mais difícil prever e tomar decisões com base nos dados coletados. 
A análise das médias pode ser bastante perigosa se você não excluir pontos fora da curva também conhecidos como ``\textit{outliers}''. Isto pode gerar dados muito positivos ou negativos que não correspondem à realidade.
   
      
\subsubsection{Motiva\c c\~ao da pesquisa} \label{subsubsec:motivacao}
   %Escrever algo motivador 
    
    De acordo com \cite{vasconcelos_2020} Curitiba e região metropolitana enfrentou um rodízio com $36$ horas com água e $36$ horas sem abastecimento. A média geral dos reservatórios da região está em $27,96\%$ da capacidade. Assim em medida a isso essa pesquisa tem como a abordagem da falta de água, essa falta que pode ser vista como uma seca, em média nos anos anteriores de 2020 a chuva tem marcado a quantia de $1.704$ mm. \cite{vasconcelos_2020} Desde 2016, quando registrou 1.704 mm de chuva, Curitiba não atingiu mais a média anual de precipitação, que é de 1.490 mm, com base em dados da estação pluviométrica do IBMET.  Apesar de abaixo da média, o mínimo registrado desde então ocorreu em 2020, com 1.158 mm.
    
    Em mediana a esta motivação podem ser melhor interpretados os dados que a SANPEAR ofereceu para prever e evitar a escassez de água que foi registrada e a anomalia que foi detectada em 2020, com o retorno das chuvas os reservatórios tinham aumentado de nível.
    