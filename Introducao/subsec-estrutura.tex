\subsection{Estrutura da Disserta\c c\~ao} \label{subsec:estrutura}


Essa dissertação está organizada em capítulos. Cada um abordando aspectos específicos da pesquisa. 
O Capítulo~\ref{sec:int},  apresentou uma contextualização do estudo, destacando a motivação e os objetivos a serem alcançados. 

O Capítulo~\ref{sec:refteo}, menciona uma visão geral das principais pesquisas e estudos relacionados às questões abordadas na pesquisa.

No Capítulo~\ref{sec:base}, são apresentados os modelos que serão utilização dos dados de séries temporais da SANEPAR, com os dados coletados.


O Capítulo~\ref{sec:result}, apresenta os resultados obtidos ao longo da pesquisa.  Os resultados de previsão são detalhados, evidenciando análises quanto aos modelos de previsão projetados.

Finalizando, o Capítulo~\ref{sec:conclusoes} apresenta os resultados da pesquisa e um cronograma.


 
 \begin{figure}[!htb]
 	\centering
 	\caption{Estrutura da dissertação}
 	\label{fig:estrutura}
 	\includegraphics[width=0.9\linewidth]{Introducao/Figuras/estrutura}
 	
 	 
 \end{figure}



