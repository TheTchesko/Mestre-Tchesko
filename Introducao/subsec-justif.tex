\subsection{Justificativa da pesquisa} \label{subsec:justif}

Ao longo desta dissertação, os seguintes aspectos são abordados visando a previsão e tomada de decisões adequadas para evitar a ocorrência futura de escassez de água.

\subsubsection{Contribui\c c\~oes} \label{subsubsec:Contribuição}

Após as perguntas de pesquisa apresentadas na subseção \ref{subsubsec:obespec}, surgem duas contribuições significativas nesta dissertação. A primeira diz respeito à previsão da demanda de água na cidade de Curitiba, abordando aspectos como consumo e gasto de energia durante períodos de pico (conforme mencionado em \ref{q5}\ref{q5:a} a \ref{q5}\ref{q5:e}).

Nesse sentido, foram utilizados métodos de previsão de séries temporais, como os modelos ARIMA, ARMA, SARIMA, ARIMAX e SARIMAX, bem como modelos mais simples derivados do ARIMA, como AR, ARX e MA. Além disso, foram explorados modelos regressivos, como LR e RFR, e modelos baseados em gradientes, como XGBoost e LightGBM. Essa variedade de modelos foi selecionada visando uma previsão precisa e eficiente, levando em consideração as demandas relacionadas ao consumo de energia e água pela empresa SANEPAR, com o objetivo de minimizar os gastos associados.

As previsões foram realizadas tanto para o curto prazo (1 a 7 dias) quanto para o longo prazo (14 a 30 dias), a fim de embasar a tomada de decisões estratégicas em relação à demanda de água. Os resultados destacaram que, no longo prazo, os modelos ARIMA tiveram um desempenho superior em comparação aos modelos baseados em gradientes. Por outro lado, os modelos de gradiente mostraram-se mais eficazes nas previsões de curto prazo, como para um dia ou uma semana. Ainda assim, os modelos ARIMA e seus derivados superaram os modelos baseados em gradientes.

A comparação entre os modelos de previsão desempenha um papel central nesta dissertação. Através do teste estatístico Ljung-Box, é possível avaliar o desempenho de cada modelo ARIMA tanto no curto prazo quanto no longo prazo. No Apêndice \ref{sec:comtb18}, apresenta-se a comparação dos modelos por meio desse teste estatístico. Além disso, nas Figuras \ref{fig:modelos-arima} e \ref{fig:violin-lr-xgb-lgbm-rf} do Apêndice \ref{sec:comtb24}, é realizada a comparação dos modelos regressivos com os modelos ARIMA. Essas análises comparativas são cruciais para a seleção do modelo mais adequado, permitindo uma tomada de decisão embasada para enfrentar o problema em questão.





