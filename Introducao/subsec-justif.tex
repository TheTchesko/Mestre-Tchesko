\subsection{Justificativa da pesquisa} \label{subsec:justif}

No decorrer dessa dissertação ocorre da seguinte forma, para que possa ser previsto e para que seja evitado a efetiva falta d'água, e como pode ser solucionado esse problema para não voltar a acontecer.

\subsubsection{Contribui\c c\~oes} \label{subsubsec:Contribuição}

Seguindo as questões de pesquisa feito na subseção \ref{subsubsec:obespec} tem duas contribuições, a primeira levando em conta a demanda d'água na cidade de Curitiba, entre a \ref{q1} a \ref{q4} é feito a previsão da demanda d'água, as outras ficam em como é o consumo d'água na cidade e gasto com energia no período de pico, mostrado na \ref{q5}\ref{q5:a} a \ref{q5}\ref{q5:e}.

Assim usando os métodos escolhido de previsão de series temporais, como os modelos ARIMA e ARIMA atualizado, como os modelos ARMA, SARIMA, ARIMAX e SARIMAX, outros modelos mais simples que vem do modelo ARIMA, como, por exemplo, os modelos AR, ARX e MA para previsão mais precisa como na \ref{q5} em diante os modelos regressivo ou modelos de gradiente, modelos regressivo testado aqui foi os modelos LR e floresta aleatória, para os modelos de gradiente foi usado XGBoost e Ligth GBM se torna uma opção mais viável na hora de tomar a decisão em meio aos gastos de energia e água que a empresa SANEPAR teve e com o intuito de minimizar esses gasto. Foi estabelecido os horizonte de previsão para que possa ser tomado a melhor decisão a respeito da demanda d'água.

Em ambas das contribuições foi realizado o tabelamento tanto em curto prazo (1 a 30 dias, um mês) até longo prazo (30 a 60 dias, dois meses). Para que assim o melhor modelo tanto em curto quanto em longo prazo seja mostrado e evidenciado. Os modelos ARIMA para o problema em questão em horizonte de previsão de longo prazo se sai melhor que os modelos de reforço de gradiente, modelos de gradiente é mais viável em previsão de curto prazo, por exemplo de 1 dia a frente até uma semana. E ainda sim os modelos ARIMA ou que os modelos que se vem dele supera os gradiente.

