\subsection{Justificativa da pesquisa} \label{subsec:justif}

No decorrer dessa dissertação ocorre da seguinte forma, para que possa ser previsto e para que seja evitado a efetiva falta d'água, e como pode ser solucionado esse problema para não voltar a acontecer.

\subsubsection{Contribui\c c\~oes} \label{subsubsec:Contribuição}

%%Ler tese do Matheus

%A água como oxigênio têm uma importância significativa na vida humana, visando isso pode ser notado que sem ela eventualmente não existiria a humanidade, pois segundo \citeonline{walter} A água é a principal substância da vida. O corpo humano é composto de 48 a 54\% de água para pessoas adultas. Com o envelhecimento, a porcentagem de água no corpo humano diminui.
%
%Tendo isso em mente a água que temos hoje pode ser um risco em acabar, como prova \citeonline{vasconcelos_2020} comenta isso no jornal Brasil fato, como os dados dessa pesquisa, vai até o mesmo ano da publicação desse artigo.

Seguindo as questões de pesquisa feito na subseção \ref{subsubsec:obespec} tem duas contribuições, a primeira levando em conta a demanda d'água na cidade de Curitiba, entre a \ref{q1} a \ref{q4} é previso a demanda d'água, a outra fica em como é o consumo d'água na cidade e gasto com energia no período de pico, mostrado na \ref{q5}\ref{q5:a} a \ref{q5}\ref{q5:e}.

Assim usando os métodos escolhido de previsão de series temporais, como os modelos ARIMA e ARIMA atualizado, como os modelos SARIMA, ARIMAX e SARIMAX, outros modelos mais simples que vem do modelo ARIMA, como, por exemplo, os modelos AR, ARX e MA para previsão mais precisa como na \ref{q5} em diante os modelos regressivo ou modelos de gradiente, modelos regressivo testado aqui foi os modelos LR e floresta aleatória, para os modelos de gradiente foi usado XGBoost e Ligth GBM se torna uma opção mais viável na hora de tomar a decisão em meio aos gastos que a empresa SANEPAR teve e com o intuito de minimizar esses gasto.

Em ambas das contribuições foi realizado o tabelamento tanto em curto prazo (30 dias, um mês) até longo prazo (60 dias, dois meses). Para que assim o melhor modelo tanto em curto quanto em longo prazo seja mostrado e evidenciado.