\subsection{Justificativa da pesquisa} \label{subsec:justif}

Ao longo desta dissertação, os seguintes aspectos são abordados visando a previsão e tomada de decisões adequadas para evitar a ocorrência futura de escassez de água.

\subsubsection{Contribui\c c\~oes} \label{subsubsec:Contribuição}

Após as perguntas de pesquisa apresentadas na subseção \ref{subsubsec:obespec}, surgem duas contribuições significativas nesta dissertação. A primeira diz respeito à previsão da demanda de água na cidade de Curitiba, abordando aspectos como consumo e gasto de energia durante períodos de pico \citeonline{arima_python,xgboost_intro}.

Segundo estudos recentes, os modelos ARIMA desempenham um papel fundamental na análise de séries temporais \citeonline{arima_python}. De acordo com pesquisas, os modelos ARIMA são amplamente utilizados na previsão de séries temporais devido à sua capacidade de capturar padrões complexos e comportamentos de longo prazo \citeonline{arima_python}.

Conforme relatos, o modelo XGBoost tem sido aplicado com sucesso em problemas de previsão de séries temporais \citeonline{xgboost_intro}. Estudos demonstraram que o XGBoost é uma poderosa ferramenta para lidar com desafios de previsão em séries temporais \citeonline{xgboost_intro}. De acordo com especialistas, o LightGBM tem ganhado destaque como um modelo eficiente para previsão de séries temporais \citeonline{lightgbm_forecasting}. Pesquisas recentes destacam o desempenho promissor do LightGBM na análise e previsão de séries temporais \citeonline{lightgbm_forecasting}. De acordo com \citeonline{decisiontree}, o algoritmo de árvore de decisão é um dos mais populares e eficazes na área de classificação. Amplamente empregado em mineração de dados e aprendizado de máquina, destaca-se por sua simplicidade e interpretabilidade. A representação gráfica das árvores de decisão facilita sua compreensão e interpretação. Além disso, esses modelos possuem eficiência computacional e são capazes de lidar tanto com dados categóricos quanto numéricos. Já \citeonline{random_forest_regression} enfatizam a importância do uso de \textit{random forest regression} na previsão de séries temporais.

\begin{quoting}[rightmargin=0cm,leftmargin=4cm]
	\begin{singlespace}
		{\footnotesize \noindent As redes neurais recorrentes (RNNs) são uma classe de redes neurais artificiais que permitem o processamento de dados sequenciais. Elas são amplamente utilizadas em mineração de dados e aprendizado de máquina devido à sua capacidade de modelar dependências temporais. As RNNs têm sido aplicadas com sucesso em diversas tarefas de processamento de linguagem natural, como modelagem de linguagem, tradução automática e análise de sentimentos \citeonline{rnn}.}
		
		{\footnotesize \noindent As redes neurais convolucionais (CNNs) são uma classe de redes neurais profundas amplamente utilizadas em tarefas de reconhecimento de imagens e vídeos. As CNNs têm sido aplicadas com sucesso em diversas tarefas de processamento de linguagem natural, como classificação de texto, reconhecimento de entidades nomeadas e análise de sentimentos \citeonline{cnn}.}
		
		{\footnotesize \noindent As unidades recorrentes com portões (GRUs) são um tipo de rede neural recorrente que foi introduzido como uma alternativa às unidades de memória de longo e curto prazo (LSTMs). As GRUs têm se mostrado eficazes em muitas tarefas de processamento de linguagem natural, como tradução automática, classificação de texto e análise de sentimentos \citeonline{gru}.}
		
		{\footnotesize \noindent A memória de longo prazo de curto prazo (LSTM) é um tipo de rede neural recorrente que foi introduzida para resolver o problema do gradiente desvanecido nas RNNs padrão. As LSTMs têm se mostrado eficazes em muitas tarefas de processamento de linguagem natural, como tradução automática, classificação de texto e análise de sentimentos \citeonline{lstm}.}
	\end{singlespace}
\end{quoting}


De acordo com \citeonline{DBLP:journals/corr/abs-2107-02248}, constata-se que as redes GRU demonstram uma vantagem em relação às redes LSTM em termos de complexidade e desempenho. Além disso, as Redes Neurais Recorrentes (RNN), incluindo tanto GRU quanto LSTM, são observadas como tendo uma capacidade competitiva em tarefas de aprendizagem de sequências. Portanto, os resultados sugerem que as redes GRU, pertencentes à categoria de RNN, são mais adequadas para a aprendizagem de sequências simbólicas que exigem memória seletiva e adaptativa.

No entanto, de acordo com as constatações dessa pesquisa, o modelo RNN (especificamente o GRU) é identificado como o melhor modelo para a tarefa em questão. Isso se baseia na habilidade do GRU em eficientemente memorizar e generalizar sequências, resultando em um desempenho superior em comparação a outros modelos, incluindo Redes Neurais Artificiais (ANN), Convolutional Neural Networks (CNN) e modelos baseados em gradientes, como XGBoost, LightGBM, entre outros. Dessa forma, conclui-se que o modelo RNN, e mais especificamente o GRU, se destaca como a escolha mais apropriada para a aprendizagem e previsão de sequências simbólicas relevantes para o contexto investigado.

É importante ressaltar que essas conclusões se baseiam nos resultados e análises específicos fornecidos pelo estudo citado, e podem variar conforme as características particulares do conjunto de dados e do problema abordado.
