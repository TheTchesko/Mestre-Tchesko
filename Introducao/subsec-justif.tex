\subsection{Justificativa da Pesquisa} \label{subsec:justif}

Ao longo desta dissertação, os seguintes aspectos são abordados visando a previsão e tomada de decisões.

\subsubsection{Contribui\c c\~oes} \label{subsubsec:Contribuição}


A dissertação fundamenta-se em modelos previamente não explorados neste contexto, como GRU, LSTM, XGBOOST, LGBM, Transformer, RNN, ANN e CNN. A primeira contribuição aborda a previsão da demanda de água em Curitiba, considerando elementos como consumo e consumo de energia durante picos.

Vários desses modelos na literatura não foram aplicados neste tema de demanda de água, utilizando-os para comparação entre si e em relação aos modelos que já foram trabalhados nesse contexto de demanda d'água, a fim de certificar se esses modelos podem ou não ser utilizados nesse contexto.

Os modelos do tipo ARIMA e suas variantes foram aplicados neste tema, como demonstrado por \citep{2-s2.0-85099424908, 2-s2.0-85069459067}. Alguns outros modelos, apesar de suas vantagens, ainda não foram devidamente aplicados, como é o caso do modelo de RNN \cite{2-s2.0-85067419084}, que se mostrou significativamente superior aos outros 19 modelos listados ao longo deste trabalho.

%A primeira contribuição diz respeito à previsão da demanda de água na cidade de Curitiba, abordando aspectos como consumo e gasto de energia durante períodos de pico.
%
%Segundo estudos recentes, os modelos ARIMA desempenham um papel fundamental na análise de séries temporais. Os modelos ARIMA são utilizados na previsão de séries temporais devido à sua capacidade de capturar padrões complexos e comportamentos de longo prazo \cite{arima_python}.
%
%Conforme relatos, o modelo XGBoost tem sido aplicado com sucesso em problemas de previsão de séries temporais. Estudos demonstraram que o XGBoost é uma ferramenta para lidar com desafios de previsão em séries temporais \cite{xgboost_intro}. O LightGBM tem ganhado destaque como um modelo eficiente para previsão de séries temporais \cite{WELLENS20221482}. De acordo com \citeonline{decisiontree}, o algoritmo de árvore de decisão é um dos mais populares e eficazes na área de classificação. Alguns modelos baseados em árvores possuem eficiência computacional e são capazes de lidar tanto com dados categóricos quanto numéricos. Já \citeonline{random_forest_regression} enfatiza a importância do uso de \textit{Random Forest Regression} na previsão de séries temporais.
%
%As RNNs (Redes Neurais Recorrentes) são redes neurais artificiais que possibilitam o processamento de dados sequenciais. Elas são amplamente utilizadas em mineração de dados e aprendizado de máquina devido à sua habilidade em modelar dependências temporais \cite{rnn}. Assim como outras redes neurais artificiais, as RNNs são capazes de lidar com a detecção de anomalias de forma mais eficaz do que modelos mais básicos de séries temporais, como o modelo ARIMA e seus predecessores. Enquanto o ARIMA pode identificar anomalias nos dados, as RNNs podem ajustar e otimizar os neurônios para aumentar a eficácia na detecção dessas anomalias.
%
%As CNNs (Redes Neurais Convolucionais) representam uma classe de redes neurais profundas amplamente utilizadas em tarefas de reconhecimento de imagens e vídeos \cite{cnn}. Embora sejam uma opção vantajosa para previsões de tempo, assim como as RNNs, as CNNs são especialistas em dados bidimensionais. Isso pode resultar em erros maiores quando comparadas às RNNs em contextos de dados sequenciais.
%
%As GRUs (Unidades Recorrentes com Portões)  são um tipo de rede neural recorrente introduzido como alternativa às LSTMs (Unidades de Memória de Longo e Curto Prazo) \cite{gru}. As LSTMs, por sua vez, foram desenvolvidas para resolver o problema do gradiente desvanecido em RNNs padrão \cite{lstm}. Embora sejam semelhantes às RNNs, as GRUs e LSTMs incorporam técnicas específicas para melhorar a capacidade das RNNs em lidar com dependências temporais a longo prazo. Cada uma delas possui particularidades que as tornam mais adequadas para diferentes cenários de previsão, com a GRU sendo especialmente útil para resolver problemas relacionados à dissipação de gradientes.
%
%
%Constata-se que as redes GRU demonstram uma vantagem em relação às redes LSTM em termos de complexidade e desempenho. As RNN, incluindo tanto GRU quanto LSTM, são observadas como tendo uma capacidade competitiva em tarefas de aprendizagem de sequências. Portanto, os resultados sugerem que as redes GRU, pertencentes à categoria de RNN, são mais adequadas para a aprendizagem de sequências simbólicas que exigem memória seletiva e adaptativa \cite{DBLP:journals/corr/abs-2107-02248}.
%
%No entanto, de acordo com as constatações dessa pesquisa, o modelo RNN especificamente o GRU é identificado como o melhor modelo para a tarefa em questão. Isso se baseia na habilidade do GRU em eficientemente memorizar e generalizar sequências, resultando em um desempenho superior em comparação a outros modelos, incluindo ANN (do inglês \textit{Artificial Neural Network}), CNN e modelos baseados em gradientes, tais como XGBoost e LightGBM. Dessa forma, conclui-se que o modelo RNN, e mais especificamente o GRU, se destaca como a escolha mais apropriada para a aprendizagem e previsão de séries temporais.


