\subsection{Justificativa da pesquisa} \label{subsec:justif}

No decorrer desta dissertação o seguinte para que se possa prever e tomar a decisão mais correta para evitar a real escassez de água e como este problema pode ser resolvido para que ele não ocorra novamente.

\subsubsection{Contribui\c c\~oes} \label{subsubsec:Contribuição}

Após as perguntas de pesquisa feitas na subseção \ref{subsubsec:obespec} tem duas contribuições, a primeira considerando a demanda de água na cidade de Curitiba entre \ref{q1} a \ref{q4} é feita a previsão da demanda de água, as outras estão em como é o consumo de água na cidade e o gasto de energia no período de pico mostrado em \ref{q5}\ref{q5:a} a \ref{q5}\ref{q5:e}.


Assim, utilizando os métodos escolhidos de previsão de séries temporais, tais como os modelos ARIMA e ARIMA atualizados, tais como os modelos ARMA, SARIMA, ARIMAX e SARIMAX, outros modelos mais simples que vêm do modelo ARIMA, tais como os modelos AR, ARX e MA para uma previsão mais precisa como no \ref{q5} em diante os modelos regressivos ou modelos gradientes. Os modelos regressivos testados aqui foram o LR e modelos florestais aleatórios, para os modelos de gradiente XGBoost e Ligth GBM foi usado para se tornar uma opção mais viável no momento de tomar a decisão em meio aos gastos de energia e água que a empresa SANEPAR tinha e a fim de minimizar esses gastos. O horizonte de previsão foi estabelecido para que a melhor decisão pudesse ser tomada em relação à demanda de água.

Em ambas as contribuições a tabulação foi feita tanto a curto prazo (1 a 7 dias, uma semana) quanto a longo prazo (14 a 30 dias, um mês). Para que o melhor modelo, tanto a curto prazo como a longo prazo, seja mostrado e evidenciado. Os modelos ARIMA para o problema em questão no horizonte de previsão a longo prazo têm melhor desempenho do que os modelos de aumento de gradiente os modelos de gradiente são mais viáveis na previsão a curto prazo, por exemplo, de um dia para uma semana. E ainda assim, os modelos ARIMA ou os modelos que provêm dele superam os modelos de gradiente.

