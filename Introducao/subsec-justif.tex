\subsection{Justificativa da Pesquisa} \label{subsec:justif}

Ao longo desta dissertação, os seguintes aspectos são abordados visando a previsão e tomada de decisões adequadas para evitar a ocorrência futura de escassez de água.

\subsubsection{Contribui\c c\~oes} \label{subsubsec:Contribuição}

As perguntas de pesquisa apresentadas na subseção \ref{subsubsec:obespec}, surgem duas contribuições significativas nesta dissertação. A primeira diz respeito à previsão da demanda de água na cidade de Curitiba, abordando aspectos como consumo e gasto de energia durante períodos de pico \citeonline{arima_python,xgboost_intro}.

Segundo estudos recentes, os modelos ARIMA desempenham um papel fundamental na análise de séries temporais. Os modelos ARIMA são amplamente utilizados na previsão de séries temporais devido à sua capacidade de capturar padrões complexos e comportamentos de longo prazo \cite{arima_python}.

Conforme relatos, o modelo XGBoost tem sido aplicado com sucesso em problemas de previsão de séries temporais. Estudos demonstraram que o XGBoost é uma poderosa ferramenta para lidar com desafios de previsão em séries temporais \cite{xgboost_intro}. O LightGBM tem ganhado destaque como um modelo eficiente para previsão de séries temporais \cite{lightgbm_forecasting}. De acordo com \citeonline{decisiontree}, o algoritmo de árvore de decisão é um dos mais populares e eficazes na área de classificação. Amplamente empregado em mineração de dados e aprendizado de máquina, destaca-se por sua simplicidade e interpretabilidade. A representação gráfica das árvores de decisão facilita sua compreensão e interpretação. Além disso, esses modelos possuem eficiência computacional e são capazes de lidar tanto com dados categóricos quanto numéricos. Já \citeonline{random_forest_regression} enfatiza a importância do uso de \textit{Random Forest Regression} na previsão de séries temporais.

As RNNs (Redes Neurais Recorrentes) são redes neurais artificiais que permitem o processamento de dados sequenciais. Elas são amplamente utilizadas em mineração de dados e aprendizado de máquina devido à sua capacidade de modelar dependências temporais. As RNNs têm sido aplicadas com sucesso em diversas tarefas de processamento de linguagem natural, como modelagem de linguagem, tradução automática e análise de sentimentos \cite{rnn}.
		
As CNNs (Redes Neurais Convolucionais)  são uma classe de redes neurais profundas amplamente utilizadas em tarefas de reconhecimento de imagens e vídeos. As CNNs têm sido aplicadas com sucesso em diversas tarefas de processamento de linguagem natural, como classificação de texto, reconhecimento de entidades nomeadas e análise de sentimentos \cite{cnn}.
		
As GRUs (Unidades Recorrentes com Portões) são um tipo de rede neural recorrente que foi introduzido como uma alternativa às LSTMs (Unidades de Memória de Longo e Curto Prazo). As GRUs têm se mostrado eficazes em muitas tarefas de processamento de linguagem natural, como tradução automática, classificação de texto e análise de sentimentos \cite{gru}.
		
A LSTM é um tipo de rede neural recorrente que foi introduzida para resolver o problema do gradiente desvanecido nas RNNs padrão. As LSTMs têm se mostrado eficazes em muitas tarefas de processamento de linguagem natural, como tradução automática, classificação de texto e análise de sentimentos \cite{lstm}.


Constata-se que as redes GRU demonstram uma vantagem em relação às redes LSTM em termos de complexidade e desempenho. As RNN, incluindo tanto GRU quanto LSTM, são observadas como tendo uma capacidade competitiva em tarefas de aprendizagem de sequências. Portanto, os resultados sugerem que as redes GRU, pertencentes à categoria de RNN, são mais adequadas para a aprendizagem de sequências simbólicas que exigem memória seletiva e adaptativa \cite{DBLP:journals/corr/abs-2107-02248}.

No entanto, de acordo com as constatações dessa pesquisa, o modelo RNN especificamente o GRU é identificado como o melhor modelo para a tarefa em questão. Isso se baseia na habilidade do GRU em eficientemente memorizar e generalizar sequências, resultando em um desempenho superior em comparação a outros modelos, incluindo ANN (do inglês \textit{Artificial Neural Network}), CNN e modelos baseados em gradientes, como XGBoost, LightGBM, entre outros. Dessa forma, conclui-se que o modelo RNN, e mais especificamente o GRU, se destaca como a escolha mais apropriada para a aprendizagem e previsão de séries temporais.


