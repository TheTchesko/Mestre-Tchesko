\subsection{Descri\c c\~ao do problema} \label{subsec:descricao}

Esta subseção discute as variáveis no conjunto de dados e como elas serão previstas.

\begin{itemize}
	\item Bombas de sucção (B1, B2 e B3) – valor máximo da frequência 60 Hz
	
	\item[] Variáveis importantes: Fluxo, pressão e nível
	
	\item Nível do Reservatório (Câmara 1) LT01 $ (m^3) $ - \textbf{PREVER}
	
	\item Vazão de entrada (FT01) $ (m^3/h) $
	
	\item Vazão de gravidade (FT02) $ (m^3/h) $
	
	\item Vazão de recalque (FT03) $ (m^3/h) $
	
	\item Pressão de Sucção (PT01SU) (mca)
	
	\item Pressão de Recalque (PT02RBAL) (mca)
\end{itemize}

Na pesquisa será utilizada a variável LT01, que é o nível do reservatório, este nível é de grande importância como mostram as Figuras \ref{fig:dados-todos} e \ref{fig:2020-a-frente}.

\subsection{Procedimentos metodol{\'o}gicos} \label{subsec:metod}

Esta parte trata de como a dissertação será realizada em cada etapa da análise.
   
    \subsubsection{Etapas da pesquisa}\label{subsubsec:etp}
    A pesquisa seguiu as seguintes etapas:
    \begin{figure}[H]
    	\centering
    	\caption{Mapa das Etapas}
    	\label{fig:etapas}
    	\includegraphics[width=1\linewidth]{Introducao/Figuras/Etapas}
    	
    	Fonte: Elaboração própria
    \end{figure}
    \begin{enumerate}[start=1, label = {\textbf{Etapa} \arabic* } ]
    	\item Análise exploratória dos dados – EDA ( do inglês \textit{Exploratory Data Analysis}) \label{etp:1}
    	
    	A exploração de dados na EDA é fundamental para entender melhor os dados que estão sendo trabalhados, como, por exemplo, excluir valores ausentes, saber como os dados estão separados em horas ou dias e, assim, tomar a melhor decisão a ser trabalhada com os dados, usar gráficos de linha na análise para observar a convergência dos dados e as anomalias que podem ocorrer.
    	
        	
    	\item O que vai ser usado como variáveis previsoras e qual será a variável a ser predita (MISO) \label{etp:2}
    	
    	Nessa etapa, tem o papel de relacionar as variáveis ao que será previsto, como os modelos de variáveis exógenas que são usados aqui nos modelos SARIMAX, ARX e ARIMAX do tipo ARIMAS. Cada modelo tem a interação de mais variáveis do que o modelo ARIMA básico ou seus derivados AR, MA e SARIMA. O conhecimento de quais variáveis estão incluídas na modelagem do problema torna a modelagem mais abrangente quando o horizonte de previsão é estendido além dos dados.
    	
       	
    	\item Fazer a decomposição STL (do inglês \textit{Seasonal-Trend Decomposition}) Sazonalidade, Tendência e Resíduo \label{etp:3}
    	
        O algoritmo STL executa suavização na série de tempo usando LOESS em dois loops; o loop interno itera entre a suavização sazonal e de tendência e o loop externo minimiza o efeito de valores atípicos. Durante o loop interno, o componente sazonal é calculado primeiro e removido para calcular o componente de tendência. O restante é calculado subtraindo os componentes sazonais e de tendência da série de tempo.
        
        Os três componentes da análise STL se relacionam com a série de tempo bruta da seguinte forma:
        
        \begin{eqnarray}
        	y_i &=& s_i + t_i + r_i
        \end{eqnarray}
    	
    	Onde:
    	
    	\begin{itemize}
    		\item $y_i = O$ valor da série de tempo no ponto $i$.
    		\item $s_i = O$ valor do componente sazonal no ponto $i$.
    		\item $t_i = O$ valor do componente de tendência no ponto $i$.
    		\item $ri = O$ valor do componente restante no ponto $i$.
    	\end{itemize}
    	\item Verifique a média e o desvio padrão de cada um desses conjuntos para obter a divisão mais apropriada dos dados. Dividir o conjunto de dados em treinamento, validação e teste. 70\% para treinamento e validação e 30\% para testes a partir daí, dos 70\% divididos em 80\% para treinamento e 20\% para validação.\label{etp:4}
    	
    	
    	\item Estratégia de previsão (recursiva e iterada-método direto) \label{etp:5}
    	
    	A estratégia recursiva envolve o uso de um modelo de uma etapa várias vezes, onde a previsão para a etapa de tempo anterior é usada como uma entrada para fazer uma previsão na etapa de tempo seguinte.
    	
    	No caso de prever a demanda de água para os próximos dias, desenvolveríamos um modelo de previsão de uma etapa. Este modelo seria então usado para prever o dia 1, então esta previsão seria usada como um \textit{input} de observação para prever o dia 2.
    	 
    	Por Exemplo
    	
    	\begin{eqnarray}
    	prediction(t+1) &=& model_1(obs(t-1), obs(t-2), \ldots, obs(t-n))\\
    	prediction(t+2) &=& model_2(obs(t-2), obs(t-3), \ldots, obs(t-n))   	
    	\end{eqnarray}
    	
    	\citeonline{machinemaster} como as previsões são usadas no lugar das observações, a estratégia recursiva permite que os erros de previsão se acumulem de tal forma que o desempenho possa se degradar rapidamente à medida que o horizonte de tempo de previsão aumenta.
    	
    	
    	\item Horizonte de previsão (1 passo ou n passos a frente) \label{etp:6}
    	
    	Nessa etapa, o tipo de horizonte foi escolhido de forma a mudar entre os dias, prevendo um passo à frente, uma semana, duas semanas e um mês.
    	
    	
    	\item Modelos de previsão e métricas de desempenho \label{etp:7}
    	
    	Os modelos discutidos aqui são os modelos clássicos de previsão, juntamente com os modelos de regressão gradiente. Os modelos são AR, ARX, ARMA, ARIMA, SARIMA, SARIMAX e ARIMAX, seguidos pelos modelos de regressão LR, XGBRegressor, Random Forest Regressor e LGBMRegressor. Esses modelos adotados foram escolhidos pela revisão sistemática realizada na dissertação.
    	
    	As métricas usadas em toda a dissertação são as métricas RMSE, MAE e MAPE, encontradas na revisão e uma das mais usadas até hoje, na subseção \ref{subsec:metrica} é explicado em mais detalhes cada uma delas.
    	
    	
    	
    	%\item Ajustar os hiperparâmetros dos modelos de previsão Hiperparâmetro ajusta a priori (ex: número de neurônios da rede neural), e parâmetro (pesos da rede neural) ajusta durante o processo. \label{etp:8}
    	
    	
    	\item Aplicar os modelos de previsão e fazer comparativo baseado em testes de significância estatística (\textit{Friedman e Nemenyi}) \label{etp:9}
    	
    	
    	O teste de Friedman é o teste não paramétrico usado para comparar dados de amostras vinculadas, ou seja, quando o mesmo indivíduo é avaliado mais de uma vez. 
    	ou seja, quando o mesmo indivíduo é avaliado mais de uma vez. 
    	O teste de Friedman não usa os dados numéricos diretamente, mas sim as classificações ocupadas pelos dados após a classificação de cada grupo separadamente. 
    	separadamente. Após a classificação, a hipótese de igualdade da soma das classificações de cada grupo é testada. 

		O teste consiste em fazer comparações em pares com o intuito de verificar qual dos fatores que diferem entre si. No entanto, o teste de Nemenyi é muito conservador e pode não encontrar diferença significativa entre os pares testados.
    	
    \end{enumerate}






    