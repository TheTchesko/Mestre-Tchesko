\subsection{Procedimentos metodol{\'o}gicos} \label{subsec:metod}

Esta parte trata de como a dissertação será realizada em cada etapa da análise.
   
    \subsubsection{Etapas da pesquisa}\label{subsubsec:etp}
    A pesquisa seguiu as seguintes etapas:
    \begin{figure}[H]
    	\centering
    	\caption{Mapa das Etapas}
    	\label{fig:etapas}
    	\includegraphics[width=0.7\linewidth]{Introducao/Figuras/Etapas}
    	
    	Fonte: Elaboração própria
    \end{figure}
    \begin{enumerate}[start=1, label = {\textbf{Etapa} \arabic* } ]
    	\item Análise exploratória dos dados – EDA ( do inglês \textit{Exploratory Data Analysis}) \label{etp:1}
    	\item O que vai ser usado como variáveis previsoras e qual será a variável a ser predita (MISO) \label{etp:2}
    	\item Fazer a decomposição STL (do inglês \textit{Seasonal-Trend Decomposition}) Sazonalidade, Tendência e Resíduo \label{etp:3}
    	\item Verifique a média e o desvio padrão de cada um desses conjuntos para obter a divisão mais apropriada dos dados. Dividir o conjunto de dados em treinamento, validação e teste. 70\% para treinamento e validação e 30\% para testes a partir daí, dos 70\% divididos em 80\% para treinamento e 20\% para validação.\label{etp:4}
    	\item Estratégia de previsão (recursiva e iterada-método direto) \label{etp:5}
    	\item Horizonte de previsão (1 passo ou n passos a frente) \label{etp:6}
    	\item Modelos de previsão e métricas de desempenho \label{etp:7}
    	%\item Ajustar os hiperparâmetros dos modelos de previsão Hiperparâmetro ajusta a priori (ex: número de neurônios da rede neural), e parâmetro (pesos da rede neural) ajusta durante o processo. \label{etp:8}
    	\item Aplicar os modelos de previsão e fazer comparativo baseado em testes de significância estatística (\textit{Friedman e Nemenjy}) \label{etp:9}
    \end{enumerate}






    