\subsection{Objetivo geral} \label{subsec:objetivos}

%Escrever melhor dando mais efancie no que vou fazer.
 
O objetivo desta pesquisa é identificar o melhor modelo de séries temporais para abordar o problema da escassez de água que ocorreu em Curitiba. Ao longo da dissertação, foram avaliados diversos modelos de regressão, com foco especial nos modelos baseados em \textit{gradient boosting}, considerados eficazes na literatura para a previsão de séries temporais. Os principais modelos explorados incluem o ARIMA e suas variantes atualizadas. Além da previsão, também serão realizadas análises de anomalias nos dados, visando compreender as causas subjacentes a essas ocorrências.
    
    
    \subsubsection{Objetivos espec\'ificos e quest\~ao de pesquisa} \label{subsubsec:obespec}
    
Neste estudo, busca-se identificar e compreender possíveis anomalias nos dados, bem como investigar as causas por trás dessas ocorrências. O objetivo é responder às perguntas de pesquisa relacionadas a essas anomalias.

\begin{enumerate}[start=1, label={\textbf{Q} \arabic*}]
	\item \label{q1}  A pressão é suficiente para a demanda diária? 
	\item \label{q2} Quanta água deve ter no reservatório para evitar o acionamento das bombas no horário de pico ($18$ as $21$ h)? Quanto maior a frequência de funcionamento da bomba maior a demanda. Valor máximo $ 60 $ Hz. 
	\item \label{q3} Qual a vazão ótima para atender a demanda? Quanta pressão para atender a demanda? 
	\item \label{q4} Ponto de equilíbrio entre demanda e vazão e ter um armazenamento sem necessidade de acionar as bombas no período do custo energético mais caro ($18$ as $21$ horas).
	\item \label{q5} Se a SANEPAR ativar as bombas de sucção das $18$ às $21$ horas ela tem o maior custo energético, isto é, ela paga mais caro pela energia neste período.	
	 
	\begin{enumerate}[label=\alph*.]
	\item \label{q5:a} Qual o nível que deve estar no reservatório para não ser necessário a SANEPAR ativar as bombas das $18$ às $21$ horas sem faltar água para a população?
	Verificar a média das vazões nos horários críticos (onde tem a maior demanda $18$ às $21$ horas) para as diferentes estações do ano (Outono, Inverno, Primavera, Verão). 
	\item \label{q5:b} Existe tendência, padrão, sazonalidade para os dados destes 3 anos do Bairro Alto?
	\item \label{q5:c}Identificar quais os horários de maior demanda das $18$ às $21$?
	\item \label{q5:d} Quanto tenho que armazenar previamente no reservatório para não acionar as bombas no horário de pico?
	\item \label{q5:e} Se a vazão cresce e a pressão decresce temos uma ANOMALIA na rede (com base no histórico).	
	\end{enumerate}
\end{enumerate}
