\subsection{Objetivo Geral} \label{subsec:objetivos}

O objetivo desta dissertação de mestrado é desenvolver séries temporais para auxiliar na tomada de decisões em situações de escassez de água no Bairro Alto, em Curitiba. A ideia é utilizar modelos de séries temporais como suporte para a melhor gestão desse problema enfrentado pela cidade.

Diversos modelos de regressão serão avaliados, com foco especial em modelos de redes neurais e o Prophet. Destaca-se a ênfase em modelos de \textit{gradient boosting}, redes neurais artificiais, além do ARIMA e suas variações contemporâneas. 

%O objetivo desta dissertação de mestrado é desenvolver séries temporais para que a melhor decisão possa ser tomada no problema da falta de água no Bairro Alto, na cidade de Curitiba. Com modelos de séries temporais, pode-se apoiar-se neles para tomar a melhor decisão em relação a esse problema que a cidade estava enfrentando.
%
%Serão avaliados diversos modelos de regressão, com destaque para os modelos de redes neurais e o Prophet. É importante mencionar que serão enfatizados os modelos de \textit{gradient boosting}, redes neurais artificiais, além do ARIMA e suas variações mais contemporâneas. Na mesma visão, foi realizadas análises de anomalias nas séries temporais avaliadas do Bairro Alto em Curitiba, buscando compreender as causas subjacentes a essas ocorrências.
    
    
\subsubsection{Objetivos Espec\'ificos} \label{subsubsec:obespec}
    
%O objetivo específico é comparar modelos de séries temporais e técnicas de otimização baseadas em otimização bayesiana usando o algoritmo de TPE (do inglês \textit{Parzen Tree Estimation}). A análise se concentrará no desempenho de diferentes modelos de séries temporais em termos de precisão, eficiência e capacidade de prever padrões em conjuntos de dados específicos. Além disso, serão exploradas e implementadas estratégias de otimização baseadas em otimização bayesiana usando TPE para melhorar os hiperparâmetros desses modelos. O foco principal será identificar qual combinação de modelo de série temporal e configurações de otimização proporciona os melhores resultados, com o objetivo de aprimorar a precisão das previsões e otimizar o processo de modelagem.

%\begin{enumerate}
%	\item Comparação de Modelos e Técnicas de Otimização
%	
%Compara modelos de séries temporais.
%Avalia técnicas de otimização baseadas em otimização bayesiana utilizando o algoritmo de TPE (do inglês \textit{Parzen Tree Estimation}).
%
%\item Desempenho dos Modelos de Séries Temporais
%
%Avalia o desempenho dos diferentes modelos de séries temporais.
%Analisa a precisão, eficiência e capacidade de previsão desses modelos em conjuntos de dados específicos.
%
%\item Implementação de Estratégias de Otimização
%
%Explora estratégias de otimização baseadas em otimização bayesiana utilizando o algoritmo TPE.
%Implementa técnicas de otimização para ajustar hiperparâmetros dos modelos de séries temporais.
%
%\item Identificação da Melhor Combinação de Modelo e Otimização
%
%Identifica a combinação mais eficaz de modelo de séries temporais e configurações de otimização.
%Aprimora a precisão das previsões e otimiza o processo de modelagem.
%\end{enumerate}
    
\begin{enumerate}
	\item Comparação de Modelos e Técnicas de Otimização
	
	Compara modelos de séries temporais.
	Avalia técnicas de otimização baseadas em otimização bayesiana utilizando o algoritmo de TPE (do inglês  \textit{Tree-structured Parzen Estimator}).
	
	\item  Desempenho dos Modelos de Séries Temporais
	
	Avalia o desempenho dos diferentes modelos de séries temporais.
	Analisa a precisão, eficiência e capacidade de previsão desses modelos em conjuntos de dados específicos.
	
	\item  Implementação de Estratégias de Otimização
	
	Explora estratégias de otimização baseadas em otimização bayesiana utilizando o algoritmo TPE.
	Implementa técnicas de otimização para ajustar hiperparâmetros dos modelos de séries temporais.
	
	\item Identificação da Melhor Combinação de Modelo e Otimização
	
	Identifica a combinação mais eficaz de modelo de séries temporais e configurações de otimização.
	Aprimora a precisão das previsões e otimiza o processo de modelagem.
\end{enumerate}
    
%Neste estudo, busca-se identificar e compreender possíveis anomalias nos dados, para que assim tome a melhor decisão, bem como investigar as causas dessas ocorrências. Entre os objetivos específicos está responder às perguntas de pesquisa relacionadas a essas questões.
%
%\begin{enumerate}[start=1, label={\textbf{Q} \arabic*}]
%	\item \label{q1} Qual é a adequação da pressão atual para atender à demanda diária?
%	\item \label{q2} Qual é o volume mínimo de água necessário no reservatório para evitar o acionamento das bombas durante o horário de pico? 
%	\item \label{q3} Qual é a vazão ótima para atender à demanda diária?
%	\item \label{q4} Como encontrar o ponto de equilíbrio entre a demanda e a vazão?
%	\item \label{q5} Qual é o impacto do acionamento das bombas durante o horário de pico?
%	 
%	\begin{enumerate}[label=\alph*]
%	\item \label{q5:a} Qual é o nível ideal no reservatório para evitar a ativação das bombas da SANEPAR durante o período de maior demanda, das 18h às 21h, sem comprometer o abastecimento de água para a população?  
%	\item \label{q5:b} Existe tendência, padrão, sazonalidade para os dados destes três anos do Bairro Alto?
%	\item \label{q5:c} Identificar quais os horários de maior demanda de abastecimento de água das $18$ às $21$?
%	\item \label{q5:d} Quanto deve-se armazenar previamente no reservatório para não acionar as bombas no horário de pico?
%	\item \label{q5:e} Se a vazão cresce e a pressão decresce existe uma ANOMALIA na rede de abastecimento de água (com base no histórico).	
%	\end{enumerate}
%\end{enumerate}
