\subsection{Trabalhos Relacionados}\label{subsec:estudo-de-caso-base}


A previsão da demanda de água é uma preocupação fundamental para muitas organizações e autoridades responsáveis pelo abastecimento de água. A análise de séries temporais é uma abordagem comumente usada para prever padrões futuros com base em dados históricos. Neste estudo de caso, será explorado como a análise de séries temporais pode ser aplicada para prever a demanda de água ao longo do tempo.



Na subseção \ref{subsubsec:obespec} estão as perguntas de pesquisa que serão abordadas no estudo de caso, da pergunta \ref{q1} à \ref{q5}, com as ramificações da \ref{q5}.
Na subseção \ref{subsec:descricao}, são apresentadas as variáveis contidas no conjunto de dados coletado no período de $2018$ a $2020$, durante uma grave falta de água que afetou a cidade. Devido a essa situação, foi implementado um rodízio de abastecimento de água para os residentes. Os dados foram coletados em intervalos de uma hora, levando em consideração cada variável, com ênfase na variável-alvo, denominada LT01, que representa o nível do reservatório.

O conjunto de dados possui um total de $26.306$ linhas e $9$ colunas. Durante a coleta dos dados, verificou-se que eles apresentam padrões sazonais, indicando variações recorrentes ao longo do tempo. Além disso, constatou-se que o consumo diário foi significativamente afetado no ano de 2020, diferindo dos anos anteriores, nos quais as mudanças não foram tão significativas.

Ao longo do trabalho realizado, pôde-se observar na subseção \ref{subsec:detec} que foi realizada uma análise gráfica do problema antes da aplicação de qualquer método. A detecção de anomalias mostrou-se desafiadora, porém não impossível de ser realizada. Essa detecção permitiu a análise da presença de sazonalidade nos dados. A decomposição STL foi utilizada para essa finalidade, conforme descrito na etapa \ref{etp:3} e detalhado na subseção \ref{subsubsec:stl}, onde são apresentadas as decomposições realizadas.

É fundamental lembrar que, durante a análise exploratória, os dados sofreram algumas alterações. Por exemplo, a média diária foi calculada em vez de ser considerada a nível horário, resultando em uma redução do conjunto de dados de 26.306 linhas para 1.096 linhas. A decomposição STL foi aplicada nos formatos aditivo e multiplicativo, e ambas as abordagens estão ilustradas nas Figuras \ref{fig:stl-aditiva} e \ref{fig:stl}, respectivamente.
Adicionalmente, na subseção \ref{subsubsec:stl}, foi realizada a verificação da estacionariedade da série. O teste de DF (do inglês \textit{Dickey-Fuller}) foi empregado para auxiliar na tomada de decisões, e os resultados demonstraram que a série em análise é estacionária, conforme evidenciado pelo teste DF.


Dentro da análise, foram incluídos uma variedade de modelos para melhor capturar a natureza dos dados e aprimorar as previsões. Esses modelos incluem:
RNN, que leva em conta as dependências sequenciais nos dados para prever valores futuros.
LSTM, um tipo de RNN que lida especialmente bem com sequências longas e dependências de longo prazo.
GRU, outra variante de RNN que equilibra o poder de modelagem e a eficiência computacional.
Transformer, um modelo amplamente utilizado para tarefas de processamento de linguagem natural e sequências, que também pode ser adaptado para previsões sequenciais.
Prophet, um modelo de previsão desenvolvido pelo Facebook que lida bem com dados sazonais e tendências.
\textit{Decision Tree Regressor}, um modelo baseado em árvore de decisão que segmenta os dados em subgrupos para fazer previsões.
Esses modelos são complementados com abordagens tradicionais como:

Modelos de séries temporais univariados, incluindo AR, MA, ARMA, ARIMA e SARIMA, que levam em consideração a sazonalidade dos dados.
Modelos de séries temporais multivariados, como ARX, ARIMAX e SARIMAX, que incorporam variáveis exógenas para melhorar as previsões.
Foram explorados modelos de aprendizado de máquina supervisionados:

LR (\textit{Regressão Linear}), que estabelece relações lineares entre variáveis para fazer previsões.
RFR (\textit{Random Forest Regressor}), um \textit{ensemble} de árvores de decisão que captura complexas relações nos dados.
LightGBM e XGBoost, modelos baseados em \textit{gradient boosting} que são reconhecidos por sua eficácia na previsão e tomada de decisões. O XGBoost é particularmente conhecido por seu desempenho superior em várias métricas de avaliação.



\noindent\textbf{Ajuste do modelo}


Ao ajustar o modelo para a base de dados, foi feita uma alteração na ordem do modelo sugerido pelo \textbf{pmdarima}. A escolha foi trocar o modelo SARIMAX(1,1,1)(2,1,0,12) para SARIMAX(7,1,7)(2,1,0,12). Essa decisão foi tomada com base na observação de um ajuste mais preciso aos dados, evidenciado pela redução nos resíduos e uma melhor captura das características da série temporal. Além disso, considerando o conhecimento do problema e as características específicas dos dados, foi identificado que padrões mais complexos requeriam ordens mais altas para serem adequadamente capturados. Dessa forma, foi realizado um processo iterativo de experimentação e avaliação para determinar o modelo SARIMAX(7,1,7)(2,1,0,12) como o mais adequado para a base de dados em questão. É importante ressaltar que o desempenho do novo modelo será avaliado por meio de diagnósticos adicionais e análise dos resultados obtidos.

Os modelos RNN, LSTM e GRU foram ajustados minuciosamente por meio da técnica de otimização de hiperparâmetros do Optuna, permitindo uma exploração adaptativa e eficiente do espaço de configurações. Essa abordagem exclusiva do Optuna resultou em modelos sequenciais com aprimoramento notável na capacidade preditiva. Parâmetros vitais, como taxa de aprendizado, tamanho da camada oculta e número de unidades, foram otimizados de forma eficaz através do Optuna \cite{DBLP}.

O RFR apresentou melhorias notáveis após o ajuste com o Optuna. A otimização realizada pelo Optuna permitiu identificar uma combinação de hiperparâmetros ideal para o RFR, resultando em um significativo aprimoramento no desempenho preditivo desse modelo.
Considerando que o modelo LR não demonstrou melhorias significativas, uma decisão foi tomada para substituí-lo pelo modelo \textit{Decision Tree Regressor}. Este último foi ajustado empregando o Optuna, buscando encontrar a configuração de hiperparâmetros ideal para o modelo de árvore de decisão. Essa decisão foi respaldada pelo fato de que o Optuna havia demonstrado ser uma ferramenta eficaz para otimização de hiperparâmetros, como evidenciado pelas melhorias observadas no RFR e em outros modelos \cite{DBLP}.

Dessa forma, os modelos RNN, LSTM, GRU, XGBRegressor, LGBMRegressor e o Decision Tree Regressor foram todos otimizados com sucesso utilizando o Optuna, resultando em previsões mais robustas e confiáveis. No entanto, os modelos Transformer e Prophet mantiveram suas configurações originais devido à ausência de melhorias substanciais após tentativas de otimização com o Optuna.

O \textbf{Optuna} oferece uma abordagem de otimização de hiperparâmetros mais avançada e eficaz em comparação com outras técnicas amplamente utilizadas, como o GridSearchCV, BayesSearchCV e RandomizedSearchCV. Enquanto essas abordagens tradicionais têm suas vantagens, o Optuna leva a otimização de hiperparâmetros a um nível superior.
Existem geralmente dois tipos de métodos de amostragem: a amostragem relacional, que explora as correlações entre os parâmetros, e a amostragem independente, que recolhe amostras de cada parâmetro de forma independente. A amostragem independente não é necessariamente uma opção ingênua, porque alguns algoritmos de amostragem como o TPE A eficácia em termos de custos da amostragem relacional e independente depende do ambiente e da tarefa. O Optuna apresenta ambos, e pode lidar com vários métodos de amostragem independente independentes, incluindo TPE, bem como métodos de amostragem relacional como o CMA-ES. No entanto, há que ter algumas palavras de precaução para a implementação da amostragem relacional num definido por execução \cite{DBLP}.

\noindent\textbf{Avalia\c c\~ao do modelo}


A avaliação da precisão dos modelos de previsão é uma etapa fundamental no processo de modelagem. Diversas métricas podem ser utilizadas para esse propósito, como o sMAPE, o MAE e o RRMSE. Essas métricas têm sido amplamente adotadas na literatura de previsão e são consideradas indicadores confiáveis para mensurar a qualidade das previsões.

O MAPE é uma métrica bastante utilizada na avaliação de modelos de previsão, especialmente quando há variações significativas nos dados ou quando se deseja comparar a precisão de diferentes modelos. O MAPE calcula o erro médio percentual entre as previsões e os valores reais, fornecendo uma medida relativa da precisão do modelo \cite{zhang2016}.
O uso do erro médio absoluto (MAE) apresenta vantagens na avaliação do desempenho médio de um modelo, em comparação com o erro quadrático médio (RMSE) \cite{willmott2005advantages}.
Destacam a importância do RMSE na avaliação de modelos e argumentam contra a exclusão dessa métrica na literatura \cite{jones2017}.


O RRMSE é uma métrica de avaliação altamente eficaz para medir a precisão relativa de modelos de regressão. Eles destacam que sua normalização em relação à média dos valores reais permite uma interpretação intuitiva e facilita a comparação entre diferentes modelos. Segundo os autores, o RRMSE é amplamente utilizado na literatura devido à sua capacidade de fornecer uma medida robusta e padronizada da precisão dos modelos de regressão \cite{lopes2020evaluation}. O MAPE é amplamente utilizado na avaliação de modelos de previsão, especialmente quando há variações significativas nos dados ou quando se deseja comparar a precisão de diferentes modelos \cite{peng2017effective}.

O sMAPE é uma métrica amplamente utilizada para avaliar a precisão de modelos de previsão. Eles afirmam que o sMAPE possui algumas vantagens, como a consideração da simetria dos erros percentuais e a interpretação intuitiva como uma medida de precisão relativa \cite{nguyen2020toxicological}.

O MAE e o RMSE são métricas amplamente adotadas na análise de previsões, pois fornecem uma medida direta do desvio absoluto e do desvio quadrático médio entre as previsões e os valores observados. O MAE é particularmente útil quando se busca uma medida de erro que não seja sensível a valores extremos, enquanto o RMSE penaliza de forma mais significativa os erros maiores, oferecendo uma visão mais abrangente da precisão do modelo \cite{jones2017}.

O sMAPE é uma métrica de avaliação popular para comparar a precisão de diferentes modelos de previsão. Eles destacam que o sMAPE é particularmente útil quando os valores de demanda têm diferentes magnitudes, pois captura os erros relativos em uma escala percentual. Além disso, o sMAPE possui uma interpretação intuitiva e facilita a comparação entre modelos de previsão \cite{hyndman2006effect}.

\subsubsection{Estudo de Caso 1}

\noindent\textbf{Estudo de Caso 1: Adequação da Pressão e Vazão em uma Rede de Distribuição de Água}

\eqref{q1} Adequação da pressão atual para atender à demanda diária: Neste estudo de caso, o modelo SARIMAX (H) foi utilizado para avaliar a adequação da pressão atual em uma rede de distribuição de água, considerando a demanda diária \cite{2-s2.0-85099424908}. O objetivo foi prever a pressão na rede com base em dados históricos, permitindo que fosse realizada uma análise crítica da capacidade do sistema em atender às necessidades dos consumidores.

\eqref{q2} Volume mínimo de água no reservatório para evitar o acionamento das bombas: Para determinar o volume mínimo de água necessário no reservatório para evitar o acionamento das bombas durante o horário de pico, foi empregado um modelo Decision Tree Regressor (I) \cite{2-s2.0-85054695177}. Este modelo ajudou a identificar regras e padrões que guiam a tomada de decisão sobre o nível de armazenamento ideal.

\eqref{q3} Vazão ótima para atender à demanda diária: O estudo também buscou encontrar a vazão ótima para atender à demanda diária. Para isso, utilizou-se o modelo XGBRegressor (K) para otimizar a vazão na rede de distribuição, considerando as flutuações na demanda ao longo do dia \cite{2-s2.0-85130441623}.

\subsubsection{Estudo de Caso 2}

\noindent\textbf{Estudo de Caso 2: Impacto do Acionamento das Bombas durante o Horário de Pico em uma Rede de Distribuição de Água}

\eqref{q5} Impacto do acionamento das bombas durante o horário de pico: Neste segundo estudo de caso, analisou-se o impacto do acionamento das bombas durante o horário de pico em uma rede de distribuição de água.

\ref{q5}\eqref{q5:a} Nível ideal no reservatório e variação das vazões nos horários críticos: Utilizou-se o modelo ARIMA (E) \cite{2-s2.0-85069459067} para prever o nível ideal no reservatório e analisar as variações das vazões nos horários críticos, levando em consideração as diferentes estações do ano.

\ref{q5}\eqref{q5:b} Tendência, padrão e sazonalidade nos dados do Bairro Alto: Para identificar tendências, padrões e sazonalidades nos dados de três anos do Bairro Alto, empregou-se o modelo de decomposição STL, reconhecido por sua eficácia na modelagem de séries temporais com essas características.

\ref{q5}\eqref{q5:c} Identificação dos horários de maior demanda: A identificação dos horários de maior demanda entre as 18h e as 21h foi realizada com o uso da RNN (P) \cite{2-s2.0-85067419084}.

\ref{q5}\eqref{q5:d} Tendência, padrão e sazonalidade nos dados do Bairro Alto: Para identificar tendências, padrões e sazonalidades nos dados de três anos do Bairro Alto, empregou-se o modelo decomposição STL, reconhecido por sua eficácia na modelagem de séries temporais com essas características. Volume de armazenamento no reservatório para evitar o acionamento das bombas: Determinar a quantidade de água a ser armazenada previamente no reservatório para evitar o acionamento das bombas durante o horário de pico envolveu o modelo LGBMRegressor (L).

\ref{q5}\eqref{q5:e} Tendência, Padrão e Sazonalidade nos Dados do Bairro Alto: Para identificar tendências, padrões e sazonalidades nos dados de três anos do Bairro Alto, empregou-se o modelo STL, reconhecido por sua eficácia na modelagem de séries temporais com essas características. Detecção de anomalias na rede com base no histórico): Para detectar anomalias na rede com base no histórico de vazão e pressão, utilizou-se novamente o modelo ARX (B) \cite{2-s2.0-85051469381}.










