\subsection{Descri\c c\~ao do Problema} \label{subsec:descricao}
A descrição do problema, centrada no abastecimento de água, é crucial. Ela apresenta variáveis-chave, como Bombas de Sucção (B1, B2 e B3), cuja frequência máxima é de $60$ Hz, Nível do Reservatório (Câmara 1) representado por LT01 $(m^3)$, Vazão de entrada (FT01) em $(m^3/h)$, Vazão de gravidade (FT02) em $(m^3/h)$, Vazão de recalque (FT03) em $(m^3/h)$, Pressão de Sucção (PT01SU) medida em metros de coluna d'água (mca) e Pressão de Recalque (PT02RBAL) também em metros de coluna d'água (mca).

A pesquisa fará uso da variável LT01, que representa o nível do reservatório e desempenha um papel de extrema importância. A separação dos dados foi feita por hora a hora, mesmo que os dados obtidos da SANEPAR sejam de $2018$ a $2020$, sendo que o ano de $2020$ causou muitas irregularidades. É possível remover esse ano para melhor trabalhar com os dados.

Mesmo havendo $9$ variáveis nesse conjunto de dados, poderia-se trabalhar com $1$ para previsão e as outras $8$ como variáveis exógenas. No entanto, todas as variáveis podem ter correlação com o tanque, mas nem todas são necessárias, causando ruído na série temporal. Com isso em mente, foram retiradas as variáveis B3 e FT02 restando assim as variáveis de previsão com as variáveis que tiveram correlação significativa.

A dimensão dos dados fornecidos pela SANEPAR foi de $26.306$ linhas e $9$ colunas. Essas colunas representam as variáveis listadas anteriormente, com a exclusão das duas variáveis B3 e FT02, resultando em apenas $6$ variáveis no formato de variáveis exógenas e uma variável para previsão. Também é relevante observar que o ano de $2020$, devido às muitas anomalias nos dados, foi removido para mitigar a variação nos dados ao longo do tempo. Com essa abordagem, restam $17.522$ observações, com o intervalo temporal compreendido entre $2018$ e $2019$. Essa decisão foi tomada para evitar que o modelo sofra excessivamente com variações temporais.

%A descrição do problema destaca a importância do abastecimento de água, enfatizando variáveis-chave como Bombas de sucção (B1, B2 e B3), Nível do Reservatório (Câmara 1) LT01, Vazão de entrada (FT01), Vazão de gravidade (FT02), Vazão de recalque (FT03), Pressão de Sucção (PT01SU) e Pressão de Recalque (PT02RBAL). O objetivo claro da previsão, especialmente utilizando a variável LT01, é enfatizado.
%
%A escolha de uma granularidade horária, apesar do período de 2018 a 2020, é explicada devido às irregularidades causadas em 2020. A exclusão desse ano é justificada para melhorar o manuseio dos dados.
%
%A decisão de trabalhar com 1 variável para previsão (LT01) e as outras 8 como variáveis exógenas é explicada, destacando a possibilidade de ruído na série temporal se todas as variáveis forem mantidas. A escolha específica de retirar as variáveis B3 e FT02 é fundamentada na busca por uma série temporal mais precisa.



