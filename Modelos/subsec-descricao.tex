\subsection{Descri\c c\~ao do Problema} \label{subsec:descricao}
A descrição do problema, com foco no abastecimento de água, é essencial. Ela apresenta variáveis-chave, incluindo LT01, e estabelece claramente o objetivo da previsão. Sem essa clareza, o uso de modelos de previsão é difícil de justificar. Definir metas de previsão antes de aplicar modelos é fundamental.


\begin{itemize}
	\item Bombas de sucção (B1, B2 e B3) – valor máximo da frequência $60$ Hz
	
	\item Nível do Reservatório (Câmara 1) LT01 $ (m^3) $ 
	
	\item Vazão de entrada (FT01) $ (m^3/h) $
	
	\item Vazão de gravidade (FT02) $ (m^3/h) $
	
	\item Vazão de recalque (FT03) $ (m^3/h) $
	
	\item Pressão de Sucção (PT01SU) (mca)
	
	\item Pressão de Recalque (PT02RBAL) (mca)
\end{itemize}

A pesquisa fará uso da variável LT01, que representa o nível do reservatório e desempenha um papel de extrema importância.
A separação dos dados foi feito por hora a hora, mesmo que os dados que obteve da SANEPAR foi de 2018 à 2020 sendo 2020 causando muitas irregularidade é de um forma que tem como retirar esse ano para que assim os dados seja melhor trabalhado. 

Mesmo havendo 9 variáveis nesse conjunto de dados, poderia-se trabalhar com 1 para previsão e as outras 8 como variáveis exógenas. No entanto, todas as variáveis podem ter correlação com o tanque, mas nem todas são necessárias, causando assim ruído na série temporal. Com isso em mente, foram retiradas as variáveis B3, FT02, FT03 e PT02, restando assim as variáveis de previsão com as variáveis que tiveram correlação significativa.