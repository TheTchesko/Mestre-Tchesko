\subsection{Trabalhos Relacionados}\label{subsec:estudo-de-caso-base}


A previsão da demanda de água é uma preocupação fundamental para muitas organizações e autoridades responsáveis pelo abastecimento de água. A análise de séries temporais é uma abordagem comumente usada para prever padrões futuros com base em dados históricos. Neste estudo de caso, será explorado como a análise de séries temporais pode ser aplicada para prever a demanda de água ao longo do tempo.



\subsubsection{Defini\c c\~ao do problema}



Na subseção \ref{subsubsec:obespec} estão as perguntas de pesquisa que serão abordadas no estudo de caso, da pergunta \ref{q1} à \ref{q5}, com as ramificações da \ref{q5}.

\subsubsection{Coleta de dados}


Na subseção \ref{subsec:descricao}, são apresentadas as variáveis contidas no conjunto de dados coletado no período de 2018 a 2020, durante uma grave falta de água que afetou a cidade. Devido a essa situação, foi implementado um rodízio de abastecimento de água para os residentes. Os dados foram coletados em intervalos de uma hora, levando em consideração cada variável, com ênfase na variável-alvo, denominada LT01, que representa o nível do reservatório.

O conjunto de dados possui um total de 26.306 linhas e 9 colunas. Durante a coleta dos dados, verificou-se que eles apresentam padrões sazonais, indicando variações recorrentes ao longo do tempo. Além disso, constatou-se que o consumo diário foi significativamente afetado no ano de 2020, diferindo dos anos anteriores, nos quais as mudanças não foram tão significativas.



\subsubsection{An\'alise explorat\'oria dos dados}



Ao longo do trabalho realizado, pôde-se observar na subseção \ref{subsec:detec} que foi realizada uma análise gráfica do problema antes da aplicação de qualquer método. A detecção de anomalias mostrou-se desafiadora, porém não impossível de ser realizada. Essa detecção permitiu a análise da presença de sazonalidade nos dados. A decomposição STL foi utilizada para essa finalidade, conforme descrito na etapa \ref{etp:3} e detalhado na subseção \ref{subsubsec:stl}, onde são apresentadas as decomposições realizadas.

É fundamental lembrar que, durante a análise exploratória, os dados sofreram algumas alterações. Por exemplo, a média diária foi calculada em vez de ser considerada a nível horário, resultando em uma redução do conjunto de dados de 26.306 linhas para 1.096 linhas. A decomposição STL foi aplicada nos formatos aditivo e multiplicativo, e ambas as abordagens estão ilustradas nas Figuras \ref{fig:stl-aditiva} e \ref{fig:stl}, respectivamente.

Adicionalmente, na subseção \ref{subsubsec:stl}, foi realizada a verificação da estacionariedade da série. O teste de Dickey-Fuller (DF) foi empregado para auxiliar na tomada de decisões, e os resultados demonstraram que a série em análise é estacionária, conforme evidenciado pelo teste DF.



\subsubsection{Escolha do modelo}



Dentro da análise, foram incluídos uma variedade de modelos para melhor capturar a natureza dos dados e aprimorar as previsões. Esses modelos incluem:

\begin{description}
	\item [RNN (Rede Neural Recorrente),] que leva em conta as dependências sequenciais nos dados para prever valores futuros.

	\item [LSTM (Long Short-Term Memory),] um tipo de RNN que lida especialmente bem com sequências longas e dependências de longo prazo.

	\item [GRU (Gated Recurrent Unit),] outra variante de RNN que equilibra o poder de modelagem e a eficiência computacional.

	\item [Transformer,] um modelo amplamente utilizado para tarefas de processamento de linguagem natural e sequências, que também pode ser adaptado para previsões sequenciais.

	\item [Prophet,] um modelo de previsão desenvolvido pelo Facebook que lida bem com dados sazonais e tendências.

	\item [Decision Tree Regressor,] um modelo baseado em árvore de decisão que segmenta os dados em subgrupos para fazer previsões.

Esses modelos são complementados com abordagens tradicionais como:

	\item [Modelos de séries temporais univariados,] incluindo AR, MA, ARMA, ARIMA e SARIMA, que levam em consideração a sazonalidade dos dados.

	\item [Modelos de séries temporais multivariados,] como ARX, ARIMAX e SARIMAX, que incorporam variáveis exógenas para melhorar as previsões.

Além disso, foram explorados modelos de aprendizado de máquina supervisionados:

	\item [LR (\textit{Regressão Linear}),] que estabelece relações lineares entre variáveis para fazer previsões.

	\item [RFR (\textit{Random Forest Regressor}),] um \textit{ensemble} de árvores de decisão que captura complexas relações nos dados.

	\item [LightGBM e XGBoost,] modelos baseados em \textit{gradient boosting} que são reconhecidos por sua eficácia na previsão e tomada de decisões. O XGBoost é particularmente conhecido por seu desempenho superior em várias métricas de avaliação.

\end{description}

A inclusão desses modelos permite uma abordagem abrangente à previsão, aproveitando as vantagens específicas de cada método para lidar com diferentes aspectos dos dados e melhorar as previsões em um conjunto diversificado de cenários.



\subsubsection{Divis\~ao dos dados}


Para obter a divisão mais adequada dos dados, verificam-se a média e o desvio padrão de cada um desses conjuntos. O conjunto de dados é dividido em três partes: treinamento, validação e teste. Nessa divisão, utiliza-se inicialmente 70\% dos dados para treinamento e validação, e os 30\% restantes para teste. Em seguida, a porção de treinamento e validação é subdividida em 80\% para treinamento e 20\% para validação.

\subsubsection{Ajuste do modelo}
Nesta etapa, você aplicará o modelo selecionado aos dados de treinamento. Ajuste os parâmetros do modelo com o objetivo de minimizar os erros de previsão. Dependendo do modelo escolhido, você pode usar técnicas de otimização para encontrar os melhores parâmetros.

Ao ajustar o modelo para a base de dados, foi feita uma alteração na ordem do modelo sugerido pelo pmdarima. A escolha foi trocar o modelo SARIMAX(1,1,1)(2,1,0,12) para SARIMAX(7,1,7)(2,1,0,12). Essa decisão foi tomada com base na observação de um ajuste mais preciso aos dados, evidenciado pela redução nos resíduos e uma melhor captura das características da série temporal. Além disso, considerando o conhecimento do problema e as características específicas dos dados, foi identificado que padrões mais complexos requeriam ordens mais altas para serem adequadamente capturados. Dessa forma, foi realizado um processo iterativo de experimentação e avaliação para determinar o modelo SARIMAX(7,1,7)(2,1,0,12) como o mais adequado para a base de dados em questão. É importante ressaltar que o desempenho do novo modelo será avaliado por meio de diagnósticos adicionais e análise dos resultados obtidos.

Os modelos RNN, LSTM e GRU foram ajustados minuciosamente por meio da técnica de otimização de hiperparâmetros do Optuna, permitindo uma exploração adaptativa e eficiente do espaço de configurações. Essa abordagem exclusiva do Optuna resultou em modelos sequenciais com aprimoramento notável na capacidade preditiva. Parâmetros vitais, como taxa de aprendizado, tamanho da camada oculta e número de unidades, foram otimizados de forma eficaz através do Optuna.

O Random Forest Regressor (RFR) apresentou melhorias notáveis após o ajuste com o Optuna. A otimização realizada pelo Optuna permitiu identificar uma combinação de hiperparâmetros ideal para o RFR, resultando em um significativo aprimoramento no desempenho preditivo desse modelo.

No entanto, o modelo Transformer e o Prophet não puderam ser otimizados com sucesso por meio do Optuna. Apesar das tentativas de ajuste, as melhorias significativas não foram observadas em seus resultados. Estes modelos mantiveram suas configurações padrão, uma vez que os esforços de otimização com o Optuna não resultaram em ganhos substanciais.

Considerando que o modelo LR (Regressão Linear) não demonstrou melhorias significativas, uma decisão foi tomada para substituí-lo pelo modelo Decision Tree Regressor. Este último foi ajustado empregando o Optuna, buscando encontrar a configuração de hiperparâmetros ideal para o modelo de árvore de decisão. Essa decisão foi respaldada pelo fato de que o Optuna havia demonstrado ser uma ferramenta eficaz para otimização de hiperparâmetros, como evidenciado pelas melhorias observadas no RFR e em outros modelos.

Dessa forma, os modelos RNN, LSTM, GRU, XGBRegressor, LGBMRegressor e o Decision Tree Regressor foram todos otimizados com sucesso utilizando o Optuna, resultando em previsões mais robustas e confiáveis. No entanto, os modelos Transformer e Prophet mantiveram suas configurações originais devido à ausência de melhorias substanciais após tentativas de otimização com o Optuna.

O \textbf{Optuna} oferece uma abordagem de otimização de hiperparâmetros mais avançada e eficaz em comparação com outras técnicas amplamente utilizadas, como o GridSearchCV, BayesSearchCV e RandomizedSearchCV. Enquanto essas abordagens tradicionais têm suas vantagens, o Optuna leva a otimização de hiperparâmetros a um nível superior.

O \textbf{GridSearchCV}, por exemplo, é uma técnica exaustiva que explora todas as combinações possíveis de hiperparâmetros. Embora seja abrangente, essa abordagem pode se tornar computacionalmente intensiva e demorada, especialmente quando o espaço de busca é grande. Além disso, o GridSearchCV não considera a interação entre os hiperparâmetros, o que pode limitar sua capacidade de encontrar combinações ótimas.

O \textbf{BayesSearchCV} melhora a eficiência em relação ao GridSearchCV usando otimização bayesiana, mas ainda enfrenta limitações. Ele utiliza um processo de amostragem sequencial com um modelo probabilístico para explorar o espaço de hiperparâmetros de maneira mais inteligente. No entanto, o BayesSearchCV ainda pode exigir muitas iterações para encontrar configurações ideais, especialmente em espaços de busca complexos.

O \textbf{RandomizedSearchCV}, por outro lado, busca aleatoriamente subconjuntos do espaço de hiperparâmetros. Isso pode ser útil para espaços de busca vastos, mas ainda não garante uma exploração eficaz do espaço para encontrar as melhores configurações.

Aqui é onde o Optuna se destaca. Utilizando algoritmos avançados de otimização, o Optuna oferece uma abordagem mais inteligente e eficiente para encontrar as configurações ideais de hiperparâmetros. Ele emprega técnicas de amostragem adaptativa, como o algoritmo TPE (Tree-structured Parzen Estimator), para direcionar a pesquisa para as áreas mais promissoras do espaço de hiperparâmetros. Isso permite uma exploração mais eficaz e uma descoberta mais rápida das configurações que maximizam a métrica de avaliação.

Além disso, o Optuna é altamente flexível e pode ser facilmente integrado a diversos algoritmos e frameworks de aprendizado de máquina. Sua abordagem adaptativa e inteligente proporciona economia de recursos computacionais e, ao mesmo tempo, aumenta as chances de encontrar configurações de hiperparâmetros que levem a um desempenho superior do modelo.

Em resumo, o Optuna supera as abordagens tradicionais, como o GridSearchCV, BayesSearchCV e RandomizedSearchCV, oferecendo uma otimização de hiperparâmetros mais eficiente, inteligente e adaptativa. Sua capacidade de explorar de maneira direcionada o espaço de hiperparâmetros resulta em uma descoberta mais rápida das configurações ideais, impulsionando o desempenho dos modelos de aprendizado de máquina de forma significativa.

\subsubsection{Avalia\c c\~ao do modelo}


A avaliação da precisão dos modelos de previsão é uma etapa fundamental no processo de modelagem. Diversas métricas podem ser utilizadas para esse propósito, como o sMAPE, o MAE e o RRMSE. Essas métricas têm sido amplamente adotadas na literatura de previsão e são consideradas indicadores confiáveis para mensurar a qualidade das previsões.

De acordo com \citeonline{zhang2016}, o MAPE é uma métrica bastante utilizada na avaliação de modelos de previsão, especialmente quando há variações significativas nos dados ou quando se deseja comparar a precisão de diferentes modelos. O MAPE calcula o erro médio percentual entre as previsões e os valores reais, fornecendo uma medida relativa da precisão do modelo.

De acordo com \citeonline{willmott2005advantages}, o uso do erro médio absoluto (MAE) apresenta vantagens na avaliação do desempenho médio de um modelo, em comparação com o erro quadrático médio (RMSE).

\citeonline{jones2017} destacam a importância do RMSE na avaliação de modelos e argumentam contra a exclusão dessa métrica na literatura.


O RRMSE é uma métrica de avaliação altamente eficaz para medir a precisão relativa de modelos de regressão. Eles destacam que sua normalização em relação à média dos valores reais permite uma interpretação intuitiva e facilita a comparação entre diferentes modelos. Segundo os autores, o RRMSE é amplamente utilizado na literatura devido à sua capacidade de fornecer uma medida robusta e padronizada da precisão dos modelos de regressão \cite{lopes2020evaluation}. Segundo \citeonline{peng2017effective}, o MAPE é amplamente utilizado na avaliação de modelos de previsão, especialmente quando há variações significativas nos dados ou quando se deseja comparar a precisão de diferentes modelos.

O sMAPE é uma métrica amplamente utilizada para avaliar a precisão de modelos de previsão. Eles afirmam que o sMAPE possui algumas vantagens, como a consideração da simetria dos erros percentuais e a interpretação intuitiva como uma medida de precisão relativa \cite{nguyen2020toxicological}.

Além disso, \citeonline{jones2017} afirmam que o MAE e o RMSE são métricas amplamente adotadas na análise de previsões, pois fornecem uma medida direta do desvio absoluto e do desvio quadrático médio entre as previsões e os valores observados. O MAE é particularmente útil quando se busca uma medida de erro que não seja sensível a valores extremos, enquanto o RMSE penaliza de forma mais significativa os erros maiores, oferecendo uma visão mais abrangente da precisão do modelo.

O sMAPE é uma métrica de avaliação popular para comparar a precisão de diferentes modelos de previsão. Eles destacam que o sMAPE é particularmente útil quando os valores de demanda têm diferentes magnitudes, pois captura os erros relativos em uma escala percentual. Além disso, o sMAPE possui uma interpretação intuitiva e facilita a comparação entre modelos de previsão \cite{hyndman2006effect}.


Portanto, ao utilizar essas métricas, o pesquisador estará seguindo uma prática comum e fundamentada na literatura. O sMAPE permitirá avaliar a precisão relativa das previsões, enquanto o MAE e o RRMSE fornecerão uma medida direta dos desvios absolutos e quadráticos, respectivamente. Essas métricas fornecerão uma base sólida para a avaliação dos modelos de previsão utilizados na pesquisa.


\subsubsection{Previs\~oes Futuras}


Com base nos modelos AR, ARX, MA, ARMA, ARIMA, ARMAX, SARIMA, SARIMAX,  XGBRegressor, LGBMRegressor, RNN, LSTM, GRU, Transformer, Prophet e Decision Tree Regressor, que foram cuidadosamente aplicados e avaliados, é possível afirmar que uma vez que a precisão desses modelos tenha sido satisfatória, eles podem ser utilizados para fazer previsões futuras. Ao aplicar esses modelos aos dados futuros disponíveis, ele poderá estimar a demanda de água para diferentes horizontes de previsão, como um dia, uma semana, duas semanas e um mês.

Essas previsões fornecerão informações valiosas para o planejamento e gerenciamento eficiente dos recursos hídricos. Ao ter conhecimento antecipado da demanda de água esperada nos próximos períodos, será possível adotar medidas adequadas para garantir o suprimento apropriado de água, prevenir escassez ou desperdício e realizar um planejamento eficaz para a distribuição e uso dos recursos hídricos.

Com base nos resultados significativos obtidos por esses modelos durante o processo de validação, o pesquisador terá confiança em aplicá-los para previsões futuras de curto prazo. Essas previsões permitirão a compreensão das tendências e variações na demanda de água ao longo de diferentes períodos, capacitando os responsáveis pela gestão dos recursos hídricos a tomar decisões informadas e estratégicas.

Portanto, uma vez que os modelos tenham sido devidamente avaliados e demonstrado sua eficácia, eles podem ser utilizados para fazer previsões precisas da demanda de água em horizontes de previsão de um dia, uma semana, duas semanas e um mês, contribuindo para a gestão e planejamento eficiente dos recursos hídricos.

\subsubsection{Monitoramento e Ajuste Cont\'inuo}


É importante destacar que todas as questões de pesquisa abordadas neste estudo estão fundamentadas no fator dos horários de pico e nas anomalias que ocorreram durante o período analisado. O comportamento da demanda de água durante os horários de maior consumo e as anomalias observadas foram aspectos-chave que motivaram a realização desta pesquisa.

Ao investigar os efeitos dos horários de pico e das anomalias na demanda de água, o estudo teve como objetivo compreender melhor os padrões de consumo, identificar possíveis causas para as variações significativas na demanda e desenvolver modelos de previsão mais precisos. A análise desses aspectos contribuiu para uma melhor compreensão dos desafios enfrentados no abastecimento de água e na gestão dos recursos hídricos durante os períodos críticos.

Considerando a importância desses fatores na formulação das questões de pesquisa, as análises realizadas e os modelos desenvolvidos buscaram fornecer insights e informações relevantes para aprimorar a capacidade de previsão e planejamento do abastecimento de água, especialmente durante os horários de pico e diante de anomalias observadas.

\subsubsection{Principais Conclus\~ao}


Ao longo deste estudo de caso, foram resolvidas as questões de pesquisa levantadas por meio da aplicação da análise de séries temporais para prever a demanda de água. A abordagem adotada demonstrou ser eficaz na obtenção de insights valiosos para o gerenciamento do abastecimento hídrico.

Foi constatado que a análise de séries temporais é uma ferramenta promissora para prever a demanda de água, permitindo tomar decisões informadas e embasadas nesse contexto. Por meio da modelagem e aplicação de diversos modelos, como ARIMA, SARIMA e outros, foi possível analisar e interpretar os dados históricos de maneira precisa, obtendo previsões confiáveis.

Durante o estudo, foram levantadas questões relacionadas à sazonalidade da demanda de água, influência de fatores externos imprevisíveis e mudanças no comportamento dos consumidores. Através da adaptação das técnicas de análise de séries temporais, foi possível abordar essas questões de forma eficiente e obter respostas relevantes para o gerenciamento do abastecimento de água.

Ao longo do processo, foram identificadas anomalias e flutuações na demanda de água, bem como tendências sazonais específicas. Por meio da análise dos resultados obtidos com os modelos aplicados, foi possível ajustar e aprimorar as previsões, tornando-as mais acuradas e confiáveis.

Em suma, este estudo de caso demonstrou que a análise de séries temporais é uma abordagem eficaz para prever a demanda de água, permitindo uma gestão mais eficiente do abastecimento hídrico. Ao adaptar e aplicar as técnicas adequadas aos dados específicos e às características do contexto, foram resolvidas as questões de pesquisa propostas e obtidos resultados significativos.

Essas descobertas têm o potencial de contribuir para a tomada de decisões embasadas no planejamento e no gerenciamento da demanda de água, visando a sustentabilidade e a eficiência dos recursos hídricos.

\section{Resultados} \label{sec:result}

Neste capítulo, é fornecida uma síntese e uma visão geral dos resultados obtidos até o momento. É apresentado um resumo sucinto das principais realizações e descobertas que foram alcançadas até agora.


\subsection{Planejamento do Problema} \label{subsec:planexp}

Assim como apresentado na seção \ref{subsubsec:etp}, os passos da dissertação delinearam o processo pelo qual cada modelo foi construído e os métodos utilizados para responder às questões de pesquisa abordadas na seção \ref{subsubsec:obespec}. Esses passos proporcionaram uma cronologia lógica das etapas realizadas ao longo do tempo com os dados da SANEPAR, ilustrando o progresso e os resultados alcançados até o momento.


\subsubsection{An\'alise Explorat\'oria dos dados (EDA)}

A partir de \ref{etp:1} é realizado o EDA para o processamento de dados obtidos até agora, com EDA será respondido. De acordo com a \citeonline{Yu2016} Na era dos grandes dados, coletamos volumes de dados em massa caóticos, não estruturados e multimídia através de vários canais. Como descobrir as regras, modelos analíticos e hipóteses destes dados se tornou o novo desafio. A análise exploratória de dados foi promovida por John Tukey para encorajar os estatísticos a explorar os dados e possivelmente formular hipóteses que poderiam levar a uma maior coleta de dados e experimentos. Em contraste com a análise inicial de dados, a análise exploratória de dados (EDA) é uma abordagem para analisar conjuntos de dados para resumir suas principais características, muitas vezes com métodos visuais. Muitas técnicas de EDA têm sido adotadas em grandes análises de dados.

Olhando o \ref{q1} relacionando a demanda com a variável prevista e a pressão para a variável PT01 na Figura \ref{fig:person} pode-se ver que ambas estão trabalhando igualmente, quase uma correlação perfeita de $r=1$, então para esta pergunta basta olhar para a correlação Pearson na Figura \ref{fig:person}. 

Para \ref{q2} uma tabela é feita para responder melhor a esta pergunta


\begin{table}[H]
	\centering
	\caption{Descrição estatística dos dados com o filtro aplicado das 18h às 21h}\label{tb:est}
	\begin{tabular}{@{}cccccccccc@{}}
		\toprule
		\textbf{18 a 21h}  & \textbf{B1} & \textbf{B2} & \textbf{B3} & \textbf{LT01} & \textbf{FT01} & \textbf{FT02} & \textbf{FT03} & \textbf{PT01} & \textbf{PT02} \\ \midrule
		\textbf{Contagem} & 366         & 366         & 366         & 366           & 366           & 366           & 366           & 366           & 366           \\
		\textbf{Média}    & 43,87       & 22,26       & 8,70        & 3,34          & 164,83        & 133,08        & 102,01        & 4,23          & 17,29         \\
		\textbf{STD}      & 23,22       & 18,47       & 17,81       & 0,69          & 114,60        & 67,99         & 47,55         & 0,81          & 8,59          \\
		\textbf{Min}      & 0           & 0           & 0           & 0,99          & 0,07          & 0             & 0             & 1,88          & 0             \\
		\textbf{25\%}     & 37,93       & 0           & 0           & 2,87          & 64,31         & 131,06        & 107,92        & 3,69          & 16,77         \\
		\textbf{50\%}     & 57,99       & 30,92       & 0           & 3,41          & 201,37        & 146,17        & 121,40        & 4,22          & 22,46         \\
		\textbf{75\%}     & 57,99       & 37,25       & 0           & 3,86          & 268,61        & 158,71        & 127,07        & 4,85          & 22,52         \\
		\textbf{Max}      & 59,99       & 57,33       & 53,74       & 4,40          & 379,20        & 285,56        & 170,56        & 5,66          & 24,23         \\ \bottomrule
	\end{tabular}
	
	Fonte: Elaboração própria a partir de dados da SANEPAR (2018 a 2020)
\end{table}

Na tabela \ref{tb:est} o desvio padrão é dado pela sigla STD que vem do inglês \textit{standard deviation} também observando para responder ao \ref{q2} assim como toda empresa de tratamento de água é feito um acionamento automático chamado trava de segurança para que o tanque não chegue a zero e falte água em todos os lugares adjacentes que é abastecida por esta água, este mínimo que o tanque pode alcançar é $1.459 m^3\Longleftrightarrow 1459 $ litros e as bombas serão ativadas em sua potência máxima para evitar a ativação das bombas o nível do tanque tem que estar na faixa de $[3.843,4.256]\ m^3$ bomba 1 ainda estaria funcionando para completar o nível. Em casos de pico, o mais ideal, mas não o mais rentável, é outro tanque de reserva nesses momentos e instalar uma tubulação para conectar uma à outra. Durante o dia, ambos estariam abastecendo e à noite, por gravidade, ficariam com o mesmo nível até que o consumo atingisse um nível para acionar as bombas.  



%\begin{figure}[H]
%	\centering
%	\caption{Solução para o acionamento das bombas}
%	\label{fig:esquema}
%	\includegraphics[width=1\linewidth]{Resultados/Figuras/esquema}
%	Fonte: Elaboração própria 
%\end{figure}
%
%Na Figura \ref{fig:esquema} um esquema prático para evitar a escassez de água e o consumo em horários de pico. Este é um esquema muito simples de como a hora do dia pode ser melhorada para o armazenamento de água.

Na \ref{q3} o tanque tem como máximo nos dados $4,256 m^3$ de dano em litros $4256$L para atender esta demanda e manter o tanque quase cheio ou sempre cheio o fluxo de entrada tem que estar entre $[238,302] \ m^3/h$ fluxo de gravidade tem que estar entre $[126,182] \ m^3/h$ fluxo de retorno entre $[110,144] \ m^3/h$ pressão de sucção entre $[1.92,4.24] mca$, pressão de retorno entre $[21,24] \ mca$.

Para \ref{q4}, o ponto de equilíbrio para não iniciar as bombas seria o fluxo FT01 $211 m^3/h$ FT02 $114 m^3/h$ FT03 $100m^3/h$ e o nível do tanque a $3,545 m^3$.

Ao \ref{q5}\ref{q5:a} o tanque deve estar a um nível de $4,00 m^3$ para que não precise funcionar com bombas nas horas de pico. 

\subsubsection{M\'ultiplas entradas e sa\'ida \'unica (MISO)}

Nesta \ref{etp:2} os modelos que foram mais cobertos no decorrer da dissertação são os modelos ARIMA ou aqueles derivados deste modelo e os modelos regressivos fora do LR têm múltiplas entradas e uma saída da variável que se prevê a LT01, as outras variáveis servem como suporte para melhorar os modelos do tipo ARX ou modelos com variáveis exógenas. Os modelos ARIMA sem a variável exógena são apenas uma entrada semelhante com LR.

\subsubsection{Decomposi\c c\~ao STL}\label{subsubsec:stl}

A decomposição sazonal e de tendência utilizando o procedimento de Loess (STL) é uma técnica amplamente utilizada para decompor séries temporais em seus componentes sazonais, de tendência e restantes. De acordo com \citeonline{Theodosiou20111178}, o método STL realiza a decomposição aditiva dos dados por meio de uma sequência de aplicações do Loess mais suave, onde regressões polinomiais ponderadas localmente são aplicadas em cada ponto do conjunto de dados, tendo como variáveis explicativas os valores mais próximos do ponto cuja resposta está sendo estimada.

A decomposição STL é especialmente útil para identificar e isolar padrões sazonais e de tendência presentes nas séries temporais. Ela permite a separação dos componentes sazonais, que ocorrem em intervalos regulares ao longo do tempo, da componente de tendência, que indica a direção geral dos dados ao longo do tempo. A decomposição também resulta em uma componente restante, que representa a variação não explicada pelos componentes sazonais e de tendência.

Ao aplicar a decomposição STL, a série temporal pode ser expressa como a soma dos componentes sazonais, de tendência e restantes. Essa técnica é útil para análise e modelagem de séries temporais, pois proporciona uma compreensão mais clara dos padrões de variação presentes nos dados.

A decomposição STL é formalmente definida como:

\begin{eqnarray}
	y_t=f\left(S_t, T_t, R_t\right)&=&\left\{\begin{array}{l}
		y_t=S_t+T_t+R_t \quad \text { aditivo } \\
		y_t=S_t T_t R_t \quad \text { multiplicativo }
	\end{array}\right. \label{eq:stl}
\end{eqnarray}

\begin{figure}[htp!]
	\centering
	\caption{Decomposição STL aditiva dos dados coletados}
	\label{fig:stl-aditiva}
	\includegraphics[width=0.9\linewidth]{"Resultados/Figuras/STL aditiva"}
	
	\fonte{Elaboração própria a partir de dados da SANEPAR (2018 a 2020)}
\end{figure}


\begin{figure}[H]
	\centering
	\caption{Decomposição STL multiplicativa dos dados coletados}
	\label{fig:stl}
	\includegraphics[width=0.9\linewidth]{Resultados/Figuras/STL}
	
	\fonte{Elaboração própria a partir de dados da SANEPAR (2018 a 2020)}
\end{figure}

Na resposta à pergunta \ref{q5}\ref{q5:b}, as Figuras \ref{fig:stl-aditiva} e \ref{fig:stl} fornecem informações sobre a presença de tendência, sazonalidade e resíduos na série temporal.

Através da decomposição, é possível analisar se a série apresenta tendência, sazonalidade e resíduos. Ao observar as Figuras \ref{fig:stl-aditiva} e \ref{fig:stl}, é evidente que os dados exibem ambos os padrões. Isso indica que a série é estacionária, como confirmado pelo seguinte teste.

Teste de Dickey-Fuller (DF) Aumentado:
\begin{itemize}
	\item Estatística de teste ADF: $-4.248$
	\item Valor de p: $0.001$
	\item Atrasos utilizados: $21.000$
	\item Observações: $1074.000$
	\item Valor crítico (1\%): $-3.436$
	\item Valor crítico (5\%): $-2.864$
	\item Valor crítico (10\%): $-2.568$
\end{itemize}

Com base na forte evidência contra a hipótese nula, podemos rejeitar a hipótese nula. Isso indica que os dados não possuem raiz unitária e são estacionários em \ref{q5}\ref{q5:c}. Identificar as horas de pico entre 18h e 21h não é uma tarefa fácil. No entanto, ao observar a Figura \ref{fig:hist}, podemos notar um aumento na demanda durante essas horas durante o ano de 2020.
	
	
	\begin{figure}[H]
		\centering
		\caption{Violino no nível do reservatório}
		\label{fig:hist}
		\includegraphics[width=0.9\linewidth]{Resultados/Figuras/viol}
		
		\fonte{Elaboração própria a partir de dados da SANEPAR (2018 a 2020)}
	\end{figure}
	
Conforme mencionado na subseção \ref{subsubsec:motivacao}, as anomalias climáticas ocorridas em 2020, especialmente a falta de chuvas, tiveram um impacto significativo nos resultados. Isso contribuiu para as mudanças observadas na demanda de água ao longo desse período.

Com relação à pergunta \ref{q5}\ref{q5:d}, durante as horas de pico, é necessário que o nível do tanque esteja dentro da faixa de $[3.545,4.256] m^3$ para evitar o acionamento das bombas. Manter o nível do tanque dentro dessa faixa permitirá que o sistema opere de forma eficiente, atendendo à demanda sem a necessidade de acionar as bombas.
	
	
	\begin{figure}[H]
		\centering
		\caption{Violino da vazão de recalque}
		\label{fig:ft03}
		\includegraphics[width=0.9\linewidth]{Resultados/Figuras/ft03}
		
		\fonte{Elaboração própria a partir de dados da SANEPAR (2018 a 2020)}
	\end{figure}
	
Para responder à pergunta \ref{q5}\ref{q5:e}, a Figura \ref{fig:ft03} ilustra como a vazão pode ser afetada pelo nível do tanque. É interessante observar que a vazão de recalque tem um impacto mais significativo no nível do tanque em comparação com as outras vazões. Isso ocorre porque a vazão de recalque está associada à injeção de água diretamente no tanque por meio da bomba localizada próxima à base do tanque. Por outro lado, as demais vazões apresentam alguns valores ausentes, o que limita sua influência na análise geral.	
	


De acordo com o \citeonline{Reisen2017115}, o teste DF tem as seguintes equações

\begin{eqnarray}
	z_t&=& y_t+\theta \beta_t, \qquad t=1,\ldots, T, \label{eq:df3}\\	
	\hat{\rho}_{\mathrm{DF}}-1&=&\frac{\sum_{t=1}^T z_{t-1} \Delta z_t}{\sum_{t=1}^T z_{t-1}^2} \label{eq:regdf}
\end{eqnarray}

De \eqref{eq:regdf} onde $\Delta z_t=z_t-z_{t-1}$. Sob a hipótese nula $\left(H_0\right)$ : `` $\rho=1$'', as estatísticas do teste DF e suas distribuições limitantes são dadas da seguinte forma:


\begin{eqnarray}
	T\left(\hat{\rho}_{\mathrm{DF}}-1\right)=T \frac{\sum_{t=1}^T z_{t-1} \Delta z_t}{\sum_{t=1}^T z_{t-1}^2}
\end{eqnarray}
e


\begin{eqnarray}
	\hat{\tau}_{\mathrm{DF}}&=&\frac{\hat{\rho}_{\mathrm{DF}}-1}{\hat{\sigma}_{\mathrm{DF}}\left(\sum_{t=1}^T z_{t-1}^2\right)^{-1 / 2}} \label{eq:df}
\end{eqnarray}

De \eqref{eq:df} onde $\hat{\sigma}_{\mathrm{DF}}^2=T^{-1} \sum_{t=1}^T\left(\Delta z_t-\left(\hat{\rho}_{\mathrm{DF}}-1\right) z_{t-1}\right)^2 .$



Suponha que $\left(z_t\right)_{1 \leq t \leq T}$ são dadas por \eqref{eq:df3}, então quando $\rho=1$,


\begin{eqnarray}
	T\left(\hat{\rho}_{\mathrm{DF}}-1\right) \stackrel{d}{\longrightarrow} \frac{W(1)^2-1}{2 \int_0^1 W(r)^2 \mathrm{~d} r}-\left(\frac{\theta}{\sigma}\right)^2 \frac{\pi}{\int_0^1 W(r)^2 \mathrm{~d} r}, \text { como } T \rightarrow \infty \\
	\hat{\tau}_{\mathrm{DF}} \stackrel{d}{\longrightarrow}\left[1+2(\theta / \sigma)^2 \pi\right]^{-1 / 2}\left\{\frac{W(1)^2-1}{2\left(\int_0^1 W(r)^2 \mathrm{~d} r\right)^{1 / 2}}-\frac{(\theta / \sigma)^2 \pi}{\left(\int_0^1 W(r)^2 \mathrm{~d} r\right)^{1 / 2}}\right\} \\
	\quad \operatorname{como} T \rightarrow \infty\label{eq:df2}
\end{eqnarray}

A partir de \eqref{eq:df2}, onde$\stackrel{d}{\longrightarrow}$ denota convergência na distribuição e onde $\{W(r), r \in[0,1]\}$ denota o movimento Browniano padrão.

O ACF (do inglês \textit{Auto-Correlation Function}) é uma medida estatística utilizada para identificar a presença de correlação serial em uma série temporal. Ele calcula a autocorrelação entre os valores da série em diferentes defasagens, ou seja, a correlação entre os valores atuais e os valores passados da série. 

O ACF é útil para analisar a dependência temporal dos dados e identificar padrões de sazonalidade, tendência ou outros efeitos temporais. Através do ACF, é possível avaliar se a série exibe autocorrelação significativa em defasagens específicas, o que pode indicar a presença de não estacionariedade ou estrutura temporal que precisa ser considerada na análise ou modelagem da série temporal.

\begin{figure}[H]
	\centering
	\caption{Autocorrelação e Autocorrelação parcial}
	\label{fig:acf}
	\includegraphics[width=0.9\linewidth]{Resultados/Figuras/acf} 
	
\end{figure}	
\begin{figure}[H]
	\centering
	\includegraphics[width=0.9\linewidth]{Resultados/Figuras/pacf}
	
	\fonte{Elaboração própria a partir de dados da SANEPAR (2018 a 2020)}
\end{figure}

Na Figura \ref{fig:acf}, é possível observar a diferença entre a autocorrelação e a autocorrelação parcial (PACF). A autocorrelação mede a correlação entre os valores da série temporal em diferentes defasagens, levando em consideração tanto a correlação direta quanto a correlação indireta. Por outro lado, a autocorrelação parcial mede apenas a correlação direta entre os valores, eliminando a influência das defasagens intermediárias.

O intervalo de confiança padrão de 95\% é representado pela marca azul na Figura. As observações que estão fora desse intervalo são consideradas estatisticamente correlacionadas, indicando a presença de padrões ou estrutura na série temporal.

A correlação visualizada na Figura \ref{fig:acf} é fundamental para a interpretação do teste DF. Em uma série de ruído branco, os valores são completamente aleatórios e não apresentam correlação significativa. Portanto, quando há correlação presente na série, isso indica a existência de padrões ou dependências entre os valores, o que pode ser explorado para a modelagem e previsão da série temporal.

\begin{figure}[htp!]
	\centering
	\caption{Ruído branco}
	\label{fig:ruido-branco}
	\includegraphics[width=0.9\linewidth]{Resultados/Figuras/ruido-branco}
	
	\fonte{Elaboração própria a partir de dados da SANEPAR (2018 a 2020)}
\end{figure}

Na Figura \ref{fig:ruido-branco}, é possível observar uma série temporal que pode ser caracterizada como ruído branco. Uma série temporal é considerada ruído branco se suas variáveis forem independentes e distribuídas de forma idêntica, com média zero. Isso implica que todas as variáveis possuem a mesma variância ($\sigma^2$) e que cada valor não possui correlação com os demais valores da série.

Além disso, é importante destacar o comprimento dos zeros na variável prevista, o que conclui a etapa \ref{etp:3}.

\subsubsection{Separa\c c\~ao dos dados}\label{subsubsec:divisao}

Na \ref{etp:4} tem um esquema de como os dados foram divididos em treinamento, teste e validação, esta prática é comum para profissionais de aprendizagem de máquinas porque além de não poder processar os dados de uma só vez se lidar com dados em uma escala menor eles podem até rodar, mas tudo depende da máquina que está rodando o processamento de dados cada modelo particular usa uma certa coleção do seu computador para processar se, Por exemplo, você está trabalhando com um modelo de aprendizado profundo que é mais comum no processamento de imagens Nvidia sempre inovou com suas GPUs e trazendo mais poder ao processamento, com o recente lançamento da placa de vídeo de $4090$ um sonho do consumidor de jogos e de profissionais de aprendizado profundo e de máquinas.

Em resumo, se o computador que foi realizado o processamento era um computador não tão bom, você ainda pode estar pensando que ele estaria processando sem a inovação que foi estabelecida ao longo dos anos, o computador que foi realizado os cálculos dos modelos era em partes um processador de computador $i5-3330 $ e um notebook com $i7-5500 $ ambos com 4 fios (em português: fio de execução ou encadeamento de execução) e o notebook com apenas 2 núcleos o $i5 $ contém 4 núcleos. Cada um deles tem suas especificações para ser o melhor em algum momento, mas sabendo que não é preciso a última geração para fazer tal processamento. É a vontade de compreender e aplicar cada um deles.

A divisão mais básica que você tem na literatura foi realizada aqui separando os dados de $70\%$ para treinamento e os dados restantes de $30\%$ para testes os dados de $70\%$ têm uma divisão adicional que leva $80\%$ dos dados de $70\%$ para treinamento novamente e os dados de $20\%$ para validação tendo esta fórmula aplicada em linguagem de programação para que não precise ser contada toda vez que o modelo for alterado.


\textbf{Modelagem e sele\c c\~ao do modelo}
a estratégia recursiva é mencionada por \citeonline{PETROPOULOS2022705} como uma abordagem eficaz na previsão de séries temporais de múltiplos passos. De acordo com o autor, essa estratégia envolve o uso de previsões anteriores como entradas para prever os próximos passos da série temporal. A abordagem recursiva tem demonstrado potencial para melhorar a acurácia das previsões de séries temporais de longo prazo.

Na Etapa \ref{etp:5}, discute-se a previsão dos dados em uma janela de horizonte de previsão estendida, abrangendo diferentes períodos de tempo, como uma hora, seis horas, doze horas e um dia. Essa estratégia de previsão recorrente permite a comparação entre modelos de regressão e modelos ARIMA em diferentes horizontes temporais.

Essa abordagem é vantajosa, pois cada modelo possui suas próprias características e desempenho ao lidar com previsões de curto prazo, como um dia, e previsões de prazo mais longo, como um dia. Ao utilizar uma janela de previsão mais ampla, é possível observar e avaliar melhor as diferenças entre os modelos e analisar seu desempenho em horizontes de tempo variados.

Além desses modelos, vários outros foram implementados no documento, tais como DTR, RFR, XGBRegressor, LGBMRegressor, LSTM, GRU, Prophet, RNN, Transformer, CNN e ANN, a fim de obter o melhor resultado para a previsão de séries temporais de abastecimento de água.

\subsubsection{Horizonte}

Na \ref{etp:6}, o horizonte de previsão foi personalizado com base no método recursivo de previsão de série temporal e na previsão do nível do tanque LT01. Os passos para a previsão à frente foram 1, 7, 14 e 30 dias. Uma estratégia com uma janela menor já foi executada, mas para a comparação dos modelos, esta janela foi mais adequada.

\subsubsection{Previs\~ao e Avalia\c c\~ao}\label{subsubsec:modelos}

A partir da etapa \ref{etp:7}, foram empregadas três métricas amplamente utilizadas na literatura para avaliar e comparar os modelos ARIMA e os modelos de regressão. Essas métricas foram detalhadas na seção \ref{subsec:metrica}.

Ao analisar os modelos desenvolvidos, observou-se que o modelo DTR obteve o melhor desempenho tanto para previsões de curto prazo, com uma janela de modelagem de 24 horas, quanto durante as horas de pico, que ocorrem entre 18h e 21h. Além disso, os modelos MA, AR, SARIMA, ARIMA, SARIMAX, ARIMAX, ARX, LGBMRegressor, XGBRegressor, RFR, RNN, ANN, CNN, GRU, LSTM, Prophet e Transformer também apresentaram resultados satisfatórios, seguindo uma ordem decrescente de desempenho.

No contexto de previsões de longo prazo, como nos casos de 30 dias, procedeu-se à avaliação dos modelos ARMA, AR, MA, ARIMA, ARIMAX, ARX, SARIMA, SARIMA, XGBRegressor, RFR, LGBMRegressor, DTR, RNN, ANN, CNN, GRU, LSTM, Prophet e Transformer. Mais uma vez, observou-se que os modelos que incorporam variáveis exógenas parecem apresentar uma capacidade superior de previsão em relação aos demais modelos. Essa tendência é claramente evidenciada nas Figuras de \ref{fig:1-AR-ARX-MA24} a \ref{fig:prophet} e nas Tabelas de \ref{tb:apd-trn} a \ref{tb:apd-int}, onde os valores menores estão destacados em \textbf{negrito} para facilitar a análise. Vale destacar que o modelo de rede neural recorrente (RNN) se sobressaiu tanto nos conjuntos de treinamento quanto na avaliação geral, consolidando-se como o modelo mais eficaz nas previsões realizadas.


\subsubsection{Teste de Signific\^ancia}

Na \ref{etp:9}, o teste escolhido foi de \textit{Friedman e Nemenjy} no teste de Nemenyi precisa ser para obter a diferença entre as classificações médias (linha do meio da tabela de classificação) entre todos os classificadores (comparando pares de classificadores). Se esta diferença for maior ou igual a um CD (distância crítica), pode-se dizer que estes dois classificadores são significativamente diferentes um do outro. O CD é calculado como:


\begin{eqnarray}
	C D&=&q_\alpha \sqrt{\frac{k(k+1)}{6 N}}\label{eq:neme}
\end{eqnarray}

De \eqref{eq:neme} o termo $q_\alpha$ é obtido de ($\alpha=0,05$):

\begin{table}[H]
	\centering
	\caption{Teste Nemenyi}
	\begin{tabular}{@{}clllllllll@{}}
		\toprule
		\multicolumn{1}{l}{\textbf{Nemenyi}} & \multicolumn{1}{c}{\textbf{0}} & \multicolumn{1}{c}{\textbf{1}} & \multicolumn{1}{c}{\textbf{2}} & \multicolumn{1}{c}{\textbf{3}} & \multicolumn{1}{c}{\textbf{4}} & \multicolumn{1}{c}{\textbf{5}} & \multicolumn{1}{c}{\textbf{6}} & \multicolumn{1}{c}{\textbf{7}} & \multicolumn{1}{c}{\textbf{8}} \\ \midrule
		\textbf{0}                           & 1,000                          & 0,001                          & 0,001                          & 0,001                          & 0,001                          & 0,001                          & 0,001                          & 0,001                          & 0,001                          \\
		\textbf{1}                           & 0,001                          & 1,000                          & 0,001                          & 0,001                          & 0,001                          & 0,001                          & 0,001                          & 0,001                          & 0,157                          \\
		\textbf{2}                           & 0,001                          & 0,001                          & 1,000                          & 0,847                          & 0,001                          & 0,001                          & 0,001                          & 0,001                          & 0,001                          \\
		\textbf{3}                           & 0,001                          & 0,001                          & 0,847                          & 1,000                          & 0,001                          & 0,001                          & 0,001                          & 0,001                          & 0,001                          \\
		\textbf{4}                           & 0,001                          & 0,001                          & 0,001                          & 0,001                          & 1,000                          & 0,001                          & 0,001                          & 0,001                          & 0,001                          \\
		\textbf{5}                           & 0,001                          & 0,001                          & 0,001                          & 0,001                          & 0,001                          & 1,000                          & 0,001                          & 0,001                          & 0,001                          \\
		\textbf{6}                           & 0,001                          & 0,001                          & 0,001                          & 0,001                          & 0,001                          & 0,001                          & 1,000                          & 0,001                          & 0,001                          \\
		\textbf{7}                           & 0,001                          & 0,001                          & 0,001                          & 0,001                          & 0,001                          & 0,001                          & 0,001                          & 1,000                          & 0,001                          \\
		\textbf{8}                           & 0,001                          & 0,157                          & 0,001                          & 0,001                          & 0,001                          & 0,001                          & 0,001                          & 0,001                          & 1,000                          \\ \bottomrule
	\end{tabular}
	
	Fonte: Elaboração própria a partir de dados da SANEPAR (2018 a 2020)
\end{table}

O teste de Nemenyi (Nemenyi, 1963) é um teste \textit{post-hoc}, ou seja, é um teste de comparação múltipla que é usado após a aplicação de teste não paramétricos com três ou mais fatores.

Para calcular a estatística de teste $F_r$ de Friedman cria-se inicialmente uma tabela com os dados, colocando-se em cada linha uma amostra e cada coluna correspondendo a uma condição de teste. A seguir, as amostras ao longo das condições são ordenadas, da melhor situação para a pior. Se não houver empates, usa-se a equação \eqref{eq:fr} para determinar a estatística de teste $F_r$:

\begin{eqnarray}
	F_r&=&\left[\frac{12}{n k(k+1)} \sum_{i=1}^k R_i{ }^2\right]-3 n(k+1)\label{eq:fr}
\end{eqnarray}

Na equação \eqref{eq:fr} $n$ é o número de linhas (ou amostras) $k$ é o número de colunas (ou condições) e $R_i$ é a soma das fileiras da coluna (ou condição) $i$.   
Seguindo a equação \eqref{eq:fr} tem o seguinte resultado nos dados da pesquisa.



$statistic=8015.611,\ \ pvalue=0.0$ com o números de 26306 linhas x 9 colunas.


\subsubsection{Compara\c c\~ao dos modelos}

A fim de ver melhor como cada modelo se comporta, os modelos foram comparados com base em um gráfico de violino, e assim observar qual dos modelos era o melhor.


\begin{figure}[H]
	\centering
	\caption{Comparação dos modelos ARIMAS}
	\includegraphics[width=0.9\linewidth]{Resultados/Figuras/modelos-arima}
	
	\label{fig:modelos-arima}
	
	Fonte: Elaboração própria a partir de dados da SANEPAR (2018 a 2020)
\end{figure}


\begin{figure}[H]
	\centering
	\caption{Comparação de modelos de regressão }
	\includegraphics[width=0.9\linewidth]{Resultados/Figuras/violin-LR-XGB-LGBM-RF}
	
	\label{fig:violin-lr-xgb-lgbm-rf}
	
	Fonte: Elaboração própria a partir de dados da SANEPAR (2018 a 2020)
\end{figure}

Em comparação com os modelos apresentadosnas Figuras \ref{fig:modelos-arima} e \ref{fig:violin-lr-xgb-lgbm-rf} os modelos que podem ser observados que são os melhores levando em conta a modelagem dos dados nos modelos ARIMA os melhores são AR, ARX, MA, ARMA, ARIMAX e SARIMAX devido aos \textit{outliers} e ao limite inferior de alguns modelos que olham para os modelos de gradiente e regressão pode-se notar que eles eram semelhantes devido às técnicas de otimização matemática Grid Search (do inglês pesquisa grande) e Randomized Search (do inglês pesquisa aleatória) que permitiram o aperfeiçoamento do método utilizado. Em um horizonte de previsão pequeno, LR prevê melhor que os outros modelos, mas em um horizonte de previsão maior, XGBoost e Light GBM estão prevendo com melhor precisão. A floresta aleatória também está prevendo com precisão apenas atrás do XGBoost em previsões de longo prazo.

O método Ljung box é um método que pode ser estimado nos modelos ARIMAS de longo prazo se, a longo prazo, eles ainda irão prever eficientemente nos dados de longo prazo os modelos que melhor prevêem são os modelos ARX, ARIMAX e SARIMAX com as variáveis exógenas para modelos não lineares que podem aguentar mais tempo de previsão do que os outros modelos ARIMA.  
  

\subsection{Estudo de Caso Emp\'irico Resultado}\label{subsec:estudo-reslt}


A previsão da demanda de água é uma preocupação fundamental para muitas organizações e autoridades responsáveis pelo abastecimento de água. Neste estudo de caso, explorou-se como a análise de séries temporais pode ser aplicada para prever a demanda de água ao longo do tempo.

A análise de séries temporais é uma abordagem comumente utilizada para prever padrões futuros com base em dados históricos. No estudo, foram aplicadas técnicas de modelagem e previsão, permitindo obter insights valiosos sobre a demanda de água futura. Diversos modelos, como ARIMA e SARIMA, foram empregados para analisar os dados históricos e gerar previsões confiáveis.

Ao longo do estudo, identificaram-se sazonalidades na demanda de água, bem como padrões de consumo que variam ao longo do tempo. Essas informações são essenciais para o planejamento adequado do abastecimento de água, permitindo uma alocação eficiente dos recursos e uma resposta adequada às flutuações de demanda.

A aplicação da análise de séries temporais na previsão da demanda de água proporciona uma base sólida para a tomada de decisões informadas. Com base nos resultados obtidos, é possível ajustar estratégias de gerenciamento, antecipar picos de demanda e otimizar o uso dos recursos hídricos disponíveis.

Em suma, este estudo demonstrou que a análise de séries temporais é uma abordagem eficaz para prever a demanda de água ao longo do tempo. Ao fornecer insights precisos e confiáveis, essa técnica contribui para o planejamento e o gerenciamento eficiente do abastecimento de água, promovendo a sustentabilidade e a utilização racional dos recursos hídricos.



\subsubsection{Descri\c c\~ao do sistema de abastecimento de \'agua}



Neste estudo, foram realizadas análises e modelagens utilizando a abordagem de séries temporais para prever a demanda diária de água em uma determinada cidade para os próximos seis meses. Os resultados obtidos forneceram insights valiosos sobre a demanda futura e contribuíram para um melhor planejamento do abastecimento hídrico. A seguir, apresentam-se as principais conclusões para cada uma das perguntas de pesquisa:

\ref{q1}: Qual é a adequação da pressão atual para atender à demanda diária?

Após análise dos dados e das métricas utilizadas, conclui-se que a pressão atual é adequada para atender à demanda diária. Durante o período analisado, não foram identificadas situações de pressão insuficiente que afetassem o fornecimento de água.

\ref{q2}: Qual é o volume mínimo de água necessário no reservatório para evitar o acionamento das bombas durante o horário de pico?

Com base na frequência de funcionamento das bombas e na demanda durante o horário de pico, determinou-se que é necessário manter um volume mínimo de água no reservatório, correspondente a 5285,90 litros, para evitar o acionamento das bombas nesse período.

\ref{q3}: Qual é a vazão ótima para atender à demanda diária?

Após análise e modelagem dos dados, identificou-se que a vazão ótima para atender à demanda varia conforme o período do dia e as características sazonais. A pressão necessária para atender à demanda é de 3,60 PSI (pound-force per square inch) na sucção.

\ref{q4}: Como encontrar o ponto de equilíbrio entre a demanda e a vazão?

Após análise e modelagem dos dados, foi constatado que não existe um ponto de equilíbrio entre a demanda e a vazão no reservatório. No entanto, identificou-se um volume mínimo de reserva de 3.545 litros que permite manter um armazenamento adequado no reservatório sem a necessidade de acionar as bombas durante o período de maior custo energético.

Embora essa estimativa de volume mínimo seja importante para garantir o abastecimento contínuo durante o período de pico, é importante ressaltar que não há um equilíbrio perfeito entre a demanda e a vazão nos dados analisados. Portanto, é necessário considerar estratégias adicionais, como otimização do sistema de abastecimento e gerenciamento eficiente dos recursos hídricos, para atender de forma adequada às necessidades da população.

\ref{q5}: Qual é o impacto do acionamento das bombas durante o horário de pico?

Confirmou-se que a ativação das bombas de sucção durante o período de 18h às 21h resulta em um maior custo energético para a SANEPAR. Portanto, é recomendado evitar o acionamento das bombas durante esse período, utilizando estratégias de armazenamento e gerenciamento eficientes.

Em suma, os resultados obtidos neste estudo fornecem informações valiosas para o planejamento e gerenciamento do abastecimento de água. A abordagem de séries temporais mostrou-se eficaz na previsão da demanda futura e na identificação de estratégias para otimizar o uso dos recursos hídricos. Essas conclusões têm o potencial de contribuir para uma gestão mais eficiente e sustentável do abastecimento de água, garantindo o atendimento adequado às necessidades da população.



\subsubsection{An\'alise explorat\'oria dos dados}



Ao longo do trabalho realizado, pôde-se observar, na subseção \ref{subsec:detec}, que foi realizada uma análise gráfica do problema antes da aplicação de qualquer método. A detecção de anomalias mostrou-se desafiadora, porém não impossível de ser realizada. Essa detecção permitiu a análise da presença de sazonalidade nos dados. A decomposição STL foi utilizada para essa finalidade, conforme descrito na etapa \ref{etp:3} e detalhado na subseção \ref{subsubsec:stl}, onde são apresentadas as decomposições realizadas.

É fundamental lembrar que, durante a análise exploratória, os dados sofreram algumas alterações. Por exemplo, foi calculada a média diária em vez de ser considerado o nível horário, resultando em uma redução do conjunto de dados de 26.306 linhas para 1.096 linhas. A decomposição STL foi aplicada nos formatos aditivo e multiplicativo, e ambas as abordagens estão ilustradas nas Figuras \ref{fig:stl-aditiva} e \ref{fig:stl}, respectivamente.

Adicionalmente, na subseção \ref{subsubsec:stl}, foi realizada a verificação da estacionariedade da série. O teste de Dickey-Fuller (DF) foi empregado para auxiliar na tomada de decisões, e os resultados demonstraram que a série em análise é estacionária, conforme evidenciado pelo teste DF.

Essa análise exploratória dos dados permitiu ao pesquisador obter insights sobre os padrões e tendências presentes nas variáveis estudadas, auxiliando na compreensão do comportamento do sistema de abastecimento de água durante o período analisado.



\subsubsection{Quest\~oes de pesquisa 1 a 4}\label{subsubsec:quest-est}


As questões de pesquisa levantadas neste estudo foram cuidadosamente abordadas e respondidas ao longo da análise. A seguir, apresenta-se as respostas para cada uma das questões:

\ref{q1} Com base nos resultados obtidos, conclui-se que as pressões atuais das variáveis \textbf{PRESSÃO DE SUCÇÃO - PT01} e \textbf{PRESSÃO DE RECALQUE - PT02} são adequadas para atender à demanda diária. O percentil 10 das pressões de sucção ($3,48$ mca) indica que apenas 10\% dos valores estão abaixo desse limite, o que sugere que a pressão de sucção geralmente se mantém em níveis adequados para o funcionamento adequado do sistema. Da mesma forma, o percentil 90 das pressões de recalque ($24.02$ mca) indica que apenas 10\% dos valores estão acima desse limite, evidenciando que a pressão de recalque também se mantém dentro dos padrões necessários para atender à demanda diária.

Esses resultados indicam que as pressões de sucção e de recalque estão em conformidade com as exigências do sistema, fornecendo a pressão necessária para o adequado abastecimento de água.

\ref{q2} Com base na frequência de funcionamento das bombas e na demanda durante o horário de pico, determinou-se que é necessário manter um volume mínimo de água no reservatório, correspondente a 5285,90 litros, para evitar o acionamento das bombas nesse período.

A vazão ótima para atender à demanda diária do tanque é determinada pelas faixas de fluxo de entrada, gravidade e retorno, juntamente com as faixas de pressão de sucção e retorno. Com base nas informações fornecidas na pergunta \ref{q3}, para manter o tanque quase cheio ou sempre cheio, as seguintes faixas de vazão devem ser consideradas:

\begin{itemize}
	\item Fluxo de entrada: entre $238 \ m^3/h$ e $302 \ m^3/h$.
	\item Fluxo de gravidade: entre $126 \ m^3/h$ e $182 \ m^3/h$.
	\item Fluxo de retorno: entre $110 \ m^3/h$ e $144 \ m^3/h$.
	\item Pressão de sucção: entre $1,92 \ mca$ e $4,24 \ mca$.
	\item Pressão de retorno: entre $21 \ mca$ e $24 \ mca$.
\end{itemize}

Essas faixas de vazão e pressão garantem que a demanda diária do tanque seja atendida de forma adequada, mantendo o nível de água próximo ao máximo e garantindo a pressão necessária para o funcionamento adequado do sistema de abastecimento de água.


Para responder à pergunta \ref{q4} sobre o ponto de equilíbrio entre a demanda e a vazão, o sistema alcança o equilíbrio quando a vazão da FT01 é de 211 $m^3/h$, a vazão da FT02 é de 114 $m^3/h$, a vazão da FT03 é de 100 $m^3/h$ e o nível do tanque está em 3.545 $m^3$. Nesse ponto de equilíbrio, as bombas não precisam ser acionadas, o que indica que o sistema de abastecimento de água está em uma condição estável. Esses valores de vazão e nível do tanque permitem atender à demanda diária sem a necessidade de tomar medidas adicionais.


\subsubsection{Quest\~ao de pesquisa 5}

\ref{q5} Confirmou-se que a ativação das bombas de sucção durante o período de 18h às 21h resulta em um maior custo energético para a SANEPAR. Portanto, é recomendado evitar o acionamento das bombas durante esse período, utilizando estratégias de armazenamento e gerenciamento eficientes.

\ref{q5:a} Verificou-se que, para evitar o acionamento das bombas durante o horário de pico (18h às 21h) sem comprometer o abastecimento de água para a população, é necessário manter o nível do reservatório acima de 4.000 litros.

\ref{q5:b} Ao analisar os dados dos últimos 3 anos do Bairro Alto, identificou-se a presença de tendências sazonais e padrões de consumo de água. Essas informações são valiosas para compreender os padrões de demanda e planejar o abastecimento de forma mais eficiente.

\ref{q5:c} Observou-se que os horários de pico, nesse caso, correspondem aos períodos em que há maior consumo de água. Esses horários são críticos para o abastecimento, pois a demanda é significativamente maior, exigindo uma gestão cuidadosa dos recursos hídricos nesse intervalo de tempo. É importante monitorar e garantir que haja suprimento adequado nesses horários para atender à demanda da população.



\begin{figure}[!htb]
	\centering
	\caption{Demanda Média das Variáveis de Fluxo}
	\includegraphics[width=0.9\linewidth]{Resultados/Figuras/grafico-barras-demanda}
	
	\label{fig:grafico-barras-demanda}
	
	\fonte{Elaboração própria a partir de dados da SANEPAR (2018 a 2020)}
\end{figure}

O gráfico de barras apresentado na Figura \ref{fig:grafico-barras-demanda} mostra a demanda média das variáveis de fluxo (Vazão de Entrada-FT01, Vazão de Gravidade-FT02 e Vazão de Recalque-FT03) durante o intervalo das 18h às 21h. Cada barra representa a média da demanda para cada variável em um horário específico dentro desse intervalo. A altura de cada barra indica a magnitude da demanda média para a respectiva variável. Essa visualização permite que sejam identificados os horários em que as variáveis de fluxo apresentaram maior demanda, o que é útil para o planejamento e gerenciamento adequado do sistema.

A questão de pesquisa \ref{q5}\ref{q5:c} foi respondida através da análise dos dados, permitindo a identificação dos horários de maior demanda durante o período das 18h às 21h. A tabela a seguir apresenta os resultados para as três variáveis estudadas: vazão de entrada-FT01, vazão de gravidade-FT02 e vazão de recalque-FT03.




\begin{table}[!htb]
	\centering
	\caption{Demanda de água}\label{tb:dem}
	\begin{tabular}{@{}ccc@{}}
		\toprule
		\textbf{Variável}         & \textbf{Horário de Maior Demanda} & \textbf{Valor da Demanda} \\ \midrule
		Vazão de entrada - FT01   & 2020/10/08 21:00:00               & $383,87 m^3/h$                   \\
		Vazão de gravidade - FT02 & 2020/10/20 18:00:00               & $326,17 m^3/h$                    \\
		Vazão de recalque - FT03  & 2020/11/26 19:00:00               & $194,35 m^3/h$                    \\ \bottomrule
	\end{tabular}
	
	
	\fonte{Elaboração própria a partir de dados da SANEPAR (2018 a 2020)}
\end{table}

Os resultados destacam os horários específicos em que cada variável apresentou maior demanda dentro do intervalo das 18h às 21h, fornecendo insights importantes para o planejamento e gerenciamento adequado do sistema. A tabela \ref{tb:dem} resume essas informações.


\ref{q5}\ref{q5:d} Durante as horas de pico, é necessário que o nível do reservatório esteja mantido dentro da faixa de $[3.545, 4.256] \ m^3$ para evitar o acionamento das bombas. Manter o nível do reservatório dentro dessa faixa permitirá que o sistema opere de forma eficiente, atendendo à demanda de água sem a necessidade de acionar as bombas.

\ref{q5}\ref{q5:e} É importante destacar que a vazão de recalque exerce um impacto mais significativo no nível do reservatório em comparação com as outras vazões. Essa diferença se deve ao fato de que a vazão de recalque está diretamente relacionada à injeção de água no reservatório por meio da bomba localizada próxima à sua base. Em contraste, as demais vazões possuem alguns valores ausentes, o que limita sua influência na análise geral do sistema.



\subsubsection{Discuss\~ao geral e conclus\~oes}

Nesta seção, serão discutidos os principais resultados e conclusões deste estudo sobre a previsão da demanda de água usando a abordagem de séries temporais. Ao longo da análise e interpretação dos dados, foram identificados padrões sazonais e tendências na demanda de água, além de estratégias para otimizar o abastecimento e o gerenciamento dos recursos hídricos.

Durante a análise exploratória dos dados, observou-se que a demanda de água apresenta flutuações ao longo do tempo, com variações sazonais significativas. A decomposição STL foi uma ferramenta útil para identificar essas sazonalidades e tendências, fornecendo uma visão mais detalhada do comportamento do sistema de abastecimento de água.

Com base nas questões de pesquisa, pode-se concluir que a pressão atual do sistema é adequada para atender à demanda diária, sem ocorrência de pressão insuficiente que possa prejudicar o fornecimento de água aos consumidores. Além disso, determinou-se um volume mínimo de reserva no reservatório, levando em consideração a frequência de operação das bombas e a demanda durante o horário de pico. Essa reserva mínima visa evitar o acionamento das bombas nesse período, contribuindo para a eficiência energética e reduzindo os custos operacionais.

A análise também permitiu identificar a vazão ótima para atender à demanda diária, considerando as flutuações sazonais e as diferentes partes do dia. No entanto, observou-se que não há um equilíbrio perfeito entre a demanda e a vazão nos dados analisados. Portanto, recomenda-se explorar estratégias adicionais, como otimização do sistema de abastecimento e gerenciamento eficiente dos recursos hídricos, a fim de aprimorar ainda mais a eficiência e a sustentabilidade do abastecimento de água.

Os resultados obtidos neste estudo demonstram a aplicação efetiva da análise de séries temporais na previsão da demanda de água e na otimização do abastecimento hídrico. Eles fornecem insights valiosos para o planejamento e o gerenciamento eficiente do sistema de abastecimento de água, contribuindo para a sustentabilidade e a utilização racional dos recursos hídricos.

Ao considerar os resultados e as conclusões deste estudo, é recomendado que medidas adicionais sejam adotadas para aprimorar ainda mais a eficiência e a sustentabilidade do abastecimento de água. Isso pode envolver a implementação de estratégias de conservação de água, o desenvolvimento de fontes alternativas de abastecimento e a promoção de conscientização sobre o uso responsável da água entre os consumidores.

Em suma, este estudo fornece uma base sólida para a tomada de decisões informadas no planejamento e gerenciamento do abastecimento de água. A análise de séries temporais mostrou-se uma ferramenta eficaz para prever a demanda futura e identificar estratégias para otimizar o uso dos recursos hídricos. Essas conclusões têm o potencial de contribuir para uma gestão mais eficiente e sustentável do abastecimento de água, garantindo o atendimento adequado às necessidades da população e o cuidado com o meio ambiente.


\section{Conclus\~oes} \label{sec:conclusoes}

Na dissertação realizada, foi conduzido um estudo abrangente sobre a previsão da demanda de água por meio da análise de séries temporais. Através da análise exploratória dos dados e da aplicação da decomposição STL, foram identificados padrões sazonais e tendências na demanda de água, fornecendo insights valiosos para o planejamento e gerenciamento eficiente do sistema de abastecimento de água.

Com base nos resultados obtidos, conclui-se que a abordagem de séries temporais é uma ferramenta eficaz para prever a demanda futura de água. Os resultados também indicaram a importância de considerar as flutuações sazonais e as diferentes partes do dia ao determinar a vazão ótima e o volume mínimo de reserva no reservatório.

Apesar dos avanços alcançados nesta pesquisa, é importante ressaltar que existem algumas limitações a serem consideradas. Primeiramente, a análise foi baseada em dados históricos de demanda de água de uma única região, limitando a generalização dos resultados para outras áreas geográficas. Além disso, o estudo não levou em conta fatores externos, como mudanças climáticas ou eventos imprevistos, que podem influenciar a demanda de água.

Para pesquisas futuras, sugere-se abordar essas limitações e expandir o escopo do estudo. Uma proposta seria coletar dados de demanda de água de diferentes regiões e considerar variáveis climáticas e socioeconômicas para aprimorar a precisão das previsões. Além disso, seria interessante explorar técnicas de modelagem mais avançadas, como redes neurais artificiais ou métodos de aprendizado de máquina, a fim de melhorar ainda mais a precisão e eficiência das previsões.

Outra proposta futura seria investigar estratégias adicionais para o gerenciamento eficiente dos recursos hídricos, como a implementação de sistemas de reúso de água, a promoção de práticas de conservação e o desenvolvimento de fontes alternativas de abastecimento. Essas medidas podem contribuir para a sustentabilidade do abastecimento de água e reduzir a dependência de recursos naturais limitados.

Em resumo, esta dissertação proporcionou insights valiosos para a previsão da demanda de água e o gerenciamento eficiente do abastecimento hídrico. Apesar das limitações encontradas, as conclusões desta pesquisa fornecem uma base sólida para futuros estudos e aprimoramentos no campo da gestão dos recursos hídricos, visando garantir um abastecimento de água adequado, sustentável e resiliente às demandas futuras.

\subsection{Propostas Futuras}

Apesar dos resultados promissores evidenciados por esta pesquisa, é essencial que se reconheçam suas limitações e que se instigue a exploração de novos horizontes em pesquisas subsequentes. Uma análise mais profunda e abrangente pode ser realizada, investigando modelos de redes neurais mais avançados. Além disso, a implementação de técnicas de otimização matemática mais refinadas, como o uso do método \textit{Covariance Matrix Adaptation Evolution Strategy} (CMAES), pode ser considerada. Seria prudente incluir cuidadosamente variáveis exógenas em todos os modelos pertinentes, como o uso de variáveis climáticas e dados de precipitação do tempo.
Implementa modelos que utilizam sistemas \textit{fuzzy} para aprimorar a previsão do tanque. Usa essa previsão juntamente com modelos existentes na literatura, como a otimização \textit{Bayesian Optimization Algorithm} (BOA), que não foi abordada neste contexto.



 


