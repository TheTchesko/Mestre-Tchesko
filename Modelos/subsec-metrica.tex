\subsection{M\'etricas de Erros}\label{subsec:metrica}


A métrica de Erro Quadrático Médio (MSE) é amplamente utilizada no campo do aprendizado de máquina para avaliar a qualidade dos modelos de previsão. O MSE é calculado pela média da soma dos quadrados das diferenças entre os valores reais e os valores previstos. Sua fórmula é a seguinte:

\begin{eqnarray}
	MSE &=& \frac{1}{n} \sum_{i=1}^{n} (y_i - \hat{y}_i)^2 \label{eq:mse}
\end{eqnarray}

Nessa fórmula, $n$ representa o número de amostras, $y_i$ é o valor real correspondente à amostra $i$ e $\hat{y}_i$ é o valor previsto para a mesma amostra. O MSE é calculado como a média das diferenças ao quadrado entre os valores reais e os valores previstos.

A utilização do MSE fornece uma medida quantitativa da precisão do modelo, pois penaliza de forma mais significativa os erros maiores. Ao elevar as diferenças ao quadrado, a métrica enfatiza a importância de minimizar as discrepâncias entre os valores reais e os valores previstos. Dessa forma, quanto menor o valor do MSE, melhor é o desempenho do modelo em termos de previsão.

Portanto, o MSE é uma métrica fundamental para avaliar a qualidade dos modelos de previsão e é amplamente utilizada para comparar diferentes algoritmos e abordagens de aprendizado de máquina.

\subsubsection{Erro quadr\'atico m\'edio raiz (RMSE)}

O RMSE é uma métrica amplamente empregada na avaliação de modelos de previsão em séries temporais. Ele é calculado tomando a raiz quadrada do MSE, conforme mostrado na seguinte fórmula:

\begin{eqnarray}
	RMSE &=& \sqrt{\frac{1}{n} \sum_{i=1}^{n} (y_i - \hat{y}_i)^2} \label{eq:rmse}
\end{eqnarray}

Na equação \eqref{eq:rmse}, $n$ representa o número de amostras, $y_i$ é o valor real correspondente à amostra $i$, e $\hat{y}_i$ é o valor previsto para a mesma amostra. O RMSE fornece uma medida da dispersão média entre os valores reais e os valores previstos pelo modelo.

Uma das vantagens de utilizar o RMSE é que, ao computar a raiz quadrada, o erro passa a ter a mesma escala da variável de interesse. Isso permite uma interpretação mais fácil dos resultados, sendo que um valor baixo de RMSE indica um bom desempenho do modelo, já que o erro se aproxima de zero.

O RMSE possui algumas características positivas. Ele penaliza de forma significativa os valores discrepantes, caso seja necessário para o modelo. Além disso, o erro resultante está nas mesmas unidades da série temporal, facilitando a interpretação. O RMSE pode ser considerado uma combinação das melhores características do MSE e do Erro Absoluto Médio (MAE).

No entanto, o RMSE também apresenta algumas desvantagens. Ele tem uma interpretabilidade reduzida, uma vez que os erros ainda são elevados ao quadrado. Além disso, o RMSE é dependente da escala dos dados, o que impede sua comparação direta com modelos de séries temporais que utilizam unidades diferentes.

Apesar das limitações, o RMSE é uma métrica amplamente utilizada para avaliar modelos de previsão em séries temporais. Ele fornece uma medida da dispersão média entre os valores reais e previstos, auxiliando na compreensão do desempenho do modelo e na comparação com outras abordagens.

\subsubsection{Erro Absoluto M\'edio (MAE)}

O Erro Absoluto Médio (MAE) é amplamente utilizado como uma métrica para avaliar o desempenho de modelos de previsão. Em vez de calcular a média das diferenças entre os valores reais e previstos, o MAE calcula a média dos valores absolutos dessas diferenças, garantindo que os erros positivos e negativos não se anulem.

O MAE mede o desvio médio das previsões em relação aos valores reais e é uma métrica intuitiva e fácil de interpretar, representando a magnitude média dos erros em relação à escala dos dados. Por exemplo, um MAE de 2 significa que, em média, as previsões têm um desvio absoluto de 2 unidades em relação aos valores reais.

Uma das vantagens do MAE é a sua insensibilidade a valores extremos, pois trata os erros de forma absoluta. No entanto, como o MAE não considera a magnitude dos erros individuais, pode não refletir adequadamente a gravidade de desvios significativos em relação aos valores reais.

Para superar essa limitação, uma alternativa é o Erro Médio Absoluto Percentual (MAPE). O MAPE expressa o MAE como uma porcentagem em relação aos valores reais, proporcionando uma medida relativa de erro. Essa métrica é especialmente útil quando se deseja avaliar o desempenho de um modelo em relação à magnitude dos dados.

Em resumo, o MAE é uma métrica simples e fácil de interpretar, que mede o desvio médio das previsões em relação aos valores reais. O MAPE, por sua vez, fornece uma medida relativa de erro, expressa como uma porcentagem dos valores reais. A escolha entre essas métricas depende do contexto do problema e dos requisitos específicos de avaliação.

O cálculo do MAE é realizado utilizando o valor absoluto da diferença entre o valor real e o valor previsto, e em seguida, divide-se pela quantidade $n$ de amostras. Isso resulta no erro médio absoluto. A equação do MAE é dada por:

\begin{eqnarray}
	M A E &=& \dfrac{1}{n} \sum\left|y_i-\hat{y}_i\right|\label{eq:mae}
\end{eqnarray}

Sua interpretação é similar ao RMSE, em que o erro é expresso na mesma escala ou ordem de grandeza da variável estudada.

\subsubsection{Erro Percentual Absoluto M\'edio (MAPE)}

Com certeza! Aqui está uma versão aprimorada do texto:

O MAPE é uma métrica que expressa o erro de previsão como uma porcentagem relativa ao valor observado. Ele é calculado somando as diferenças entre o valor real e o valor previsto (representando o erro), dividido pelo valor observado.

O MAPE é calculado usando a seguinte fórmula:

\begin{eqnarray}
	MAPE &=& \dfrac{1}{n} \sum\left|\frac{y_i - \hat{y}_i}{y_i}\right|\label{eq:mape}
\end{eqnarray}

No entanto, surge um problema quando o valor observado $y_i$ é igual a zero, pois é matematicamente impossível dividir por zero. O MAPE é uma medida de erro em que valores menores indicam um melhor desempenho de previsão.

Uma alternativa ao MAPE é calcular $1 - \textbf{MAPE}$, que representa a porcentagem de acerto.

O Erro Médio Percentual Absoluto é a diferença percentual entre o valor real e o valor previsto. É comumente usado como uma métrica de referência para avaliar o desempenho de modelos de previsão.

Prós:

\begin{itemize}
	\item Fácil de interpretar
	\item Independente de escala, permitindo comparações entre diferentes séries temporais
\end{itemize}

Contras:

\begin{itemize}
	\item Erro infinito se o valor real estiver próximo ou igual a zero
	\item Previsões mais baixas estão propensas a ter um erro de 100\%, enquanto previsões mais altas podem ter um erro infinito, o que resulta em um viés de subprevisão.
\end{itemize}