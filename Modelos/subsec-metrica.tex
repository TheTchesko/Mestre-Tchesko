\subsection{M\'etricas de Avalia\c c\~ao de Modelos}\label{subsec:metrica}

A métrica de Erro Quadrático Médio (MSE) é amplamente utilizada no campo do aprendizado de máquina para avaliar a qualidade dos modelos de previsão. O MSE é calculado pela média da soma dos quadrados das diferenças entre os valores reais e os valores previstos,

\begin{eqnarray}
	MSE &=& \frac{1}{n} \sum_{i=1}^{n} (y_i - \hat{y}_i)^2 \label{eq:mse}
\end{eqnarray}

\noindent onde, $n$ representa o número de amostras, $y_i$ é o valor real correspondente à amostra $i$ e $\hat{y}_i$ é o valor previsto para a mesma amostra. O MSE é calculado como a média das diferenças ao quadrado entre os valores reais e os valores previstos.

O MSE também é referido como uma perda quadrática porque a penalização é quadrada e não diretamente proporcional ao erro. Os \textit{outliers} recebem mais peso quando o erro é elevado ao quadrado, criando um gradiente suave para erros menores. Os algoritmos de otimização beneficiam desta penalização para erros enormes, uma vez que ajuda a obter os valores ideais para os parâmetros. Dado que os erros são elevados ao quadrado, o MSE nunca pode ser negativo, e o valor do erro pode estar em qualquer lugar entre 0 e infinito. Com erros crescentes, o MSE cresce exponencialmente, e o valor do MSE de um bom modelo será próximo de zero \cite{jadon2022comprehensive}.

\subsubsection{Erro Quadr\'atico M\'edio Raiz (RMSE)}

O RMSE é uma métrica amplamente empregada na avaliação de modelos de previsão em séries temporais. Ele é calculado tomando a raiz quadrada do MSE, conforme segue,

\begin{eqnarray}
	RMSE &=& \sqrt{\dfrac{1}{n} \sum_{i=1}^{n} (y_i - \hat{y}_i)^2} \label{eq:rmse}
\end{eqnarray}

\noindent onde \eqref{eq:rmse}, $n$ representa o número de amostras, $y_i$ é o valor real correspondente à amostra $i$, e $\hat{y}_i$ é o valor previsto para a mesma amostra. O RMSE fornece uma medida da dispersão média entre os valores reais e os valores previstos pelo modelo.

Uma das vantagens de utilizar o RMSE é que, ao computar a raiz quadrada, o erro passa a ter a mesma escala da variável de interesse. Isso permite uma interpretação mais fácil dos resultados, sendo que um valor baixo de RMSE indica um bom desempenho do modelo, já que o erro se aproxima de zero.

O RMSE possui algumas características positivas. Ele penaliza de forma significativa os valores discrepantes, caso seja necessário para o modelo. Além disso, o erro resultante está nas mesmas unidades da série temporal, facilitando a interpretação. O RMSE pode ser considerado uma combinação das melhores características do MSE e do Erro Absoluto Médio (MAE).

O RMSE também apresenta algumas desvantagens. Ele tem uma interpretabilidade reduzida, uma vez que os erros ainda são elevados ao quadrado. Além disso, o RMSE é dependente da escala dos dados, o que impede sua comparação direta com modelos de séries temporais que utilizam unidades diferentes.

Apesar das limitações, o RMSE é uma métrica amplamente utilizada para avaliar modelos de previsão em séries temporais. Ele fornece uma medida da dispersão média entre os valores reais e previstos, auxiliando na compreensão do desempenho do modelo e na comparação com outras abordagens.

\subsubsection{Raiz do Erro M\'edio Quadr\'atico Relativo (RRMSE)}\label{subsub:rrmse}

O RRMSE é uma variante do RMSE sem dimensões. O erro quadrático médio (RMSE) é uma medida de erro quadrático médio relativo que foi escalado em relação ao valor real e depois normalizado pelo valor da raiz quadrada média. Enquanto as medidas originais medidas originais restringem o RMSE, o RRMSE pode ser usado para  comparar várias abordagens de medição. Um RRMSE acontece quando as suas previsões se revelam erradas, e o erro é expresso pelo RRMSE de forma relativa ou em percentagem. RRMSE A exatidão do modelo é excelente quando a pontuação do modelo é inferior a 10\%, boa quando a pontuação do modelo se situa entre 10\% e 20\%, razoável quando a pontuação do modelo está entre 20\% e 30\%, e má quando a pontuação do modelo é superior a 30\% O RRMSE pode ser expresso como,

\begin{eqnarray}
	R R M S E&=&\sqrt{\frac{\frac{1}{N} \sum_{i=1}^N\left(y_i-\hat{y}_i\right)^2}{\sum_{i=1}^N\left(\hat{y}_i\right)^2}}
\end{eqnarray}

\noindent onde, $N$ é o número de amostras de dados, $y_i$ é o valor real, e $\hat{y}_i$ é o valor previsto.

\noindent\textbf{Vantagens do RRMSE :}


 O RRMSE pode ser utilizado para comparar diferentes técnicas de medição
 técnicas de medição \cite{jadon2022comprehensive}.


\noindent\textbf{Desvantagens do RRMSE:}


O RRMSE pode esconder a imprecisão nos resultados da experiência \cite{jadon2022comprehensive}.


É essencial que sejam consideradas essas vantagens e desvantagens ao utilizar o RRMSE como métrica de avaliação. É recomendado o uso de várias métricas em conjunto para obter uma visão mais completa do desempenho do modelo de regressão.



\subsubsection{Erro Absoluto M\'edio (MAE)}

O Erro Absoluto Médio (MAE) é amplamente utilizado como uma métrica para avaliar o desempenho de modelos de previsão. Em vez de calcular a média das diferenças entre os valores reais e previstos, o MAE calcula a média dos valores absolutos dessas diferenças, garantindo que os erros positivos e negativos não se anulem.

O MAE mede o desvio médio das previsões em relação aos valores reais e é uma métrica intuitiva e fácil de interpretar, representando a magnitude média dos erros em relação à escala dos dados. Por exemplo, um MAE de 2 significa que, em média, as previsões têm um desvio absoluto de 2 unidades em relação aos valores reais.

Uma das vantagens do MAE é a sua insensibilidade a valores extremos, pois trata os erros de forma absoluta. No entanto, como o MAE não considera a magnitude dos erros individuais, pode não refletir adequadamente a gravidade de desvios significativos em relação aos valores reais.

Para superar essa limitação, uma alternativa é o Erro Médio Absoluto Percentual (MAPE). O MAPE expressa o MAE como uma porcentagem em relação aos valores reais, proporcionando uma medida relativa de erro. Essa métrica é especialmente útil quando se deseja avaliar o desempenho de um modelo em relação à magnitude dos dados.


O cálculo do MAE é realizado utilizando o valor absoluto da diferença entre o valor real e o valor previsto, e em seguida, divide-se pela quantidade $n$ de amostras. Isso resulta no erro médio absoluto. A equação do MAE é dada por:

\begin{eqnarray}
	M A E &=& \dfrac{1}{n} \sum\left|y_i-\hat{y}_i\right|\label{eq:mae}
\end{eqnarray}

Sua interpretação é similar ao RMSE, em que o erro é expresso na mesma escala ou ordem de grandeza da variável estudada.

\subsubsection{Erro Percentual Absoluto M\'edio (MAPE)}

O Erro Percentual Absoluto Médio (MAPE) é uma métrica que expressa o erro de previsão como uma porcentagem relativa ao valor observado. Ele é calculado somando as diferenças entre o valor real e o valor previsto (representando o erro), dividido pelo valor observado.
O MAPE é calculado usando a seguinte fórmula:

\begin{eqnarray}
	MAPE &=& \dfrac{1}{n} \sum\left|\frac{y_i - \hat{y}_i}{y_i}\right|\label{eq:mape}
\end{eqnarray}

No entanto, surge um problema quando o valor observado $y_i$ é igual a zero, pois é matematicamente impossível dividir por zero. O MAPE é uma medida de erro em que valores menores indicam um melhor desempenho de previsão.
Uma alternativa ao MAPE é calcular $1 - \text{MAPE}$, que representa a porcentagem de acerto.
O Erro Percentual Absoluto Médio é comumente usado como uma métrica de referência para avaliar o desempenho de modelos de previsão.

\noindent\textbf{Vantagens do MAPE:}


Fácil de interpretar.
Independente de escala, permitindo comparações entre diferentes séries temporais


\noindent\textbf{Desvantagens do MAPE:}

Erro infinito se o valor real estiver próximo ou igual a zero.
Previsões mais baixas estão propensas a ter um erro de 100\%, enquanto previsões mais altas podem ter um erro infinito, o que resulta em um viés de subprevisão.
Essa métrica são amplamente utilizadas na avaliação de modelos de previsão em diferentes áreas e ajudam a quantificar a qualidade das previsões realizadas pelos modelos.

\subsubsection{Erro Percentual Absoluto M\'edio Sim\'etrico (sMAPE)}


O sMAPE (do inglês \textit{Symmetric Mean Absolute Percentage Error}), ou Erro Médio Percentual Absoluto Simétrico, é outra métrica comumente utilizada para avaliar a precisão de modelos de previsão. Aqui estão as vantagens e desvantagens do sMAPE:

\noindent\textbf{Vantagens do sMAPE:}


Interpretação intuitiva: O sMAPE é expresso como uma porcentagem, facilitando a compreensão da precisão relativa do modelo. Valores menores indicam uma melhor precisão.	
Simetria: Ao contrário do MAPE (do inglês \textit{Mean Absolute Percentage Error}), o sMAPE é simétrico em relação aos valores previstos e reais. Isso significa que ele considera igualmente as discrepâncias de subestimação e superestimação.	
Robustez contra valores nulos: O sMAPE é adequado para lidar com valores nulos nos dados, pois a divisão por zero é evitada no cálculo da métrica.


\noindent\textbf{Desvantagens do sMAPE:}


Sensibilidade a valores extremos: O sMAPE é sensível a valores extremos nos dados. Se houver valores discrepantes que não representem a tendência geral, eles podem influenciar significativamente a métrica.	
Assimetria em torno de zero: Embora o sMAPE seja simétrico em relação aos valores previstos e reais, ele não é simétrico em torno de zero. Isso pode causar interpretações inconsistentes, especialmente quando os valores reais são próximos de zero.



\begin{eqnarray}
	sMAPE &=& \dfrac{1}{n} \sum_{i=1}^{n} \dfrac{2|y_i - \hat{y}_i|}{(|y_i| + |\hat{y}_i|)} \times 100\label{eq:smape}
\end{eqnarray}


\noindent onde $y_i$ representa o valor real, $\hat{y}_i$ representa o valor previsto e $n$ é o número total de amostras.
Ao utilizar o sMAPE como métrica de avaliação, é importante considerar esses prós e contras. Além disso, recomenda-se o uso de várias métricas em conjunto para obter uma visão abrangente do desempenho do modelo de previsão.




