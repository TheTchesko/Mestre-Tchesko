\newpage
%\phantomsection
\pdfbookmark[1]{Lista de Abreviaturas e Siglas}{los}
\section*{Lista de Abreviaturas e Siglas}
%\addcontentsline{toc}{section}{Lista de Abreviaturas e Siglas}



\begin{tabular}{cp{0.65\textwidth}}
	AdaBoost & Impulso ou Estímulo Adaptativo (do inglês \textit{Adaptive Boosting}) \\
	ANN & Rede Neural Artificial (do inglês \textit{Artificial Neural Network}) \\
	AR & Auto-Regressivo\\
	ARIMA & Média Móvel Integrada Auto-Regressiva (do inglês \textit{Autoregressive Integrated Moving Average}) \\
	ARIMAX & Média Móvel Integrada Auto-Regressiva com Regressores Exógenos (do inglês \textit{Autoregressive Integrated Moving Average with Exogenous Regressors})\\
	ARMA & Média Móvel Auto-Regressiva (do inglês \textit{Autoregressive Moving Average}) \\
	ARX & Auto-Regressivo com Variável Exógena (do inglês \textit{Autoregressive with Exogenous Inputs})\\
	BrownBoost & Algoritmo de Aumento\\
	CNN & Rede Neural Convolucional (do inglês \textit{Convolutional Neural Network ou ConvNet})\\
	DBN & Rede de Crenças Profundas (do inglês \textit{Deep Belief Network}) \\
	DTR & Regressor de Árvore de Decisão (do inglês \textit{Decision tree regressor}) \\
	EFB & Pacote de Características Exclusivas (do inglês \textit{Exclusive Feature Bundling})\\
	FT & Flow Transmitter (Transmissor de Fluxo)\\
	GRU & Unidade de Recorrência Gated (do inglês \textit{Gated Recurrent Unit}) \\
	Hz & Hertz\\
	INMET & Instituto Nacional de Meteorologia\\
	LGBMRegressor & Regressão Ligth GBM\\
	Light GBM & Máquina de Impulso de Gradiente Leve (do inglês \textit{Light Gradient Boosting Machine}) \\
	LogitBoost & Técnicas de Regressão Logística\\
	LPBoost & Reforço da Programação Linear (do inglês \textit{Linear Programming Boosting}) \\
	LR & Regressão Linear (do inglês \textit{Linear Regression})\\
	LSTM & Memória de Longo Curto Prazo (do inglês \textit{Long Short-Term Memory})		
\end{tabular}

\begin{tabular}{cp{0.65\textwidth}}
	$m^3$ & Metros Cúbicos\\
	$m^3/h$ & Metros Cúbicos por Hora\\
	MA & Média Móvel (do inglês \textit{Moving Average})\\
	MadaBoost & Modificando o Sistema de Ponderação do AdaBoost\\
	MAE & Erro Médio Absoluto (do inglês \textit{Mean Absolute Error})\\
	MAPE & Erro Percentual Médio Absoluto (do inglês \textit{Mean Absolute Percentage Error})\\
	mca & Metros Coluna de Água\\
	ML & Aprendizado de Máquina (do inglês \textit{Machine Learning})\\
	mm & Milímetros\\
	MSE & Erro Médio Quadrático (do inglês \textit{Mean Squared Error})\\
	PR & Estado do Paraná\\
	RBAL & Recalque Bairro Alto\\
	RFR & Regressão de Floresta Aleatória (do inglês \textit{Random Forest Regression})\\
	RMSE & Erro de Raiz Média Quadrática (do inglês \textit{Root Mean Squared Error})\\
	RNN & Rede Neural Recorrente (do inglês \textit{Recurrent Neural Network})\\
	RRMSE & Raiz do Erro Médio Quadrático Relativo (do inglês \textit{Root of the Relative Mean Square Error})\\
	SANEPAR & Companhia de Saneamento do Paraná \\
	SARIMA & Auto-Regressivos Integrados de Médias Móveis com Sazonalidade (do inglês \textit{Seasonal Auto-Regressive Integrated Moving Averages}) \\
	SARIMAX & Média Móvel Integrada Auto-Regressiva Sazonal com Regressores Exógenos (do inglês \textit{Seasonal Auto-Regressive Integrated Moving Average with Exogenous Regressors}) \\
	sMAPE &  Erro Percentual Absoluto Médio Simétrico (do inglês \textit{Symmetric Mean Absolute Percentage Error})\\
	SVM-VAR & Máquinas de Vetor de Suporte - Vetores Auto-Regressivos\\
	TotalBoost & Impulso Total\\
	Transformer & Transformador \\
	XGBRegressor & Regressão XGBoost\\
	XGBoost & Impulso Gradiente Extremo (do inglês \textit{eXtreme Gradient Boosting})
\end{tabular}
