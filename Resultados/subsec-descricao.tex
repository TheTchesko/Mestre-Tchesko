%\subsubsection{Detec\c c\~ao de Anomalias} \label{subsec:detec}
%
%
%
%A detecção de anomalias em séries temporais representa um desafio significativo para os previsores, pois requer habilidade em identificar mudanças nos dados, mesmo quando não estão claramente evidentes. Nesse contexto, a coleta de dados realizada ao longo do tempo pela empresa SANEPAR revela anomalias mais expressivas do que inicialmente imaginado. A escassez de água que afetou a cidade de Curitiba se prolongou por um longo período de tempo, como é evidenciado pelos gráficos de linha utilizados na etapa. Esses gráficos oferecem uma representação das variações nos níveis de água ao longo do tempo, auxiliando na compreensão da extensão do problema e na necessidade de uma abordagem adequada.
%
%As Figuras \ref{fig:dados-todos} e \ref{fig:2020-a-frente} ilustram as variações e padrões observados nos dados ao longo do tempo, destacando a importância de explorá-los de maneira apropriada a fim de compreender as anomalias e embasar a tomada de decisões. Os dados coletados possuem uma dimensão de $26.306$ linhas e $9$ colunas, o período da amostragem é nos anos de 2018 à 2020 e essa ampla quantidade de dados será utilizada nos modelos descritos na subseção mencionada para que seja possível prever e analisar as anomalias evidenciadas.
%
%\begin{figure}[H]
%	\centering
%	\caption{Dados completos com uma frequência média de 24 horas}
%	\label{fig:dados-todos}
%	\includegraphics[width=0.9\linewidth]{"Introducao/Figuras/dados todos"}
%	
%	
%\end{figure}
%
%\begin{figure}[H]
%	\centering
%	\caption{Dados da SANEPAR no ano de 2020}\label{fig:2020-a-frente}
%	\includegraphics[width=0.9\linewidth]{"Introducao/Figuras/2020 a frente"}
%
%	
%	
%\end{figure}





