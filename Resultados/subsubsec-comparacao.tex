\subsubsection{Compara\c c\~ao dos modelos}

Com o objetivo de obter uma análise mais aprofundada do desempenho de cada modelo, foi realizada uma comparação por meio de um gráfico de violino. Dessa forma, pôde-se observar qual dos modelos apresentava o melhor desempenho.


\begin{figure}[H]
	\centering
	\caption{Análise comparativa dos modelos utilizando gráficos de violino}
	\begin{subfigure}{0.9\textwidth}
		\includegraphics[width=\linewidth]{Resultados/Figuras/modelos-arima}
		\caption{Comparação dos modelos ARIMA}
		\label{fig:modelos-arima}
	\end{subfigure}
	
	\begin{subfigure}{0.9\textwidth}
		\includegraphics[width=\linewidth]{Resultados/Figuras/violin-LR-XGB-LGBM-RF}
		\caption{Comparação de modelos de regressão}
		\label{fig:violin-lr-xgb-lgbm-rf}
	\end{subfigure}
	
	\fonte{Elaboração própria a partir de dados da SANEPAR (2018 a 2020)}
\end{figure}

Ao comparar os modelos apresentados nas Figuras \ref{fig:modelos-arima} e \ref{fig:violin-lr-xgb-lgbm-rf}, é possível observar quais são os modelos que se destacam, levando em consideração a modelagem dos dados. Os modelos ARIMA que mostram melhor desempenho são o AR, ARX, MA, ARMA, ARIMAX e SARIMAX, devido à sua capacidade de lidar com \textit{outliers} e limites inferiores em alguns modelos. No caso dos modelos baseados em gradientes e regressão, é perceptível que eles exibem resultados semelhantes, graças às técnicas de otimização matemática conhecidas como Grid Search e Randomized Search, que permitem aprimorar os métodos utilizados.

Quando se trata de um horizonte de previsão curto, o modelo de LR apresenta melhor desempenho em comparação com os demais. No entanto, em horizontes de previsão mais longos, os modelos XGBoost e Light GBM demonstram maior precisão. A Random Forest também é capaz de realizar previsões precisas, ficando ligeiramente atrás do XGBoost em previsões de longo prazo.

Para avaliar a eficiência dos modelos ARIMA em previsões de longo prazo, utiliza-se o método conhecido como Ljung-Box. Os modelos que mostram melhor desempenho nesse contexto são o ARX, ARIMAX e SARIMAX, os quais incorporam variáveis exógenas. Esses modelos não lineares têm capacidade de previsão mais robusta em horizontes de tempo mais distantes, em comparação com os outros modelos ARIMA.
