\subsubsection{Separa\c c\~ao dos dados}\label{subsubsec:divisao}

Na etapa \ref{etp:4}, os dados foram divididos em conjuntos de treinamento, teste e validação. Essa prática é comum entre profissionais de aprendizado de máquina, pois permite avaliar o desempenho do modelo em conjuntos de dados diferentes.

Em relação ao processamento de modelos de aprendizado profundo, é importante mencionar as inovações trazidas pela empresa Nvidia ao longo dos anos, especialmente no campo do processamento de imagens. O lançamento da placa de vídeo GeForce RTX 4090 tem sido bastante aguardado tanto por gamers quanto por profissionais que lidam com aprendizado de máquina.

No contexto do estudo, foram utilizados dois computadores para realizar os cálculos dos modelos. Um deles é equipado com um processador Intel Core i5-3330 e o outro é um notebook com um processador Intel Core i7-5500. Ambos os processadores possuem 4 threads, sendo que o notebook possui 2 núcleos físicos e o i5 possui 4 núcleos físicos. Cada processador tem suas especificações e desempenho adequados a diferentes necessidades. Vale ressaltar que não é obrigatório utilizar as últimas gerações de processadores para realizar esses processamentos, e sim compreender e aplicar corretamente os recursos disponíveis.

Quanto à divisão dos dados, foi adotada uma estratégia básica em que 70\% dos dados foram destinados ao conjunto de treinamento e os 30\% restantes foram reservados para o conjunto de teste. Dentro dos 70\% de treinamento, foi realizada uma subdivisão em que 80\% desses dados foram usados novamente para treinamento e os 20\% restantes foram utilizados para validação. Essa abordagem foi implementada em linguagem de programação para facilitar o processo e evitar a necessidade de recalculá-la a cada modificação do modelo.