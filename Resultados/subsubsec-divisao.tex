\subsubsection{Separa\c c\~ao dos dados}\label{subsubsec:divisao}

Na \ref{etp:4} tem um esquema de como os dados foram divididos em treinamento, teste e validação, esta prática é comum para profissionais de aprendizagem de máquinas porque além de não poder processar os dados de uma só vez se lidar com dados em uma escala menor eles podem até rodar, mas tudo depende da máquina que está rodando o processamento de dados cada modelo particular usa uma certa coleção do seu computador para processar se, Por exemplo, você está trabalhando com um modelo de aprendizado profundo que é mais comum no processamento de imagens Nvidia sempre inovou com suas GPUs e trazendo mais poder ao processamento, com o recente lançamento da placa de vídeo de $4090$ um sonho do consumidor de jogos e de profissionais de aprendizado profundo e de máquinas.

Em resumo, se o computador que foi realizado o processamento era um computador não tão bom, você ainda pode estar pensando que ele estaria processando sem a inovação que foi estabelecida ao longo dos anos, o computador que foi realizado os cálculos dos modelos era em partes um processador de computador $i5-3330 $ e um notebook com $i7-5500 $ ambos com 4 fios (em português: fio de execução ou encadeamento de execução) e o notebook com apenas 2 núcleos o $i5 $ contém 4 núcleos. Cada um deles tem suas especificações para ser o melhor em algum momento, mas sabendo que não é preciso a última geração para fazer tal processamento. É a vontade de compreender e aplicar cada um deles.

A divisão mais básica que você tem na literatura foi realizada aqui separando os dados de $70\%$ para treinamento e os dados restantes de $30\%$ para testes os dados de $70\%$ têm uma divisão adicional que leva $80\%$ dos dados de $70\%$ para treinamento novamente e os dados de $20\%$ para validação tendo esta fórmula aplicada em linguagem de programação para que não precise ser contada toda vez que o modelo for alterado.
