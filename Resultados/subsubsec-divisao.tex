\textbf{Separa\c c\~ao dos Dados}
na etapa \ref{etp:4}, os dados foram divididos em conjuntos de treinamento, teste e validação. Essa prática é comum entre profissionais de aprendizado de máquina, pois permite avaliar o desempenho do modelo em conjuntos de dados diferentes \cite{raschka2015practical, geron2017hands_on}.

Quanto à divisão dos dados, foi adotada uma estratégia básica em que $70\%$ dos dados foram destinados ao conjunto de treinamento e os $30\%$ restantes foram reservados para o conjunto de teste. Dentro dos $70\%$ de treinamento, foi realizada uma subdivisão em que $80\%$ desses dados foram usados novamente para treinamento e os $20\% $ restantes foram utilizados para validação. Essa abordagem foi implementada em linguagem de programação para facilitar o processo e evitar a necessidade de recalculá-la a cada modificação do modelo.