\subsubsection{An\'alise Explorat\'oria dos dados (EDA)}

A partir do passo \ref{etp:1}, foi realizado o EDA (do inglês \textit{Exploratory Data Analysis}) para processar os dados obtidos até o momento. O EDA permite responder às questões de pesquisa levantadas. Conforme mencionado por \citeonline{Yu2016}, na era dos grandes dados, é desafiador descobrir as regras, modelos analíticos e hipóteses por trás dos volumes massivos de dados caóticos, não estruturados e multimídia coletados por meio de vários canais. A análise exploratória de dados foi promovida por John Tukey como uma abordagem para explorar os dados, resumir suas principais características e formular hipóteses que possam direcionar a coleta adicional de dados e experimentos. No contexto de grandes análises de dados, várias técnicas de EDA têm sido adotadas.

Ao analisar a pergunta \ref{q1}, que relaciona a demanda com a variável prevista e a pressão para a variável PT01, pode-se observar na Figura \ref{fig:person} que ambas as variáveis apresentam uma correlação quase perfeita, com um coeficiente de correlação de Pearson ($r$) igual a 1. Portanto, para responder a essa pergunta, basta observar a correlação de Pearson na Figura \ref{fig:person}.

\begin{figure}[H]
	\centering
	\caption{Correlação de Pearson}
	\label{fig:person}
	\includegraphics[width=0.9\linewidth]{Apendices/Figuras/modelagem-24h/person}
	
	\fonte{Elaboração própria a partir de dados da SANEPAR (2018 a 2020)}
\end{figure}

A Figura \ref{fig:person} ilustra a correlação entre as variáveis no conjunto de dados em questão. Essa imagem representa graficamente a relação entre as variáveis e é usada para demonstrar a existência de uma correlação forte entre elas. Com base nessa análise, é possível responder à pergunta de pesquisa \ref{q1}, pois a correlação entre as variáveis é significativa.

Para responder à pergunta \ref{q2}, é criada uma tabela para fornecer uma resposta mais completa.


\begin{table}[H]
	\centering
	\caption{Descrição estatística dos dados com o filtro aplicado das 18h às 21h}\label{tb:est}
	\begin{tabular}{@{}cccccccccc@{}}
		\toprule
		\textbf{18 a 21h}  & \textbf{B1} & \textbf{B2} & \textbf{B3} & \textbf{LT01} & \textbf{FT01} & \textbf{FT02} & \textbf{FT03} & \textbf{PT01} & \textbf{PT02} \\ \midrule
		\textbf{Contagem} & 4385    & 4385     & 4385     & 4385      & 4385       & 4385       & 4385       & 4385       & 4385       \\
		\textbf{Média}      & 51,94       & 27,81       & 6,41        & 3,24          & 112,68        & 132,93        & 112,41        & 4,11          & 20,80         \\
		\textbf{STD}       & 17,14       & 17,61       & 16,77       & 0,70          & 132,59        & 44,78         & 31,33         & 0,76          & 6,14          \\
		\textbf{Min}       & 0           & 0           & 0           & 0,29          & 0             & 0             & 0             & 0,88          & 0             \\
		\textbf{25\%}      & 57,84       & 0           & 0           & 2,79          & 0,12          & 123,96        & 111,66        & 3,62          & 21,72         \\
		\textbf{50\%}      & 57,99       & 34,91       & 0           & 3,30          & 0,12          & 136,00        & 118,82        & 4,15          & 22,05         \\
		\textbf{75\%}      & 57,99       & 38,02       & 0           & 3,78          & 264,27        & 148,20        & 125,63        & 4,66          & 23,02         \\
		\textbf{Max}       & 59,99       & 59,99       & 59,99       & 4,40          & 383,87        & 326,17        & 194,35        & 5,68          & 28,08         \\ \bottomrule
	\end{tabular}
	
	\fonte{Elaboração própria a partir de dados da SANEPAR (2018 a 2020)}
\end{table}



Na Tabela \ref{tb:est}, o desvio padrão é representado pela sigla STD, que corresponde à expressão em inglês ``\textit{standard deviation}''. Além disso, em resposta à pergunta \ref{q2}, é importante mencionar que, assim como em qualquer empresa de tratamento de água, é utilizado um mecanismo de acionamento automático chamado "trava de segurança" para evitar que o nível do tanque chegue a zero e haja falta de água nos locais abastecidos por esse tanque. O nível mínimo que o tanque pode alcançar é de $5.29 m^3$ (equivalente a 5,29 litros). As bombas são ativadas em sua potência máxima para evitar que sejam acionadas quando o nível do tanque. No entanto, a bomba 1 ainda estaria operando para completar o nível do tanque caso ele esteja dentro dessa faixa.

Em situações de demanda de pico, uma abordagem ideal, embora não necessariamente a mais econômica, seria ter um tanque de reserva adicional e instalar uma tubulação que os conecte. Durante o dia, ambos os tanques seriam abastecidos e, à noite, por meio da ação da gravidade, eles manteriam o mesmo nível até que o consumo atinja um ponto em que as bombas sejam acionadas. Essa estratégia permite um abastecimento contínuo e eficiente de água.


Na pergunta \ref{q3}, observa-se que o tanque tem uma capacidade máxima de $4,256 m^3$, o que equivale a $4.256$ litros. Para atender a essa demanda e manter o tanque quase cheio ou sempre cheio, é necessário que o fluxo de entrada esteja na faixa de $[238, 302] \ m^3/h$, o fluxo de gravidade esteja entre $[126, 182] \ m^3/h$, o fluxo de retorno esteja entre $[110, 144] \ m^3/h$, a pressão de sucção esteja entre $[1.92, 4.24] \ mca$ e a pressão de retorno esteja entre $[21, 24] \ mca$.

Para responder à pergunta \ref{q4}, o ponto de equilíbrio, onde as bombas não precisam ser acionadas, ocorre quando o fluxo de FT01 é de $211 \ m^3/h$, FT02 é de $114 \ m^3/h$, FT03 é de $100 \ m^3/h$ e o nível do tanque está em $3.545 \ m^3$.
No que diz respeito à pergunta \ref{q5}\ref{q5:a}, o nível do tanque deve ser de $4,00 \ m^3$ para evitar o funcionamento das bombas durante as horas de pico.