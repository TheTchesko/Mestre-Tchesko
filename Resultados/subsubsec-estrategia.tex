\subsubsection{Modelagem e Sele\c c\~ao do Modelo}\label{subsubsec:est}

A estratégia recursiva é mencionada por \citeonline{PETROPOULOS2022705} como uma abordagem eficaz na previsão de séries temporais de múltiplos passos. De acordo com o autor, essa estratégia envolve o uso de previsões anteriores como entradas para prever os próximos passos da série temporal. A abordagem recursiva tem demonstrado potencial para melhorar a acurácia das previsões de séries temporais de longo prazo.

Na Etapa \ref{etp:5}, discute-se a previsão dos dados em uma janela de horizonte de previsão estendida, abrangendo diferentes períodos de tempo, como um dia, uma semana, duas semanas e um mês. Essa estratégia de previsão recorrente permite a comparação entre modelos de regressão e modelos ARIMA em diferentes horizontes temporais.

Essa abordagem é vantajosa, pois cada modelo possui suas próprias características e desempenho ao lidar com previsões de curto prazo, como um dia, e previsões de prazo mais longo, como um mês. Ao utilizar uma janela de previsão mais ampla, é possível observar e avaliar melhor as diferenças entre os modelos e analisar seu desempenho em horizontes de tempo variados.