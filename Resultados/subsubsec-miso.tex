\subsubsection{M\'ultiplas Entradas e Sa\'ida \'Unica (MISO)}

Na etapa \ref{etp:2}, foi explorado o modelo MISO (do inglês \textit{Multiple Inputs, Single Output}) na dissertação. O modelo ARIMA, juntamente com suas variantes e extensões, foi amplamente estudado durante a pesquisa, assim como modelos regressivos que envolvem múltiplas variáveis de entrada e uma variável de saída, neste caso, a LT01. As demais variáveis foram utilizadas como suporte para melhorar o modelo do tipo ARIMAX ou modelos com variáveis exógenas. Quando aplicado sem o uso de variáveis exógenas, o modelo ARIMA apresenta apenas uma entrada, semelhante ao modelo de LR. No entanto, ao incluir variáveis exógenas, o modelo se torna MISO, permitindo uma modelagem mais abrangente e considerando a interação de várias variáveis para prever a variável de interesse.