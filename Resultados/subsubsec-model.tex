\subsubsection{Modelos de previs\~ao e m\'etricas de desempenho}\label{subsubsec:modelos}

A partir da etapa \ref{etp:7}, foram utilizadas três métricas amplamente empregadas na literatura para a previsão e comparação de modelos ARIMA e modelos de regressão. Essas métricas foram detalhadas na seção \ref{subsec:metrica}.

Ao analisar os modelos desenvolvidos, foi observado que o modelo de regressão linear (LR) obteve o melhor desempenho tanto na previsão de curto prazo, considerando uma modelagem de 24 horas, quanto nas horas de pico entre 18h e 21h. Os modelos MA, AR, SARIMA, ARIMA, SARIMAX, ARIMAX, ARX, LGBMRegressor, XGBRegressor e RFR também apresentaram um desempenho satisfatório, seguindo uma ordem de melhor para pior.

Para previsões de longo prazo, como no caso dos 30 dias, foram avaliados os modelos ARMA, AR, MA, ARIMA, ARIMAX, ARX, SARIMA, SARIMA, XGBRegressor, RFR, LGBMRegressor e LR, novamente seguindo a ordem de melhor desempenho. No entanto, ao analisar os resultados graficamente nos apêndices, foi observado que os modelos que incorporam variáveis exógenas parecem ter uma capacidade de previsão superior em relação aos demais modelos. Essa tendência pode ser visualizada nas Figuras de \ref{fig:1-AR-ARX-MA24} a \ref{fig:60-ARIMAX-SARIMA-SARIMAX24} e nas Tabelas de \ref{tb:1-24trn} a \ref{tb:60-24cm}.
