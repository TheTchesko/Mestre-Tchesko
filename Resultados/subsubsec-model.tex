\subsubsection{Previs\~ao e Avalia\c c\~ao}\label{subsubsec:modelos}

A partir da etapa \ref{etp:7}, foram empregadas três métricas amplamente utilizadas na literatura para avaliar e comparar os modelos ARIMA e os modelos de regressão. Essas métricas foram detalhadas na seção \ref{subsec:metrica}.

Ao analisar os modelos desenvolvidos, observou-se que o modelo DTR obteve o melhor desempenho tanto para previsões de curto prazo, com uma janela de modelagem de 24 horas, quanto durante as horas de pico, que ocorrem entre 18h e 21h. Além disso, os modelos MA, AR, SARIMA, ARIMA, SARIMAX, ARIMAX, ARX, LGBMRegressor, XGBRegressor, RFR, RNN, ANN, CNN, GRU, LSTM, Prophet e Transformer também apresentaram resultados satisfatórios, seguindo uma ordem decrescente de desempenho.

No contexto de previsões de longo prazo, como nos casos de 30 dias, procedeu-se à avaliação dos modelos ARMA, AR, MA, ARIMA, ARIMAX, ARX, SARIMA, SARIMA, XGBRegressor, RFR, LGBMRegressor, DTR, RNN, ANN, CNN, GRU, LSTM, Prophet e Transformer. Mais uma vez, observou-se que os modelos que incorporam variáveis exógenas parecem apresentar uma capacidade superior de previsão em relação aos demais modelos. Essa tendência é claramente evidenciada nas Figuras de \ref{fig:1-AR-ARX-MA24} a \ref{fig:prophet} e nas Tabelas de \ref{tb:apd-trn} a \ref{tb:apd-int}, onde os valores menores estão destacados em \textbf{negrito} para facilitar a análise. Vale destacar que o modelo de rede neural recorrente (RNN) se sobressaiu tanto nos conjuntos de treinamento quanto na avaliação geral, consolidando-se como o modelo mais eficaz nas previsões realizadas.
