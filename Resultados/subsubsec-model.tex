\subsubsection{Modelos de previs\~ao e m\'etricas de desempenho}\label{subsubsec:modelos}

A partir da \ref{etp:7}, as métricas utilizadas aqui foram vistas na seção \ref{subsec:metrica} e três das métricas mais utilizadas na literatura para previsão e comparação dos modelos ARIMA e modelos de regressores foram utilizadas aqui.

Em comparação com os modelos feitos, pode-se ver que o modelo LR em um passo à frente tem tanto na modelagem de 24 horas como nas horas de pico entre 18 e 21 horas foi o modelo que melhor se saiu na previsão logo após os modelos MA, AR, SARIMA, ARIMA, SARIMAX, ARIMAX, ARX, LGBMRegressor, XGBRegressor e Random Forest Regressor para o curto prazo estes modelos estão em ordem do melhor para o pior.

Já em períodos mais longos, como foi feito em 30 dias os modelos ARMA, AR, MA, ARIMA, ARIMAX, ARX, SARIMA, SARIMA, XGBRegressor, Random Forest Regressor, LGBMRregressor e LR, seguindo a mesma lógica do melhor ao pior. Mas também olhando graficamente os modelos que foram feitos os modelos com variáveis exógenas parecem prever melhor do que os outros modelos apenas olhando os dados nos apêndices tanto quando os Figuras de \ref{fig:1-AR-ARX-MA24} a \ref{fig:60-ARIMAX-SARIMA-SARIMAX24} como as Tabelas \ref{tb:1-24trn} a \ref{tb:60-24cm}.   
