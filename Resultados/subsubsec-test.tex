\textbf{Teste de signific\^ancia}
na etapa \ref{etp:8}, realizou-se o teste de Friedman e o teste de Nemenyi para comparar as classificações médias entre os diversos classificadores. O teste de Nemenyi é uma ferramenta de comparação múltipla frequentemente empregada após a aplicação de testes não paramétricos com três ou mais fatores.

A matriz de comparação entre os classificadores, apresentada na Tabela \ref{tb:nemeyi}, exibe os valores de comparação múltipla de Nemenyi, onde as entradas evidenciam as diferenças significativas entre os pares de classificadores.

A Tabela \ref{tb:nemeyi} apresenta os resultados do teste de Nemenyi, um método utilizado para comparar as classificações médias entre diferentes classificadores após a aplicação de testes não paramétricos com três ou mais fatores. Cada célula da tabela mostra os valores de comparação múltipla de Nemenyi, que indicam as diferenças significativas entre os pares de classificadores. O valor na interseção da linha $i$ e da coluna $j$ representa a diferença significativa entre os classificadores $i$ e $j$.

\begin{table}[!htb]
	\centering
	\caption{Teste Nemenyi}\label{tb:nemeyi}
	\begin{tabular}{@{}clllllllll@{}}
		\toprule
		\multicolumn{1}{l}{\textbf{Nemenyi}} & \multicolumn{1}{c}{\textbf{B1}} & \multicolumn{1}{c}{\textbf{B2}} & \multicolumn{1}{c}{\textbf{B3}} & \multicolumn{1}{c}{\textbf{LT01}} & \multicolumn{1}{c}{\textbf{FT01}} & \multicolumn{1}{c}{\textbf{FT02}} & \multicolumn{1}{c}{\textbf{FT03}} & \multicolumn{1}{c}{\textbf{PT01}} & \multicolumn{1}{c}{\textbf{PT02}} \\ \midrule
		\textbf{B1}                         & 1.000                          & 0.001                          & 0.001                          & 0.001                          & 0.001                          & 0.001                          & 0.001                          & 0.001                          & 0.001                          \\
		\textbf{B2}                         & 0.001                          & 1.000                          & 0.001                          & 0.001                          & 0.001                          & 0.001                          & 0.001                          & 0.001                          & 0.001                          \\
		\textbf{B3}                         & 0.001                          & 0.001                          & 1.000                          & 0.001                          & 0.001                          & 0.001                          & 0.001                          & 0.001                          & 0.001                          \\
		\textbf{LT01}                       & 0.001                          & 0.001                          & 0.001                          & 1.000                          & 0.001                          & 0.001                          & 0.001                          & 0.001                          & 0.001                          \\
		\textbf{FT01}                       & 0.001                          & 0.001                          & 0.001                          & 0.001                          & 1.000                          & 0.001                          & 0.131                          & 0.001                          & 0.001                          \\
		\textbf{FT02}                       & 0.001                          & 0.001                          & 0.001                          & 0.001                          & 0.001                          & 1.000                          & 0.001                          & 0.001                          & 0.001                          \\
		\textbf{FT03}                       & 0.001                          & 0.001                          & 0.001                          & 0.001                          & 0.131                          & 0.001                          & 1.000                          & 0.001                          & 0.001                          \\
		\textbf{PT01}                       & 0.001                          & 0.001                          & 0.001                          & 0.001                          & 0.001                          & 0.001                          & 0.001                          & 1.000                          & 0.001                          \\
		\textbf{PT02}                       & 0.001                          & 0.001                          & 0.001                          & 0.001                          & 0.001                          & 0.001                          & 0.001                          & 0.001                          & 1.000                          \\ \bottomrule
	\end{tabular}
\end{table}



No contexto do estudo, os resultados da análise comparativa revelaram diferenças estatisticamente significativas entre vários pares de classificadores, como indicado pelas entradas da tabela. Isso sugere que pelo menos um modelo é considerado estatisticamente superior aos demais, com base nas comparações realizadas.

O valor crítico CD foi utilizado para determinar se dois classificadores eram significativamente diferentes entre si. Esse valor é calculado com base no valor crítico obtido da Tabela \ref{tb:nemeyi} de teste de Nemenyi, o número de classificadores e o número total de amostras. O valor CD é uma métrica que auxilia na interpretação das diferenças entre os classificadores, ajudando a identificar quais pares de classificadores apresentam diferenças estatisticamente significativas.

Os resultados da pesquisa indicaram a existência de evidências estatísticas que sugerem a superioridade de pelo menos um modelo em relação aos demais. Além disso, a análise de comparação significativa entre os modelos revelou pares de classificadores que apresentam diferenças estatisticamente significativas em seus desempenhos. Essas informações são valiosas para a seleção e avaliação dos modelos de classificação, permitindo uma compreensão mais precisa das diferenças de desempenho entre os classificadores avaliados no estudo. Na Tabela \ref{tab:metrics} é mostrado como cada modelos foi comparado entre si em 24 passos à frente.

\begin{table}[!htb]
	\centering
	\caption{Métricas de avaliação dos modelos}
	\label{tab:metrics}
	\small
	
	\begin{tabular}{cccc}
		\hline
		
		\textbf{Modelo} & \textbf{sMAPE} & \textbf{MAE} & \textbf{RRMSE} \\
		\hline
		Prophet & 25,67 & 0,844 & 0,975 \\
		Transformer & 16,39 & 0,544 & 0,324 \\
		ANN & 26,22 & 0,980 & 0,531 \\
		CNN & 26,22 & 0,980 & 0,531 \\
		\textbf{RNN} & \textbf{0,235} & \textbf{0,008} & \textbf{0,003} \\
		LSTM & 73,75 & 4,010 & 6,068 \\
		GRU & 103,57 & 7,443 & 2,485 \\
		AR & 6,66 & 0,428 & 0,169 \\
		ARX & 6,79 & 0,434 & 0,173 \\
		MA & 6,57 & 0,423 & 0,166 \\
		ARMA & 6,66 & 0,428 & 0,169 \\
		ARIMA & 6,60 & 0,424 & 0,167 \\
		SARIMA & 6,71 & 0,432 & 0,170 \\
		ARIMAX & 6,82 & 0,436 & 0,173 \\
		SARIMAX & 6,80 & 0,435 & 0,173 \\
		DTR & 6,77 & 0,577 & 0,158 \\
		RFR & 12,09 & 0,886 & 0,316 \\
		XGBRegressor & 12,41 & 0,916 & 0,323 \\
		LGBMRegressor & 12,21 & 0,897 & 0,319 \\
		\hline
	\end{tabular}
\end{table}




\noindent\textbf{Modelo com menor valor em cada métrica:}	
Primeiramente, os diversos modelos de previsão de séries temporais foram avaliados para um horizonte de previsão de um dia. Para cada métrica (\textbf{sMAPE}, \textbf{MAE} e \textbf{RRMSE}), identificou-se o modelo que apresentou o menor valor.
A métrica \textbf{sMAPE} apontou que o modelo \textbf{RNN} obteve o menor valor.
Quanto à métrica \textbf{MAE}, novamente o modelo \textbf{RNN} demonstrou o menor valor.
A métrica \textbf{RRMSE} também indicou que o modelo \textbf{RNN} teve o menor valor.


\noindent\textbf{Evidências estatísticas de que pelo menos um modelo é superior:}
Para validar estatisticamente as diferenças entre os modelos, foi realizado um teste estatístico denominado \textbf{Teste de Friedman}. Esse teste avalia o desempenho dos modelos em todas as métricas simultaneamente. O resultado do teste de Friedman revelou \textbf{evidências estatísticas} que pelo menos um dos modelos apresenta superioridade estatística em relação aos demais, considerando um nível de significância de $0.05$.
	
\noindent\textbf{Comparação significativa entre modelos - Teste de Nemenyi:}	
A fim de determinar quais modelos apresentam diferenças estatisticamente significativas entre si, foi conduzido o \textbf{teste de comparações múltiplas de Nemenyi}. Esse teste avalia todos os pares possíveis de modelos e identifica quais deles possuem diferenças estatisticamente significativas. Os resultados indicaram \textbf{diferenças estatisticamente significativas} entre vários pares de modelos. Especificamente:

O modelo \textbf{RNN} apresentou diferenças significativas em relação aos modelos \textbf{LSTM} e \textbf{GRU}.
O modelo \textbf{LSTM} apresentou diferenças significativas em relação ao modelo \textbf{RNN}.
O modelo \textbf{GRU} exibiu diferenças significativas em relação ao modelo \textbf{RNN}.
Com base na análise estatística de Friedman e no teste de comparações múltiplas de Nemenyi, conclui-se que o modelo \textbf{RNN} apresenta o melhor desempenho geral em relação às métricas consideradas (\textbf{sMAPE}, \textbf{MAE} e \textbf{RRMSE}) para um horizonte de previsão de um dia, utilizando os dados completos.


