\subsubsection{Teste de Signific\^ancia}

Na etapa \ref{etp:9}, foi utilizado o teste de Friedman e Nemenyi para comparar as classificações médias entre os classificadores. O teste de Nemenyi é um teste de comparação múltipla utilizado após a aplicação de testes não paramétricos com três ou mais fatores.

\begin{table}[H]
	\centering
	\caption{Teste Nemenyi}
	\begin{tabular}{@{}clllllllll@{}}
		\toprule
		\multicolumn{1}{l}{\textbf{Nemenyi}} & \multicolumn{1}{c}{\textbf{0}} & \multicolumn{1}{c}{\textbf{1}} & \multicolumn{1}{c}{\textbf{2}} & \multicolumn{1}{c}{\textbf{3}} & \multicolumn{1}{c}{\textbf{4}} & \multicolumn{1}{c}{\textbf{5}} & \multicolumn{1}{c}{\textbf{6}} & \multicolumn{1}{c}{\textbf{7}} & \multicolumn{1}{c}{\textbf{8}} \\ \midrule
		\textbf{0}                           & 1,000                          & 0,001                          & 0,001                          & 0,001                          & 0,001                          & 0,001                          & 0,001                          & 0,001                          & 0,001                          \\
		\textbf{1}                           & 0,001                          & 1,000                          & 0,001                          & 0,001                          & 0,001                          & 0,001                          & 0,001                          & 0,001                          & 0,157                          \\
		\textbf{2}                           & 0,001                          & 0,001                          & 1,000                          & 0,847                          & 0,001                          & 0,001                          & 0,001                          & 0,001                          & 0,001                          \\
		\textbf{3}                           & 0,001                          & 0,001                          & 0,847                          & 1,000                          & 0,001                          & 0,001                          & 0,001                          & 0,001                          & 0,001                          \\
		\textbf{4}                           & 0,001                          & 0,001                          & 0,001                          & 0,001                          & 1,000                          & 0,001                          & 0,001                          & 0,001                          & 0,001                          \\
		\textbf{5}                           & 0,001                          & 0,001                          & 0,001                          & 0,001                          & 0,001                          & 1,000                          & 0,001                          & 0,001                          & 0,001                          \\
		\textbf{6}                           & 0,001                          & 0,001                          & 0,001                          & 0,001                          & 0,001                          & 0,001                          & 1,000                          & 0,001                          & 0,001                          \\
		\textbf{7}                           & 0,001                          & 0,001                          & 0,001                          & 0,001                          & 0,001                          & 0,001                          & 0,001                          & 1,000                          & 0,001                          \\
		\textbf{8}                           & 0,001                          & 0,157                          & 0,001                          & 0,001                          & 0,001                          & 0,001                          & 0,001                          & 0,001                          & 1,000                          \\ \bottomrule
	\end{tabular}
	
	Fonte: Elaboração própria a partir de dados da SANEPAR (2018 a 2020)
\end{table}

Para calcular a estatística de teste $F_r$ de Friedman, inicialmente cria-se uma tabela com os dados, onde cada linha representa uma amostra e cada coluna representa uma condição de teste. Em seguida, as amostras são ordenadas ao longo das condições, da melhor situação para a pior. Se não houver empates, a estatística de teste $F_r$ é calculada utilizando a seguinte fórmula:

\begin{equation}
	F_r = \left(\frac{12}{n k(k+1)} \sum_{i=1}^k R_i^2\right) - 3n(k+1)
\end{equation}

Nessa fórmula, $n$ é o número de linhas (ou amostras), $k$ é o número de colunas (ou condições) e $R_i$ é a soma das fileiras da coluna (ou condição) $i$.

Além disso, o valor crítico CD (Critical Difference) é utilizado para determinar se dois classificadores são significativamente diferentes um do outro. O CD é calculado usando a fórmula que mencionei anteriormente:

\begin{equation}
	CD = q_\alpha \sqrt{\frac{k(k+1)}{6N}}
\end{equation}

Na fórmula do CD, $q_\alpha$ é o valor crítico obtido da tabela de teste de Nemenyi, $k$ é o número de classificadores e $N$ é o número total de amostras.

De acordo com essa equação, os resultados da pesquisa foram os seguintes:

$statistic=8015.611,\ \ p-value=0.0$ com um total de 26.306 linhas por 9 colunas.
