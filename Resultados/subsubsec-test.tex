\subsubsection{Teste de Signific\^ancia}

Na \ref{etp:9}, o teste escolhido foi de \textit{Friedman e Nemenjy} no teste de Nemenyi precisa ser para obter a diferença entre as classificações médias (linha do meio da tabela de classificação) entre todos os classificadores (comparando pares de classificadores). Se esta diferença for maior ou igual a um CD (distância crítica), pode-se dizer que estes dois classificadores são significativamente diferentes um do outro. O CD é calculado como:


\begin{eqnarray}
	C D&=&q_\alpha \sqrt{\frac{k(k+1)}{6 N}}\label{eq:neme}
\end{eqnarray}

De \eqref{eq:neme} o termo $q_\alpha$ é obtido de ($\alpha=0,05$):

\begin{table}[H]
	\centering
	\caption{Teste Nemenyi}
	\begin{tabular}{@{}clllllllll@{}}
		\toprule
		\multicolumn{1}{l}{\textbf{Nemenyi}} & \multicolumn{1}{c}{\textbf{0}} & \multicolumn{1}{c}{\textbf{1}} & \multicolumn{1}{c}{\textbf{2}} & \multicolumn{1}{c}{\textbf{3}} & \multicolumn{1}{c}{\textbf{4}} & \multicolumn{1}{c}{\textbf{5}} & \multicolumn{1}{c}{\textbf{6}} & \multicolumn{1}{c}{\textbf{7}} & \multicolumn{1}{c}{\textbf{8}} \\ \midrule
		\textbf{0}                           & 1,000                          & 0,001                          & 0,001                          & 0,001                          & 0,001                          & 0,001                          & 0,001                          & 0,001                          & 0,001                          \\
		\textbf{1}                           & 0,001                          & 1,000                          & 0,001                          & 0,001                          & 0,001                          & 0,001                          & 0,001                          & 0,001                          & 0,157                          \\
		\textbf{2}                           & 0,001                          & 0,001                          & 1,000                          & 0,847                          & 0,001                          & 0,001                          & 0,001                          & 0,001                          & 0,001                          \\
		\textbf{3}                           & 0,001                          & 0,001                          & 0,847                          & 1,000                          & 0,001                          & 0,001                          & 0,001                          & 0,001                          & 0,001                          \\
		\textbf{4}                           & 0,001                          & 0,001                          & 0,001                          & 0,001                          & 1,000                          & 0,001                          & 0,001                          & 0,001                          & 0,001                          \\
		\textbf{5}                           & 0,001                          & 0,001                          & 0,001                          & 0,001                          & 0,001                          & 1,000                          & 0,001                          & 0,001                          & 0,001                          \\
		\textbf{6}                           & 0,001                          & 0,001                          & 0,001                          & 0,001                          & 0,001                          & 0,001                          & 1,000                          & 0,001                          & 0,001                          \\
		\textbf{7}                           & 0,001                          & 0,001                          & 0,001                          & 0,001                          & 0,001                          & 0,001                          & 0,001                          & 1,000                          & 0,001                          \\
		\textbf{8}                           & 0,001                          & 0,157                          & 0,001                          & 0,001                          & 0,001                          & 0,001                          & 0,001                          & 0,001                          & 1,000                          \\ \bottomrule
	\end{tabular}
	
	Fonte: Elaboração própria a partir de dados da SANEPAR (2018 a 2020)
\end{table}

O teste de Nemenyi (Nemenyi, 1963) é um teste \textit{post-hoc}, ou seja, é um teste de comparação múltipla que é usado após a aplicação de teste não paramétricos com três ou mais fatores.

Para calcular a estatística de teste $F_r$ de Friedman cria-se inicialmente uma tabela com os dados, colocando-se em cada linha uma amostra e cada coluna correspondendo a uma condição de teste. A seguir, as amostras ao longo das condições são ordenadas, da melhor situação para a pior. Se não houver empates, usa-se a equação \eqref{eq:fr} para determinar a estatística de teste $F_r$:

\begin{eqnarray}
	F_r&=&\left[\frac{12}{n k(k+1)} \sum_{i=1}^k R_i{ }^2\right]-3 n(k+1)\label{eq:fr}
\end{eqnarray}

Na equação \eqref{eq:fr} $n$ é o número de linhas (ou amostras) $k$ é o número de colunas (ou condições) e $R_i$ é a soma das fileiras da coluna (ou condição) $i$.   
Seguindo a equação \eqref{eq:fr} tem o seguinte resultado nos dados da pesquisa.



$statistic=8015.611,\ \ pvalue=0.0$ com o números de 26306 linhas x 9 colunas.
