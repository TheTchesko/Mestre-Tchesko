{\selectlanguage{english}
\begin{abstract}
	\noindent  In a competitive scenario, accurate demand forecasting has become an increasingly strategic tool for various organizational sectors. In this context, time series forecasting plays a significant role in decision-making. Recently, the capital city of Paraná faced a severe healthcare crisis, with periods of shortages that caused serious instability in housing supply for many families.	
	In addressing this problem, the methods found through the literature review conducted in this work were used for prediction purposes. The chosen methods serve as a decision-making approach for water demand. Each selected method is capable of dealing with the problem in a different way and providing viable solutions for decision-making.	
	With the objective of mitigating the issue faced by the Paraná Sanitation Company (SANEPAR) in 2020 and avoiding surprises in the near future, this work aims to enhance water usage. Although the isolated event that occurred in that year may not repeat itself in subsequent years, it is important to seek improvements in water resource management.	
	Methods derived from the autoregressive integrated moving average (ARIMA) model, including those with exogenous variables and considering data seasonality, prove to be the most effective forecasting models for modeling data with temporal variations. Although each method has its own characteristics, they are all based on the initial ARIMA model.	
	Boosting models, such as XGBoost (Extreme Gradient Boosting), followed by the simple linear regression (LR) model, are considered the best models for time series due to their use of the gradient boosting approach for predictions. This choice is based on error metrics, where a lower value indicates better decision-making capability. The metrics used in this article include mean absolute percentage error (MAPE), mean absolute error (MAE), and root mean squared error (RMSE). In time series, these metrics are commonly used to evaluate the effectiveness of prediction models under different circumstances and forecast horizons. The XGBoost model achieved an MAPE error of \SI{0.264}{\percent}, the lowest among the evaluated models, while the LR model had the highest error of \SI{5}{\percent} in the longest forecast horizon (one month). The moving averages (MA) model obtained an MAPE error of \SI{0.113}{\percent}, while the LR model had an error of \SI{5}{\percent}. Therefore, the LR model may be more efficient for smaller datasets, working with a reduced volume of data, while errors increase as the forecast horizon lengthens.

    \noindent \textbf{Keywords:} Forecasting, Water savings, Time series, Systematic literature review.
\end{abstract}
}
