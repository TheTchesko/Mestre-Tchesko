{\selectlanguage{english}
	\begin{abstract}
		
		
%		\noindent The study, situated in the context of water supply in Curitiba, focuses on the effectiveness of forecasting water demand in Bairro Alto from $2018$ to $2020$. The central question investigated is how to anticipate water demand for more efficient planning in the context of scarcity faced by the residents. The purpose of the work is to contribute to the effective control of water resources, utilizing forecasting models, with an emphasis on improving water supply in a competitive environment.
%		Models such as \textit{Auto-Regressive Integrated Moving Average} (ARIMA), \textit{Decision Tree} (DT), \textit{eXtreme Gradient Boosting} (XGBoost), and \textit{Recurrent Neural Network} (RNN) are explored for time series forecasting, with a comparative analysis of effectiveness. The need for a new or better solution arises from the water scarcity in Bairro Alto, justifying the search for more efficient methods of demand forecasting.
%		The proposed solution involves the application of machine learning models such as ARIMA, DT, XGBoost, and especially RNN in forecasting water demand. The basic methodology includes applying these models to the data collected by SANEPAR (Sanitation Company of Paraná).
%		The features responding to the initial questions are assessed through metrics such as \textit{Symmetric Mean Absolute Percentage Error} (sMAPE), \textit{Mean Absolute Error} (MAE), and \textit{Root Relative Mean Square Error} (RRMSE), highlighting that the RNN model consistently demonstrated the lowest errors in all analyses. It is concluded that the proposed approach significantly contributes to water demand forecasting, providing a more efficient and sustainable planning of water supply in Bairro Alto.
%		

%\noindent This study, situated in the context of water supply in Curitiba, exemplifies a comprehensive approach to address the growing challenges of water demand. Focusing specifically on the effectiveness of demand forecasting in Bairro Alto, the primary objective is to ensure that the existing infrastructure can meet the continually growing needs of the population, avoiding issues of inadequate supply. The analysis of time series forecasting models is conducted to provide a precise analysis of water demand prediction.
%Utilizing data collected by the Companhia de Saneamento do Paraná during the years $2018$ to $2019$, this study proposes a significant contribution to the effective control of water resources. The emphasis is on a variety of forecasting models, including Auto-Regressive, Auto-Regressive with Exogenous Inputs, Moving Average, Auto-Regressive Moving Average, Seasonal Auto-Regressive Integrated Moving Average , ARIMA with Exogenous Inputs, and Seasonal ARIMA with Exogenous Inputs.
%In addition to these models, the study particularly highlights the Recurrent Neural Network (RNN), which consistently demonstrated the smallest errors in all analyses. Exploring forecast horizons of one hour ahead, six hours ahead, twelve hours ahead, and twenty-four hours ahead, the detailed results show the different performances of these models.
%For 1-hour ahead forecasts, the RNN achieves a Symmetric Mean Absolute Percentage Error (SMAPE) of $0.1647$, Mean Absolute Error (MAE) of $0.0057$, and Root Relative Mean Square Error (RRMSE) of $0.0022$.
%At the 6-hour ahead horizon, the RNN demonstrates an SMAPE of $0.1356$, MAE of $0.0046$, and RRMSE of $0.0022$, highlighting its accuracy in this time interval.
%When considering 12-hour ahead forecasts, the RNN achieves an SMAPE of $0.1343$, MAE of $0.0046$, and RRMSE of $0.0021$, indicating its consistency in longer time horizons.
%For 24-hour ahead projections, the RNN maintains remarkable performance with an SMAPE of $0.2231$, MAE of $0.0077$, and RRMSE of $0.0028$.
%These results underscore the reliability of the RNN in water demand forecasting across different time scales, thereby providing efficient and sustainable water supply planning in the region.

\noindent The study, set in the context of water supply in Curitiba, focuses on the effectiveness of demand forecasting in Bairro Alto, using data collected by Companhia de Saneamento do Paraná during the years $2018$ to $2020$. The central issue investigated is the effectiveness of water supply forecasts, aimed at ensuring that the existing infrastructure is capable of meeting the growing needs of the population, preventing problems of inadequate supply. The purpose of this study is to contribute to the effective control of water resources using forecasting models, with an emphasis on improving water supply.
Forecasting models such as Auto-Regressive (AR), Auto-Regressive with Exogenous Inputs, Moving Average, Auto-Regressive Moving Average, Seasonal Auto-Regressive Integrated Moving Average, ARIMA with Exogenous Inputs, and Seasonal ARIMA with Exogenous Inputs, Decision Tree (DT), eXtreme Gradient Boosting, and Recurrent Neural Network for time series forecasting, with a comparative analysis of the effectiveness of the forecasting models.
The performance of the forecasting models is evaluated using metrics including Symmetric Mean Absolute Percentage Error (SMAPE), Mean Absolute Error (MAE) and Root Relative Mean Square Error (RRMSE). It can be seen that, overall, the DT model performed better in short-term forecasts, standing out with results of SMAPE $13.50$, MAE $0.58$ and RRMSE $0.16$ for the $6$ hour ahead forecast. With $1$ hours ahead, the result was SMAPE $7.83$, MAE $0.36$ and RRMSE $0.20$, second only to the MAE and RRMSE values of the AR model, which had the lowest values in this forecast horizon. However, the DT model generally proved superior to the other models up to the $24$ hour or one-day ahead forecast. In the one-day-ahead forecast, Prophet stood out, with results of SMAPE $5.05$, MAE $0.17$ and RRMSE $0.19$.
It can be concluded that the proposed approach contributes significantly to water demand forecasting, providing efficient and sustainable water supply planning in Bairro Alto. In addition, forecasting makes it possible to anticipate and prevent possible water shortages, predicting future demand and enabling the adoption of proactive measures to avoid supply interruptions.

		\hspace{1cm}
		
		\noindent \textbf{Keywords:} Forecasting, Time series, Water supply, Machine learning, Artificial neural networks.
	\end{abstract}
}
