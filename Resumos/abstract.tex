{\selectlanguage{english}
	\begin{abstract}
		
		
		\noindent The study, situated in the context of water supply in Curitiba, focuses on the effectiveness of forecasting water demand in Bairro Alto from $2018$ to $2020$. The central question investigated is how to anticipate water demand for more efficient planning in the context of scarcity faced by the residents. The purpose of the work is to contribute to the effective control of water resources, utilizing forecasting models, with an emphasis on improving water supply in a competitive environment.
		Models such as \textit{Auto-Regressive Integrated Moving Average} (ARIMA), \textit{Decision Tree} (DT), \textit{eXtreme Gradient Boosting} (XGBoost), and \textit{Recurrent Neural Network} (RNN) are explored for time series forecasting, with a comparative analysis of effectiveness. The need for a new or better solution arises from the water scarcity in Bairro Alto, justifying the search for more efficient methods of demand forecasting.
		The proposed solution involves the application of machine learning models such as ARIMA, DT, XGBoost, and especially RNN in forecasting water demand. The basic methodology includes applying these models to the data collected by SANEPAR (Sanitation Company of Paraná).
		The features responding to the initial questions are assessed through metrics such as \textit{Symmetric Mean Absolute Percentage Error} (sMAPE), \textit{Mean Absolute Error} (MAE), and \textit{Root Relative Mean Square Error} (RRMSE), highlighting that the RNN model consistently demonstrated the lowest errors in all analyses. It is concluded that the proposed approach significantly contributes to water demand forecasting, providing a more efficient and sustainable planning of water supply in Bairro Alto.
		
		
		\hspace{1cm}
		
		\noindent \textbf{Keywords:} Forecasting, Time series, Water supply, Machine learning, Artificial neural networks.
	\end{abstract}
}
