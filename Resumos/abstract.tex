{\selectlanguage{english}
\begin{abstract}
	\noindent  In a competitive scenario, assertive demand forecasting has increasingly become a strategic tool for several organizational branches.	
	In this context, time series forecasting has played a very important role in decision making. Recently, the capital city of Paraná faced a serious health crisis, with periods of shortages that generated serious instability in the supply of housing for many families.
	In the decision making of this problem in question, the methods found by the review that was done during this work are used to be predicted. The methods adopted here are a form of decision making for the water demand issue. Thus, each method chosen can better deal with the problem and get a more feasible solution for decision making.	
	In order to mitigate and make the best decision for the problem that the sanitation company of Paraná (SANEPAR) faced in the year 2020, so that it does not occur again or be caught by surprise in a near future that may arise. With the isolated fact that occurred in the year in question and that may not be repeated in future years, this work aims to improve the use of water. 
	The methods derived from the autoregressive integrated moving average model (ARIMA), with exogenous variables and with seasonality in the data are the most effective predictors for modeling the data with variation between each method, but all based on the initial ARIMA method.
	The so-called boosting models (or gradient regression tree), followed by the simple linear regression (LR) model, are considered the best models for time series because they use the gradient form of prediction.
	This is obtained in some error metrics, the smaller the error, the better for decision making. The metrics adopted in this paper are the mean absolute percentage error (MAPE), the mean absolute error (MAE) and the root mean square error (RMSE). In time series, these metrics are more frequent, with better or more effective forecasting models in some circumstances, with no future horizon forecast. The extreme gradient boost model (XGBoost) has an error of $0.264\%$ in the MAPE metric just looking at this metric, and the LR model has the largest error of $5\%$ at the longest forecast horizon (one month), the moving average (MA) model appears with an error of $0.113\%$ and the LR model with an error of $5\%$. Thus, the LR model for a smaller data set may be more efficient than the other models because it works with a small volume of data and the errors increase as the horizon increases.

    \noindent \textbf{Keywords:} Forecasting, Water savings, Time series, Systematic literature review.
\end{abstract}
}
