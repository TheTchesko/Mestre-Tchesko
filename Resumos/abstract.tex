{\selectlanguage{english}
\begin{abstract}
	\noindent This study addresses the strategic importance of accurate water demand forecasting as a tool for efficient water resource management in a competitive landscape. The identified problem is the lack of accurate predictions, which hinders strategic decision-making in water supply. The proposed solution is the use of advanced time series forecasting models to improve demand prediction accuracy. Based on a comprehensive review of existing literature, different methods and approaches used in water supply time series forecasting are analyzed.	
	The state-of-the-art is explored to identify the most effective models and best practices in the field. Specific methods and products are proposed based on the state-of-the-art, considering exogenous variables, data seasonality, and utilizing autoregressive integrated moving average (ARIMA) models, boosting techniques like XGBoost (Extreme Gradient Boosting) and LightGBM (Light Gradient Boosting Machine), and linear regression. Additionally, the use of models based on Random Forest Regression (RFR) is also considered.	
	The results obtained through the application of these proposed methods and products are analyzed and compared using performance metrics such as symmetric mean absolute percentage error (sMAPE), mean absolute error (MAE), and root relative mean squared error (RRMSE). These findings provide valuable insights into the effectiveness of time series forecasting models in water supply and contribute to more informed and efficient decision-making in this field.
	

    \noindent \textbf{Keywords:} Forecasting, Water savings, Time series, Systematic literature review.
\end{abstract}
}
