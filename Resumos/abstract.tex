{\selectlanguage{english}
\begin{abstract}
	\noindent This study addresses the importance of accurate water demand forecasting as a strategic tool for efficient water resource management in a competitive scenario.
	The identified problem is the lack of precise forecasts, which hinders strategic decision-making in water supply.
	The proposed solution is the use of advanced time series forecasting models to improve the accuracy of demand forecasts.
	Based on a comprehensive review of the existing literature, different methods and approaches used in water supply-related time series forecasting are analyzed. The state of the art is explored to identify the most effective models and best practices in the field.
	Building upon the state of the art, specific methods and products are proposed for water demand forecasting. These methods consider exogenous variables, data seasonality, and utilize autoregressive integrated moving average (ARIMA) models and boosting techniques such as XGBoost (Extreme Gradient Boosting) and linear regression.
	The results obtained through the application of these proposed methods and products are analyzed and compared using performance metrics such as mean absolute percentage error (MAPE), mean absolute error (MAE), and root mean square error (RMSE). These results provide valuable insights into the effectiveness of time series forecasting models in water supply and contribute to more informed and efficient decision-making in this field.

    \noindent \textbf{Keywords:} Forecasting, Water savings, Time series, Systematic literature review.
\end{abstract}
}
