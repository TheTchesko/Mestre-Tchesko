{\selectlanguage{english}
\begin{abstract}
\noindent  Prediction of time series is very important for decision making, making models that can predict and thus improve decision making based on the models.
This dissertation will address the problem of water demand that occurred in the city of Curitiba in the state of Paraná, there was in the period of the data collected in the years 2018 to 2020, aiming for the year 2020 that was the year that occurred the highest water demand, causing the reservoirs to suffer from this, several factors such as, for example, the rainfall that was not enough to meet this year in question. 
In the decision making process of this problem in question, some methods found in the review that was conducted during the master's degree program are used, so that it can be predicted in some prediction horizons. The horizons addressed here are a way to solve the issue of water demand and thus validate the models to see which one is the most efficient.
In order to mitigate and solve the problem that SANEPAR faced in the year 2020, so that it doesn't occur again or that it doesn't catch us unprepared in the next event that may arise. With the isolated event that happened in the year in question and may not be repeated in future years, this work aims to improve the use of water.     
The ARIMA methods and the updated ARIMA with, for example, some models derived from the ARIMA models, thus listing the models are AR, ARX, MA, ARMA, ARIMA, SARIMA, SARIMAX and ARIMAX, as each model has its particularity the models with exogenous variables may seem graphically better to be predicted than the ARIMA models without the exogenous variables. These models are called traditional models of weather forecast, to start processing data, in the models of gradient reinforcement are the best models to forecast with the lowest errors, in hypothesis is what seeks in this work. The models called boost or gradient regression tree, the following LR, XGboost and Light GBM random forest regression models were used, these models for time series are listed as the best models, because some of them use the gradient prediction form.     
It is obtained in some error metrics, the smaller the error the better for decision making. The metrics adopted in this work is MAPE, MAE and RMSE, in time series these metrics are more frequent than others and to not put several in this work we adopted only these three, having with forecast models better or more effective in some circumstances with in predicting no future horizontem, or just the series forecast obtained in the data the XGBoost model has $0.079\%$ error in the MAPE metric just looking at it for now, and the LR with having the largest error of $21\%$ in the largest forecast horizon (60 days) the ARMA model comes with $12.54\%$ error and the LR model with $11153.594\%$.

    \noindent \textbf{Keywords:} Forecasting, Water savings, Time series, Time series.
\end{abstract}
}
