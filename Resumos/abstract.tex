{\selectlanguage{english}
\begin{abstract}
\noindent  Time series forecasting is very important for decision making.
In this dissertation the problem of water demand that occurred in the city of Curitiba in the state of Paraná will be addressed, there was a period of data collected in the years 2018 to 2020, aiming for the year 2020 that was the year that occurred the highest water demand, causing the reservoirs to suffer from this, several factors. 
In the decision making of this problem in question, some methods found in the review that was conducted during this work are used, to be predicted in some forecast horizons, the horizon addressed here is a way to solve the issue of water demand and thus validate the models to see which one is the most efficient, the horizon adopted was the forecast of 1, 7, 14 and 30 days ahead, so that each method will deal with the data over time.
In order to mitigate and solve the problem that SANEPAR faced in the year 2020, so that it doesn't occur again or that it doesn't catch us unprepared in the next event that may arise. With the isolated event that happened in the year in question and may not be repeated in future years, this work aims to improve the use of water.     
The methods derived from ARIMA models, thus listing the models are AR, ARX, MA, ARMA, ARIMA, SARIMA, SARIMAX and ARIMAX, as each model has its own particularity the models with exogenous variables may seem graphically better to be predicted than the ARIMA models without exogenous variables. Gradient boosting models are the best models to predict with the lowest errors. The models called boosting or gradient regression tree, the following models LR, XGboost random forest regression and Light GBM were used, these models for time series are listed as the best models because some of them use the gradient way of predicting.     
It is obtained in some error metrics, the smaller the error the better for decision making. The metrics adopted in this work is MAPE, MAE and RMSE, in time series these metrics are more frequent, with better or more effective forecasting models in some circumstances with in forecasting no future horizon, The XGBoost model has $0.013\%$ error in the MAPE metric just analyzing over this metric, and the LR model has the highest error of $21\%$ in the longest forecast horizon (30 days), the MA model comes with $11.57\%$ error and the LR model with $548.59\%$ error. Thus, the LR model for a smaller data set can be more efficient than the other models, since it works with a small volume of data and the errors get higher as the horizon increases.

    \noindent \textbf{Keywords:} Forecasting, Water economy, Time series, Time series analysis.
\end{abstract}
}
