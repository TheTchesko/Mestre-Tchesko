{\selectlanguage{english}
\begin{abstract}
	\noindent This study addresses the strategic importance of accurate water demand forecasting as a tool for efficient water resource management in a competitive scenario. The identified problem is the lack of precise predictions, which hinders strategic decision-making in water supply. The proposed solution is the use of advanced time series forecasting models to enhance demand prediction accuracy. Based on a comprehensive review of existing literature, different methods and approaches used in water supply time series forecasting are analyzed. The state of the art is explored to identify the most effective models and best practices in the field. Building upon the state of the art, specific methods and products for water demand forecasting are proposed. These methods take into consideration exogenous variables, data seasonality, and utilize autoregressive integrated moving average (ARIMA) models, boosting techniques such as XGBoost (Extreme Gradient Boosting) and LightGBM (Light Gradient Boosting Machine), as well as linear regression. Additionally, the use of Random Forest Regression (RFR) models is also considered. The results obtained through the application of these proposed methods and products are analyzed and compared using performance metrics such as mean absolute percentage error (MAPE), mean absolute error (MAE), and root mean square error (RMSE). These findings provide valuable insights into the effectiveness of time series forecasting models in water supply and contribute to more informed and efficient decision-making in this field.
	

    \noindent \textbf{Keywords:} Forecasting, Water savings, Time series, Systematic literature review.
\end{abstract}
}
