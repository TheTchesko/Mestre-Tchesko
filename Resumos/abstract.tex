{\selectlanguage{english}
\begin{abstract}
	\noindent This study explores time series forecasting for decision making related to water demand, in order to help effectively control water resources in a competitive environment. Related to water supply. The proposed approach involves the use of time series forecasting models to improve the accuracy of water demand estimates.
	In this study, the supply of water by SANEPAR (Companhia de Saneamento do Paraná) to the Bairro Alto neighborhood in the city of Curitiba is explored.	
	Artificial neural network models such as GRU (Gated Recurrent Unit), LSTM (Long Short-Term Memory), RNN (Recurrent Neural Network), \textit{Transformer} and Facebook Prophet are evaluated. In addition, ARIMA (Autoregressive Integrated Moving Average) models, boosting techniques such as XGBoost (eXtreme Gradient Boosting) and LightGBM (Light Gradient Boosting Machine), linear regression and RFR (Random Forest Regression). The effectiveness of the models is assessed using metrics such as sMAPE (Symmetric Mean Absolute Percentage Error), MAE (Mean Absolute Error) and RRMSE (Relative Root Mean Square Error).
	The analysis and comparison of all the cases showed that the RNN model obtained the lowest error in all the metrics evaluated, including SMAPE, MAE and RRMSE. It is interesting to note that the performance of the RNN model was exceptional, with prediction errors consistently lower than 1\% in all analyses. 
	
\hspace{1cm}

    \noindent \textbf{Keywords:} Time series forecasting, Water supply, Machine learning, Artificial neural networks, Forecasting models.
\end{abstract}
}
