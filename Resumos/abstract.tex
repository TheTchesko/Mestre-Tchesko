{\selectlanguage{english}
\begin{abstract}
	\noindent This study addresses the strategic importance of accurate water demand forecasting as a tool for efficient water resource management in a competitive scenario. The identified challenge is the lack of precise forecasts, hindering strategic decisions in water supply. The proposed approach involves using advanced time series forecasting models to enhance the accuracy of demand predictions.	
	Through an extensive literature review, various methods and approaches used in time series forecasting for water supply are explored. These methods encompass neural network models like GRU (Gated Recurrent Unit), LSTM (Long Short-Term Memory), RNN (Recurrent Neural Network), and Transformer, alongside the specialized model Prophet, as well as Decision Tree Regression.	
	Building on this knowledge, specific methods and products are suggested, considering exogenous variables, seasonality, and employing integrated autoregressive moving average models (ARIMA), boosting techniques such as XGBoost and LightGBM, linear regression, and Random Forest Regression (RFR). The performance of these approaches is evaluated using metrics such as sMAPE, MAE, and RRMSE, providing insights into the effectiveness of forecasting models for water supply.	
	These findings contribute to a more informed and efficient decision-making process in this domain, shedding light on the effectiveness of time series forecasting models for water supply.
	

    \noindent \textbf{Keywords:} Forecasting, Water savings, Time series, Systematic literature review.
\end{abstract}
}
