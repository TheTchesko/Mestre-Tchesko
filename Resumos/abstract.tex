{\selectlanguage{english}
\begin{abstract}
	\noindent This study explores the importance of forecasting time series for making decisions related to water demand, with a view to effectively controlling water resources in a competitive environment. The challenge lies in obtaining reliable time series to assist in water supply decisions. The proposed approach involves using time series forecasting models to improve the accuracy of demand estimates.
	There are several approaches discussed in the literature for analyzing and forecasting time series in the field of water supply. In this study, the case of SANEPAR (Companhia de Saneamento do Paraná) is examined as a representative example. However, what makes this study unique is the introduction and customized evaluation of neural network models, such as GRU (Gated Recurrent Unit), LSTM (Long Short-Term Memory), RNN (Recurrent Neural Network) and Transformer in customized form for this problem, in addition to the Facebook Prophet model, which have not been applied to this context until now. The decision tree regression technique is also explored. These innovative methods expand the possibilities in water demand forecasting.
	Based on this knowledge, specific methods and products are analyzed, taking into account external factors and seasonality, as well as using ARIMA models, boosting techniques such as XGBoost (eXtreme Gradient Boosting) and LightGBM (Light Gradient Boosting Machine), linear regression and RFR (Random Forest Regression). The effectiveness of these approaches is evaluated by metrics such as sMAPE (Symmetric Mean Absolute Percentage Error), MAE (Mean Absolute Error) and RRMSE (Relative Root Mean Square Error), providing information on the ability of the forecasting models to supply water.
	These results are useful for making more informed decisions in the context of the SANEPAR company. They provide information on how time series forecasting models perform in relation to water supply.
	The analysis and comparison of all the cases showed that the RNN model obtained the lowest error in all the metrics evaluated, such as SMAPE, MAE and RRMSE. It is interesting to note that the performance of the RNN model was exceptional, with prediction errors consistently below 1\% in all analyses. This highlights that it is the most efficient and accurate model in all the applications evaluated.
	
\hspace{1cm}

    \noindent \textbf{Keywords:} Time series forecasting, Water savings, Time series, Forecasting models.
\end{abstract}
}
