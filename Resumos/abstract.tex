{\selectlanguage{english}
\begin{abstract}
%	\noindent This study explores time series forecasting for decision making related to water demand, in order to help effectively control water resources in a competitive environment. Related to water supply. The proposed approach involves the use of time series forecasting models to improve the accuracy of water demand estimates.
%	In this study, the supply of water by SANEPAR (Companhia de Saneamento do Paraná) to the Bairro Alto neighborhood in the city of Curitiba is explored.	
%	Artificial neural network models such as GRU (Gated Recurrent Unit), LSTM (Long Short-Term Memory), RNN (Recurrent Neural Network), \textit{Transformer} and Facebook Prophet are evaluated. In addition, ARIMA (Autoregressive Integrated Moving Average) models, boosting techniques such as XGBoost (eXtreme Gradient Boosting) and LightGBM (Light Gradient Boosting Machine), linear regression and RFR (Random Forest Regression). The effectiveness of the models is assessed using metrics such as sMAPE (Symmetric Mean Absolute Percentage Error), MAE (Mean Absolute Error) and RRMSE (Relative Root Mean Square Error).
%	The analysis and comparison of all the cases showed that the RNN model obtained the lowest error in all the metrics evaluated, including SMAPE, MAE and RRMSE. It is interesting to note that the performance of the RNN model was exceptional, with prediction errors consistently lower than 1\% in all analyses. 

\noindent The study, situated in the context of water supply in Curitiba, focuses on the effectiveness of forecasting water demand in Bairro Alto from $2018$ to $2020$. The central question investigated is how to anticipate water demand for more efficient planning in the context of scarcity faced by the residents. The purpose of the work is to contribute to the effective control of water resources, utilizing forecasting models, with an emphasis on improving water supply in a competitive environment.
Models such as ARIMA (Auto-Regressive Integrated Moving Average), DTR (Decision Tree Regressor), XGBoost (eXtreme Gradient Boosting), and RNN (Recurrent Neural Network) are explored for time series forecasting, with a comparative analysis of effectiveness. The need for a new or better solution arises from the water scarcity in Bairro Alto, justifying the search for more efficient methods of demand forecasting.
The proposed solution involves the application of machine learning models such as ARIMA, DTR, XGBoost, and especially RNN in forecasting water demand. The basic methodology includes applying these models to the data collected by SANEPAR (Sanitation Company of Paraná).
The features responding to the initial questions are assessed through metrics such as sMAPE (Symmetric Mean Absolute Percentage Error), MAE (Mean Absolute Error), and RRMSE (Root Relative Mean Square Error), highlighting that the RNN model consistently demonstrated the lowest errors in all analyses. It is concluded that the proposed approach significantly contributes to water demand forecasting, providing a more efficient and sustainable planning of water supply in Bairro Alto.

	
\hspace{1cm}

    \noindent \textbf{Keywords:} Time series forecasting, Water supply, Machine learning, Artificial neural networks, Forecasting models.
\end{abstract}
}
