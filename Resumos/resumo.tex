\begin{abstract} 
	
	\noindent Este estudo aborda a importância estratégica da previsão precisa da demanda de água como uma ferramenta para a gestão eficiente dos recursos hídricos em um cenário competitivo. O problema identificado é a falta de previsões precisas, o que dificulta a tomada de decisões estratégicas no abastecimento de água. A solução proposta é o uso de modelos avançados de previsão de séries temporais para melhorar a precisão das previsões de demanda. Com base em uma revisão abrangente da literatura existente, diferentes métodos e abordagens utilizados na previsão de séries temporais no contexto do abastecimento de água são analisados.	
	O estado da arte é explorado para identificar os modelos mais eficazes e as melhores práticas na área. Métodos e produtos específicos são propostos com base no estado da arte, levando em consideração variáveis exógenas, sazonalidade dos dados e utilizando modelos autorregressivos integrados de médias móveis (ARIMA), técnicas de boosting como XGBoost (Extreme Gradient Boosting), LightGBM (Light Gradient Boosting Machine) e regressão linear. Além disso, também é considerado o uso de modelos baseados em Random Forest Regression (RFR).	
	Os resultados obtidos por meio da aplicação desses métodos e produtos propostos são analisados e comparados utilizando métricas de desempenho, como o erro percentual absoluto médio simétrico (sMAPE), o erro absoluto médio (MAE) e a raiz do erro médio quadrático relativo (RRMSE). Essas descobertas fornecem informações valiosas sobre a eficácia dos modelos de previsão de séries temporais no abastecimento de água e contribuem para uma tomada de decisão mais informada e eficiente nessa área.




    \noindent \textbf{Palavras-chave:} Previsão, Economia de água, Séries temporais, Modelos de Previsão.
\end{abstract}

