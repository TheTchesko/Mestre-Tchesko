\begin{abstract} 
	
\noindent	CONTEXTO; Diante de um cenário competitivo, a previsão assertiva da demanda tem se tornado cada vez mais uma ferramenta estratégica para vários ramos organizacionais.
	PROBLEMA; Dentro deste contexto, a previsão de séries temporais tem desempenhado um papel muito crítico na tomada de decisões. Recentemente, a capital do Paraná enfrentou uma grave crise de saúde, com períodos de escassez que geraram grave instabilidade no abastecimento das casas de muitas famílias. Esta dissertação aborda o problema da previsão da demanda de água que ocorreu na cidade de Curitiba, coletando os períodos dos anos 2018 a 2020, visando o ano 2020, que foi o ano em que ocorreu a maior demanda de água, fazendo com que os reservatórios sofressem com isso, vários fatores.
	SOLUÇÃO;
	ESTADO-DA-ARTE;
	MÉTODO(S)/PRODUTO(S) PROPOSTO(S);
	RESULTADOS.
	
	   
     \noindent Diante de um cenário competitivo, a previsão assertiva da demanda tem se tornado cada vez mais uma ferramenta estratégica para vários ramos organizacionais. Dentro deste contexto, a previsão de séries temporais tem desempenhado um papel muito crítico na tomada de decisões. Recentemente, a capital do Paraná enfrentou uma grave crise de saúde, com períodos de escassez que geraram grave instabilidade no abastecimento das casas de muitas famílias. Esta dissertação aborda o problema da previsão da demanda de água que ocorreu na cidade de Curitiba, coletando os períodos dos anos 2018 a 2020, visando o ano 2020, que foi o ano em que ocorreu a maior demanda de água, fazendo com que os reservatórios sofressem com isso, vários fatores. 
     Na tomada de decisão deste problema em questão, alguns métodos encontrados na revisão que foi realizada durante este trabalho são utilizados, para serem previstos em alguns horizontes de previsão, o horizonte aqui abordado é uma forma de resolver a questão da demanda de água e assim validar os modelos para ver qual deles é o mais eficiente, o horizonte adotado foi a previsão de 1, 7, 14 e 30 dias à frente, de modo que cada método irá lidar com os dados ao longo do tempo.
     A fim de mitigar e resolver o problema que a SANEPAR enfrentou no ano 2020, para que não ocorra novamente ou para que não nos pegue despreparados no próximo evento que possa surgir. Com o evento isolado que aconteceu no ano em questão e que pode não se repetir em anos futuros, este trabalho tem como objetivo melhorar o uso da água.     
     Os métodos derivados dos modelos ARIMA, listando assim os modelos são AR, ARX, MA, ARMA, ARIMA, SARIMA, SARIMAX e ARIMAX, pois cada modelo tem sua particularidade os modelos com variáveis exógenas podem parecer graficamente melhores de serem previstos do que os modelos ARIMA sem variáveis exógenas. Os modelos com aumento gradual são os melhores modelos para prever com os menores erros. Os modelos chamados boosting ou árvore de regressão de gradiente, os seguintes modelos LR, XGboost random forest regression e Light GBM foram usados, estes modelos para séries temporais são listados como os melhores modelos porque alguns deles usam a forma de previsão de gradiente.     
     É obtido em algumas métricas de erro, quanto menor o erro, melhor para a tomada de decisão. As métricas adotadas neste trabalho são MAPE, MAE e RMSE, em séries temporais estas métricas são mais freqüentes, com modelos de previsão melhores ou mais eficazes em algumas circunstâncias sem previsão de horizonte futuro, O modelo XGBoost tem erro de $0,013\%$ na métrica MAPE apenas analisando sobre esta métrica, e o modelo LR tem o maior erro de $21\%$ no maior horizonte de previsão (30 dias), o modelo MA vem com erro de $11,57\%$ e o modelo LR com erro de $548,59\%$. Assim, o modelo LR para um conjunto de dados menor pode ser mais eficiente do que os outros modelos, já que trabalha com um pequeno volume de dados e os erros ficam maiores à medida que o horizonte aumenta.
        
     

    \noindent \textbf{Palavras-chave:} Previsão, Economia de água, Séries temporais, Análise de séries temporais.
\end{abstract}

