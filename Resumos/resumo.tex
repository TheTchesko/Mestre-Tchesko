\begin{abstract} 
	\noindent Este estudo explora a previsão de séries temporais para a tomada de decisões relacionadas à demanda de água, visando de alguma forma auxiliar no controle eficaz dos recursos hídricos em um ambiente competitivo. Relacionado ao fornecimento de água. A abordagem proposta envolve a utilização de modelos de previsão de séries temporais para melhorar a precisão das estimativas de demanda de água.
	Neste estudo, o abastecimento de água pela SANEPAR (Companhia de Saneamento do Paraná) para o Bairro Alto da cidade de Curitiba é explorado.	
	Avaliando-se modelos de redes neurais artificiais, como GRU (do inglês \textit{Gated Recurrent Unit}), LSTM (do inglês \textit{Long Short-Term Memory}), RNN (do inglês \textit{Recurrent Neural Network}), \textit{Transformer} e Facebook Prophet. Além disso, modelos do tipo ARIMA (do inglês \textit{Auto-Regressive Integrated Moving Average}), técnicas de \textit{boosting} como XGBoost (do inglês \textit{eXtreme Gradient Boosting}) e LightGBM (do inglês \textit{Light Gradient Boosting Machine}), regressão linear e RFR (do inglês \textit{Random Forest Regression}). A eficácia dos modelos é  avaliada por meio de métricas como sMAPE (do inglês \textit{Symmetric Mean Absolute Percentage Error}), MAE (do inglês \textit{Mean Absolute Error}) e RRMSE (do inglês \textit{Root Relative Mean Square Error}).
	A análise e comparação de todos os casos, ficou evidente que o modelo RNN obteve o menor erro em todas as métricas avaliadas, incluindo-se SMAPE, MAE e RRMSE. É interessante notar que o desempenho do modelo RNN foi excepcional, com erros de previsão consistentemente menor que 1\% em todas as análises. 
	
\hspace{1cm}


    \noindent \textbf{Palavras-chave:} Previsão de séries temporais, Abastecimento de água, Aprendizado de Máquina, Redes Neurais Artificiais, Modelos de Previsão.
\end{abstract}

