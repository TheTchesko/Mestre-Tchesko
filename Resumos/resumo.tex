\begin{abstract} 
	
	\noindent Em um cenário competitivo, a previsão assertiva de demanda tem se tornado cada vez mais uma ferramenta estratégica para diversos ramos organizacionais.	
	Nesse contexto, a previsão de séries temporais tem desempenhado um papel muito importante na tomada de decisões. Recentemente, a capital paranaense enfrentou uma grave crise na área da saúde, com períodos de desabastecimento que geraram séria instabilidade na oferta de moradia para muitas famílias.
	Na tomada de decisão desse problema em questão, os métodos encontrados pela revisão que foi feita durante este trabalho são usados para serem previstos. Os métodos aqui adotados são uma forma de tomada de decisão para a questão da demanda de água. Assim, cada método escolhido pode lidar melhor com o problema e obter uma solução mais viável para a tomada de decisão.	
	Com o objetivo de mitigar e tomar a melhor decisão para o problema que a Companhia de Saneamento do Paraná (SANEPAR) enfrentou no ano de 2020, para que não ocorra novamente ou seja pega de surpresa em um futuro próximo que possa surgir. Com o fato isolado que ocorreu no ano em questão e que pode não se repetir nos anos futuros, este trabalho tem como objetivo melhorar o uso da água. 
	Os métodos derivados do modelo autorregressivo integrado de média móvel (ARIMA), com variáveis exógenas e com sazonalidade nos dados são os preditores mais eficazes para modelar os dados com variação entre cada método, mas todos baseados no método ARIMA inicial.
	Os chamados modelos de boosting (ou árvore de regressão gradiente), seguidos pelo modelo de regressão linear simples (LR), são considerados os melhores modelos para séries temporais porque usam a forma gradiente de previsão.
	Isso é obtido em algumas métricas de erro, quanto menor o erro, melhor para a tomada de decisões. As métricas adotadas neste artigo são o erro percentual absoluto médio (MAPE), o erro absoluto médio (MAE) e a raiz do erro quadrático médio (RMSE). Em séries temporais, essas métricas são mais frequentes, com modelos de previsão melhores ou mais eficazes em algumas circunstâncias, sem previsão de horizonte futuro. O modelo de impulso de gradiente extremo (XGBoost) tem um erro de $ 0,264\%$ na métrica MAPE apenas olhando para essa métrica, e o modelo LR tem o maior erro de $ 5\%$ no horizonte de previsão mais longo (um mês), o modelo de média móvel (MA) aparece com um erro de $ 0,113\%$ e o modelo LR com um erro de $ 5\%$. Assim, o modelo LR para um conjunto de dados menor pode ser mais eficiente do que os outros modelos porque trabalha com um pequeno volume de dados e os erros aumentam à medida que o horizonte aumenta.
	
	 

    \noindent \textbf{Palavras-chave:} Previsão, Economia de água, Séries temporais, Revisão sistemática da literatura.
\end{abstract}

