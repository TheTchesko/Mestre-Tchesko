\begin{abstract}    
     \noindent Diante de um cenário competitivo, uma previsão de demanda assertiva tem sido cada vez mais uma ferramente estratégica para diversos ramos organizacionais. Dentro deste contexto, a previsão de séries temporais tem desempenhado um papel muito crítico para a tomada de decisão. Recentemente, a capital paranaense enfrentou uma grave crise sanitária, com períodos de escassez que gerou uma severa instabilidade no suprimento dos lares de muitas famílias. A presente dissertação aborda o problema da previsão demanda de água que ocorreu na cidade de Curitiba por meio da coleta dos períodos dos anos de 2018 a 2020, visando o ano de 2020 que foi o ano que ocorreu a maior demanda d'água, fazendo com que os reservatórios sofressem com isso, vários fatores. 
     Na tomada de decisão desse problema em questão é usado alguns métodos encontrado na revisão que foi realizado no decorrer desse trabalho, para ser previsto em alguns horizontes de previsão, os horizonte abordado aqui é uma forma que poder resolver a questão da demanda d'água e com isso validar os modelos para ver qual deles é o mais eficiente, horizonte adotado foi de previsão de 1, 7, 14 e 30 dias a frente, assim seja cada método vai lidar com os dados no decorrer do tempo.
     De modo a amenizar e solucionar o problema que a empresa SANEPAR enfrentou no ano de 2020, para não ocorrer mais ou que não pegue desprevenido no próximo evento que pode surgir. Com o evento isolado que aconteceu no ano em questão e não possa se repetir nos anos futuros, esse trabalho visa a melhoria do usa d'água.     
     Os métodos derivados dos modelos ARIMA, assim listando os modelos é AR, ARX, MA, ARMA, ARIMA, SARIMA, SARIMAX e ARIMAX, como cada modelo tem sua particularidade os modelos de variáveis exógenas pode parecer graficamente melhor de ser previsto do que os modelos de ARIMA sem a variáveis exógenas. Nos modelos de reforço de gradiente é os melhores modelos para se prever com os erros mais baixo. Os modelos chamado de reforço ou árvore de regressão de gradiente, foi usado os seguintes modelos LR, regressão florestal aleatória XGboost e Light GBM, esses modelos para série temporal é listado como os melhores modelos, pois alguns deles usa a forma de prever de gradiente.     
     É obtido em algumas métricas de erros, quanto menor o erro melhor para a tomada de decisão. As métricas adotado nesse trabalho é MAPE, MAE e RMSE, em série temporal essas métricas são mais frequente, com modelos de previsão melhor ou mais eficaz em algumas circunstâncias com na previsão de nenhum horizonte futuro, ou apensa a previsão da série obtida nos dados o modelo XGBoost tem erro de $0,013\%$ na métrica MAPE só analisando encima dessa métrica, e o LR com o maior erro de $21\%$  no horizonte maior de previsão (30 dias) o modelo MA vem com o erro de $11,57\%$ e o modelo LR com $548,59\%$. Assim o modelo LR para um conjunto de dados menor ele pode ser mais eficiente do que os outros modelos, pois trabalha com pouco volume de dados e os erros ficam mais alto conforme vai aumentando o horizonte.

    \noindent \textbf{Palavras-chave:} Previsão, Economia d'água, Séries temporais, Análise de séries temporais.
\end{abstract}

