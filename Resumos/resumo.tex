\begin{abstract} 
%	\noindent O estudo, inserido no contexto do abastecimento de água em Curitiba, concentra-se na eficácia da previsão da demanda no Bairro Alto através dos dados coletados pela SANEPAR  (Companhia de Saneamento do Paraná) durante os anos de $2018$ a $2020$. A questão central investigada é as previsões de abastecimento de água para ajudar a garantir que a infraestrutura existente seja capaz de atender às necessidades crescentes da população, evitando problemas de oferta inadequada. Assim, o propósito deste estudo é contribuir para o controle eficaz dos recursos hídricos, utilizando modelos de previsão, com ênfase na melhoria do abastecimento d'água. São explorados modelos de previsão tais como \textit{Auto-Regressive} , \textit{Auto-Regressive with Exogenous Inputs}, \textit{Moving Average}, \textit{Auto-Regressive Moving Average}, \textit{Seasonal Auto-Regressive Integrated Moving Average}, ARIMA com \textit{Exogenous Inputs}, e \textit{Seasonal} ARIMA com \textit{Exogenous Inputs}, \textit{Decision Tree} (DT), \textit{eXtreme Gradient Boosting} e \textit{Recurrent Neural Network} para a previsão de séries temporais, com análise comparativa de eficácia dos modelos de previsão. O desempenho dos modelos de previsão são avaliadas por meio de métricas que incluem \textit{Symmetric Mean Absolute Percentage Error} (SMAPE), \textit{Mean Absolute Error} (MAE) e \textit{Root Relative Mean Square Error} (RRMSE), destacando que o modelo DT e Prophet demonstrou consistentemente os menores erros em todas as análises, como previsão de curto prazo o modelo DT se mostrou melhor com os resultados de SMAPE 13,50,  MAE 0,58 e RRMSE 0,16 com a previsão de 6 horas à frente e com 1 hora à frente o resultado foi de SMAPE 7,83, MAE 0,36 e RRMSE 0,20, só ficando atras do resultado do MAE e RRMSE do modelo AR que apresento os valores mais baixo nesse horizonte de previsão, mas o modelo DT no geral até a previsão de 24 horas ou um dia à frente se mostrou superior aos outros modelos, na previsão de um dia à frente o Prophet se mostrou melhor com os resultados de SMAPE 5,05, MAE 0,17 e RRMSE 0,19 . Conclui-se que a abordagem proposta contribui significativamente para a previsão da demanda de água, proporcionando um planejamento eficiente e sustentável do abastecimento hídrico no Bairro Alto, além do que a previsão permite antecipar e prevenir possíveis escassezes de água, prevendo a demanda futura, é possível adotar medidas proativas para evitar interrupções no fornecimento.

%\noindent Este estudo, inserido no contexto do abastecimento de água em Curitiba, exemplifica uma abordagem abrangente para enfrentar os desafios crescentes da demanda de água. Focando especificamente na eficácia da previsão da demanda no Bairro Alto, o objetivo primordial é garantir que a infraestrutura existente possa atender às necessidades em constante crescimento da população, evitando problemas de oferta inadequada. A análise dos modelos de previsão de séries temporais é conduzida para proporcionar uma análise precisa da previsão da demanda de água.
%Utilizando dados coletados pela Companhia de Saneamento do Paraná durante os anos de $2018$ a $2019$, este estudo propõe uma contribuição significativa para o controle eficaz dos recursos hídricos. A ênfase recai sobre uma variedade de modelos de previsão, incluindo \textit{Auto-Regressive} , \textit{Auto-Regressive with Exogenous Inputs}, \textit{Moving Average}, \textit{Auto-Regressive Moving Average}, \textit{Seasonal Auto-Regressive Integrated Moving Average}, ARIMA com \textit{Exogenous Inputs}, e \textit{Seasonal} ARIMA com \textit{Exogenous Inputs}.
%Além desses modelos, o estudo destaca especialmente o \textit{Recurrent Neural Network} (RNN), que demonstrou consistentemente os menores erros em todas as análises.
%Explorando horizontes de previsão de uma hora à frente, seis horas à frente, doze horas à frente e vinte e quatro horas à frente, os resultados detalhados mostram as diferentes performances desses modelos.
%Para previsões de 1 hora à frente, o RNN atinge um \textit{Symmetric Mean Absolute Percentage Error} (SMAPE) de $0,1647$, \textit{Mean Absolute Error} (MAE) de $0,0057$ e \textit{Root Relative Mean Square Error} (RRMSE) de $0,0022$.
%No horizonte de 6 horas à frente, o RNN demonstra um SMAPE de 0,1356, MAE de $0,0046$ e RRMSE de $0,0022$, destacando sua precisão nesse intervalo de tempo.
%Ao considerar previsões de 12 horas à frente, o RNN alcança um SMAPE de $0,1343$, MAE de $0,0046$ e RRMSE de $0,0021$, indicando sua consistência em horizontes temporais mais extensos.
%Para projeções de 24 horas à frente, o RNN mantém um desempenho notável, com um SMAPE de $0,2231$, MAE de $0,0077$ e RRMSE de $0,0028$.
%Esses resultados ressaltam a confiabilidade do RNN na previsão da demanda de água em diferentes escalas de tempo, fornecendo assim um planejamento eficiente e sustentável do abastecimento hídrico na região.

\noindent O estudo, inserido no contexto do abastecimento de água em Curitiba, concentra-se na eficácia da previsão da demanda no Bairro Alto, por meio dos dados coletados pela SANEPAR (Companhia de Saneamento do Paraná) durante os anos de $2018$ a $2020$. A questão central investigada é a eficácia das previsões de abastecimento de água, visando assegurar que a infraestrutura existente seja capaz de atender às crescentes necessidades da população, prevenindo problemas de oferta inadequada. O propósito deste estudo é contribuir para o controle eficaz dos recursos hídricos, utilizando modelos de previsão, com ênfase na melhoria do abastecimento d'água.
São explorados modelos de previsão, tais como \textit{Auto-Regressive} (AR), \textit{Auto-Regressive with Exogenous Inputs}, \textit{Moving Average}, \textit{Auto-Regressive Moving Average}, \textit{Seasonal Auto-Regressive Integrated Moving Average}, ARIMA com \textit{Exogenous Inputs}, e \textit{Seasonal} ARIMA com \textit{Exogenous Inputs}, \textit{Decision Tree} (DT), \textit{eXtreme Gradient Boosting}, e \textit{Recurrent Neural Network} para a previsão de séries temporais, com análise comparativa de eficácia dos modelos de previsão.
O desempenho dos modelos de previsão é avaliado por meio de métricas que incluem \textit{Symmetric Mean Absolute Percentage Error} (SMAPE), \textit{Mean Absolute Error} (MAE) e \textit{Root Relative Mean Square Error} (RRMSE). Nota-se que, no geral, o modelo DT se apresentou melhor nas previsões de curto prazo, destacando-se com resultados de SMAPE $13,50$, MAE $0,58$ e RRMSE $0,16$ para a previsão de $6$ horas à frente. Com $1$ hora à frente, o resultado foi de SMAPE $7,83$, MAE $0,36$ e RRMSE $0,20$, ficando atrás apenas dos valores de MAE e RRMSE do modelo AR, que apresentou os valores mais baixos nesse horizonte de previsão. No entanto, o modelo DT, no geral, até a previsão de $24$ horas ou um dia à frente, se mostrou superior aos outros modelos. Na previsão de um dia à frente, o Prophet se destacou, apresentando resultados de SMAPE $5,05$, MAE $0,17$ e RRMSE $0,19$.
Conclui-se que a abordagem proposta contribui significativamente para a previsão da demanda de água, proporcionando um planejamento eficiente e sustentável do abastecimento hídrico no Bairro Alto. Além disso, a previsão permite antecipar e prevenir possíveis escassezes de água, prevendo a demanda futura e possibilitando a adoção de medidas proativas para evitar interrupções no fornecimento.

	
	\hspace{1cm}
	
	
	\noindent \textbf{Palavras-chave:} Previsão, Séries Temporais, Abastecimento de Água, Aprendizado de Máquina, Redes Neurais Artificiais.
\end{abstract}

