\begin{abstract} 
	
	\noindent Este estudo aborda a relevância estratégica da previsão precisa da demanda de água para uma gestão eficaz dos recursos hídricos em um cenário competitivo. O desafio identificado é a carência de previsões confiáveis, dificultando decisões estratégicas no fornecimento de água. A abordagem proposta consiste em empregar modelos avançados de previsão de séries temporais para aprimorar a precisão das estimativas de demanda.	
	Por meio de uma revisão da literatura, diversos métodos e abordagens empregados na previsão de séries temporais para abastecimento de água são explorados. Esses métodos incluem modelos de redes neurais como GRU (Gated Recurrent Unit), LSTM (Long Short-Term Memory), RNN (Recurrent Neural Network) e Transformer, além do modelo especializado Prophet, bem como a Regressão em Árvore de Decisão.	
	Com base nesse conhecimento, métodos e produtos específicos são propostos, considerando variáveis exógenas, sazonalidade e utilizando modelos autorregressivos integrados de médias móveis (ARIMA), técnicas de boosting como XGBoost e LightGBM, regressão linear e Random Forest Regression (RFR). A performance dessas abordagens é avaliada por métricas como sMAPE, MAE e RRMSE, proporcionando insights sobre a eficácia dos modelos de previsão para o abastecimento de água.	
	Essas descobertas contribuem para uma tomada de decisão mais embasada e eficiente nessa área, informando sobre a eficácia dos modelos de previsão de séries temporais para o abastecimento de água.



    \noindent \textbf{Palavras-chave:} Previsão, Economia de água, Séries temporais, Modelos de Previsão.
\end{abstract}

