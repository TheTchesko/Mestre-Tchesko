\begin{abstract} 
	\noindent O estudo, inserido no contexto do abastecimento de água em Curitiba, concentra-se na eficácia da previsão da demanda no Bairro Alto através dos dados coletados pela SANEPAR  (Companhia de Saneamento do Paraná) durante os anos de $2018$ a $2020$. A questão central investigada é as previsões de abastecimento de água para ajudar a garantir que a infraestrutura existente seja capaz de atender às necessidades crescentes da população, evitando problemas de oferta inadequada. Assim, o propósito deste estudo é contribuir para o controle eficaz dos recursos hídricos, utilizando modelos de previsão, com ênfase na melhoria do abastecimento d'água. São explorados modelos de previsão tais como \textit{Auto-Regressive Integrated Moving Average} (ARIMA), \textit{Decision Tree} (DT), \textit{eXtreme Gradient Boosting} (XGBoost) e \textit{Recurrent Neural Network} (RNN) para a previsão de séries temporais, com análise comparativa de eficácia dos modelos de previsão. O desempenho dos modelos de previsão são avaliadas por meio de métricas que incluem \textit{Symmetric Mean Absolute Percentage Error} (SMAPE), \textit{Mean Absolute Error} (MAE) e \textit{Root Relative Mean Square Error} (RRMSE), destacando que o modelo RNN demonstrou consistentemente os menores erros em todas as análises. Conclui-se que a abordagem proposta contribui significativamente para a previsão da demanda de água, proporcionando um planejamento eficiente e sustentável do abastecimento hídrico no Bairro Alto, além do que a previsão permite antecipar e prevenir possíveis escassezes de água, prevendo a demanda futura, é possível adotar medidas proativas para evitar interrupções no fornecimento.
	
	\hspace{1cm}
	
	
	\noindent \textbf{Palavras-chave:} Previsão, Séries Temporais, Abastecimento de Água, Aprendizado de Máquina, Redes Neurais Artificiais.
\end{abstract}

