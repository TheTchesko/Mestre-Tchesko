\begin{abstract} 
	
	\noindent Este estudo aborda a importância da previsão precisa da demanda de água como uma ferramenta estratégica para a gestão eficiente dos recursos hídricos em um cenário competitivo.	
	O problema identificado é a falta de previsões precisas, o que dificulta a tomada de decisões estratégicas no abastecimento de água.	
	A solução proposta é a utilização de modelos avançados de previsão de séries temporais para melhorar a precisão das previsões de demanda.	
	A partir de uma revisão abrangente da literatura existente, são analisados diferentes métodos e abordagens utilizados na previsão de séries temporais no contexto do abastecimento de água. O estado da arte é explorado para identificar os modelos mais eficazes e as melhores práticas na área.	
	Com base no estado da arte, são propostos métodos e produtos específicos para a previsão de demanda de água. Esses métodos levam em consideração variáveis exógenas, sazonalidade dos dados e utilizam modelos autorregressivos integrados de médias móveis (ARIMA) e técnicas de boosting, como o XGBoost (Extreme Gradient Boosting), e regressão linear.	
	Os resultados obtidos por meio da aplicação desses métodos e produtos propostos são analisados e comparados com métricas de desempenho, como erro percentual absoluto médio (MAPE), erro absoluto médio (MAE) e raiz quadrada do erro quadrático médio (RMSE). Esses resultados fornecem informações importantes sobre a eficácia dos modelos de previsão de séries temporais no abastecimento de água e contribuem para a tomada de decisões mais informadas e eficientes nessa área.




    \noindent \textbf{Palavras-chave:} Previsão, Economia de água, Séries temporais, Modelos de Previsão.
\end{abstract}

