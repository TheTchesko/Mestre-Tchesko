\begin{abstract} 
%	\noindent Este estudo explora a previsão de séries temporais para a tomada de decisões relacionadas à demanda d'água, visando de alguma forma auxiliar no controle eficaz dos recursos hídricos em um ambiente competitivo.
%	Relacionado ao fornecimento de água.
%	A abordagem desse estudo envolve o problema de abastecimento d'água que afetou a cidade de Curitiba no Bairro Alto entre os anos de $2018$ e $2020$. Durante esse período, os habitantes enfrentaram a escassez de água, sendo necessário implementar rodízios, alternando períodos com e sem fornecimento. Os dados utilizados foram coletados pela companhia de saneamento do Paraná (SANEPAR).	
%	A previsão da demanda de água ao longo do tempo é essencial para um planejamento sustentável e eficiente do abastecimento hídrico, especialmente no contexto urbano, como é o caso da cidade de Curitiba, no estado do Paraná. Neste estudo, adotou-se alguns modelos de previsão da área de aprendizado de máquina e o modelo clássico ARIMA (\textit{Auto-Regressive Integrated Moving Average}) para realizar previsões diárias da demanda de água ao longo do tempo.	
%	Os modelos de aprendizado de máquina foram aplicados na previsão de séries temporais com dados coletados pela SANEPAR. Cada modelo tem suas particularidades, mas os modelos de aprendizado de máquina podem ser otimizados da mesma forma que os modelos clássicos do tipo ARIMA e RNN (\textit{Recurrent Neural Network}). Os dados coletados pela SANEPAR, utilizados para previsão, referem-se ao abastecimento d'água no Bairro Alto durante o período de 2018 a 2019.	
%	A eficácia dos modelos é  avaliada por meio de métricas como sMAPE (do inglês \textit{Symmetric Mean Absolute Percentage Error}), MAE (do inglês \textit{Mean Absolute Error}) e RRMSE (do inglês \textit{Root Relative Mean Square Error}).	
%	A análise e comparação de todos os casos, ficou evidente que o modelo RNN obteve o menor erro em todas as métricas avaliadas, incluindo-se SMAPE, MAE e RRMSE. É interessante notar que o desempenho do modelo RNN foi excepcional, com erros de previsão consistentemente menor que 1\% em todas as análises. 
	
	
	\noindent O estudo, inserido no contexto do abastecimento de água em Curitiba, concentra-se na eficácia da previsão da demanda no Bairro Alto durante os anos de $2018$ a $2020$. A questão central investigada é como antecipar a demanda de água para um planejamento mais eficiente no contexto de escassez enfrentada pelos habitantes. O propósito do trabalho é contribuir para o controle eficaz dos recursos hídricos, utilizando modelos de previsão, com ênfase na melhoria do abastecimento d'água em um ambiente competitivo.	
	São explorados modelos como ARIMA (do inglês \textit{Auto-Regressive Integrated Moving Average}), DTR (do inglês \textit{Decision tree regressor}), XGBoost (do inglês \textit{eXtreme Gradient Boosting}) e RNN (do inglês \textit{Recurrent Neural Network}) para a previsão de séries temporais, com uma análise comparativa de eficácia. A necessidade de uma solução nova ou melhor surge da escassez de água enfrentada no Bairro Alto, justificando a busca por métodos  eficientes de previsão da demanda.	
	A solução proposta, envolve a aplicação de modelos de aprendizado de máquina, como ARIMA, DTR, XGBoost e especialmente RNN, na previsão da demanda de água. A metodologia básica inclui a aplicação desses modelos aos dados coletados pela SANEPAR  (Companhia de Saneamento do Paraná).	
	As características que respondem às questões iniciais são avaliadas por meio de métricas como sMAPE (do inglês \textit{Symmetric Mean Absolute Percentage Error}), MAE (do inglês \textit{Mean Absolute Error}) e RRMSE (do inglês \textit{Root Relative Mean Square Error}), destacando que o modelo RNN demonstrou consistentemente os menores erros em todas as análises. Conclui-se que a abordagem proposta contribui significativamente para a previsão da demanda de água, proporcionando um planejamento eficiente e sustentável do abastecimento hídrico no Bairro Alto.
	
	
	%	A abordagem proposta envolve a utilização de modelos de previsão de séries temporais para melhorar a precisão das estimativas de demanda de água.
	%	Neste estudo, o abastecimento de água pela SANEPAR (Companhia de Saneamento do Paraná) para o Bairro Alto da cidade de Curitiba é explorado.	
	%	Avaliando-se modelos de redes neurais artificiais, como GRU (do inglês \textit{Gated Recurrent Unit}), LSTM (do inglês \textit{Long Short-Term Memory}), RNN (do inglês \textit{Recurrent Neural Network}), \textit{Transformer} e Facebook Prophet. Além disso, modelos do tipo ARIMA (do inglês \textit{Auto-Regressive Integrated Moving Average}), técnicas de \textit{boosting} como XGBoost (do inglês \textit{eXtreme Gradient Boosting}) e LightGBM (do inglês \textit{Light Gradient Boosting Machine}), regressão linear e RFR (do inglês \textit{Random Forest Regression}).
\hspace{1cm}


    \noindent \textbf{Palavras-chave:} Previsão de Séries Temporais, Abastecimento de Água, Aprendizado de Máquina, Redes Neurais Artificiais, Modelos de Previsão.
\end{abstract}

