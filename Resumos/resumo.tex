\begin{abstract} 
	
	\noindent Em um cenário competitivo, a previsão assertiva de demanda tem se tornado cada vez mais uma ferramenta estratégica para diversas áreas organizacionais. Nesse contexto, a previsão de séries temporais desempenha um papel importante na tomada de decisões. Recentemente, a capital paranaense enfrentou uma grave crise na área da saúde, com períodos de desabastecimento que geraram instabilidade na oferta de moradia para muitas famílias.		
	Na abordagem para solucionar esse problema, foram utilizados métodos encontrados na revisão bibliográfica realizada durante este trabalho. Os métodos escolhidos são utilizados como uma forma de tomada de decisão para a demanda de água. Cada método escolhido tem a capacidade de lidar com o problema de maneira diferente e oferecer soluções viáveis para a tomada de decisão.		
	Com o objetivo de mitigar e tomar as melhores decisões para o problema enfrentado pela Companhia de Saneamento do Paraná (SANEPAR) em 2020 e evitar surpresas no futuro, este trabalho visa aprimorar o uso da água. Embora o evento isolado ocorrido nesse ano possa não se repetir nos próximos anos, é importante buscar melhorias na gestão dos recursos hídricos.		
	Os métodos derivados do modelo autorregressivo integrado de médias móveis (ARIMA), incluindo aqueles com variáveis exógenas e considerando a sazonalidade dos dados, são os modelos de previsão mais eficazes para modelar os dados com variações temporais, embora cada método tenha suas peculiaridades, todos são baseados no modelo ARIMA inicial.		
	Os modelos de boosting, como XGBoost (\textit{Impulso Extremo de Gradiente}), seguidos pelo modelo de regressão linear simples (LR), são considerados os melhores modelos para séries temporais devido ao uso da abordagem de gradient boosting para previsões. Essa escolha é baseada em métricas de erro, em que um menor valor indica uma melhor capacidade de tomada de decisão. As métricas adotadas neste artigo são o erro percentual absoluto médio (MAPE), o erro absoluto médio (MAE) e a raiz quadrada do erro quadrático médio (RMSE). Em séries temporais, essas métricas são comumente utilizadas para avaliar a eficácia dos modelos de previsão em diferentes circunstâncias e horizontes de previsão. O modelo XGBoost apresentou um erro de \SI{0.264}{\percent} no MAPE, o menor entre os modelos avaliados, enquanto o modelo LR obteve o maior erro de \SI{5}{\percent} no horizonte de previsão mais longo (um mês). O modelo de médias móveis (MA) obteve um erro de \SI{0.113}{\percent} no MAPE, enquanto o modelo LR apresentou um erro de \SI{5}{\percent}. Assim, o modelo LR pode ser mais eficiente para conjuntos de dados menores, trabalhando com um volume reduzido de dados, enquanto os erros aumentam à medida que o horizonte de previsão aumenta.

	
	 

    \noindent \textbf{Palavras-chave:} Previsão, Economia de água, Séries temporais, Revisão sistemática da literatura.
\end{abstract}

