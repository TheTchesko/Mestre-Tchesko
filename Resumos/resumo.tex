\begin{abstract} 
	\noindent Este estudo explora a importância da previsão de séries temporais para a tomada de decisões relacionadas à demanda de água, visando um controle eficaz dos recursos hídricos em um ambiente competitivo. O desafio reside na obtenção de séries temporais confiáveis que auxiliem nas decisões sobre o fornecimento de água. A abordagem proposta envolve a utilização de modelos de previsão de séries temporais para melhorar a precisão das estimativas de demanda.
	Existem várias abordagens discutidas na literatura para a análise e previsão de séries temporais no campo do abastecimento de água. Neste estudo, o caso da SANEPAR (Companhia de Saneamento do Paraná) é examinado como exemplo representativo. No entanto, o que torna este estudo único é a introdução e avaliação personalizada de modelos de redes neurais, como GRU (do inglês \textit{Gated Recurrent Unit}), LSTM (do inglês \textit{Long Short-Term Memory}), RNN (do inglês \textit{Recurrent Neural Network}) e Transformer na forma personalizada para esse problema, além do modelo do Facebook Prophet, que não foram aplicados a esse contexto até então. A técnica de regressão em árvore de decisão também é explorada. Esses métodos inovadores expandem as possibilidades na previsão da demanda por água.
	Com base nesse conhecimento, métodos e produtos específicos são analisados, levando em consideração fatores externos e sazonalidade, além de usar modelos ARIMA (do inglês \textit{Auto-Regressive Integrated Moving Average}), técnicas de \textit{boosting} como XGBoost (do inglês \textit{eXtreme Gradient Boosting}) e LightGBM (do inglês \textit{Light Gradient Boosting Machine}), regressão linear e RFR (do inglês \textit{Random Forest Regression}). A eficácia dessas abordagens é avaliada por métricas como sMAPE (do inglês \textit{Symmetric Mean Absolute Percentage Error}), MAE (do inglês \textit{Mean Absolute Error}) e RRMSE (do inglês \textit{Root Relative Mean Square Error}), fornecendo informações sobre a capacidade dos modelos de previsão no fornecimento de água.
	Esses resultados são úteis para tomar decisões mais informadas no contexto da empresa SANEPAR. Eles fornecem informações sobre como os modelos de previsão de séries temporais se saem em relação ao abastecimento de água.
	A análise e comparação de todos os casos, ficou evidente que o modelo RNN obteve o menor erro em todas as métricas avaliadas, como SMAPE, MAE e RRMSE. É interessante notar que o desempenho do modelo RNN foi excepcional, com erros de previsão consistentemente abaixo de 1\% em todas as análises. Isso destaca que ele é o modelo mais eficiente e preciso em todas as aplicações avaliadas.

\hspace{1cm}


    \noindent \textbf{Palavras-chave:} Previsão de séries temporais, Economia de água, Séries temporais, Modelos de Previsão.
\end{abstract}

