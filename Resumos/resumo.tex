\begin{abstract}    
     \noindent Previsão de séries temporais é muito importante para a tomada de decisão, fazendo modelos que pode ser previsto e melhorar assim a tomada de decisão baseado nos modelos.
     Nessa dissertação será abordado o problema de demanda de água que ocorreu na cidade de Curitiba no estado do Paraná, houve no período dos dados coletados nos anos de 2018 a 2020, visando o ano de 2020 que foi o ano que ocorreu a maior demanda de água, fazendo com que os reservatórios sofressem com isso, vários fatores, como por exemplo a chuva que não foi devidamente suficiente para atender nesse ano em questão. 
     Na tomada de decisão desse problema em questão é usado alguns métodos encontrado na revisão que foi realizado no decorrer do mestrado, para que possa ser previsto em alguns horizonte de previsão, os horizonte abordado aqui é uma forma de poder resolver a questão da demanda de água e junto com isso validar os modelos para ver qual deles é o mais eficiente, horizonte adotado foi de previsão de 1, 10, 30 e 60 dias a frente, assim pode ser visto como cada método vai lidar com os dados no decorrer dos anos.
     Afim de amenizar e solucionar o problema que a empresa SANEPAR enfrentou no ano de 2020, para que não ocorra mais ou que não pegue desprevenido no próximo evento que pode surgir. Com o evento isolado que aconteceu no ano em questão e não possa se repetir nos anos futuros, esse trabalho visa a melhoria do usa d'água.     
     Os métodos ARIMA e os ARIMA atualizados com por exemplo alguns modelos derivado dos modelos ARIMA, assim listando os modelos é AR, ARX, MA, ARMA, ARIMA, SARIMA, SARIMAX e ARIMAX, como cada modelo tem sua particularidade os modelos de variáveis exógenas pode parecer graficamente melhor de ser previsto do que os modelos de ARIMA sem a variáveis exógenas. Esses modelos são chamado de modelos tradicionais de previsão de tempo, para inicio de um processamento de dados, nos modelos de reforço de gradiente é os melhores modelos para se prever com os erros mais baixo, em hipótese é o que busca nesse trabalha. Os modelos chamado de reforço ou árvore de regressão de gradiente, foi usado os seguintes modelos LR, regressão florestal aleatória (do inglês random forest regression) XGBRegressor (XGboost) e LGBMRegressor (Light GBM), esses modelos para série temporal é listado como os melhores modelos, pois alguns deles usa a forma de prever de gradiente.     
     É obtido em algumas métricas de erros, quanto menor o erro melhor para a tomada de decisão. As métricas adotado nesse trabalho é MAPE, MAE e RMSE, em serie temporal essas métricas são mais frequente do que outras e para não colocar várias nesse trabalho adotamos apenas essas três, tendo com modelos de previsão melhor ou mais eficaz em algumas circunstância com na previsão de nenhum horizontem futuro ou apensa a previsão da série obtida nos dados o modelo XGBoost tem erro de $0,079\%$ na métrica MAPE só olhando para ela por enquanto, e o LR com tendo o maior erro de $21\%$  no horizonte maior de previsão (60 dias) o modelo ARMA vem com o erro de $12,54\%$ e o modelo LR com $11153,594\%$.

    \noindent \textbf{Palavras-chave:} Previsão, Economia de água, Séries temporais, Série cronológica.
\end{abstract}

