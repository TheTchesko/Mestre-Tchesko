\begin{abstract}    
     \noindent CONTEXTO -- Previsão de séries temporais é muito importante para a tomada de decisão, fazendo modelos que pode ser previsto e melhorar assim a tomada de decisão baseado nos modelos.
     
     PROBLEMA -- Nessa dissertação será abordado o problema de demanda de água que ocorreu na cidade de Curitiba -- Paraná, houve no período dos dados coletados nos anos de 2018 a 2020, visando o ano de 2020 que foi o ano que ocorreu a maior demanda de água, fazendo com que os reservatórios sofressem com isso, vários fatores, como por exemplo a chuva que não foi devidamente suficiente para atender nesse ano em questão. 
     
     SOLUÇÃO -- Na tomada de decisão desse problema em questão é usado alguns métodos encontrado na revisão que foi realizado no decorrer do mestrado, para que possa ser previsto em alguns horizonte de previsão, os horizonte abordado aqui é uma forma de poder resolver a questão da demanda de água e junto com isso validar os modelos para ver qual deles é o mais eficiente, horizonte adotado foi de previsão de 1, 10, 30 e 60 dias a frente, assim pode ser visto como cada método vai lidar com os dados no decorrer dos anos.
     
     ESTADO DA ARTE -- 
     
     MÉTODO(S)/PRODUTO(S) PROPOSTO(S) --
     
     RESULTADOS -- 
     \\
    \\
    \\
    
    \noindent \textbf{Palavras-chave:} Previsão, Economia de água, Séries temporais, Série cronológica.
\end{abstract}

