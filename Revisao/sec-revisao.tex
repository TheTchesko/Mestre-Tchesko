\section{Referencial}\label{sec:refteo}


Este capítulo apresenta o referencial teórico que serviu de base para a elaboração desta dissertação. Embora os resultados obtidos possam ser considerados mais modestos em comparação a uma tese, eles ainda são relevantes para o trabalho realizado aqui. A revisão bibliográfica realizada consiste em uma análise abrangente e crítica das principais fontes de literatura relacionadas ao tema em questão. Por meio dessa revisão, busca-se obter uma compreensão aprofundada do estado atual do conhecimento na área e identificar lacunas ou oportunidades de pesquisa. Os insights e informações extraídos da literatura são fundamentais para embasar a fundamentação teórica, a metodologia e a análise dos resultados desta dissertação. Dessa forma, a revisão bibliográfica desempenha um papel crucial no embasamento teórico e na contextualização do trabalho, fornecendo um sólido alicerce para o desenvolvimento e contribuição desta pesquisa.


 

\subsection{Detec\c cão de anomalias} \label{subsec:detec}


Detectar anomalias em séries temporais representa um desafio significativo para os previsores, pois requer habilidade em identificar mudanças nos dados mesmo quando não estão claramente evidentes. Nesse contexto, a coleta de dados realizada ao longo do tempo pela empresa SANEPAR revela anomalias mais expressivas do que inicialmente imaginado. A escassez de água que afetou a cidade de Curitiba se prolongou por vários dias, como evidenciado pelos gráficos de linha utilizados na etapa de trabalho mencionada (\ref{etp:1}). Esses gráficos oferecem uma representação visual clara das variações nos níveis de água ao longo do tempo, auxiliando na compreensão da extensão do problema e na necessidade de uma abordagem adequada.

\begin{figure}[H]
	\centering
	\caption{Dados completos com uma frequência média de 24 horas}
	\label{fig:dados-todos}
	\includegraphics[width=0.9\linewidth]{"Introducao/Figuras/dados todos"}
	
	Fonte: Elaboração própria a partir de dados da SANEPAR (2018 a 2020)
\end{figure}

\begin{figure}[H]
	\centering
	\caption{Plotagem de dados para o ano de 2020}
	\label{fig:2020-a-frente}
	\includegraphics[width=0.9\linewidth]{"Introducao/Figuras/2020 a frente"}
	
	Fonte: Elaboração própria a partir de dados da SANEPAR (2018 a 2020)
\end{figure}

Os dados coletados possuem uma dimensão de $26306$ linhas  $9$ colunas. Essa ampla quantidade de dados será utilizada nos modelos descritos na subseção \ref{subsec:metod} para que seja possível prever e analisar as anomalias evidenciadas nas Figuras \ref{fig:dados-todos} e \ref{fig:2020-a-frente}. Essas figuras ilustram visualmente as variações e padrões observados nos dados ao longo do tempo, destacando a importância de explorá-los de maneira apropriada a fim de compreender as anomalias e embasar a tomada de decisões.








\subsection{Revis\~ao sistem\'atica da literatura} \label{subsec:revisão}

As séries temporais aparecem em vários campos do conhecimento, tais como Economia (preços de estoque diários, taxa de desemprego mensal, produção industrial), Medicina (eletrocardiograma, eletroencefalograma), Epidemiologia (número mensal de novos casos de meningite), Meteorologia (chuvas, temperatura diária, velocidade do vento), etc. Ao longo dos anos tem usado ferramentas computacionais para tornar esta previsão mais eficiente, com aprendizagem de máquinas e algumas características que podem ser aplicadas em linguagem computacional através da linguagem \textit{python e R}, as melhores linguagens para trabalhar com séries temporais hoje em dia.

Para entender melhor este conceito de série temporal, suponhamos que um maratonista que esteja correndo há vários anos e uma pessoa sedentária se submeta a uma corrida de, no máximo, $5$ km, ambos corram ao mesmo tempo para que tenham um monitor de frequência cardíaca para que possa ser monitorado por um médico se você pegar os dados desde o início e compará-los com o final da corrida, o maratonista terá uma série mais estacionária porque ele tem o hábito de correr regularmente enquanto a pessoa sedentária terá uma série não estacionária como mostrado na Figura \ref{fig:series}.


\begin{figure}[H]
	\centering
	\caption{Exemplo de séries temporais}
	\label{fig:series}
	\includegraphics[width=1\linewidth]{Revisao/Figuras/séries}
	
	Fonte: \cite{brandão_2020}
\end{figure}


Na figura \ref{fig:series} observa-se que o eixo $x$ representa os dados observados e $t$ para o tempo percorrido.
Além disso, as séries temporais são processos estocásticos por leis probabilísticas, o que significa que há a possibilidade de ser pensado como um conjunto de todas as trajetórias possíveis na Figura \ref{fig:series} é capaz de ser observado para uma variável alvo. Por exemplo, se você lançar um dado qualquer valor inteiro entre 1 e 6, mas apenas um número ocorrerá. Da mesma forma, em séries temporais existem infinitas possibilidades, entre elas apenas uma de acordo com as características que atenderam a esse período e que de fato ocorrerão.

\begin{figure}[H]
	\centering
	\caption{Processo estocástico}
	\label{fig:serie}
	\includegraphics[width=0.8\linewidth]{Revisao/Figuras/serie}
	
	\fonte{Adaptado de \citeonline{pinheiro_2022}}
\end{figure}

Com $Y(t)$ os dados fictícios e $Tempo \ (t)$ a linha do tempo da Figura \ref{fig:series}.

De repente é pensado como um conjunto de todas as trajetórias possíveis que poderiam ser para observar uma variável.


Esta revisão sistemática da literatura, com o tema abordado até agora é sobre séries temporais, considerando o contexto aqui exposto este tema pode ser de grande relevância em diversas áreas, como mostrado na Figura \ref{fig:areas}. Realizando esta análise de séries temporais nos últimos 6 anos para poder observar as melhores realizações neste tema abordado aqui um curto período, mas tendo o tempo não muito a favor, então teve a opção de deixar este tempo específico para buscar artigos.

O objetivo desta revisão é analisar uma literatura menor, mas muito relevante. Como a própria série temporal procura analisar e modelar a dependência e considerando a ordem apresentada nas bases, por exemplo, os maiores autores e o ano de atividade que mais publicaram nos países que têm o maior número de publicações na apresentação das palavras-chave que serão mostradas, o objetivo é rever cada coisa que pode ser usada em uma aplicação de aprendizagem de máquina.

Em todos os artigos observados que tem uma contribuição científica neste trabalho é a análise do conceito de série temporal com o melhor uso das palavras-chave mesmo não tendo uma grande relação na aprendizagem de máquinas podem ser usados estes artigos como base para outros pesquisadores, aqui algumas análises muito simples para alguns leitores. Entretanto, é um ponto de partida para muitos que não conhecem o conceito de séries cronológicas ou revisão sistemática da literatura.


\subsection{Problematiza\c c\~ao da Revis\~ao} \label{subsec: problematização da revisão}

Nesta subseção, é discutido um problema de pesquisa que pode ser compreendido por diversos leitores. A Figura \ref{fig:serie-temporal} apresenta um mapa conceitual das publicações, destacando a importância dos autores como base para esta revisão. Os modelos propostos por esses autores são fundamentais para abordar o problema em questão, uma vez que a previsão em séries temporais é um desafio de grande significado por si só.

\begin{figure}[H]
	\centering
	\caption{Mapa conceitual do problema de pesquisa}
	\label{fig:serie-temporal}
	\includegraphics[width=0.9\linewidth]{Revisao/Figuras/"Série temporal"}
	
	Fonte: Elaboração própria 
\end{figure}

O mapa conceitual apresentado na Figura \ref{fig:serie-temporal} ilustra a relação entre as palavras-chave que estão relacionadas ao problema em questão, proporcionando uma visão clara do que será abordado ao longo do trabalho. Esse mapa contribui para a identificação dos principais tópicos de pesquisa e das questões que serão exploradas posteriormente.

\begin{enumerate}[start=1, label = {\textbf{Q} \arabic* } ]
	\item \label{questão:rev1}Quais os autores que mais publicam sobre o assunto de séries temporais?
	\item \label{questão:rev2}Quais os países que mais publicam sobre o assunto? 
	\item \label{questão:rev3}Quais as áreas que mais publicam sobre o tema?
	\item \label{questão:rev4}Quais são as obras mais influentes na análise de séries temporais?
\end{enumerate}

\subsection{Metodologia}\label{subsec:met da revisão}

Nesta subseção, é fornecida uma explicação detalhada de como a revisão foi conduzida, abrangendo desde a análise do banco de dados até a conclusão final da revisão. São apresentados os passos e critérios adotados para a seleção dos artigos, bem como os procedimentos utilizados para a extração e análise dos dados. A subseção visa esclarecer de forma clara e objetiva todo o processo metodológico empregado durante a realização da revisão.

\begin{figure}[H]
	\centering
	\caption{Etapas da Revisão.}
	\label{fig:rsl}
	\includegraphics[width=0.9\linewidth]{Revisao/Figuras/RSL}
	
	Fonte: Adaptado de \citeonline{MARTINS201671}
\end{figure}


\begin{enumerate}[start=1, label={\textbf{Etapa} \arabic*}]
	
	\item \label{etp:rev-1} A Figura \ref{fig:rsl} apresenta uma adaptação da metodologia proposta por \citeonline{MARTINS201671} para a realização desta revisão sistemática. Inicialmente, foram realizadas buscas nos bancos de dados Scopus, Web of Science e Lens, selecionando algumas bases relevantes para o tema da pesquisa.
	
	
\textbf{Scopus campo de busca}

\textbf{\textit{TITLE-ABS-KEY (``time series forecasting")  AND  TITLE-ABS-KEY (``time series analysis")  AND  ( LIMIT-TO ( DOCTYPE ,  ``ar" ) )  AND  ( LIMIT-TO ( LANGUAGE ,  ``English" ) )  AND  ( LIMIT-TO ( PUBYEAR ,  2022 )  OR LIMIT-TO ( PUBYEAR ,  2021 )  OR  LIMIT-TO ( PUBYEAR ,  2020 )  OR  LIMIT-TO ( PUBYEAR ,  2019 )  OR  LIMIT-TO ( PUBYEAR ,  2018 )  OR  LIMIT-TO ( PUBYEAR ,  2017 ) )}}

\textbf{Web of Science campo de busca}

\textit{\textbf{``times series forecasting" (All Fields) and ``time series analysis" (All Fields)}} (Publication Years: 2022 or 2021 or 2020 or 2019 or 2018 or 2017) (Document Types: Articles) (Languages: English)

\textbf{Lens campo de busca}

\textit{\textbf{Scholarly Works (11) = ( ``time series forecasting" ) AND ( ( ``time series analysis" ) AND ( ``nonlinear forecasting" ) ) }}
Filters: Year Published = ( 2016 - 2022  ) Publication Type = ( journal article  )\\
	
	Para todas as bases de busca, foram considerados os últimos 6 anos, com exceção do Lens, que retornava poucos artigos. Nesta etapa, foram utilizadas palavras-chave que se adequam melhor à pesquisa, como \textit{time series forecasting and time series analysis and nonlinear forecasting}.
	
	\item \label{etp:rev-2} No cruzamento das palavras-chave, obteve-se um número considerável de artigos, sem restringir a área em que cada um pode ser publicado. A Tabela \ref{tb1} apresenta a tabulação dos resultados obtidos, sem excluir duplicatas, que serão tratadas na seção \ref{subesec:resul da revisão}.
	
	\item \label{etp:rev-3} Nesta etapa, é realizada uma avaliação preliminar de cada artigo obtido, sem aplicar nenhum filtro anual nas buscas. Analisar todos os artigos dessa maneira resultaria em um número elevado, por exemplo, no banco de dados Scopus seriam 498 artigos, na Web of Science seriam 140 artigos e no Lens, que retorna poucos artigos, seriam 11 artigos, totalizando 649 artigos sem remover duplicatas. É importante ressaltar que esses artigos passaram apenas pelo filtro de idioma inglês e de serem artigos, visando aprimorar a busca e a tomada de decisões. Ao aplicar o filtro dos últimos 6 anos, obteve-se um número mais gerenciável de artigos para análise. Levando em consideração a diferença entre essa estimativa apresentada na Tabela \ref{tb1} e a quantidade de artigos restantes após a remoção de duplicatas, temos menos de 356 artigos para análise. É válido lembrar que, ao remover as duplicatas, esse número pode diminuir ainda mais, atingindo o objetivo proposto neste trabalho.
	
	\item \label{etp:rev-4} Nesta etapa, é realizada uma análise mais aprofundada do conteúdo dos artigos selecionados, levando em consideração as áreas de especialização e correlação com séries temporais. Como esta revisão está inserida no contexto de um programa de mestrado em Engenharia de Produção e Sistemas, vale a pena analisar a correlação com áreas como Matemática. A Figura \ref{fig:areas} mostra que as áreas mais relevantes para a pesquisa são \textbf{Informática, Engenharia e Matemática}, representando 50\% das publicações. Portanto, a pesquisa está alinhada com a utilização de conceitos matemáticos básicos para realizar uma estimativa do número de artigos que podem ser eliminados. Estima-se que cerca de 481 artigos possam ser excluídos, porém essa estim
	
	ativa não possui uma base sólida. Utilizando o software Mendeley Desktop para obter o número exato de artigos sem duplicatas, chegou-se a um total de 308 artigos.
	
\end{enumerate}

\subsection{Resultados da Busca de Revis\~ao}\label{subesec:resul da revisão}


Nesta seção, são apresentados os resultados da pesquisa, utilizando um software para melhor aproveitamento de cada banco de dados utilizado no trabalho. Inicialmente, é realizada uma análise no \textit{software VOSviewer}.

\begin{figure}[!htb]
	\centering
	\caption{Palavras-chave mais populares na Scopus}
	\label{fig:scopus-09-08}
	\includegraphics[width=0.8\linewidth]{Revisao/Figuras/"scopus 09-08"}
	
	\fonte{Elaboração própria a partir de dados da Scopus (2016 a 2022)}
\end{figure}

A Figura \ref{fig:scopus-09-08} mostra uma lista das palavras mais frequentemente utilizadas como sinônimos ou em conjunto com "time series analysis" nos artigos. A análise da base de dados do Scopus é feita com uma ferramenta que exibe as palavras-chave relacionadas em cada campo de busca, proporcionando uma visão abrangente das correlações com as palavras-chave principais.

Nesse primeiro momento, são obtidas 3.484 palavras-chave, sendo que 212 delas atingem o limite estabelecido. É importante destacar que as palavras-chave base utilizadas são ``\textit{time series forecasting and time series analysis}'' no Scopus.

\begin{figure}[!htb]
	\centering
	\caption{Palavras-chave mais populares na Web of Science}
	\label{fig:web-09-08}
	\includegraphics[width=0.8\linewidth]{Revisao/Figuras/"web 09-08"}
	
	
	
	\fonte{Elaboração própria a partir de dados da Web of Science (2016 a 2022)}
\end{figure}

A análise do banco de dados Web of Science, apresentada na Figura \ref{fig:web-09-08}, também é realizada por meio de uma ferramenta que mostra as palavras-chave relacionadas em cada campo de busca. Mais uma vez, é possível obter uma visão ampla das correlações com as palavras-chave principais.

Nesse primeiro momento, são obtidas 305 palavras-chave, sendo que 13 delas atingem o limite estabelecido. É importante ressaltar que as palavras-chave base utilizadas são \textit{``time series forecasting and time series analysis''} na Web of Science.

O banco de dados Lens não é apresentado aqui, pois, embora seja uma excelente fonte, não retornou muitos resultados na pesquisa realizada. O site do Lens retorna apenas 11 artigos com os filtros aplicados. Na \ref{etp:rev-1} apresenta o campo de busca utilizado nessa pesquisa, resultando nos 11 artigos encontrados.


\begin{table}[!htb]
	\caption{Cruzamento de palavras-chave através da aplicação de filtros de ano e de linguagem}\label{tb1}
	\centering
	\begin{tabular}{@{}cp{2cm}p{1cm}p{2cm}p{1cm}p{2cm}p{2cm}p{2cm}@{}}
		\toprule
		Bases                             & \multicolumn{5}{c}{Palavras Chaves}                                                         & Resultado \\ \midrule
		\multirow{2}{*}{Scopus}           & time   series forecasting & AND & time   series analysis    &     &                         & 490       \\
		& nonlinear forecasting     & AND & time   series forecasting &     &                         & 8         \\
		\multirow{2}{*}{Web   of Science} & time   series forecasting & AND & time   series analysis    &     &                         & 126       \\
		& nonlinear forecasting     & AND & time   series forecasting &     &                         & 14        \\
		Lens                              & time   series forecasting & AND & time   series analysis    & AND & nonlinear   forecasting & 11        \\
		\multicolumn{6}{c}{Total}                                                                                                       & 649       \\ \bottomrule
	\end{tabular}
	
	\fonte{Elaboração própria a partir de dados da Scopus, Lens e Web of Science (2016 a 2022)}
\end{table}


A Tabela \ref{tb1} apresenta as palavras-chave utilizadas em cada base de dados, juntamente com o número de artigos encontrados inicialmente. No entanto, é importante ressaltar que esses dados ainda não foram processados para remover duplicatas. Após a utilização do \textit{software Mendeley} para eliminar as duplicações, restam 308 artigos únicos, os quais serão considerados nesta revisão.


\begin{figure}[htp!]
	\centering
	\caption{Analise das quantidades de artigos em relação aos anos.}
	\label{fig:regressao-linear-dos-artigos-baseados-nos-anos}
	\includegraphics[width=0.9\linewidth]{Revisao/Figuras/"regressão linear dos artigos baseados nos anos"}
	
	\fonte{Elaboração própria a partir de dados da SANEPAR (2018 a 2020)}
\end{figure}


A Figura \ref{fig:regressao-linear-dos-artigos-baseados-nos-anos} apresenta um gráfico que ilustra a relação entre o número de artigos publicados e os anos correspondentes. Foi realizada uma análise utilizando regressão linear para examinar essa relação ao longo do tempo.

A equação de regressão linear obtida é a seguinte:

\begin{eqnarray}
	y(x) &=& 8,3571x - 16,803 \quad \text{com } R^2 = 0,3062\label{eq1}
\end{eqnarray}

Na equação \eqref{eq1}, $y(x)$ representa a equação da reta, onde $x$ é a variável independente que corresponde aos anos. O coeficiente angular da reta é de $ 8,3571$, e o coeficiente linear é de -16.803, indicando o ponto de intersecção com o eixo $y$.

O coeficiente de determinação, $R^2$, é utilizado para avaliar a proporção da variação na variável dependente (número de artigos) que pode ser explicada pela variação na variável independente (anos). Nesse caso, o valor de $R^2$ é de $0,3062$, o que indica que aproximadamente $30,62\%$ da variação nos números de artigos pode ser explicada pela passagem do tempo.

O coeficiente de determinação mede a relação entre a variável dependente e as variáveis independentes, representando a porcentagem da variação explicada pela regressão em relação à variação total. Quando $R^2$ é igual a 1, todos os pontos observados estão exatamente na reta de regressão, indicando um ajuste perfeito, ou seja, todas as variações em $y$ são totalmente explicadas pela variação em $x_n$ através da função especificada, sem desvios em torno da função estimada. Por outro lado, quando $R^2$ é igual a 0, conclui-se que as variações em $y$ são exclusivamente aleatórias e a inclusão das variáveis $x_n$ no modelo não fornece nenhuma informação sobre as variações em $y$.

A fórmula do coeficiente de determinação $R^2$ é dada pela equação:
\begin{equation}
	R^{2}=\frac{\left(\sum X \cdot Y-\frac{\sum X \cdot \sum Y}{n}\right)^{2}}{\left[\sum X^{2}-\frac{\left(\sum X\right)^{2}}{n}\right] \cdot\left[\sum Y^{2}-\frac{\left(\sum Y\right)^{2}}{n}\right]}=(r)^{2}\label{eq2}
\end{equation}
Na equação \eqref{eq2}, $X$ e $Y$ representam as coordenadas no plano cartesiano, como, por exemplo, o par ordenado $(x,y)$. Na análise realizada com a relação entre o número de artigos e os anos, obteve-se um valor de $R^2=30\%$, o que implica que a linha de regressão é influenciada pelo valor encontrado de $R^2$.

Embora seja uma análise simples da relação entre o número de artigos e os anos, essa é uma validação significativa para observar o teste F de significância, que deve ser sempre inferior a $5\%$, também conhecido como valor-p. Com base nesses valores, é possível analisar o significado da linha de regressão e observar que o ano de 2021 foi o ano em que a maioria dos artigos foi publicada sobre o tema das séries temporais.

\begin{table}[H]
	\centering
	\caption{Fator de impacto.}\label{tb2}
	\begin{tabular}{@{}cp{3cm}p{3cm}c@{}}
		\toprule
		Revista cientíica      & Quantidade de plubicação & Qualidade da revista & H-INDEX \\\midrule
		Neurocomputing         & 27                         & A1                     & 143     \\
		IEEE Access            & 18                         & A1                     & 127     \\
		Applied Soft Computing & 12                         & A1                     & 143     \\
		Energies               & 11                         & A2                     & 93      \\
		Energy                 & 11                         & A1                     & 343     \\ \bottomrule
	\end{tabular}
	
	
	\vspace{0.2cm}
	Fonte: Elaboração própria a partir de dados da Scopus, Lens e Web of Science (2016 a 2022)
\end{table}

A Tabela \ref{tb2} apresenta as revistas que mais publicam artigos sobre o tema em questão. É importante destacar que muitas dessas revistas estão localizadas fora do Brasil e têm seus nomes em inglês. No entanto, todas as revistas listadas, incluindo aquelas com um alto fator de impacto, como a categoria Q1, apresentam uma correlação significativa com as áreas de \textbf{informática, engenharia e matemática}.

Essa observação ressalta a importância dessas áreas de especialização na pesquisa sobre séries temporais, uma vez que estão fortemente representadas nas principais revistas científicas. Essas revistas desempenham um papel fundamental na disseminação do conhecimento e no avanço do campo, garantindo a qualidade e o impacto dos artigos publicados. Portanto, é valioso direcionar a atenção para essas revistas, uma vez que são reconhecidas como fontes confiáveis e respeitadas dentro da comunidade científica.


\begin{figure}[H]
	\centering
	\caption{Autores relação entre artigos publicados.	}
	\label{fig:autores-relacao-entre-artigos-publicados}
	\includegraphics[width=1\linewidth]{Revisao/Figuras/"Autores Relação entre artigos publicados"}
	\vspace{0.2cm}
	Fonte: Elaboração própria a partir de dados da Scopus (2016 a 2022)
\end{figure}

\begin{figure}[H]
	\centering
	\caption{Ligação bibliográfica entre os autores}
	\label{fig:autores}
	\includegraphics[width=1\linewidth]{Revisao/Figuras/Autores}
	
	\vspace{0.2cm}
	Fonte: Elaboração própria a partir de dados da Scopus (2016 a 2022)
\end{figure}

Em resposta à questão colocada anteriormente \eqref{questão:rev1}, foi utilizada a Figura \ref{fig:autores-relacao-entre-artigos-publicados} para visualizar de forma mais clara os autores que mais publicaram sobre o tema em análise. O gráfico apresenta um histograma que destaca os autores cujo número de publicações é maior que 4 durante o período de 2016 a 2022. Essa abordagem visa evitar a inclusão de todos os autores e destacar aqueles que tiveram uma contribuição significativa no campo, considerando o critério estabelecido de pelo menos 4 publicações. Dessa forma, é possível identificar os principais autores que se destacam nesse tópico específico, fornecendo uma visão geral da distribuição da produção científica entre os pesquisadores.


\begin{figure}[!htb]
	\centering
	\caption{Mapa mundial da publicação de artigos em todo o mundo}
	\label{fig:mapa-mundi-artigos}
	\includegraphics[width=0.87\linewidth]{Revisao/Figuras/"mapa mundi artigos"}
	
	
	\fonte{Elaboração própria a partir de dados da Scopus, Lens e Web of Sicence (2016 a 2022)}
\end{figure}

A pergunta de pesquisa \eqref{questão:rev2} foi abordada por meio da análise da Figura \ref{fig:mapa-mundi-artigos}, que apresenta os países com maior número de publicações sobre o assunto em escala, ordenados de forma decrescente. Os principais países que se destacam nessa análise são os seguintes: China, com $119$ publicações; Estados Unidos, com $67$ publicações; Índia, com $57$ publicações; Brasil, com $32$ publicações; Espanha, com $28$ publicações; Reino Unido, com $25$ publicações; Austrália, com $24$ publicações; Irã, com $18$ publicações; Malásia, com $17$ publicações; e Canadá, com $16$ publicações.

É importante ressaltar que o mapa não exibe todos os países e seus respectivos números de publicações, mas destaca aqueles com maior produção nesse contexto específico. Essa análise ajuda a identificar os países com maior contribuição científica nessa área de estudo, fornecendo insights sobre os locais onde a pesquisa sobre séries temporais tem sido mais ativa.

\begin{figure}[htpb!]
	\centering
	\caption{Áreas de aplicação do tema}
	\label{fig:areas}
	\includegraphics[width=0.8\linewidth]{Revisao/Figuras/areas}
	\vspace{0.2cm}
	
	\fonte{Elaboração própria a partir de dados da Scopus, Lens e Web of Sicence (2016 a 2022)}
\end{figure}


Para responder à pergunta de pesquisa \eqref{questão:rev3}, foi criado um gráfico circular, apresentado na Figura \ref{fig:areas}, que ilustra as áreas com maior número de publicações durante o período analisado na revisão. A Tabela \ref{tb3} complementa o gráfico, fornecendo os valores específicos de cada área e a quantidade de publicações correspondente.

O gráfico circular oferece uma representação visual clara das áreas que se destacam em termos de produção científica no campo das séries temporais. Ao examinar a tabela, é possível identificar as áreas com maior número de publicações, permitindo uma compreensão aprofundada das principais áreas de conhecimento relacionadas ao tema. Essa análise contribui para uma melhor compreensão da distribuição de publicações e áreas de pesquisa ao longo do período estudado.

\begin{table}[htpb!]
	\centering
	\caption{Áreas e seus valores respetivos de artigos em cada área.}\label{tb3}
	\begin{tabular}{@{}ll@{}}
		\toprule
		Informática                      & 240 \\ \midrule
		Engenharia                       & 174 \\
		Ciências Ambientais              & 94  \\
		Matemática                       & 67  \\
		Neurociência                     & 40  \\
		Medicina                         & 38  \\
		Ciências sociais                 & 38  \\
		Ciências dos Materias            & 34  \\
		Negócios, Gestão e Contabilidade & 33  \\
		Outros                           & 204 \\ \bottomrule
	\end{tabular}

	
	\fonte{Elaboração própria a partir de dados da Scopus, len e Web of Sicence (2016 a 2022)}
\end{table}

Na última pergunta de pesquisa, referente à \eqref{questão:rev4}, foi realizada uma investigação dos artigos mais influentes na revisão. Esses artigos retratam alguns dos métodos utilizados por renomados autores \citeonline{Golyandina2020, Kumar2021, Xie2019, Lara-Benitez2021, Ahmad2018, CarvalhoJr.2019, Tan2021, Liu2019, Liu2021, Rossi2018, Soyer, Martinovic2020a, Ursu2016, Wang2016, Shih2019a, Moon2019, Chou2018, Bergmeir2018, Boroojeni2017, Chou2018a, Coelho2017, Du2020, Sadaei2019, Salgotra2020, Tyralis2017, Vlachas2020, Yang2019a, Shen2020, Sezer2020, Chen2018, Buyuksahin2019, Li2020, Kulshreshtha2020, Samanta2020, Xu2019, Graff2017, Taieb2016}.

Esses artigos abordam diferentes métodos usados pelos autores para previsão de séries temporais e análise não-linear dessas previsões. Eles representam contribuições significativas para o avanço do conhecimento e aplicação prática das séries temporais, oferecendo insights valiosos sobre abordagens eficazes nesse campo. Ao incluir esses estudos influentes na análise, obtém-se uma visão abrangente dos métodos e técnicas mais relevantes na previsão de séries temporais.

No estudo conduzido por \citeonline{Xu2019}, um modelo híbrido foi proposto, combinando o modelo linear AR e LR com o modelo não-linear ARIMA e o modelo DBN. Essa abordagem permite capturar tanto os comportamentos lineares quanto os não-lineares de uma série temporal. Por outro lado, \citeonline{Li2020} comparou o desempenho de previsão da abordagem MAELS com outros modelos de aprendizado de máquina de última geração, como ANN, CNN, RNN, LSTM, GRU, Transformer, Prophet ARIMA e SVM-VAR. As abordagens ANN, CNN, RNN, GRU, Transformer e LSTM são capazes de lidar com dados multivariados de entrada e saída, enquanto o ARIMA utiliza informações passadas para prever o futuro com base em características como autocorrelação e médias móveis.

Dessa forma, por meio dessa revisão sistemática e análise de conteúdo, a pergunta de pesquisa formulada no início do capítulo foi respondida.
Além desses modelos mencionados, também será utilizada a versão atualizada do ARIMA nesta dissertação. Os modelos SARIMA e SARIMAX também serão comparados para determinar qual deles é o mais adequado. Além disso, serão empregados os modelos Light GBM e XGBoost. Quanto às métricas de erro, serão utilizadas MAE, sMAPE e RRMSE, que são amplamente adotadas na literatura. O coeficiente de determinação ($R^2$), mencionado na equação \eqref{eq2}, não é tão comumente utilizado para comparação de modelos de previsão futura.

\subsection{Principais conclus\~ao} \label{subsec:conclusão da revisão}

A conclusão abrangente da pesquisa de revisão revela que diversas bases de dados foram consultadas, como Scopus, Web of Science e Lens. Cada uma dessas bases proporcionou uma quantidade significativa de artigos relevantes, que foram minuciosamente analisados. Essa abordagem permitiu responder à pergunta de pesquisa formulada no início da revisão.

Apesar da base de dados Lens ser menor em comparação com as demais, também foram encontrados artigos relevantes que contribuíram para enriquecer o processo de dissertação. Além disso, o uso de software especializado foi essencial para lidar com a grande quantidade de artigos e suas inter-relações.

No âmbito específico da revisão sistemática, a análise de séries temporais recebeu uma ênfase particular, com um enfoque aprofundado e atualizado nos últimos seis anos. Os resultados obtidos foram altamente relevantes e significativos. Por meio do cruzamento de palavras-chave e da aplicação de filtros específicos, foram selecionados 308 artigos publicados entre 2016 e 2022.

Com o objetivo de refinar ainda mais a análise, foi realizado um filtro adicional com base em áreas de interesse, como matemática, engenharia e informática. Isso resultou na seleção de 481 artigos relacionados a essas áreas, excluindo aqueles de outras áreas não pertinentes.

A pesquisa de revisão realizada foi minuciosa e abrangente, proporcionando uma base sólida de artigos relevantes para o desenvolvimento da dissertação. Os resultados obtidos foram fundamentais para orientar as próximas etapas do trabalho e para alcançar uma compreensão aprofundada do tema das séries temporais.









