\subsection{Principais conclus\~ao} \label{subsec:conclusão da revisão}

A conclusão abrangente da pesquisa de revisão revela que diversas bases de dados foram consultadas, como Scopus, Web of Science e Lens. Cada uma dessas bases proporcionou uma quantidade significativa de artigos relevantes, que foram minuciosamente analisados. Essa abordagem permitiu responder à pergunta de pesquisa formulada no início da revisão.

Apesar da base de dados Lens ser menor em comparação com as demais, também foram encontrados artigos relevantes que contribuíram para enriquecer o processo de dissertação. Além disso, o uso de software especializado foi essencial para lidar com a grande quantidade de artigos e suas inter-relações.

No âmbito específico da revisão sistemática, a análise de séries temporais recebeu uma ênfase particular, com um enfoque aprofundado e atualizado nos últimos seis anos. Os resultados obtidos foram altamente relevantes e significativos. Por meio do cruzamento de palavras-chave e da aplicação de filtros específicos, foram selecionados 308 artigos publicados entre 2016 e 2022.

Com o objetivo de refinar ainda mais a análise, foi realizado um filtro adicional com base em áreas de interesse, como matemática, engenharia e informática. Isso resultou na seleção de 481 artigos relacionados a essas áreas, excluindo aqueles de outras áreas não pertinentes.

A pesquisa de revisão realizada foi minuciosa e abrangente, proporcionando uma base sólida de artigos relevantes para o desenvolvimento da dissertação. Os resultados obtidos foram fundamentais para orientar as próximas etapas do trabalho e para alcançar uma compreensão aprofundada do tema das séries temporais.