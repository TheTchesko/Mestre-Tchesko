\subsection{Principais conclus\~ao} \label{subsec:conclusão da revisão}

A pesquisa de revisão foi minuciosamente conduzida, abrangendo uma variedade de bases de dados, como Scopus, Web of Science e Lens. Cada uma dessas bases proporcionou uma quantidade significativa de artigos relevantes, os quais foram cuidadosamente analisados. Essa abordagem rigorosa permitiu que a pergunta de pesquisa formulada no início da revisão fosse respondida.

Embora a base de dados Lens seja menor em comparação com as demais, também foram encontrados artigos relevantes que contribuíram para enriquecer o processo de dissertação. Além disso, o uso de software especializado desempenhou um papel crucial ao lidar com a grande quantidade de artigos e suas inter-relações.
No contexto específico da revisão sistemática, deu-se uma ênfase particular à análise de séries temporais, com uma abordagem aprofundada e atualizada nos últimos seis anos. Os resultados obtidos foram altamente relevantes e significativos. Por meio do cruzamento de palavras-chave e da aplicação de filtros específicos, foram selecionados $308$ artigos publicados entre $2016$ a $2022$.

Com o objetivo de aprimorar ainda mais a análise, realizou-se um filtro adicional com base em áreas de interesse, como matemática, engenharia e informática. Isso resultou na seleção de 481 artigos relacionados a essas áreas, excluindo aqueles de outras áreas não pertinentes.
A pesquisa de revisão realizada foi minuciosa e abrangente, proporcionando uma base sólida de artigos relevantes para o desenvolvimento da dissertação. Os resultados obtidos foram fundamentais para orientar as próximas etapas do trabalho e alcançar uma compreensão aprofundada do tema das séries temporais.