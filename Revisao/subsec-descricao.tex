\subsection{Descri\c c\~ao do problema} \label{subsec:descricao}

Esta subseção discute as variáveis no conjunto de dados e como elas serão previstas.

\begin{itemize}
\item Bombas de sucção (B1, B2 e B3) – valor máximo da frequência 60 Hz

\item[] Variáveis importantes: Fluxo, pressão e nível

\item Nível do Reservatório (Câmara 1) LT01 $ (m^3) $ - \textbf{PREVER}

\item Vazão de entrada (FT01) $ (m^3/h) $

\item Vazão de gravidade (FT02) $ (m^3/h) $

\item Vazão de recalque (FT03) $ (m^3/h) $

\item Pressão de Sucção (PT01SU) (mca)

\item Pressão de Recalque (PT02RBAL) (mca)
\end{itemize}

Na pesquisa será utilizada a variável LT01, que é o nível do reservatório, este nível é de grande importância como mostram as Figuras \ref{fig:dados-todos} e \ref{fig:2020-a-frente}. 



\begin{figure}[H]
	\centering
	\caption{Dados completos com frequência média de 24h}
	\label{fig:dados-todos}
	\includegraphics[width=1\linewidth]{"Introducao/Figuras/dados todos"}
	
	Fonte: Elaboração própria a partir de dados da SANEPAR (2018 a 2020)
\end{figure}

\begin{figure}[H]
	\centering
	\caption{Plotagem dos dados para o ano 2020}
	\label{fig:2020-a-frente}
	\includegraphics[width=1\linewidth]{"Introducao/Figuras/2020 a frente"}
	
	Fonte: Elaboração própria a partir de dados da SANEPAR (2018 a 2020)
\end{figure}

Os dados coletados têm o tamanho de 26306 linhas × 9 colunas, para tanta relação que será usada nos modelos da subseção \ref{subsec:metod} para prever e analisar as anomalias apresentadas nas Figuras \ref{fig:dados-todos} e \ref{fig:2020-a-frente}.






