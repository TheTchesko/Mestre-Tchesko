 

\subsection{Detec\c cão de anomalias} \label{subsec:detec}



A detecção de anomalias em séries temporais representa um desafio significativo para os previsores, pois requer habilidade em identificar mudanças nos dados, mesmo quando não estão claramente evidentes. Nesse contexto, a coleta de dados realizada ao longo do tempo pela empresa SANEPAR revela anomalias mais expressivas do que inicialmente imaginado. A escassez de água que afetou a cidade de Curitiba se prolongou por vários dias, como é evidenciado pelos gráficos de linha utilizados na etapa de trabalho mencionada (\ref{etp:1}). Esses gráficos oferecem uma representação visual clara das variações nos níveis de água ao longo do tempo, auxiliando na compreensão da extensão do problema e na necessidade de uma abordagem adequada.

\begin{figure}[htp!]
	\centering
	\caption{Dados completos com uma frequência média de 24 horas}
	\label{fig:dados-todos}
	\includegraphics[width=0.9\linewidth]{"Introducao/Figuras/dados todos"}
	
	Fonte: Elaboração própria a partir de dados da SANEPAR (2018 a 2020)
\end{figure}

\begin{figure}[htp!]
	\centering
	\caption{Plotagem de dados para o ano de 2020}
	\label{fig:2020-a-frente}
	\includegraphics[width=0.9\linewidth]{"Introducao/Figuras/2020 a frente"}
	
	Fonte: Elaboração própria a partir de dados da SANEPAR (2018 a 2020)
\end{figure}



As Figuras \ref{fig:dados-todos} e \ref{fig:2020-a-frente} apresentadas ilustram visualmente as variações e padrões observados nos dados ao longo do tempo, destacando a importância de explorá-los de maneira apropriada a fim de compreender as anomalias e embasar a tomada de decisões. Os dados coletados possuem uma dimensão de $26.306$ linhas e $9$ colunas, e essa ampla quantidade de dados será utilizada nos modelos descritos na subseção mencionada para que seja possível prever e analisar as anomalias evidenciadas. Essas análises permitirão uma melhor compreensão das anomalias e orientarão as decisões tomadas.






