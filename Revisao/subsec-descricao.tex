 

\subsection{Detec\c cão de anomalias} \label{subsec:detec}


As anomalias em séries temporais é um desafio muito grande para os previsores saber quando os dados sofrem uma mudança se ela não estiver muito evidente nos dados é um exercício de concentração, com isso os dados coletados ao longo do tempo pela empresa SANEPAR traz as anomalias mais claras do que se imaginava, pois a falta de água que sofreu a cidade de Curitiba se alastrou por dias logo a frente é mostrado os gráficos de linha utilizados para a \ref{etp:1} do trabalho em questão.

\begin{figure}[H]
	\centering
	\caption{Dados completos com frequência média de 24h}
	\label{fig:dados-todos}
	\includegraphics[width=1\linewidth]{"Introducao/Figuras/dados todos"}
	
	Fonte: Elaboração própria a partir de dados da SANEPAR (2018 a 2020)
\end{figure}

\begin{figure}[H]
	\centering
	\caption{Plotagem dos dados para o ano 2020}
	\label{fig:2020-a-frente}
	\includegraphics[width=1\linewidth]{"Introducao/Figuras/2020 a frente"}
	
	Fonte: Elaboração própria a partir de dados da SANEPAR (2018 a 2020)
\end{figure}

Os dados coletados têm o tamanho de 26306 linhas × 9 colunas, para tanta relação que será usada nos modelos da subseção \ref{subsec:metod} para prever e analisar as anomalias apresentadas nas Figuras \ref{fig:dados-todos} e \ref{fig:2020-a-frente}.






